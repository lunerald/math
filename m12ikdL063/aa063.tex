\documentclass[a4paper]{article}
\usepackage[pdftex]{graphicx}
\usepackage[utf8]{inputenc}
\usepackage{enumerate}
\usepackage{amssymb}
\usepackage{href-ul}
\hypersetup{
	colorlinks=true,
	linkcolor=blue,
	urlcolor=blue}
\usepackage{geometry}
\geometry{a4paper, top=15mm, left=15mm, right=15mm, bottom=15mm,
	headsep=10mm, footskip=12mm}

\begin{document}
{\bf Kurvendiskussion mit Integralrechnung}
\begin{enumerate}[1.]

%Aufgabe 1
{\bf \item In folgender Abbildung ist der Graph $G_f$ einer ganzrationalen Funktion vierten Grades zu sehen.} Der Graph ist achsensymmetrisch zur Y-Achse, geht durch den Ursprung und besitzt in $H(1|1)$ einen Hochpunkt

\begin{minipage}{0.65\textwidth}
\begin{enumerate}[a)]
	\item Bestimmen Sie den Funktionsterm $f(x)$. [ Ergebnis: $f(x) = -x^4+ 2x^2 $ ]
	\item Berechnen Sie den Inhalt der eingeschlossenen Fläche zwischen dem Graphen von $f$ und der x-Achse.
	\item Berechnen Sie den Inhalt der eingeschlossenen Fläche zwischen dem Graphen von $f$ und der Geraden durch die beiden Hochpunkte .
\end{enumerate}
\end{minipage}
\hfill
\begin{minipage}{0.3\textwidth}
\includegraphics[width=5.5 cm]{vier063}
\end{minipage}



{\bf \item Aufgabe }
\begin{enumerate}[a)]
\item Skizzieren Sie den Graphen der Funktion $h(x) = e^{x-1} - 2 $ möglichst genau und beschreiben Sie, wie er geometrisch aus dem Graphen der Funktion $ y = e^x $ hervorgeht.
\item Geben Sie eine mögliche Stammfunktion $H$ der Funktion $h$ an.
\end{enumerate}


\begin{minipage}{0.65\textwidth}
{\bf \item Gegeben ist die Funktion $ f(x) = x \cdot e^{-x} $,
	$ D_f = \mathbb{R}$ sowie ihr Graph $G_f$.}
\begin{enumerate}[a)]
\item 
Zeigen Sie, dass $F(x) = - e^{-x} \cdot (x+1) $ eine Stammfunktion von $ f(x) $ ist!
\item
 Der Graph $G_f$, die x-Achse und die Gerade mit der Gleichung $ x= z \quad ( z > 0 ) $ schließen ein Flächenstück mit dem Inhalt $ A(z) $ ein. Ermitteln Sie den Flächeninhalt $ A(z) $ und untersuchen Sie, ob sich für $ z \rightarrow \infty $ ein endlicher Wert für den Flächeninhalt $ A(z) $ ergibt.
\end{enumerate}
\end{minipage}
\hfill
\begin{minipage}{0.3\textwidth}
\includegraphics[width=5.5 cm]{inte063}
\end{minipage}

{ \bf \item Aufgabe} 
\begin{enumerate}[a)]
\item Bestimmen Sie die exakten Werte für $ a \in [ 0 ; 2 \pi ] $ so, dass 
$ \int \limits_{0}^{a} 4 \cos{x}\, {\rm d} x = 2 $ ist.
\item Berechnen Sie den Wert des bestimmten Integrales $ \int \limits_{0}^{2} \sin{(2x)} {\rm d} x $ .
Warum stimmt der Wert dieses Integrals nicht mit dem Inhalt der Fläche überein, die für 
$ 0 \le x \le 2 $ zwischen dem zugehörigen Graphen und der x-Achse liegt?
\end{enumerate}


\begin{minipage}{0.65\textwidth}
{\bf \item  Nebenstehend sind der Graph der Funktion} $ f(x) = x^2 $ im Intervall $[ 0 ; 1 ]$ und eine
Parallele zur x-Achse durch den Punkt $ R ( k | k^2 ) $ des Graphen gezeichnet. Berechnen Sie den Wert für k, damit die Teilflächen $ A $ und $ B $ gleich groß sind!
\end{minipage}
\hfill
\begin{minipage}{0.3\textwidth}
\includegraphics[width=5.5 cm]{pafl063.pdf}
\end{minipage}


{\bf \item Gegeben ist die Funktion $f(x)= -x^2+3x$}

\begin{enumerate}[a)]
\item Bestimmen Sie die Nullstellen und die Koordinaten des Hochpunkts von $G_f$. Tragen Sie den Graphen von $f$ in ein Koordinatensystem ein (Einheit 1 cm).
\item Bestimmen Sie den Inhalt des Flächenstücks, das von der positiven x-Achse und dem Graphen von $f$ eingeschlossen wird.
\item Die Gerade $y=x$ zerlegt obiges Flächenstück in zwei Teile. Berechnen Sie das Verhältnis der beiden Teilstücke.
\item Geben Sie für die Integralfunktion $I(x)= \int\limits_0^xf(t){\rm d} t$ das Krümmumgsverhalten an und bestimmen Sie die Koordinaten des Wendepunkts.
\end{enumerate}

{\bf \item Ermitteln Sie die unbestimmten Integrale:}
$$ \int {6x \over x^2 +4} {\rm d} x \qquad \int 2x\cdot e^{x^2}{\rm d} x \qquad \int 3\cdot \cos(6x){\rm d} x $$

{\bf \item Gegeben ist die Funktion $f(x)={1\over 4}x^3 -3x^2+9x$ mit $x \in \mathbb{R}$. $G_f$ bezeichne den Graphen der Funktion $f$.}

\begin{enumerate}[a)]
\item Berechnen Sie die Art und die Lage der Extrempunkte von $G_f$. 
\item Weisen Sie nach, dass $G_f$ einen Wendepunkt besitzt und bestimmen Sie dessen Koordinaten und die Gleichung der zugehörigen Wedetangente.
\item Bestimmen Sie die Gleichung der Geraden $g$, die durch beide Extrempunkte verläuft. Fertigen Sie eine Skizze des Graphen von $f$ mit der Geraden $g$ an  und berechnen Sie den von $G_f$ und der Geraden $g$ eingeschlossenen Flächeninhalt.
\end{enumerate}

\end{enumerate}      

\fbox{
	\begin{minipage}{0.5\textwidth}
		Zur Lösung bitte \href{https://www.okuyakl.de/math/m12ikdL063/ll063.pdf}{hier klicken} oder den QR-Code scannen.\\
	Weitere Arbeitsblätter gibt es unter 
	
	\href{https://www.okuyakl.de}{www.okuyakl.de}
	\end{minipage}
	\hfill
	\begin{minipage}{0.4\textwidth}
		\includegraphics[width=1.5 cm]{../../viecher/zwe03}
		\includegraphics[width=3 cm]{qr063}
		\includegraphics[width=2 cm]{../../viecher/afanticon1}
		
	\end{minipage}}

\end{document}%Lösung-------------------------------------------
