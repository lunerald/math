\documentclass[a4paper]{article}
\usepackage[pdftex]{graphicx}
\usepackage[utf8]{inputenc}
\usepackage{enumerate}
\usepackage{amssymb}
\usepackage{href-ul}
\hypersetup{
	colorlinks=true,
	linkcolor=blue,
	urlcolor=blue}
\usepackage{geometry}
\geometry{a4paper, top=15mm, left=15mm, right=15mm, bottom=15mm,
	headsep=10mm, footskip=12mm}

\begin{document}
	{\bf Curve discussion with integral calculus}
	\begin{enumerate}[1.]
		
		%Task 1
		{\bf \item The following figure shows the graph $G_f$ of a completely rational function of the fourth degree.} The graph is axisymmetric to the Y-axis, goes through the origin and has a high point in $H(1|1)$
		
		\begin{minipage}{0.65\textwidth}
			\begin{enumerate}[a)]
				\item Determine the function term $f(x)$. [ Result: $f(x) = -x^4+ 2x^2 $ ]
				\item Calculate the contents of the enclosed area between the graph of $f$ and the x-axis.
				\item Calculate the contents of the enclosed area between the graph of $f$ and the line through the two high points.
			\end{enumerate}
		\end{minipage}
		\hfill
		\begin{minipage}{0.3\textwidth}
			\includegraphics[width=5.5 cm]{vier063}
		\end{minipage}
		
		
		
		{\bf \item Task }
		\begin{enumerate}[a)]
			\item Sketch the graph of the function $h(x) = e^{x-1} - 2 $ as accurately as possible and describe how it emerges geometrically from the graph of the function $ y = e^x $.
			\item Specify a possible antiderivative $H$ of the function $h$.
		\end{enumerate}
		
		
		\begin{minipage}{0.65\textwidth}
			{\bf \item The function $ f(x) = x \cdot e^{-x} $ is given,
				$ D_f = \mathbb{R}$ and its graph $G_f$.}
			\begin{enumerate}[a)]
				\item
				Show that $F(x) = - e^{-x} \cdot (x+1) $ is an antiderivative of $ f(x) $!
				\item
				The graph $G_f$, the x-axis and the line with the equation $ x= z \quad ( z > 0 ) $ enclose an area with the content $ A(z) $. Determine the area $ A(z) $ and examine whether $ z \rightarrow \infty $ results in a finite value for the area $ A(z) $.
			\end{enumerate}
		\end{minipage}
		\hfill
		\begin{minipage}{0.3\textwidth}
			\includegraphics[width=5.5 cm]{inte063}
		\end{minipage}
		
		{ \bf \item Task}
		\begin{enumerate}[a)]
			\item Determine the exact values for $ a \in [ 0 ; 2 \pi ] $ such that
			$ \int \limits_{0}^{a} 4 \cos{x}\, {\rm d} x = 2 $.
			\item Calculate the value of the definite integral $ \int \limits_{0}^{2} \sin{(2x)} {\rm d} x $ .
			Why does the value of this integral not agree with the content of the area that is for
			$ 0 \le x \le 2 $ lies between the corresponding graph and the x-axis?
		\end{enumerate}
		
		
		\begin{minipage}{0.65\textwidth}
			{\bf \item Below is the graph of the function} $ f(x) = x^2 $ in the interval $[ 0 ; 1 ]$ and one
			Parallel to the x-axis is drawn through the point $ R ( k | k^2 ) $ of the graph. Calculate the value for k so that the areas $ A $ and $ B $ are the same size!
		\end{minipage}
		\hfill
		\begin{minipage}{0.3\textwidth}
			\includegraphics[width=5.5 cm]{pafl063.pdf}
		\end{minipage}
		
		
		{\bf \item Given is the function $f(x)= -x^2+3x$}
		
		\begin{enumerate}[a)]
			\item Determine the zeros and the coordinates of the high point of $G_f$. Plot the graph of $f$ in a coordinate system (unit 1 cm).
			\item Determine the contents of the area enclosed by the positive x-axis and the graph of $f$.
			\item The line $y=x$ divides the above area into two parts. Calculate the ratio of the two pieces.
			\item Specify the curvature behavior for the integral function $I(x)= \int\limits_0^xf(t){\rm d} t$ and determine the coordinates of the turning point.
		\end{enumerate}
		
		{\bf \item Find the indefinite integrals:}
		$$ \int {6x \over x^2 +4} {\rm d} x \qquad \int 2x\cdot e^{x^2}{\rm d} x \qquad \int 3\cdot \cos( 6x){\rm d} x $$
		
		{\bf \item Given is the function $f(x)={1\over 4}x^3 -3x^2+9x$ with $x \in \mathbb{R}$. Let $G_f$ denote the graph of the function $f$.}
		
		\begin{enumerate}[a)]
			\item Calculate the type and location of the extreme points of $G_f$.
			\item Prove that $G_f$ has a turning point and determine its coordinates and the equation of the corresponding turning tangent.
			\item Determine the equation of the line $g$ that passes through both extreme points. Make a sketch of the graph of $f$ with the line $g$ and calculate the area enclosed by $G_f$ and the line $g$.
		\end{enumerate}
		
	\end{enumerate}
	
	\fbox{
		\begin{minipage}{0.5\textwidth}
			For the solution, please \href{https://www.okuyakl.de/math/m12ikdL063/le063.pdf}{click here} or scan the QR code.\\
			Additional worksheets are available at
			
			\href{http://www.okuyakl.com}{www.okuyakl.com}
		\end{minipage}
		\hfill
		\begin{minipage}{0.4\textwidth}
			\includegraphics[width=1.5 cm]{../../viecher/zwe03}
			\includegraphics[width=3 cm]{qre063}
			\includegraphics[width=2 cm]{../../viecher/afanticon1}
			
	\end{minipage}}
	
\end{document}