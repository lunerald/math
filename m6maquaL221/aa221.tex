
\documentclass[a4paper]{article}
\usepackage[pdftex]{graphicx}
\usepackage[utf8]{inputenc}
\usepackage{enumerate}
\usepackage{amssymb}
\usepackage{geometry}
\geometry{a4paper, top=15mm, left=15mm, right=15mm, bottom=15mm,
	headsep=10mm, footskip=12mm}
\usepackage{href-ul}
\hypersetup{
	colorlinks=true,
	linkcolor=blue,
	urlcolor=blue}

\begin{document}

{\bf Magische Quadrate }\\
Magische Quadrate haben die Eigenschaft, dass die Summe der Brüche in jeder Zeile, Spalte und Diagonale immer gleich 1 ist. Berechne durch Additionen und Subtraktionen die fehlenden Werte. Trage farbig die fehlenden Werte in die Leerfelder ein.
\vspace{20 pt}

\begin{minipage}{0.48\textwidth}

\renewcommand{\arraystretch}{3}
\begin{tabular}{|p{20 pt}|p{20 pt}|p{20 pt}|}
\hline
 \huge ${2 \over 5}$ & &   \huge ${2 \over 15}$  \\
\hline
&  \huge ${1 \over 3}$ & \\
\hline
& & \\
\hline
\end{tabular}
\end{minipage}
\hfill
\begin{minipage}{0.48\textwidth}

\renewcommand{\arraystretch}{3}
\begin{tabular}{|p{20 pt}|p{20 pt}|p{20 pt}|}
\hline
  \huge ${4 \over 9}$ & &    \\
\hline
&  \huge ${1 \over 3}$ & \\
\hline
  \huge ${1 \over 2}$& & \\
\hline
\end{tabular}
\end{minipage}

\vspace{20 pt}

\begin{minipage}{0.48\textwidth}

\renewcommand{\arraystretch}{3}
\begin{tabular}{|p{20 pt}|p{20 pt}|p{20 pt}|p{20 pt}|}
\hline
 & \huge ${3 \over 7}$ &   & \huge ${3 \over 14}$  \\
\hline
 \huge ${4 \over 21}$ &  & \huge ${5 \over 14}$ &  \huge ${1 \over 3}$\\
\hline
&  &  \huge ${5 \over 21}$ & \\
\hline
\huge ${1 \over 7}$  & &  \huge ${13 \over 42}$& \\		
\hline
\end{tabular}
\end{minipage}
\hfill
\begin{minipage}{0.48\textwidth}

\renewcommand{\arraystretch}{3}
\begin{tabular}{|p{20 pt}|p{20 pt}|p{20 pt}|p{20 pt}|}
\hline
  & \huge ${13 \over 50}$ &  & \huge ${2 \over 5}$ \\
\hline
\huge ${9 \over 25}$ & & &  \huge ${3 \over 25}$ \\
\hline
  &  \huge ${1 \over 5}$ & \huge ${7 \over 25}$& \\
\hline
&  \huge ${8 \over 25}$ &  \huge ${6 \over 25}$ & \\
\hline
\end{tabular}
\end{minipage}

\fbox{
	\begin{minipage}{0.5\textwidth}
		Zur Lösung bitte \href{https://www.okuyakl.de/math/m6maquaL221/ll221.pdf}{hier klicken} oder den QR-Code scannen.\\
	Weitere Arbeitsblätter gibt es unter 
	
	\href{https://www.okuyakl.de}{www.okuyakl.de}
	\end{minipage}
	\hfill
	\begin{minipage}{0.4\textwidth}
		\includegraphics[width=1.5 cm]{../../viecher/zwe03}
		\includegraphics[width=3 cm]{qr221}
		\includegraphics[width=2 cm]{../../viecher/afanticon1}
		
	\end{minipage}}

\end{document}%L L Ö S U N G -----------------------------------------
