\documentclass[a4paper]{article}
\usepackage[pdftex]{graphicx}
\usepackage[utf8]{inputenc}
\usepackage{enumerate}
\usepackage{amssymb}
\usepackage{href-ul}
\hypersetup{
	colorlinks=true,
	linkcolor=blue,
	urlcolor=blue}
\usepackage{geometry}
\geometry{a4paper, top=15mm, left=15mm, right=15mm, bottom=15mm,
	headsep=10mm, footskip=12mm}

\begin{document}
{\bf Vektoraddition, skalare Multiplikation und Skalarprodukt}
\begin{enumerate}[1.]
{\bf \item Berechnen Sie folgende Terme}

\renewcommand{\arraystretch}{1}
\begin{tabular}{lll}
	a) $\left(\begin{array}{c} 0 \\ 8 \\ 15 \end{array} \right)-\left(\begin{array}{c} 4 \\ -3 \\ 1 \end{array} \right) $ & b) $2\cdot \left(\begin{array}{c} 2 \\0,5\\ -7\end{array} \right)+\left(\begin{array}{c} -4 \\ 1\\ 3 \end{array} \right) $ & c) $ \left(\begin{array}{c} 8 \\ 7 \\ 2\end{array} \right) - 3\cdot \left(\begin{array}{c} -1 \\ 6 \\ 2 \end{array} \right)$ \\
\end{tabular}

{\bf \item Berechnen Sie den unbekannten Vektor}

\renewcommand{\arraystretch}{1}
\begin{tabular}{lll}
	a) $\left(\begin{array}{c} 3 \\ -2 \\ 6 \end{array} \right)+ \vec{x} = \left(\begin{array}{c} 12 \\ 8 \\ -1 \end{array} \right) $ & b) $2\cdot \vec{x} -\left(\begin{array}{c} -4 \\ 10\\ -7\end{array} \right)=-\left[\left(\begin{array}{c} -1 \\ 0\\ 3 \end{array} \right)- \left(\begin{array}{c} 2 \\ 9\\ 7 \end{array} \right)\right]$ & c) $ \left(\begin{array}{c} 0,5 \\ 2 \\ -6\end{array} \right) - {1\over 2}\cdot \vec{x} = \vec{0}$ \\
\end{tabular}
{\bf \item Berechnen Sie die fehlenden Punkte (M=Mittelpunkt)}

\begin{enumerate}[a)]
	\item $A(3|-4|9);\quad B(-1|0|1);\quad M(?|?|?)$
	\item $P(7|12|-2);\quad Q(6|-2|3);\quad M(?|?|?) $
	\item $C(-1|5|2);\quad D(?|?|?);\quad M(3|3|3)$
	\item $G(?|?|?); \quad H(0|2|4);\quad M(-7|2|0)$
\end{enumerate}

{\bf \item $A(3|2|-1)$, $B(-1|4|-7)$, und $C(8|-3|0)$ bilden ein Dreieck. Veranschaulichen Sie anhand einer Skizze} (ohne Koordinatensystem) die drei Möglichkeiten, dieses Dreieck mit einem vierten Punkt $D_1$, $D_2$ und $D_3$ zu je einem Parallelogramm zu ergänzen und berechnen Sie die Koordinaten der Punkte $D_1$, $D_2$ und $D_3$.

{\bf \item Von einem Parallelogramm $ABCD$ sind nur die Punkte} $A(0|2|0)$, $B(6|-3|2)$ und der Mittelpunkt \\ $M(3|-2|-1)$ gegeben. Berechnen Sie die Koordinaten der Punkte $C$ und $D$.

{\bf \item Die Strecke $[AB]$ mit $A(-2|1|1)$ und $B(7|-11|16)$ soll durch die Punkte $T$ und $R$} in drei gleiche Teile zerlegt werden. Berechnen Sie die Koordinaten von $T$ und $R$.

{\bf \item Bestimmen Sie die Parameter so, dass die beiden Vektoren linear abhängig sind.}

\renewcommand{\arraystretch}{1}
\begin{tabular}{lll}
	a) $\left(\begin{array}{c} 1 \\a \\ -3\end{array} \right); \left(\begin{array}{c} 3 \\ 12 \\ b\end{array} \right) $ & b) $\left(\begin{array}{c} u \\ 2 \\ 16\end{array} \right); \left(\begin{array}{c} -1,5 \\0,5 \\ v \end{array} \right) $ & c) $\left(\begin{array}{c} k \\-10 \\-2 \end{array} \right); \left(\begin{array}{c} 3 \\ 5\\ m \end{array} \right) $
\end{tabular}

{\bf \item Bestimmen Sie den Parameter so, dass die Vektoren den angegebenen Betrag haben.}

\renewcommand{\arraystretch}{1}
\begin{tabular}{lllll}
	a) $\left|\left(\begin{array}{c} -1 \\ a \\ 2 \end{array} \right)\right|= 3$ & 
	b) $\left|\left(\begin{array}{c} t \\ 5 \\ 4 \end{array} \right)\right|= 5\sqrt{2}$ &
	c) $\left|\left(\begin{array}{c} 6 \\ -2\\ s\end{array} \right)\right|= 7$ &
	d) $\left|\left(\begin{array}{c}  r \\ 0,8 \\ -0,6 \end{array} \right)\right|= 1 $ &
	e) $\left|\left(\begin{array}{c} t \\ -7 \\ t\end{array} \right)\right|= 9$ 
	    
\end{tabular}

{\bf \item Bestimmen Sie den Parameter so, dass die beiden Vektoren orthogonal sind.}

\renewcommand{\arraystretch}{1}
\begin{tabular}{lll}
	a) $\left(\begin{array}{c} v \\ v+1 \\ -1\end{array} \right); \left(\begin{array}{c} 2 \\-3 \\0\end{array} \right) $ & b) $\left(\begin{array}{c} -5 \\-2 \\u-2 \end{array} \right);\left(\begin{array}{c} 1 \\2\\ -2\end{array} \right) $ & c) $\left(\begin{array}{c} t \\3 \\2 \end{array} \right);\left(\begin{array}{c} t \\-9 \\1 \end{array} \right) $
\end{tabular}

{\bf \item Berechnen Sie die Innenwinkel des Dreiecks ABC}

\begin{enumerate}[a)]
	\item $A(0|0|0)$; \quad $B(2|2|2)$; \quad $C(4|-3|0)$
	\item $A(4|0|3)$; \quad $B(0|5|0)$; \quad $C(1|5|7)$
\end{enumerate}

\end{enumerate} 

\fbox{
	\begin{minipage}{0.5\textwidth}
		Zur Lösung bitte \href{https://www.okuyakl.de/math/m11skaL109/ll109.pdf}{hier klicken} oder den QR-Code scannen.\\
	Weitere Arbeitsblätter gibt es unter 
	
	\href{https://www.okuyakl.de}{www.okuyakl.de}
	\end{minipage}
	\hfill
	\begin{minipage}{0.4\textwidth}
		\includegraphics[width=1.5 cm]{../../viecher/zwe03}
		\includegraphics[width=3 cm]{qr109}
		\includegraphics[width=2 cm]{../../viecher/afanticon1}
		
\end{minipage}}
\end{document}