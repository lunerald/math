\documentclass[a4paper]{article}
\usepackage[pdftex]{graphicx}
\usepackage[utf8]{inputenc}
\usepackage{enumerate}
\usepackage{icomma}
\usepackage{amssymb}
\usepackage{eurosym}
\usepackage{href-ul}
\hypersetup{
	colorlinks=true,
	linkcolor=blue,
	urlcolor=blue}
\usepackage{geometry}
\geometry{a4paper, top=15mm, left=15mm, right=15mm, bottom=15mm,
	headsep=10mm, footskip=12mm}

\begin{document}
{\bf Stochastik}
\begin{enumerate}[1.]


{\bf \item Im Rahmen einer Quizsendung ist die Anzahl der von einem Kandidaten zu lösenden Aufgaben} aus dem Fachbereich Mathematik gleich der Augensumme, die von ihm bei einmaligem Würfeln zweier Spezial-Würfel erzielt wird. Die beiden Würfel tragen jeweils auf zwei Seitenflächen die Augenzahl 0, auf drei Seitenflächen die Augenzahl 1 und auf einer Seitenfläche die Augenzahl 2. 

\begin{enumerate}[a)]
\item Berechnen Sie die Wahrscheinlichkeit dafür, dass der erste Kandidat genau zwei Aufgaben aus der Mathematik lösen muss!
\item Die Zufallsgröße $ X $ beschreibt die Anzahl der von einem Kandidaten zu lösenden Aufgaben. In einer Tabelle kann die Wahrscheinlichkeit von $ X $ dargestellt werden:

\renewcommand{\arraystretch}{2}
\begin{centering}
\begin{tabular}{c|c|c|c|c|c}
$ x_i $ & 0 & 1 & 2 & 3 & 4 \\
\hline
$ P(X=x_i) $ &  &  & $ \frac{13}{36} $  &   & \\
\end{tabular}
\end{centering}\\
Ermitteln Sie die fehlenden Werte der Wahrscheinlichkeitsverteilung sowie den Erwartungswert von $ X $.


\item Um Geld für einen guten Zweck einzunehmen, bietet die Sendeleitung unter den Studiogästen ein Gewinnspiel an. Für \EUR{2} darf ein Spieler 3 Kugeln ohne Zurücklegen aus einem Behälter entnehmen, der drei rote, drei grüne und drei blaue Kugeln enthält. Haben die drei Kugeln gleiche Farbe, so gewinnt der Spieler einen bestimmten Geldbetrag $ B $, ansonsten verliert er und erhält keine Auszahlung. Anschließend werden die gezogenen Kugeln wieder in den Behälter gelegt. Zeigen Sie, dass bei einem Spiel die Wahrscheinlichkeit für einen Gewinn $ \frac{1}{28} $ beträgt.

\item Berechnen Sie, welcher Geldbetrag  $ B $ im Fall eines Gewinns die Sendeleitung auszahlen muss, damit im Mittel eine Einnahme von \EUR{1,25} pro Spiel für die Sendeleitung zur Spende für einen guten Zweck erwartet werden kann.
\end{enumerate}
%Aufgabe 2
{\bf \item Auf einem Würfel befinden sich die folgenden Augenzahlen: 1,1,4,5,5,5}
\begin{enumerate}[a)]
\item Der Würfel wird 10mal geworfen.
\begin{enumerate}[(i)]
\item Mit welcher Wahrscheinlichkeit erhält man genau fünf Einser ?
\item Mit welcher Wahrscheinlichkeit erhält man mindestens vier und höchstens acht Einser ?
\end{enumerate}
\item Nachdem man  \EUR{3,90} einbezahlt hat, darf man den Würfel einmal werfen und bekommt die geworfene Augenzahl in Euro ausbezahlt. Berechne, mit welchem Reingewinn man im Mittel bei dieser Spielregel zu rechnen hat.
\end{enumerate}

%Aufgabe 3
{\bf \item Die Firma LEUCHTFIX produziert Glühbirnen, wobei erfahrungsgemäß 3,0\% der gefertigen Glühbirnen defekt sind.}
\begin{enumerate}[a)]
\item Ein Kunde greift 12 gefertigte Glühbirnen zufällig heraus. Wie groß ist die Wahrscheinlichkeit, dass sich unter diesen Glühbirnen genau zwei defekte Lampen befinden? 
\item Wie groß müsste die Defektwahrscheinlichkeit der Produktion sein, damit man mit einer Wahrscheinlichkeit von mehr als 90\% mindestens eine defekte Glühbirne herausgreift?
\end{enumerate}

{\bf \item In einer Augenklinik hält sich jeder Patient (unabhängig von anderen Patienten) mindestens 3 Tage und höchstens 5 Tage auf.} Die Verwaltung legt für die Aufenthaltsdauer $X$ eines Patienten folgende Wahrscheinlichkeitsverteilung zugrunde:

$$
\renewcommand{\arraystretch}{2}
\begin{array}{|c|c|c|c|}
\hline
x_i & 3 & 4 & 5 \\
\hline
P(X=x_i)& 0,6 & 0,1 & 0,3 \\
\hline
\end{array}
$$
Jeder Patient zahlt für die Aufnahme in die Klinik eine einmalige Gebühr von \EUR{100} und danach für jeden Aufenthaltstag zusätzlich jeweils \EUR{450}. Bestimmen Sie die Wahrscheinlichkeitsverteilung und mit deren Hilfe den Erwartungswert und die Standardabweichung für die Zufallsgröße:
$$ Y : \textnormal{Einnahmen der Klinik pro Patient in \EUR{}}$$
\end{enumerate} 

\fbox{
	\begin{minipage}{0.5\textwidth}
		Zur Lösung bitte \href{https://www.okuyakl.de/math/m12stoL064/ll064.pdf}{hier klicken} oder den QR-Code scannen.\\
	Weitere Arbeitsblätter gibt es unter 
	
	\href{https://www.okuyakl.de}{www.okuyakl.de}
	\end{minipage}
	\hfill
	\begin{minipage}{0.4\textwidth}
		\includegraphics[width=1.5 cm]{../../viecher/zwe03}
		\includegraphics[width=3 cm]{qr064}
		\includegraphics[width=2 cm]{../../viecher/afanticon1}
	\end{minipage}}

\end{document}%Lösung-------------------------------------------
