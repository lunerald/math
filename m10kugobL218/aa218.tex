
\documentclass[a4paper]{article}
\usepackage[pdftex]{graphicx}
\usepackage[utf8]{inputenc}
\usepackage{enumerate}
\usepackage{icomma}
\usepackage{siunitx}
\sisetup{locale=DE} 
\usepackage{amssymb}
\usepackage{geometry}
\geometry{a4paper, top=15mm, left=15mm, right=15mm, bottom=15mm,
	headsep=10mm, footskip=12mm}
\usepackage{href-ul}
\hypersetup{
	colorlinks=true,
	linkcolor=blue,
	urlcolor=blue}
	
\begin{document}
%Titel
{\bf Trainingsblatt Kugeloberfläche }\\
\begin{enumerate}[1.]

%Aufgabe 1
{\bf \item Berechne die fehlenden Stücke: Oberfläche $O$ einer Kugel mit Radius $r$ und Durchmesser}
$d$. \\
a) $d=\SI{28}{\milli\meter}$ \hspace{30 pt} b) $ O=\SI{100}{\centi\meter^2}$ \hspace{30 pt} c) $ r=\SI{5,89}{\meter}$ 

%Aufgabe 2
{\bf \item Gibt es eine Kugel, die für Volumen und Oberfläche die gleiche Maßzahl besitzt?} Begründe mit Rechnung! 

%Aufgabe 3
{\bf \item Zeigen Sie:} Für alle Kugeln ist die Oberfläche des umbeschriebenen Würfels um den gleichen Prozentsatz größer als jener der Kugel.

%Aufgabe 4
{\bf \item Eine Kugel besitzt das Volumen $\SI{5800}{\centi\meter^3}$.} Berechnen Sie die Oberfläche und geben Sie die Länge und Breite eines flächengleichen Rechtecks an.

%Aufgabe 5 
\begin{minipage}{0,7\textwidth}
{\bf \item Die Reichstagskuppel ist näherungsweise eine Halbkugel mit dem Radius $r=\SI{20}{\meter}$. }
\begin{enumerate}[a)]
\item Berechnen Sie ihren Oberflächeninhalt.
\item Schätzen Sie ab, wie dick die Verglasung ist, wenn in ihr ca. 240 Tonnen Glas verbaut worden sind und die Dichte von Glas  $\varrho_{Glas}= \SI{2,4}{\tonne\per\meter^3}$ beträgt.  
\end{enumerate}
\end{minipage}
\hfill
\begin{minipage}{0,25\textwidth}
\includegraphics[width=4.5 cm]{reichst218}
\end{minipage}

%Aufgabe 6
{\bf \item Eine Kugel, eine Halbkugel und ein Würfel haben das jeweils gleiche Volumen $V=\SI{500}{\centi\meter^3}$.} Vergleichen Sie ihre Oberflächeninhalte als Verhältnis.


%Aufgabe 7
{\bf \item Der Radius eines  Ballons nimmt beim Aufsteigen in höhere Luftschichten um 10 \% zu.}
Um wieviel \% vergrößert sich seine Oberfläche?  

%Aufgabe 8
{\bf \item Eine Probe aus $\SI{250}{\gram}$ Sand wird betrachtet. Die Korngröße sei $d=\SI{1}{\milli\meter}$, die Sandkörner werden als rund angenommen;} die Dichte des Materials sei $\varrho = \SI{2,5}{\gram \per \centi\meter^3}$.
\begin{enumerate}[a)]
\item Wie lange wäre eine Reihe aus allen Körnern dieser Probe?
\item Mit Sand kann man z.B. Wasser filtern. Welche Oberfläche haben alle Körner dieser Probe zusammen?
\item Wie ändert sich die Länge der Reihe, wenn die Korngröße verdoppelt wird?
\item Wie ändert sich die Oberfläche, wenn die Korngröße verdoppelt wird?
\end{enumerate}

%Aufgabe 9
\begin{minipage}{0,7\textwidth}
	{\bf \item Sonnenflecken treten alle 11 Jahre vermehrt auf der Sonnenoberfläche auf.} Die kleinsten haben etwa die Größe des Stadtgebiets von Ingolstadt \\$d \approx \SI{15}{\kilo\meter}$, es wurden aber auch schon Sonnenflecken mit einen Durchmesser bis zu $d \approx \SI{150000}{\kilo\meter}$ beobachtet.
	\begin{enumerate}[a)]
		\item um wieviele Zehnerpotenzen kann die Fläche der als rund angenommenen Sonnenflecken variieren?  
		\item Vergleichen Sie die Fläche dieses großen Sonnenflecks mit dem Oberflächeninhalt der Erdkugel $r_E=\SI{6370}{\kilo\meter}$ . Was ist größer und wieviel mal?
	\end{enumerate}
\end{minipage}
\hfill
\begin{minipage}{0,25\textwidth}
	\includegraphics[width=4.5 cm]{sun218}
\end{minipage}

\end{enumerate} 

\fbox{
	\begin{minipage}{0.5\textwidth}
		Zur Lösung bitte \href{https://www.okuyakl.de/math/m10kugobL218/ll218.pdf}{hier klicken} oder den QR-Code scannen.\\
	Weitere Arbeitsblätter gibt es unter 
	
	\href{https://www.okuyakl.de}{www.okuyakl.de}
	\end{minipage}
	\hfill
	\begin{minipage}{0.4\textwidth}
		\includegraphics[width=1.5 cm]{../../viecher/zwe03}
		\includegraphics[width=3 cm]{qr218}
		\includegraphics[width=2 cm]{../../viecher/afanticon1}
		
	\end{minipage}}

\end{document}%Lösung------------------------------------------------
