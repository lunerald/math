
\documentclass[a4paper]{article}
\usepackage[pdftex]{graphicx}
\usepackage[utf8]{inputenc}
\usepackage{enumerate}
\usepackage{icomma}
\usepackage{amssymb}
\usepackage{amsmath}
\usepackage{geometry}
\geometry{a4paper, top=15mm, left=15mm, right=15mm, bottom=15mm,
	headsep=10mm, footskip=12mm}
\usepackage{href-ul}
\hypersetup{
	colorlinks=true,
	linkcolor=blue,
	urlcolor=blue}
% Start the document
\begin{document}
	

	{\bf Sinus- und Kosinusfunktion lassen sich aufgrund ihrer Symmetrie ineinander umwandeln.}
	
	Untersuche folgende Kärtchen durch Überlegen oder mit Hilfe eines Funktionsplotters.

	\setlength{\unitlength}{1 cm}
	\begin{picture}(9,4)
		\includegraphics[width=9 cm]{scfu126}
		\put(-0.2,0.5){$\sin x$}
		\put(0.2,2.5){$\cos x$}
	\end{picture}


\fbox{\begin{minipage}{0.3\textwidth}
$$\cos(-x)$$
		$\square \quad \sin x$\\
		$\square \quad \cos x$\\
	\end{minipage}}
	\hfill
\fbox{\begin{minipage}{0.3\textwidth}
	  $$-sin(-x)$$
		$\square \quad \sin x$\\
		$\square \quad \cos x$\\
	\end{minipage}}
	\hfill
\fbox{\begin{minipage}{0.3\textwidth}
     $$\cos({\pi \over 2}-x)$$
	   $\square \quad \sin x$\\
	   $\square \quad \cos x$\\
	\end{minipage}}

\fbox{\begin{minipage}{0.3\textwidth}
		$$\sin(\pi-x)$$
		$\square \quad \sin x$\\
		$\square \quad \cos x$\\
\end{minipage}}
\hfill
\fbox{\begin{minipage}{0.3\textwidth}
		$$\cos(2\pi+x)$$
		$\square \quad \sin x$\\
		$\square \quad \cos x$\\
\end{minipage}}
\hfill
\fbox{\begin{minipage}{0.3\textwidth}
		$$-\sin(2\pi-x)$$
		$\square \quad \sin x$\\
		$\square \quad \cos x$\\
\end{minipage}}

\fbox{\begin{minipage}{0.3\textwidth}
		$$-\cos(\pi-x)$$
		$\square \quad \sin x$\\
		$\square \quad \cos x$\\
\end{minipage}}
\hfill
\fbox{\begin{minipage}{0.3\textwidth}
		$$-sin(x +\pi)$$
		$\square \quad \sin x$\\
		$\square \quad \cos x$\\
\end{minipage}}
\hfill
\fbox{\begin{minipage}{0.3\textwidth}
		$$\sin({\pi \over 2}-x)$$
		$\square \quad \sin x$\\
		$\square \quad \cos x$\\
\end{minipage}}

\fbox{\begin{minipage}{0.3\textwidth}
		$$-\sin(x + {\pi \over 2})$$
		$\square \quad \sin x$\\
		$\square \quad \cos x$\\
\end{minipage}}
\hfill
\fbox{\begin{minipage}{0.3\textwidth}
		$$-\cos(x + {\pi \over 2})$$
		$\square \quad \sin x$\\
		$\square \quad \cos x$\\
\end{minipage}}
\hfill
\fbox{\begin{minipage}{0.3\textwidth}
		$$\sqrt{1-\cos^2x}$$
		$\square \quad |\sin x|$\\
		$\square \quad |\cos x|$\\
\end{minipage}}

\fbox{\begin{minipage}{0.3\textwidth}
		$$\sqrt{1-\sin^2x}$$
		$\square \quad \sin x$\\
		$\square \quad \cos x$\\
\end{minipage}}
\hfill
\fbox{\begin{minipage}{0.3\textwidth}
		$$\cos x \cdot \tan x$$
		$\square \quad \sin x$\\
		$\square \quad \cos x$\\
\end{minipage}}
\hfill
\fbox{\begin{minipage}{0.3\textwidth}
		$$\sin x \over \tan x$$
		$\square \quad \sin x$\\
		$\square \quad \cos x$\\
\end{minipage}}

\fbox{\begin{minipage}{0.3\textwidth}
		$$\sin(180^\circ-\varphi)$$
		$\square \quad \sin \varphi$\\
		$\square \quad \cos \varphi$\\
\end{minipage}}
\hfill
\fbox{\begin{minipage}{0.3\textwidth}
		$$\cos(360^\circ-\varphi)$$
		$\square \quad \sin \varphi$\\
		$\square \quad \cos \varphi$\\
\end{minipage}}
\hfill
\fbox{\begin{minipage}{0.3\textwidth}
		$$\cos(-\varphi)$$
		$\square \quad \sin \varphi$\\
		$\square \quad \cos \varphi$\\
\end{minipage}}

\fbox{\begin{minipage}{0.3\textwidth}
		$$\sin(90^\circ-\varphi)$$
		$\square \quad \sin \varphi$\\
		$\square \quad \cos \varphi$\\
\end{minipage}}
\hfill
\fbox{\begin{minipage}{0.3\textwidth}
		$$\cos(90^\circ-\varphi)$$
		$\square \quad \sin \varphi$\\
		$\square \quad \cos \varphi$\\
\end{minipage}}
\hfill
\fbox{\begin{minipage}{0.3\textwidth}
		$$\sin(\varphi+ 360^\circ)$$
		$\square \quad \sin \varphi$\\
		$\square \quad \cos \varphi$\\
\end{minipage}}

\fbox{
	\begin{minipage}{0.5\textwidth}
		Zur Lösung bitte \href{https://www.okuyakl.de/math/m10gozL025/ll025.pdf}{hier klicken} oder den QR-Code scannen.\\
	Weitere Arbeitsblätter gibt es unter 
	
	\href{https://www.okuyakl.de}{www.okuyakl.de}
	\end{minipage}
	\hfill
	\begin{minipage}{0.4\textwidth}
		\includegraphics[width=1.5 cm]{../../viecher/zwe03}
		\includegraphics[width=3 cm]{qr126}
		\includegraphics[width=2 cm]{../../viecher/afanticon1}
		
	\end{minipage}}

\end{document}%Lösung--------------------------------------------------------------------------------------------------
