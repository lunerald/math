\documentclass[a4paper]{article}
\usepackage[pdftex]{graphicx}
\usepackage[utf8]{inputenc}
\usepackage{enumerate}
\usepackage{icomma}
\usepackage{amssymb}
\usepackage{href-ul}
\hypersetup{
	colorlinks=true,
	linkcolor=blue,
	urlcolor=blue}
\usepackage{geometry}
\geometry{a4paper, top=15mm, left=15mm, right=15mm, bottom=15mm,
	headsep=10mm, footskip=12mm}

\begin{document}
\includegraphics[width=14 cm]{want002}
\begin{enumerate}[1.]
\vspace{0.5 cm}
\hrule
\vspace{0.5 cm}
	\begin{minipage}{0.4\textwidth}
	\item Quadratische Funktion, die durch den Ursprung geht und dort eine Tangente mit der Steigung $m=8$ hat. Der Scheitel hat die x-Koordinate $x=8$. 
\end{minipage}
\hspace{0.5 cm}
\vrule
\hspace{0.5 cm}
\begin{minipage}{0.4\textwidth}
	\item	Funktion dritten Grades, die im Ursprung eine doppelte Nullstelle hat und im Punkt $T(2|-4)$ einen Tiefpunkt aufweist .	
\end{minipage}
\vspace{0.5 cm}
\hrule
\vspace{0.5 cm}
\begin{minipage}{0.4\textwidth}
		\item Eine achsensymmetrische Funktion vierten Grades, die an der Stelle $x=1$ die x-Achse berührt und die y-Achse bei $y=-2$ schneidet. 
\end{minipage}
\hspace{0.5 cm}
\vrule
\hspace{0.5 cm}
\begin{minipage}{0.4\textwidth}
		\item Funktion möglichst kleinen Grades, die die Punkte $A(1|2)$ und $B(4|3,5)$ enthält.		
\end{minipage}
\vspace{0.5 cm}
\hrule
\vspace{0.5 cm}
\begin{minipage}{0.4\textwidth}
		\item Funktion möglichst kleinen Grades, die an der Stelle $x=4$ eine waagrechte Tangente hat und durch die Punkte $P(2|-2)$ und $Q(8|1)$ geht.
\end{minipage}
\hspace{0.5 cm}
\vrule
\hspace{0.5 cm}
\begin{minipage}{0.4\textwidth}
		\item Funktion möglichst kleinen Grades, die zur Achse $x=3$ symmetrisch ist und im Punkt $P(1|6)$ die Steigung $-4$ hat.	
\end{minipage}
\vspace{0.5 cm}
\hrule
\vspace{0.5 cm}
\begin{minipage}{0.4\textwidth}
	\item Funktion vierten Grades, die einen Terrassenpunkt $T(2|0)$ hat . Die y-Achse schneidet sie bei $y=-4$ und sie hat dort eine Steigung von 2.
\end{minipage}
\hspace{0.5 cm}
\vrule
\hspace{0.5 cm}
\begin{minipage}{0.4\textwidth}
	\item Funktion dritten Grades, die eine Nullstelle bei $x=3$ hat, im Wendepunkt die Steigung $-4$ aufweist und waagrechte Tangenten bei $x=2$ und $x=-2$ besitzt 		
\end{minipage}
\vspace{0.5 cm}
\hrule
\vspace{0.5 cm}
\begin{minipage}{0.4\textwidth}
	\item Funktion mit folgendem Schaubild:
	\includegraphics[width=7 cm]{plofa002}
\end{minipage}
\hspace{0.5 cm}
\vrule
\hspace{0.5 cm}
\begin{minipage}{0.4\textwidth}
	\item Funktion mit folgendem Schaubild:	
	\includegraphics[width=7 cm]{plofb002}
\end{minipage}
\end{enumerate} 
\fbox{
	\begin{minipage}{0.5\textwidth}
		Zur Lösung bitte \href{https://www.okuyakl.de/math/m11wanL002/ll002.pdf}{hier klicken} oder den QR-Code scannen.\\
	Weitere Arbeitsblätter gibt es unter 
	
	\href{https://www.okuyakl.de}{www.okuyakl.de}
	\end{minipage}
	\hfill
	\begin{minipage}{0.4\textwidth}
		\includegraphics[width=1.5 cm]{../../viecher/zwe03}
		\includegraphics[width=3 cm]{qr002}
		\includegraphics[width=2 cm]{cowfa002}
		
\end{minipage}}


\end{document}

