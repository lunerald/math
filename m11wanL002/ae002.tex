\documentclass[a4paper]{article}
\usepackage[pdftex]{graphicx}
\usepackage[utf8]{inputenc}
\usepackage{enumerate}
\usepackage{icomma}
\usepackage{amssymb}
\usepackage{href-ul}
\hypersetup{
	colorlinks=true,
	linkcolor=blue,
	urlcolor=blue}
\usepackage{geometry}
\geometry{a4paper, top=15mm, left=15mm, right=15mm, bottom=15mm,
	headsep=10mm, footskip=12mm}

\begin{document}
	\includegraphics[width=14 cm]{want002}
	\begin{enumerate}[1.]
		\vspace{0.5 cm}
		\hrule
		\vspace{0.5 cm}
		\begin{minipage}{0.4\textwidth}
			\item Quadratic function that passes through the origin and has a tangent there with slope $m=8$. The vertex has the x coordinate $x=8$.
		\end{minipage}
		\hspace{0.5 cm}
		\vrule
		\hspace{0.5 cm}
		\begin{minipage}{0.4\textwidth}
			\item Third degree function that has a double zero at the origin and a low point at the point $T(2|-4)$.
		\end{minipage}
		\vspace{0.5 cm}
		\hrule
		\vspace{0.5 cm}
		\begin{minipage}{0.4\textwidth}
			\item A fourth degree axisymmetric function that touches the x-axis at $x=1$ and intersects the y-axis at $y=-2$.
		\end{minipage}
		\hspace{0.5 cm}
		\vrule
		\hspace{0.5 cm}
		\begin{minipage}{0.4\textwidth}
			\item Function of the smallest possible degree that contains the points $A(1|2)$ and $B(4|3,5)$.
		\end{minipage}
		\vspace{0.5 cm}
		\hrule
		\vspace{0.5 cm}
		\begin{minipage}{0.4\textwidth}
			\item Function of the smallest possible degree that has a horizontal tangent at the point $x=4$ and passes through the points $P(2|-2)$ and $Q(8|1)$.
		\end{minipage}
		\hspace{0.5 cm}
		\vrule
		\hspace{0.5 cm}
		\begin{minipage}{0.4\textwidth}
			\item Function of the smallest possible degree, which is symmetrical about the axis $x=3$ and has the slope $-4$ at the point $P(1|6)$.
		\end{minipage}
		\vspace{0.5 cm}
		\hrule
		\vspace{0.5 cm}
		\begin{minipage}{0.4\textwidth}
			\item Fourth degree function that has a terrace point $T(2|0)$. The y-axis intersects it at $y=-4$ and it has a slope of 2 there.
		\end{minipage}
		\hspace{0.5 cm}
		\vrule
		\hspace{0.5 cm}
		\begin{minipage}{0.4\textwidth}
			\item Function of the third degree, which has a zero at $x=3$, has the slope $-4$ at the turning point and has horizontal tangents at $x=2$ and $x=-2$
		\end{minipage}
		\vspace{0.5 cm}
		\hrule
		\vspace{0.5 cm}
		\begin{minipage}{0.4\textwidth}
			\item function with the following diagram:
			\includegraphics[width=7 cm]{plofa002}
		\end{minipage}
		\hspace{0.5 cm}
		\vrule
		\hspace{0.5 cm}
		\begin{minipage}{0.4\textwidth}
			\item function with the following diagram:
			\includegraphics[width=7 cm]{plofb002}
		\end{minipage}
	\end{enumerate}
	\fbox{
		\begin{minipage}{0.5\textwidth}
			For the solution, please \href{https://www.okuyakl.de/math/m11wanL002/le002.pdf}{click here} or scan the QR code.\\
			Additional worksheets are available at
			
			\href{http://www.okuyakl.com}{www.okuyakl.com}
		\end{minipage}
		\hfill
		\begin{minipage}{0.4\textwidth}
			\includegraphics[width=1.5 cm]{../../viecher/zwe03}
			\includegraphics[width=3 cm]{qre002}
			\includegraphics[width=2 cm]{cowfa002}
			
	\end{minipage}}
	
	
\end{document}