\documentclass[a4paper]{article}
\usepackage[pdftex]{graphicx}
\usepackage[utf8]{inputenc}
\usepackage{enumerate}
\usepackage{icomma}
\usepackage{amssymb}
\usepackage{amsmath}
\usepackage{geometry}
\geometry{a4paper, top=15mm, left=15mm, right=15mm, bottom=15mm,
	headsep=10mm, footskip=12mm}
\usepackage{href-ul}
\hypersetup{
	colorlinks=true,
	linkcolor=blue,
	urlcolor=blue}
\begin{document}
{\bf Funktionen}
\begin{enumerate}[1.]

%Aufgabe 1
{\bf \item Stelle die folgende Funktion ohne Betragsstriche dar und zeichne ihren Graphen:}
$$ f(x) = - \frac{2}{3}x + \left|  \frac{4}{3}x - 4 \right| ; \quad x \in \mathbb{R} $$

{\bf \item Ordnen Sie folgende Funktionen den abgebildeten Graphen zu.\\}

\noindent
\begin{tabular}{cccccc}
$f_1(x)=x^3$ &
$f_2(x)=x^{-3}$&
$f_3(x)=3^x$ &
$f_4(x)=3^{-x}$ & 
$f_5(x)=3x$ &
$f_6(x)=\log_3{x}$\\
\end{tabular}


\noindent
\begin{minipage}{0.26\textwidth}
{\bf A}\\
\includegraphics[width=3.5 cm]{fua145}\\
\vspace{0.3 cm}\\
{\bf D}\\
\includegraphics[width=3.5 cm]{fd145}
\end{minipage}
\hfill
\begin{minipage}{0.26\textwidth}
{\bf B}\\
\includegraphics[width=3.5 cm]{fub145}\\
\vspace{0.3 cm}\\
{\bf E}\\
\includegraphics[width=3.5 cm]{fue145}
\end{minipage}
\hfill
\begin{minipage}{0.26\textwidth}
{\bf C}\\
\includegraphics[width=3.5 cm]{fc145}\\
\vspace{0.3 cm}\\
{\bf F}\\
\includegraphics[width=3.5 cm]{fuf145}
\end{minipage}

%Aufgabe 2
{\bf \item Gegeben ist die Funktion $ f(x) = (x-3)^4 + 1; \quad D_f = [ 3; \infty [ $}
 
\begin{enumerate}[a)]
\item Skizzieren Sie möglichst genau den Graphen von $f$ ohne Verwendung einer Wertetabelle.
\item Ist die Funktion in ihrem Definitionsbereich umkehrbar? Begründen Sie ihre Aussage.
\item Skizzieren Sie möglichst genau die Umkehrfunktion $f^{-1}$ von $f$.
\item Berechnen Sie die Umkehrfunktion und geben Sie ihre Definitions- und Wertemenge an.
\end{enumerate}


%Aufgabe 1
{\bf \item Eine Hyperbel $h_1$ wird beschrieben durch die Gleichung $y=a(x+1)^{-2}+e$.} Ihre Asymptoten haben die Gleichungen $y=3$ und $x=-1$. Außerdem gilt $P(1|3,5) \in h_1$.

\begin{enumerate}[a)]
\item Bestimme die Gleichung der Hyperbel $h_1$.
\item Zeige, dass der Funktionsgraph der Funktion $f(x)=x^{-2}$ achsensymmetrisch zur y-Achse ist.
\end{enumerate}

%Aufgabe 2
{\bf \item Gegeben sei die Hyperbel $h_2:~y=(x-1)^{-1}+3$ und die Gerade}\\
 $g:~y=-4x+3$. Der Punkt $M(1,75|-4)$ ist der Mittelpunkt der Strecke $[AB]$. Die Punkte $M$ und $B$ liegen auf der Geraden $g$. Der Punkt $A$ liegt sowohl auf $g$ als auch auf $h_2$.

\begin{enumerate}[a)]
\item Berechne die Koordinaten des Punktes $A$. $[\textnormal{Ergebnis}: ~A(0,5|1)]$
\item Berechne die Koordinaten des Punktes $B$.
\end{enumerate} 

{\bf \item Untersuche die folgenden Funktionen rechnerisch auf Punktsymmetrie zum Ursprung bzw. auf Achsensymmetrie zur y-Achse}


\begin{tabular}{ccccccc}
$ f(x)$&=& $(x-3)^2+6x$ &  \parbox[20 pt][2em][c]{0 cm}{} &
$g(x)$&=&$2x+\sin{x}$\\
$ h(x)$&=& $\frac{1}{4}x^4-2x^2+1$ &  \parbox[20 pt][2em][c]{0 cm}{} &
$i(x)$&=&$\frac{x^3}{x^2+4}$\\
$ j(x)$&=& $2-10^{-x}$ &  \parbox[20 pt][2em][c]{0 cm}{} &
$k(x)$&=&$5 \cdot \sqrt{x}$\\
\end{tabular}

\newpage
{\bf \item Gegeben ist die Funktion $f$ mit $f(x)= a\cdot x^n$ mit $a \in \mathbb{R},\quad n \in \mathbb{N}, $ und $ D_f= \mathbb{R}$.}\\
Welche Aussagen kann man über $a$ und $n$ machen , wenn bekannt ist, dass

\begin{enumerate}[(i)]
\item der Graph der Funktion punktsymmetrisch zum Ursprung ist und durch den III. Quadranten verläuft?
\item der Graph der Funktion achsensymmetrisch zur y-Achse ist und durch den Punkt $P(1|-1)$ verläuft?
\end{enumerate}

\end{enumerate} 

\fbox{
	\begin{minipage}{0.5\textwidth}
		Zur Lösung bitte \href{https://www.okuyakl.de/math/m10fuaL032/ll032.pdf}{hier klicken} oder den QR-Code scannen.\\
	Weitere Arbeitsblätter gibt es unter 
	
	\href{https://www.okuyakl.de}{www.okuyakl.de}
	\end{minipage}
	\hfill
	\begin{minipage}{0.4\textwidth}
		\includegraphics[width=1.5 cm]{../../viecher/zwe03}
		\includegraphics[width=3 cm]{qr032}
		\includegraphics[width=2 cm]{../../viecher/afanticon1}
		
	\end{minipage}}

\end{document}%Lösung-------------------------------------------
