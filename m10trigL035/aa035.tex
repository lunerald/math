\documentclass[a4paper]{article}
\usepackage[pdftex]{graphicx}
\usepackage[utf8]{inputenc}
\usepackage{enumerate}
\usepackage{icomma}
\usepackage{siunitx}
\sisetup{locale=DE} 
\usepackage{colortbl}
\usepackage{amssymb}
\usepackage{geometry}
\geometry{a4paper, top=15mm, left=15mm, right=15mm, bottom=15mm,
	headsep=10mm, footskip=12mm}
\usepackage{href-ul}
\hypersetup{
	colorlinks=true,
	linkcolor=blue,
	urlcolor=blue}
\begin{document}
{\bf Trigonometrie- Sinussatz spezial}
\begin{enumerate}[1.]

{\bf \item Der Punkt P liegt auf dem Einheitskreis. Bestimme die fehlende Koordinate.}
$$a) P(-1~|~ \underline{\qquad}) \quad b) P\left({1 \over 2}~|~ \underline{\qquad}\right) \quad c) P\left(\underline{\qquad}~|~ {1 \over 2}\sqrt{2}\right) 
\quad d) P\left(-{1\over 2} \sqrt{3}~|~ \underline{\qquad}\right)
 \quad e) P(-0,6~|~ \underline{\qquad})$$

{\bf \item Berechne durch Zurückführung auf spitze Winkel}
$$a) \sin 1110^\circ \quad b)\, \cos 1140^\circ \quad c)\,\tan 1485^\circ \quad d)\,\sin 765^\circ \quad e)\,\tan 1500^\circ$$

\begin{minipage}{0.6\textwidth}
{\bf \item In einem Dreieck mit $\alpha=60^\circ$, $\beta=75^\circ$ und $b=\SI{8}{\centi\meter}$} soll die kürzeste Seite so in drei Teile geteilt werden, dass die Verbindungslinien der Teilpunkte mit dem gegenüberliegenden Dreieckspunkt den Winkel $\gamma$ in drei gleich große Teile teilen. Die Zeichnung ist nicht maßstabsgetreu.\\
Berechne die Seitenlängen und die Flächeninhalte der Teildreiecke. 
\end{minipage}
\begin{minipage}{0.4\textwidth}
\includegraphics[width=6 cm]{drei248}
\end{minipage}

{\bf \item Berechne, wenn möglich, die fehlenden Größen}

\renewcommand{\arraystretch}{3}
 \begin{tabular}{|>{\columncolor[gray]{.8}}p{1 cm}|p{2 cm}|p{2 cm}|p{2 cm}|p{2 cm}|p{2 cm}|p{2 cm}|}
	\hline
	\rowcolor[gray]{.8}
&	\bf {a}&\bf  b&\bf  c& $ \alpha$ & $ \beta$ &$ \gamma$\\
	\hline
a)&	7 &  & &$80^\circ$ & &$55^\circ$ \\
	\hline
b)&	& 8&6 & & $111^\circ$ & \\
	\hline
c)&	10& & 7,2 & & & $42^\circ$ \\	
    \hline
d)&	6,1 & & &$61^\circ$ &$50^\circ$ & \\
	\hline
e)&	& 12 & & & $100^\circ$ & $75^\circ$\\
	\hline
f)&	& &3,5 &$98^\circ$ & &$46^\circ$ \\
	\hline
g)&	4,8 & 9,1 &  & $103^\circ$ & &  \\
	\hline
h)&	14 & & &$95^\circ$ & $100^\circ$ & \\
	\hline
\end{tabular}


{\bf \item Der Punkt $P(3|4)$ wird entgegen dem Uhrzeigersinn um $110^\circ$ um den Ursprung gedreht.} Wie lauten die Polarkoordinaten des Bildpunktes $P'$ ?

\end{enumerate} 

\fbox{
	\begin{minipage}{0.5\textwidth}
		Zur Lösung bitte \href{https://www.okuyakl.de/math/m10trigL035/ll035.pdf}{hier klicken} oder den QR-Code scannen.\\
	Weitere Arbeitsblätter gibt es unter 
	
	\href{https://www.okuyakl.de}{www.okuyakl.de}
	\end{minipage}
	\hfill
	\begin{minipage}{0.4\textwidth}
		\includegraphics[width=1.5 cm]{../../viecher/zwe03}
		\includegraphics[width=3 cm]{qr035}
		\includegraphics[width=2 cm]{../../viecher/afanticon1}
		
	\end{minipage}}

\end{document}%Lösung-------------------------------------------
