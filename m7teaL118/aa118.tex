\documentclass[a4paper]{article}
\usepackage[pdftex]{graphicx}
\usepackage[utf8]{inputenc}
\usepackage{enumerate}
\usepackage{eurosym}
\usepackage{amssymb}
\usepackage{icomma}
\usepackage{siunitx}
\sisetup{                     
	locale=DE} 
\usepackage{geometry}
\geometry{a4paper, top=15mm, left=15mm, right=15mm, bottom=15mm,
	headsep=10mm, footskip=12mm}
\usepackage{href-ul}
\hypersetup{
	colorlinks=true,
	linkcolor=blue,
	urlcolor=blue}
\begin{document}

{\bf Terme aufstellen mit einer Variablen}
\begin{enumerate}[1.]

{\bf \item Die Sportlehrer bestellen neue Bälle für die Turnhalle:} Fußbälle kosten je \EUR{35}, Handbälle je \EUR{30}, Volleybälle je \EUR{20} und Tennisbälle \EUR{6}. Es werden doppelt so viele Fußbälle wie Handbälle und 4 Volleybälle mehr als Handbälle gekauft. Dazu werden 12 Tennisbälle bestellt.

\begin{enumerate}[a)]
	\item Stelle einen Term mit möglichst wenig Variablen zur Berechnung der Gesamtkosten auf.
	\item Berechne die entstehenden Gesamtkosten mithilfe des in a) aufgestellten Terms, wenn 10 von den bestellten Bällen Handbälle sind.  
\end{enumerate}

\begin{minipage}{0.5\textwidth}
	{\bf \item Das Paket soll so wie in der Abbildung verschnürt werden.} Es ist doppelt so lang wie hoch und doppelt so hoch wie breit. Wie lang muss die verwendete Schnur mindestens sein (ohne Berücksichtigung von Knoten und Endstücken)?\\
	Stelle einen Term mit nur einer Variablen auf und gib die Bedeutung deiner Variable an.
\end{minipage}
\begin{minipage}{0.5\textwidth}
	\includegraphics[width=6 cm]{paket249}
\end{minipage}

{\bf \item Lunerald macht eine dreitägige Fahrradtour.} Am zweiten Tag fährt er doppelt so weit wie am ersten Tag, am dritten Tag fährt er  $\SI{15}{\kilo\meter}$ weniger als am zweiten. 
\begin{enumerate}[a)]
	\item Stelle einen Term zur Berechnung der gefahrenen Gesamtstrecke auf.
	\item Wie weit ist er am ersten Tag gefahren, wenn die Gesamtstrecke der Radtour $\SI{235}{\kilo\meter}$ betrug?
\end{enumerate} 
 

{\bf \item Liane hat sich zum 1. Mai einen jungen Hund zugelegt}. Sie gibt ihm jeden Tag zunächst $\SI{100}{\gram}$ Trockenfutter zu fressen. Alle zwei Wochen verdoppelt sie dann seine Futterration. 
\begin{enumerate}[a)]
	\item Wie viel Futter bekommt der Hund am 31. Mai ?    
	\item Wie viele $\SI{1}{\kilo\gram}$-Futtertüten hat Liane bis zum 31. Mai angebrochen?
\end{enumerate} 


{\bf \item Großfamilie Hurtzig, bestehend aus Mutter, Vater und drei Kindern} möchte mit der Autofähre über den Bodensee übersetzen. Ein Ticket für ein Kind kostet die Hälfte eines Erwachsenen-Tickets und der Fahrschein für den Familien-Van kostet \EUR{9,50}. Zusammen werden \EUR{20,70} fällig.

\begin{enumerate}[a)]
	\item Wieviel müsste Frau Hurtzig für sich alleine zahlen ?
	\item Wieviel würden zusätzlich noch beide Großeltern kosten, die 25\% Ermäßigung bekommen?  
\end{enumerate} 


{\bf \item Die in Fahrschulen unterrichtete Faustregel für die Berechnung des Bremsweges} eines Autos lautet folgendermaßen: \\
{\it ,, Teile die gefahrene Geschwindigkeit in km/h durch zehn und multipliziere das Ergebnis mit sich selbst, dann erhältst Du den Bremsweg in Meter"}\\
Fülle damit diese Tabelle aus:

\begin{center}
	\renewcommand{\arraystretch}{2}
	\begin{tabular}{|c|p{1.5 cm}|p{1.5 cm}|p{1.5 cm}|p{1.5 cm}|}
		\hline
		Geschwindigkeit in km/h & 20 & 50 & 80 &100\\
		\hline
		Bremsweg in  m	& & & & \\
		\hline
	\end{tabular}
\end{center}

{\bf \item Milan fängt eine dreijährige Lehre als Bäcker an.} Sein monatliches Einstiegsgehalt wird jedes Jahr um \EUR{150} erhöht. Am Ende der Ausbildung winkt eine Abfindung von \EUR{540}
\begin{enumerate}[a)]
	\item Stelle einen Term auf zur Berechnung von Milans insgesamt verdienten Geld. 
	\item Wie hoch wird sein Gesamtverdienst, wenn sein Einstiegsgehalt \EUR{450} beträgt?
	\item Wie hoch wird dann sein Durchschnittsgehalt?
	\item Er handelt nun zu Beginn mit der Chefin ein Weihnachtsgeld von \EUR{120} aus. Wie groß wird dann sein Gesamt- und Durchschnittsverdienst? 
\end{enumerate} 

\end{enumerate} 
\fbox{
	\begin{minipage}{0.5\textwidth}
		Zur Lösung bitte \href{https://www.okuyakl.de/math/m7teaL118/ll118.pdf}{hier klicken} oder den QR-Code scannen.\\
	Weitere Arbeitsblätter gibt es unter 
	
	\href{https://www.okuyakl.de}{www.okuyakl.de}
	\end{minipage}
	\hfill
	\begin{minipage}{0.4\textwidth}
		\includegraphics[width=1.5 cm]{../../viecher/zwe03}
		\includegraphics[width=3 cm]{qr118}
		\includegraphics[width=2 cm]{../../viecher/afanticon1}
		
	\end{minipage}}
\end{document}

