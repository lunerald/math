\documentclass[a4paper]{article}
\usepackage[pdftex]{graphicx}
\usepackage[utf8]{inputenc}
\usepackage{enumerate}
\usepackage{eurosym}
\usepackage{amssymb}
\usepackage{icomma}
\usepackage{siunitx}
\sisetup{
	locale=DE}
\usepackage{geometry}
\geometry{a4paper, top=15mm, left=15mm, right=15mm, bottom=15mm,
	headsep=10mm, footskip=12mm}
\usepackage{href-ul}
\hypersetup{
	colorlinks=true,
	linkcolor=blue,
	urlcolor=blue}
\begin{document}
	
	{\bf Set up terms with one variable}
	\begin{enumerate}[1.]
		
		{\bf \item The sports teachers order new balls for the gym:} Soccer balls cost \EUR{35} each, handballs \EUR{30} each, volleyballs \EUR{20} each and tennis balls \EUR{6}. Twice as many footballs are purchased as handballs and 4 more volleyballs are purchased than handballs. 12 tennis balls are ordered.
		
		\begin{enumerate}[a)]
			\item Create an expression with as few variables as possible to calculate the total costs.
			\item Calculate the total costs incurred using the term in a) if 10 of the balls ordered are handballs.
		\end{enumerate}
		
		\begin{minipage}{0.5\textwidth}
			{\bf \item The package should be tied as shown in the picture.} It is twice as long as it is high and twice as high as it is wide. What is the minimum length of the cord used (without taking knots and end pieces into account)?\\
			Create an expression with only one variable and state the meaning of your variable.
		\end{minipage}
		\begin{minipage}{0.5\textwidth}
			\includegraphics[width=6 cm]{paket249}
		\end{minipage}
		
		{\bf \item Lunerald goes on a three-day bike ride.} On the second day he rides twice as far as on the first day, on the third day he rides $\SI{15}{\kilo\meter}$ less than on the second.
		\begin{enumerate}[a)]
			\item Set up an expression to calculate the total distance traveled.
			\item How far did he ride on the first day if the total distance of the bike ride was $\SI{235}{\kilo\meter}$?
		\end{enumerate}
		
		
		{\bf \item Liane got a young dog on May 1st}. She initially gives him $\SI{100}{\gram}$ dry food to eat every day. She then doubles his food ration every two weeks.
		\begin{enumerate}[a)]
			\item How much food does the dog get on May 31st?
			\item How many $\SI{1}{\kilo\gram}$ food bags did Liane open by May 31st?
		\end{enumerate}
		
		
		{\bf \item The Hurtzig family, consisting of mother, father and three children} would like to take the car ferry across Lake Constance. A ticket for a child costs half the price of an adult ticket and the ticket for the family van costs €9.50. Together, \EUR{20.70} is due.
		
		\begin{enumerate}[a)]
			\item How much would Ms. Hurtzig have to pay for herself?
			\item How much extra would it cost for both grandparents who get a 25% discount?
		\end{enumerate}
		
		
		{\bf \item The rule of thumb taught in driving schools for calculating the braking distance} of a car is as follows: \\
		{\it ,, Divide the speed driven in km/h by ten and multiply the result by itself, then you get the braking distance in meters"}\\
		Use this to fill out this table:
		
		\begin{center}
			\renewcommand{\arraystretch}{2}
			\begin{tabular}{|c|p{1.5 cm}|p{1.5 cm}|p{1.5 cm}|p{1.5 cm}|}
				\hline
				Speed in km/h & 20 & 50 & 80 &100\\
				\hline
				Braking distance in m & & & & \\
				\hline
			\end{tabular}
		\end{center}
		
		{\bf \item Milan begins a three-year apprenticeship as a baker.} His monthly starting salary is increased by \EUR{150} every year. At the end of the training you will receive a severance payment of \EUR{540}
		\begin{enumerate}[a)]
			\item Create an expression to calculate Milan's total money earned.
			\item How much will his total earnings be if his starting salary is \EUR{450}?
			\item What will his average salary be then?
			\item At the beginning he negotiates a Christmas bonus of \EUR{120} with the boss. What will his total and average earnings then be?
		\end{enumerate}
		
	\end{enumerate}
	\fbox{
		\begin{minipage}{0.5\textwidth}
			For the solution, please \href{https://www.okuyakl.de/math/m7teaL118/le118.pdf}{click here} or scan the QR code.\\
			Additional worksheets are available at
			
			\href{http://www.okuyakl.com}{www.okuyakl.com}
		\end{minipage}
		\hfill
		\begin{minipage}{0.4\textwidth}
			\includegraphics[width=1.5 cm]{../../viecher/zwe03}
			\includegraphics[width=3 cm]{qre118}
			\includegraphics[width=2 cm]{../../viecher/afanticon1}
			
	\end{minipage}}
\end{document}