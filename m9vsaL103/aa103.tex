\documentclass[a4paper]{article}
\usepackage[pdftex]{graphicx}
\usepackage[utf8]{inputenc}
\usepackage{enumerate}
\usepackage{amssymb}
\usepackage{icomma}
\usepackage{colortbl}
\usepackage{siunitx}
\sisetup{locale=DE} 
\usepackage{geometry}
\geometry{a4paper, top=15mm, left=15mm, right=15mm, bottom=15mm,
	headsep=10mm, footskip=12mm}
\usepackage{href-ul}
\hypersetup{
	colorlinks=true,
	linkcolor=blue,
	urlcolor=blue}
\begin{document}

\begin{enumerate}[1.]
	\begin{minipage}{0.5\textwidth}
		{\bf Die Vierstreckensätze (Strahlensätze) }
		
		{\bf \item Berechne die fehlenden Größen}
	\end{minipage}
	\hfill
	\begin{minipage}{0.5\textwidth}
			\includegraphics[width=8 cm]{drvs103}
	\end{minipage}



	
	\renewcommand{\arraystretch}{2}
	\begin{tabular}{|>{\columncolor[gray]{.8}}p{1.3 cm}|p{1.3 cm}|p{1.3 cm}|p{1.3 cm}|p{1.3 cm}|p{1.3 cm}|p{1.3 cm}|p{1.3 cm}|p{1.3 cm}|p{1.3 cm}|}
		\hline
		\rowcolor[gray]{.8}	&$a$ &$b$ &$c$ &$d$ &$e$ & $f$ & $x$ & $y$ & $z$ \\
		\hline
		a) &  & $\SI{1,0}{\meter}$ & $\SI{24}{\deci\meter}$ & $\SI{1,2}{\meter}$ &  & $\SI{5,0}{\meter}$ &  & $\SI{18}{\deci\meter}$ &\\
	\hline
	b) & $\SI{4,0}{\deci\meter}$ & & $\SI{32}{\centi\meter}$ & $\SI{24}{\centi\meter}$ & $\SI{1,6}{\deci\meter}$ & & & & $\SI{17,5}{\centi\meter}$ \\
	\hline
	c) & $\SI{7,5}{\centi\meter}$ &  & &  & $\SI{42}{\milli\meter}$ & $\SI{3,5}{\centi\meter}$ & $\SI{70}{\milli\meter}$ &$\SI{42}{\milli\meter}$ &  \\
		\hline
		
	\end{tabular}

\vspace{0.4 cm}
\begin{minipage}{0.31\textwidth}
	{\bf \item Die Steigung von Straßen} wird in Prozent \% angegeben. Der gefahrene Höhenunterschied lässt sich damit berechnen.
	\vspace{0.4 cm}\\
	\includegraphics[width=4.5 cm]{prosch103}
	
	
\end{minipage}
\hspace{0.9 cm}
\begin{minipage}{0.6\textwidth}
		\renewcommand{\arraystretch}{2}
		\begin{tabular}{|>{\columncolor[gray]{.8}}p{2.5 cm}|p{1.5 cm}|p{1.5 cm}|p{1.5 cm}|p{1.5 cm}|}
			\hline
			\rowcolor[gray]{.8}	&a) &b) &c) &d) \\
			\hline
			Steigung p \% & 5\% & 20\% & & 12\% \\
			\hline
			Strecke &$\SI{4}{\kilo\meter}$ & &$\SI{9}{\kilo\meter}$ &$\SI{3500}{\meter}$ \\
			\hline
			Höhenunterschied & &$\SI{80}{\meter}$ &$\SI{180}{\meter}$ &\\
			\hline
		\end{tabular}
\end{minipage}
\vspace{0.4 cm}

\begin{minipage}{0.48\textwidth}
	{\bf \item  Bei einer Sonnenfinsternis erscheinen Mond und Sonne am Himmel deckungsgleich.}
	\begin{enumerate}[a)]
		\item Um wieviel Mal ist die Sonne größer als der Mond?
		\item  Welchen Durchmesser hat der Mond ?
	\end{enumerate}  
\end{minipage}
\hspace{1 cm}
\begin{minipage}{0.47\textwidth}
	\renewcommand{\arraystretch}{2}
	\begin{tabular}{|>{\columncolor[gray]{.8}}p{2.1 cm}|p{2.1 cm}|p{2.1 cm}|}
		\hline
		\rowcolor[gray]{.8}	&Mond & Sonne \\
		\hline
		Entfernung &$\SI{384000}{\kilo\meter}$ & $\SI{150e6}{\kilo\meter}$ \\
		\hline
		Durchmesser & &$\SI{1,4e6}{\kilo\meter}$ \\
		\hline
	\end{tabular}
\end{minipage}

\vspace{0.4 cm}
\begin{minipage}{0.8\textwidth}
	{\bf \item Lunerald macht eine Radtour zu den Alpen.} Am Horizont sieht er schon sein Ziel. Er streckt den Arm aus und kann den $\SI{1731}{\meter}$ hohen Herzogstand genau mit seiner Handfläche verdecken.
	\begin{enumerate}[a)]
		\item Wie weit muss Lunerald noch bis zu dem Berg fahren? Welche Angaben, die du noch zur Berechnung brauchst, kannst du hierfür leicht selbst ermitteln?
		\item Am Fuß der Berge angekommen, fährt er $\SI{8}{\kilo\meter}$ weit die Passstraße hinauf. Welchen Höhenunterschied hat er dann überwunden?
	\end{enumerate} 
\end{minipage}
\hfill
\begin{minipage}{0.2\textwidth}
	\includegraphics[width=2.5 cm]{vstg103}		
\end{minipage}
{\vspace{-0.3 cm}}

\begin{minipage}{0.5\textwidth}
{\bf \item Försterdreieck:} Die Höhe von Bäumen lässt sich indirekt bestimmen. Hierzu misst man den horizontalen Abstand zum Baum und hält sich folgende Anordnung vor die Augen:
\end{minipage}
\hfill
\begin{minipage}{0.5\textwidth}
	\includegraphics[width=8 cm]{forst100}		
\end{minipage}
\begin{enumerate}[a)]
	\item Wie hoch ist der Baum wenn $x=\SI{70}{\centi\meter}$ und $y=\SI{21}{\centi\meter}$ ist und er $\SI{30}{\meter}$ vom Betrachter entfernt ist ?
    \item Dieser Baum steht an einem Fluss. Wie breit ist dieser Fluss, wenn man die Spitze des Baumes vom gegenüberliegenden Ufer aus anpeilt und $x=\SI{70}{\centi\meter}$ und $y=\SI{7}{\centi\meter}$ ist ?
\end{enumerate} 
 


\fbox{
	\begin{minipage}{0.5\textwidth}
		Zur Lösung bitte \href{https://www.okuyakl.de/math/m9vsaL103/ll103.pdf}{hier klicken} oder den QR-Code scannen.\\
	Weitere Arbeitsblätter gibt es unter 
	
	\href{https://www.okuyakl.de}{www.okuyakl.de}
	\end{minipage}
	\hfill
	\begin{minipage}{0.4\textwidth}
		\includegraphics[width=1.5 cm]{../../viecher/zwe03}
		\includegraphics[width=3 cm]{qr103}
		\includegraphics[width=2 cm]{../../viecher/afanticon1}
		
\end{minipage}}

\end{enumerate}
\end{document}

