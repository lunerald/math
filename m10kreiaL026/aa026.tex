\documentclass[a4paper]{article}
\usepackage[pdftex]{graphicx}
\usepackage[utf8]{inputenc}
\usepackage{enumerate}
\usepackage{icomma}
\usepackage{siunitx}
\sisetup{locale=DE} 
\usepackage{amssymb}
\usepackage{geometry}
\geometry{a4paper, top=15mm, left=15mm, right=15mm, bottom=15mm,
	headsep=10mm, footskip=12mm}
\usepackage{href-ul}
\hypersetup{
	colorlinks=true,
	linkcolor=blue,
	urlcolor=blue}
\begin{document}
{\bf Kreisteile}
\begin{enumerate}[1.]

{\bf \item Ein Kreissektor hat einen Flächeninhalt $A$ von $\SI{12,5}{\centi\meter^2}$. Die Länge des Kreisbogens $b$ beträgt  $\SI{8}{\centi\meter}$.} Berechne das Maß des Winkels $\mu$.

%Aufgabe 3
\begin{minipage}{0.6\textwidth}
{\bf \item Berechne den Umfang und den Flächeninhalt der blauen Figur für $a=\SI{5}{\centi\meter}$}
\end{minipage}
\hfill
\begin{minipage}{0.35\textwidth}
\includegraphics[width=5 cm]{sektu204}
\end{minipage}

%Aufgabe 4
\begin{minipage}{0.6\textwidth}
{\bf \item Zeige rechnerisch, dass sich der Flächeninhalt der grünen Figur in Abhängigkeit von $a$ folgendermaßen darstellen lässt:}
$$ A = a^2 ( 5+ \pi ) {\rm FE}$$
\end{minipage}
\hfill
\begin{minipage}{0.35\textwidth}
\includegraphics[width=5 cm]{quak204}
\end{minipage}

\begin{minipage}{0.4\textwidth}
{\bf \item Rechne in das jeweils andere Maß um:}\\

i) $225^\circ$ \hspace{20 pt} ii) $100^\circ$ \hspace{20 pt} iii) $0,80$ \hspace{20 pt} 
iv) $ {7 \over 4}\pi$
\end{minipage}
\hspace{1 cm}
\begin{minipage}{0.4\textwidth}
		{\bf \item Ergänze die fehlenden Größen:}
	$$
	\renewcommand{\arraystretch}{2}
	\begin{array}{|c|c|c|c|}
		\hline
		\qquad \mu \qquad & \qquad r \qquad &\qquad b\qquad & \qquad A_S \qquad \\
		\hline
		50^\circ & \SI{2,3}{\deci\meter} & & \\
		\hline
		& \SI{4,5}{\centi\meter} & \SI{10}{\centi\meter} & \\
		\hline
		130^\circ & & & \SI{75}{\milli\meter^2} \\
		\hline
	\end{array}
	$$
\end{minipage}

\end{enumerate} 

\fbox{
	\begin{minipage}{0.5\textwidth}
		Zur Lösung bitte \href{https://www.okuyakl.de/math/m10kreiaL026/ll026.pdf}{hier klicken} oder den QR-Code scannen.\\
	Weitere Arbeitsblätter gibt es unter 
	
	\href{https://www.okuyakl.de}{www.okuyakl.de}
	\end{minipage}
	\hfill
	\begin{minipage}{0.4\textwidth}
		\includegraphics[width=1.5 cm]{../../viecher/zwe03}
		\includegraphics[width=3 cm]{qr026}
		\includegraphics[width=2 cm]{../../viecher/afanticon1}
		
	\end{minipage}}

\end{document}%Lösung-------------------------------------------
