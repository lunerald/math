\documentclass[a4paper]{article}
\usepackage[pdftex]{graphicx}
\usepackage[utf8]{inputenc}
\usepackage{enumerate}
\usepackage{amssymb}
\usepackage{icomma}
\usepackage{siunitx}
\sisetup{                     
	locale=DE} 
\usepackage{geometry}
\geometry{a4paper, top=15mm, left=15mm, right=15mm, bottom=15mm,
	headsep=10mm, footskip=12mm}
\usepackage{href-ul}
\hypersetup{
	colorlinks=true,
	linkcolor=blue,
	urlcolor=blue}
	
\begin{document}
{\bf Geometrische Orte}
\begin{enumerate}[1.]

\begin{minipage}{0.6\textwidth}
{\bf \item Zeichne einen Kreis $k(A; r= \SI{3}{\centi\meter})$ und daneben einen Punkt} $P$ mit $d(P;A)=\SI{5}{\centi\meter}$ (Siehe Skizze).
\begin{enumerate}[a)]
\item  Konstruiere die Tangenten $t_1$ und $t_2$ an den Kreis $k_A$ durch $P$\\
\item Gib den Fachbegriff der geometrischen Ortslinie an, die man zur Konstruktion der beiden Tangenten benötigt.
\end{enumerate}
\end{minipage}
\hfill
\begin{minipage}{0.4\textwidth}
\includegraphics[width=5.5 cm]{tang052.pdf}
\end{minipage}

{\bf \item Gib in Mengenschreibweise den gesuchten Ortsbereich an:}\\
Zur Menge $M$ gehören alle Punkte $P$, die von einer Geraden $h$ einen Abstand von $8~cm$ haben.\\
Wie bezeichnet man diesen Ortsbereich?

{\bf \item Beschreibe die in folgenden Abbildungen blau gekennzeichnete Punktmengen in mathematisch korrekter Mengenschreibweise:}

\vspace{10 pt}
\begin{minipage}{0.48\textwidth}
a)\vfill
\includegraphics[width=5 cm]{ortkreis172.pdf}
\end{minipage}
\hfill
\begin{minipage}{0.48\textwidth}
b)\vfill
\includegraphics[width=5 cm]{wihalb172.pdf}
\end{minipage}

\vspace{10 pt}
c) Zeige rechnerisch mit Hilfe der rechten Abbildung, dass die Geraden $w_1$ und $w_2$ senkrecht aufeinander stehen.

{\bf \item Gib an um welche Ortslinie bzw. welchen Ortsbereich es sich handelt.}

Wo liegt die Menge aller Punkte $P$, die $ \dots $

\begin{enumerate}[a)]
\item $ \dots $ von zwei sich schneidenden Geraden $ g_1$ und $ g_2$ den gleichen Abstand haben?\item  $ \dots $ von den parallelen Geraden $ h_1 $ und $ h_2 $ gleichen Abstand haben ?
\item  $ \dots $ von einem Punkt weniger als 2 cm entfernt sind ?
Gib diesen Ortsbereich auch in Kurzschreibweise an!
\end{enumerate}

{\bf \item Gib an, um welche Ortslinie bzw. welchen Ortsbereich es sich handelt und zeichne die Lösung farbig: }

Zeichne eine Gerade $g$ und markiere dann farbig die Menge aller Punkte  $P$, die von der Geraden den Abstand 1,5 cm haben

{\bf \item Handelt es sich bei den folgenden Orten um Ortslinien oder um Ortsbereiche?} Beschreibe diese und begründe deine Antwort.

\begin{enumerate}[a)]
\item $$ M_1 = \{ P\, | \,\overline{MP} = \SI{2,8}{\centi\meter} \}$$
\item  $$ M_1 = \{ Q \, | \, \overline{MQ} \ge r_2 \}$$
\end{enumerate}

{\bf \item Gegeben sind die Punkte $A(-4|-1)$, $B(5|-3)$ und $C(1|4)$.} Zeichne die Punkte A, B, C in ein Koordinatensystem.

Kennzeichne folgende Punktmengen farbig im Koordinatensystem
\begin{enumerate}[a)]
\item $\{ P| \angle APC >90^\circ\} $
\item $\{ P'| \overline{P'A} \ge \overline{P'B} \}$
\end{enumerate}

\begin{minipage}{0.5\textwidth}
{\bf \item Gib die schraffierte Punktmenge in Mengenschreibweise an.} Achte auf die Ränder.
\end{minipage}
\hfill
\begin{minipage}{0.45\textwidth}
\includegraphics[width=5 cm]{kreisort160.pdf}
\end{minipage}

\end{enumerate}

\fbox{
	\begin{minipage}{0.5\textwidth}
	Zur Lösung bitte \href{https://www.okuyakl.de/math/m8ortL052/ll052.pdf}{hier klicken} oder den QR-Code scannen.\\
	Weitere Arbeitsblätter gibt es unter 
	
	\href{https://www.okuyakl.de}{www.okuyakl.de}
	\end{minipage}
	\hfill
	\begin{minipage}{0.4\textwidth}
		\includegraphics[width=1.5 cm]{../../viecher/zwe03}
		\includegraphics[width=3 cm]{qr052}
		\includegraphics[width=2 cm]{../../viecher/afanticon1}
		
	\end{minipage}}
\end{document}%Lösung-------------------------------------------
