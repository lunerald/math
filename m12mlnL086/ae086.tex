\documentclass[a4paper]{article}
\usepackage[pdftex]{graphicx}
\usepackage[utf8]{inputenc}
\usepackage{enumerate}
\usepackage{amssymb}
\usepackage{icomma}
\usepackage{siunitx}
\sisetup{locale=DE}
\usepackage{href-ul}
\hypersetup{
	colorlinks=true,
	linkcolor=blue,
	urlcolor=blue}
\usepackage{geometry}
\geometry{a4paper, top=15mm, left=15mm, right=15mm, bottom=15mm,
	headsep=10mm, footskip=12mm}

\begin{document}
	{\bf pattern curve discussion ln function}
	
	\begin{enumerate}[1.]
		{\bf \item The function $$f(x)=x\cdot \ln x$$ is given
			Conduct a full curve discussion. }
		\begin{enumerate}[a)]
			\item Determine the definition set $\mathbb{D}_f$ and the behavior at infinity.
			\item Examine the function for symmetry.
			\item Calculate all zeros.
			\item Take the first, second and third derivatives.
			\item Investigate the monotonicity behavior and the type and location of the extreme point using a monotonicity table. Determine the set of values $\mathbb{W}_f$ of $f(x)$.
			\item Determine the curvature behavior and examine for turning points.
			\item Specify the term of the tangent through the zero of $f(x)$.
			\item Draw the graph $G_f$ with the tangent in a suitable coordinate system based on the knowledge gained.
			\item Show that $F(x)={1\over 2}x^2 \cdot \left(\ln x -{1\over2}\right)$ is an antiderivative of $f(x)$.
			\item The graph $G_f$, the $x$-axis and the line $x=u$ enclose an area. Calculate its content for $u \to 0$.
		\end{enumerate}
		
	\end{enumerate}
	\fbox{
		\begin{minipage}{0.5\textwidth}
			For the solution, please \href{https://www.okuyakl.de/math/m12mlnL086/le086.pdf}{click here} or scan the QR code.\\
			Additional worksheets are available at
			
			\href{http://www.okuyakl.com}{www.okuyakl.com}
		\end{minipage}
		\hfill
		\begin{minipage}{0.4\textwidth}
			\includegraphics[width=1.5 cm]{../../viecher/zwe03}
			\includegraphics[width=3 cm]{qre086}
			\includegraphics[width=2 cm]{../../viecher/afanticon1}
			
	\end{minipage}}
\end{document}