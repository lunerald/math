\documentclass[a4paper]{article}
\usepackage[pdftex]{graphicx}
\usepackage[utf8]{inputenc}
\usepackage{enumerate}
\usepackage{amssymb}
\usepackage{geometry}
\geometry{a4paper, top=15mm, left=15mm, right=15mm, bottom=15mm,
	headsep=10mm, footskip=12mm}
\usepackage{href-ul}
\hypersetup{
	colorlinks=true,
	linkcolor=blue,
	urlcolor=blue}
	
% Start the document
\begin{document}
%Titel
{\bf Trainingsblatt Normalparabeln }\\

\noindent{\bf Gegeben sind folgende Normalparabeln: $f_1(x)$ bis $f_9(x)$, von links nach rechts.}

\noindent\includegraphics[width=17 cm]{vs015}

%Aufgabe 1
{\bf Bestimme alle Parameter für eine Normalparabel und fülle die Tabelle aus:}\\
\renewcommand{\arraystretch}{2}
\begin{tabular}{|p{40 pt}|p{50 pt}|p{50 pt}|p{100 pt}|p{100 pt}|p{100 pt}|}
\hline
Funktion & Öffnung  & Scheitel & Wertemenge & Funktionsterm &Anzahl  Nullstellen  \\
   &  $\pm 1$ &  $(x|y)$ &  $\mathbb{W}=$ &  $f(x)=$ &    \\
\hline
$f_1(x)$  & & & & & \\
\hline
$f_2(x)$  & & & & & \\
\hline
$f_3(x)$  & & & & & \\
\hline
$f_4(x)$  & & & & & \\
\hline
$f_5(x)$  & & & & & \\ 
\hline
$f_6(x)$  & & & & & \\ 
\hline
$f_7(x)$  & & & & & \\ 
\hline
$f_8(x)$  & & & & & \\ 
\hline
$f_9(x)$  & & & & & \\ 
\hline
\end{tabular}

\fbox{
	\begin{minipage}{0.5\textwidth}
		Zur Lösung bitte \href{https://www.okuyakl.de/math/m9nopL015/ll015.pdf}{hier klicken} oder den QR-Code scannen.\\
	Weitere Arbeitsblätter gibt es unter 
	
	\href{https://www.okuyakl.de}{www.okuyakl.de}
	\end{minipage}
	\hfill
	\begin{minipage}{0.4\textwidth}
		\includegraphics[width=1.5 cm]{../../viecher/zwe03}
		\includegraphics[width=3 cm]{qr015}
		\includegraphics[width=2 cm]{../../viecher/afanticon1}
		
	\end{minipage}}

\end{document}%Lösung---------------------------------------------------------------
