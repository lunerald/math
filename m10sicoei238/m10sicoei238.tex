
\documentclass[a4paper]{article}
% Uncomment the following line to allow the usage of graphics (.png, .jpg)
\usepackage[pdftex]{graphicx}
% Allow the usage of utf8 characters
\usepackage[utf8]{inputenc}
\usepackage{enumerate}
\usepackage{icomma}
\usepackage{amssymb}
\usepackage{geometry}
\usepackage{colortbl}
\geometry{a4paper, top=15mm, left=15mm, right=15mm, bottom=15mm,
	headsep=10mm, footskip=12mm}

% Start the document
\begin{document}
%Titel
{\bf Trainingsblatt Sinus und Kosinus am Einheitskreis }\\
\begin{enumerate}[1.]
%Aufgabe 1
{\bf \item Trage den Winkel $\varphi$ am Einheitskreis ein. Ermittle dann durch geeignete Messungen Näherungswerte für $\sin{\varphi}$ und 
$\cos{\varphi}$. Überprüfe die Ergebnisse mit dem Taschenrechner}\\


\begin{enumerate}[a)]

\begin{minipage}{0.5\textwidth}
\item $\varphi=61^\circ$\\
\includegraphics[width=7 cm]{eikrei238}\\
$\sin{\varphi}=$ \rule{40 pt}{0.5 pt}\\
\vspace{5 pt}\\
$\cos{\varphi}=$ \rule{40 pt}{0.5 pt}\\
\end{minipage}
\hfill
\begin{minipage}{0.5\textwidth}
\item $\varphi=197^\circ$\\
\includegraphics[width=7 cm]{eikrei238}\\
$\sin{\varphi}=$ \rule{40 pt}{0.5 pt}\\
\vspace{5 pt}\\
$\cos{\varphi}=$ \rule{40 pt}{0.5 pt}\\
\end{minipage}\\

\begin{minipage}{0.5\textwidth}
\item $\varphi=322^\circ$\\
\includegraphics[width=7 cm]{eikrei238}\\
$\sin{\varphi}=$ \rule{40 pt}{0.5 pt}\\
\vspace{5 pt}\\
$\cos{\varphi}=$ \rule{40 pt}{0.5 pt}\\
\end{minipage}
\hfill
\begin{minipage}{0.5\textwidth}
\item $\varphi=137^\circ$\\
\includegraphics[width=7 cm]{eikrei238}\\
$\sin{\varphi}=$ \rule{40 pt}{0.5 pt}\\
\vspace{5 pt}\\
$\cos{\varphi}=$ \rule{40 pt}{0.5 pt}\\
\end{minipage}
\end{enumerate}

%Aufgabe 2
{\bf \item Berechne mit dem Taschenrechner die fehlenden Werte}\\

\renewcommand{\arraystretch}{2}
\begin{tabular}{|>{\columncolor[gray]{.8}}p{3.5 cm}|p{1.5 cm}|p{1.5 cm}|p{1.5 cm}|p{1.5 cm}|p{1.5 cm}|p{1.5 cm}|}
	\hline
Winkel Gradmaß	& $1^\circ$ &$7230^\circ$ & & &$225^\circ$ & \\
	\hline
Winkel Bogenmaß	& & &$1$ &${1\over 6}\pi$ & & $3,142$\\
	\hline
$\sin{\varphi}$	& & & & & &\\
	\hline
$\cos{\varphi}$	& & & & & &\\
	\hline
\end{tabular}

\newpage
%Aufgabe 3
{\bf \item Bestimme mit dem Taschenrechner und mithilfe geeigneter Eintragungen im Einheitskreis alle Winkel im Gradmaß, für die gilt:}

\begin{enumerate}[a)]

\begin{minipage}{0.5\textwidth}
\item $\sin{\alpha}=0,422$\\
\begin{tabular}{ccc}
$\Rightarrow$ & $\alpha_1$ &= \rule{50 pt}{0.5 pt}\\
\vspace{5 pt}\\
                       & $ \alpha_2$ &= \rule{50 pt}{0.5 pt}
\end{tabular} 
\end{minipage}
\hfill
\begin{minipage}{0.5\textwidth}
\includegraphics[width=7 cm]{eikrei238}
\end{minipage}\\

\begin{minipage}{0.5\textwidth}
\item $\sin{\beta}=-0,887$\\
\begin{tabular}{ccc}
$\Rightarrow$ & $\beta_1$ &= \rule{50 pt}{0.5 pt}\\
\vspace{5 pt}\\
                       & $ \beta_2$ &= \rule{50 pt}{0.5 pt}
\end{tabular} 
\end{minipage}
\hfill
\begin{minipage}{0.5\textwidth}
\includegraphics[width=7 cm]{eikrei238}
\end{minipage}\\

\begin{minipage}{0.5\textwidth}
\item $\cos{\gamma}=-0,310$\\
\begin{tabular}{ccc}
$\Rightarrow$ & $\gamma_1$ &= \rule{50 pt}{0.5 pt}\\
\vspace{5 pt}\\
                       & $ \gamma_2$ &= \rule{50 pt}{0.5 pt}
\end{tabular} 
\end{minipage}
\hfill
\begin{minipage}{0.5\textwidth}
\includegraphics[width=7 cm]{eikrei238}
\end{minipage}
\end{enumerate}

%Aufgabe 4
{\bf \item Bestimme Intervalle von Winkeln, für die gilt:}\\

\begin{enumerate}[a)]
\begin{minipage}{0.5\textwidth}
\item $\sin{\varphi} > 0 $ und  $\cos{\varphi} >0 $:\rule{50 pt}{0.5 pt}\\
\item $\sin{\varphi} < 0$ und $\cos{\varphi} <0 $: \rule{50 pt}{0.5 pt}\\
\end{minipage}
\begin{minipage}{0.5\textwidth}
\item $\sin{\varphi} > \cos{\varphi}$ : \rule{50 pt}{0.5 pt}\\
\item $\sin{\varphi} \le \cos{\varphi}$ : \rule{50 pt}{0.5 pt}
\end{minipage}
\end{enumerate}
\end{enumerate} 

%Ende
\end{document}