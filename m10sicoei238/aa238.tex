\documentclass[a4paper]{article}
\usepackage[pdftex]{graphicx}
\usepackage[utf8]{inputenc}
\usepackage{enumerate}
\usepackage{icomma}
\usepackage{tikz}
\usepackage{href-ul}
\hypersetup{
	colorlinks=true,
	linkcolor=blue,
	urlcolor=blue}
\usepackage{amssymb}
\usepackage{geometry}
\usepackage{colortbl}
\geometry{a4paper, top=15mm, left=15mm, right=15mm, bottom=15mm,
	headsep=10mm, footskip=12mm}

% Start the document
\begin{document}
%Titel
{\bf Sinus und Kosinus im Einheitskreis }\\
\begin{enumerate}[1.]
%Aufgabe 1
{\bf \item Trage den Winkel $\varphi$ am Einheitskreis ein. Ermittle dann durch geeignete Messungen Näherungswerte für $\sin{\varphi}$ und 
$\cos{\varphi}$. Überprüfe die Ergebnisse mit dem Taschenrechner}\\


\begin{enumerate}[a)]

\begin{minipage}{0.5\textwidth}
\item $\varphi=61^\circ$\\
\includegraphics[width=7 cm]{eikrei238}\\
$\sin{\varphi}=$ \rule{40 pt}{0.5 pt}\\
\vspace{5 pt}\\
$\cos{\varphi}=$ \rule{40 pt}{0.5 pt}\\
\end{minipage}
\hfill
\begin{minipage}{0.5\textwidth}
\item $\varphi=197^\circ$\\
\includegraphics[width=7 cm]{eikrei238}\\
$\sin{\varphi}=$ \rule{40 pt}{0.5 pt}\\
\vspace{5 pt}\\
$\cos{\varphi}=$ \rule{40 pt}{0.5 pt}\\
\end{minipage}\\
\hrule
\vspace{0.5 cm}
\begin{minipage}{0.5\textwidth}
\item $\varphi=322^\circ$\\
\includegraphics[width=7 cm]{eikrei238}\\
$\sin{\varphi}=$ \rule{40 pt}{0.5 pt}\\
\vspace{5 pt}\\
$\cos{\varphi}=$ \rule{40 pt}{0.5 pt}\\
\end{minipage}
\hfill
\begin{minipage}{0.5\textwidth}
\item $\varphi=137^\circ$\\
\includegraphics[width=7 cm]{eikrei238}\\
$\sin{\varphi}=$ \rule{40 pt}{0.5 pt}\\
\vspace{5 pt}\\
$\cos{\varphi}=$ \rule{40 pt}{0.5 pt}\\
\end{minipage}
\end{enumerate}

%Aufgabe 2
{\bf \item Berechne mit dem Taschenrechner die fehlenden Werte}\\

\renewcommand{\arraystretch}{2}
\begin{tabular}{|>{\columncolor[gray]{.8}}p{3.5 cm}|p{1.5 cm}|p{1.5 cm}|p{1.5 cm}|p{1.5 cm}|p{1.5 cm}|p{1.5 cm}|}
	\hline
Winkel Gradmaß	& $1^\circ$ &$7230^\circ$ & & &$225^\circ$ & \\
	\hline
Winkel Bogenmaß	& & &$1$ &${1\over 6}\pi$ & & $3,142$\\
	\hline
$\sin{\varphi}$	& & & & & &\\
	\hline
$\cos{\varphi}$	& & & & & &\\
	\hline
\end{tabular}

\newpage

%Aufgabe 4
{\bf \item Markiere die Quadranten im Einheitskreis mit "+" oder "-", in denen...}\\
\vspace{0.5 cm}
\begin{enumerate}[a)]
\begin{minipage}{0.5\textwidth}
	\item ... der Sinus positiv oder negativ ist.
	
	\includegraphics[width=7 cm]{eikrei238}
\end{minipage}
\hfill
\begin{minipage}{0.5\textwidth}
	\item ... der Kosinus positiv oder negativ ist.
	
	\includegraphics[width=7 cm]{eikrei238}
\end{minipage}
\end{enumerate}

{\bf \item Bestimme Intervalle von Winkeln, für die die folgenden Bedingungen gelten und markiere die\\ jeweiligen Bereiche im Einheitskreis:}\\

\begin{enumerate}[a)]
\begin{minipage}{0.5\textwidth}
\item $\sin{\varphi} > 0 $ und  $\cos{\varphi} >0 $:\qquad $\varphi \in$ \rule{50 pt}{0.5 pt}\\

\includegraphics[width=7 cm]{eikrei238}
\end{minipage}
\hfill
\begin{minipage}{0.5\textwidth}
\item $\sin{\varphi} < 0$ und $\cos{\varphi} <0 $:\qquad $\varphi \in$ \rule{50 pt}{0.5 pt}\\

\includegraphics[width=7 cm]{eikrei238}
\end{minipage}
\hrule
\vspace{0.5 cm}
\begin{minipage}{0.5\textwidth}
\item $\sin{\varphi} > \cos{\varphi}$ :\qquad $\varphi \in$ \rule{50 pt}{0.5 pt}\\

\includegraphics[width=7 cm]{eikrei238}
\end{minipage}
\hfill
\begin{minipage}{0.5\textwidth}
\item $\sin{\varphi} \le \cos{\varphi}$ :\qquad $\varphi \in$ \rule{50 pt}{0.5 pt}\\

\includegraphics[width=7 cm]{eikrei238}
\end{minipage}
\end{enumerate}

\newpage
%Aufgabe 3
{\bf \item Bestimme mit dem Taschenrechner und mithilfe geeigneter Eintragungen im Einheitskreis alle Winkel im Gradmaß, für die gilt:}
\vspace{0.5 cm}

\begin{enumerate}[a)]
	
	\begin{minipage}{0.5\textwidth}
		\item $\sin{\alpha}=0,422$\\
		\begin{tabular}{ccc}
			$\Rightarrow$ & $\alpha_1$ &= \rule{50 pt}{0.5 pt}\\
			
			& $ \alpha_2$ &= \rule{50 pt}{0.5 pt}
			
		\end{tabular} 
		
		\includegraphics[width=7 cm]{eikrei238}
	\end{minipage}
	\hfill
	\begin{minipage}{0.5\textwidth}
		\item $\sin{\beta}=-0,887$\\
		\begin{tabular}{ccc}
			$\Rightarrow$ & $\beta_1$ &= \rule{50 pt}{0.5 pt}\\
			& $ \beta_2$ &= \rule{50 pt}{0.5 pt}
		\end{tabular}
		
		\includegraphics[width=7 cm]{eikrei238}
	\end{minipage}\\
	\hrule
	\vspace{0.5 cm}
	\begin{minipage}{0.5\textwidth}
		\item $\cos{\gamma}=-0,310$\\
		\begin{tabular}{ccc}
			$\Rightarrow$ & $\gamma_1$ &= \rule{50 pt}{0.5 pt}\\
			& $ \gamma_2$ &= \rule{50 pt}{0.5 pt}
		\end{tabular} 
		
		\includegraphics[width=7 cm]{eikrei238}
	\end{minipage}
	\begin{minipage}{0.5\textwidth}
		\item $\cos{\delta}=0,724$\\
		\begin{tabular}{ccc}
			$\Rightarrow$ & $\delta_1$ &= \rule{50 pt}{0.5 pt}\\
			& $ \delta_2$ &= \rule{50 pt}{0.5 pt}
		\end{tabular} 
		
		\includegraphics[width=7 cm]{eikrei238}
	\end{minipage}
\end{enumerate}
\end{enumerate} 

\fbox{
	\begin{minipage}{0.5\textwidth}
		Weitere Arbeitsblätter gibt es unter 
		
		\href{https://www.okuyakl.de}{www.okuyakl.de}
	\end{minipage}
	\hfill
	\begin{minipage}{0.4\textwidth}
		\includegraphics[width=1.5 cm]{../../viecher/zwe03}
		\includegraphics[width=2 cm]{../../viecher/afanticon1}
		
\end{minipage}}
\end{document}