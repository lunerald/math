\documentclass[a4paper]{article}
\usepackage[pdftex]{graphicx}
\usepackage[utf8]{inputenc}
\usepackage{enumerate}
\usepackage{icomma}
\usepackage{siunitx}
\sisetup{locale=DE} 
\usepackage{amssymb}
\usepackage{geometry}
\geometry{a4paper, top=15mm, left=15mm, right=15mm, bottom=15mm,
	headsep=10mm, footskip=12mm}
\usepackage{href-ul}
\hypersetup{
	colorlinks=true,
	linkcolor=blue,
	urlcolor=blue}
	
\begin{document}

{\bf Satzgruppe Pythagoras für Experten}

\begin{enumerate}[1.]

{\bf \item Gegeben sei ein Dreieck $EFG$ mit rechtem Innenwinkel bei $G$ und der Höhe $h$.} Kreuze jeweils an, ob der Zusammenhang für dieses abgebildete Dreieck wahr oder falsch ist.

\begin{minipage}{0.48\textwidth}
\includegraphics[width= 5 cm]{pyth88}
\end{minipage}
\hfill
\begin{minipage}{0.48\textwidth}
\renewcommand{\arraystretch}{3}
\begin{tabular}{|c|c|c|}
\hline
Formel & Wahr & Falsch \\
\hline
$\frac{e^2}{s}=g$ & $ \square $  &  $ \square $  \\
\hline
$ g^2 - e^2 = f^2 $ & $ \square $  &  $ \square $  \\
\hline
$\frac{h^2}{s}=r$  & $ \square $  &  $ \square $  \\
\hline
$s^2=\sqrt{h^2-e^2}$ & $ \square $  &  $ \square $ \\
\hline
\end{tabular}
\end{minipage}

{\bf \item Die Cheops--Pyramide in Ägypten ist die älteste und größte der drei Pyramiden von Gizeh und wird deshalb auch als ,,Große Pyramide" bezeichnet.} Sie ist eine regelmäßige Pyramide mit quadra\-tischer Grundfläche der Seitenlänge $\SI{230}{\meter}$. Die Höhe betrug ursprünglich etwa 
$\SI{140}{\meter}$. Berechne die Länge einer Seitenkante.

{\bf \item Ein $\SI{25}{\meter}$ hoher Baum ist so abgeknickt, dass seine Spitze $\SI{5}{\meter}$ von seinem Fuß entfernt aufliegt.} Berechne, in welcher Höhe in Metern der Baum abgeknickt ist.

{\bf \item Zeige, dass jedes Dreieck mit den Seitenlängen $ 4n^2-1,~ 4n^2+1 ~\textnormal{und} ~4n; \\
n \in \mathbb{N}$ rechtwinklig ist.} (Überlege  zunächst, welcher der drei Terme die Sei\-tenlänge der Hypotenuse beschreibt).

%Aufgabe 4
\begin{minipage}{0.45\textwidth}
{\bf \item Dem Halbkreis mit dem Radius $r=\overline{AM}=\SI{8}{\centi\meter}$ ist ein rechtwinkliges Dreieck einbeschrieben.} Die Strecke $ \overline{CD} $ ist $\SI{4}{\centi\meter}$ lang. Berechne die Länge $x$ der Strecke  $\overline{MD}$
\end{minipage}
\hfill
\begin{minipage}{0.45\textwidth}
\includegraphics[width=7 cm]{thal045}
\end{minipage}

{\bf \item Ein Computerbildschirm hat eine Diagonale von $\SI{70}{\centi\meter}$}. Wie sind dessen Breite und Höhe, wenn deren Verhältnis 16 zu 9 ist?
\end{enumerate} 

\fbox{
	\begin{minipage}{0.5\textwidth}
		Zur Lösung bitte \href{https://www.okuyakl.de/math/m9pyaL045/ll045.pdf}{hier klicken} oder den QR-Code scannen.\\
	Weitere Arbeitsblätter gibt es unter 
	
	\href{https://www.okuyakl.de}{www.okuyakl.de}
	\end{minipage}
	\hfill
	\begin{minipage}{0.4\textwidth}
		\includegraphics[width=1.5 cm]{../../viecher/zwe03}
		\includegraphics[width=3 cm]{qr045}
		\includegraphics[width=2 cm]{../../viecher/afanticon1}
		
	\end{minipage}}

\end{document}%Lösung-------------------------------------------
