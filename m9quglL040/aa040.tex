\documentclass[a4paper]{article}
\usepackage[pdftex]{graphicx}
\usepackage[utf8]{inputenc}
\usepackage{enumerate}
\usepackage{amssymb}
\usepackage{color}
\usepackage{geometry}
\geometry{a4paper, top=15mm, left=15mm, right=15mm, bottom=15mm,
	headsep=10mm, footskip=12mm}
\usepackage{href-ul}
\hypersetup{
	colorlinks=true,
	linkcolor=blue,
	urlcolor=blue}
\begin{document}
{\bf Quadratische Gleichungen und Parameteraufgaben}
\begin{enumerate}[1.]

{\bf \item Ergänze:}
$$\frac{1}{2} \cdot (8x+ \dots)^2 = \dots x^2 + 8xy + \dots $$

{\bf \item Berechne die Lösungsmenge und stelle sie graphisch an der Zahlengeraden dar:}
$$ a) \quad -4x^2+2x \ge -2 \hspace{25 pt} b) \quad -\frac{1}{2}x^2+2x >2 $$

{\bf \item Gib die Anzahl der Lösungen in Abhängigkeit von $a\in \mathbb{R}$ an.}
$$ ax^2-\frac{3}{2}x + 1=0$$

{\bf \item Für welche Steigung $m$ ist die Gerade Tangente an die Parabel?}

\begin{tabular}{ccl}
$g:~y$&=&$mx+\frac{3}{2}$\\
$p:~y$&=&$x^2-\frac{3}{2}x+\frac{3}{2}$
\end{tabular}

\noindent Veranschauliche durch eine einfache Freihandskizze die Lage von Gerade und Parabel für $m=1$.


{\bf \item Gegeben ist die Gerade} $$h: \quad y=\frac{9}{16}x-1,25 $$ und die Geradenschar 
$$g_a: -4y=-ax-5$$.
\begin{enumerate}[a)]
	\item Zeige, dass sich die x-Koordinaten der Schnittpunkte $S_a(x|y)$ der Geraden $h$ mit der Geradenschar $g_a$ wie folgt in Abhängigkeit von $a$ darstellen lassen: 
	$$x= \frac{40}{9-4a}$$
	\item Für welchen Wert von $a$ gibt es keinen Schnittpunkt?
	\item Berechne den Wert von $a$ für $x=-8$. 
\end{enumerate}



\end{enumerate} 

\fbox{
	\begin{minipage}{0.5\textwidth}
			Zur Lösung bitte \href{https://www.okuyakl.de/math/m9quglL040/ll040.pdf}{hier klicken} oder den QR-Code scannen.\\
		Weitere Arbeitsblätter gibt es unter 
		
		\href{https://www.okuyakl.de}{www.okuyakl.de}
	\end{minipage}
	\hfill
	\begin{minipage}{0.4\textwidth}
		\includegraphics[width=1.5 cm]{../../viecher/zwe03}
		\includegraphics[width=3 cm]{qr040}
		\includegraphics[width=2 cm]{../../viecher/afanticon1}
		
	\end{minipage}}

\end{document}%Lösung-------------------------------------------
