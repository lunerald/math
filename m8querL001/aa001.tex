
\documentclass[landscape]{article}
\usepackage[pdftex]{graphicx}
\usepackage[utf8]{inputenc}
\usepackage{enumerate}
\usepackage{amssymb}
\usepackage{icomma}
\usepackage[a4paper]{geometry}
\geometry{a4paper, top=15mm, left=15mm, right=15mm, bottom=15mm,
	headsep=10mm, footskip=12mm}
\usepackage{href-ul}
\hypersetup{
	colorlinks=true,
	linkcolor=blue,
	urlcolor=blue}
	
\begin{document}

{\bf Trainingsblatt Extremwertbestimmung mit der quadratischen Ergänzung }\\
\begin{enumerate}[1.]

\begin{minipage}{10 cm}
{\bf \item Ausgangsform $ T(x)= x^2 + px + q$:}\\
$
\renewcommand{\arraystretch}{1,5}
\begin{array}{rcl}
T(x)&=& x^2 + 3x + 6 \\
      &=& x^2 + 2\cdot 1,5x +6 \\
      &=& x^2 +  2\cdot 1,5x + 1,5^2 - 1,5^2 +6 \\
      &=& (x^2 + 2 \cdot 1,5x + 1,5^2) - 1,5^2 +6 \\
      &=& (x^2 + 2 \cdot 1,5x + 1,5^2)+ 3,75 \\
      &=&  (x + 1,5)^2 + 3,75
\end{array}
$
$$\Rightarrow T_{min}=3,75 \quad \textnormal{für} ~x=-1,5$$
\vspace{35 pt}

{\bf Führe die Extremwertbestimmung mittels} \\
quadratischer Ergänzung nach dieser\\
Methode für folgende Terme durch:
\begin{enumerate}[a)]
\item $T(x)= x^2 -8x + 10 $
\item $T(x)= x^2 + 6x +15 $
\item $T(x)= x^2 -3x $
\item $T(x)= x(x-4) $
\item $T(x)= -10x +x^2 +30 $
\item $T(x)= (x+3)(x-1)$
\end{enumerate}
\end{minipage}
\hfill
\begin{minipage}{10 cm}
{\bf \item Ausgangsform $ T(x)= ax^2 + bx + c$:}\\
$
\renewcommand{\arraystretch}{1,5}
\begin{array}{rcl}
T(x)&=& -2x^2 + 12x - 7 \\
      &=& -2(x^2 - 6x)-7 \\
      &=& -2(x^2- 2\cdot 3x) -7\\
      &=& -2(x^2 -  2\cdot 3x + 3^2 - 3^2) -7 \\
      &=& -2(x^2 - 2 \cdot 3x + 3^2) +2 \cdot 3^2 -7 \\
      &=& -2(x^2 - 2 \cdot 3x + 3^2) + 11 \\
      &=& -2(x - 3)^2 + 11
\end{array}
$
$$\Rightarrow T_{max}=11 \quad \textnormal{für} ~x=3$$

{\bf Führe die Extremwertbestimmung mittels} \\
quadratischer Ergänzung nach dieser\\
Methode für folgende Terme durch:
\begin{enumerate}[a)]
\item $T(x)= -x^2 -4x + 1 $
\item $T(x)= 3x^2 + 18x + 42 $
\item $T(x)=0,5x^2 -3x $
\item $T(x)= -4x(3+x)+9 $
\item $T(x)= {3 \over 2} x -{1 \over 4}x^2 -{9 \over 4} $
\item $T(x)= (2x+4)(x-2)$
\end{enumerate}
\end{minipage}

\end{enumerate} 

\fbox{
	\begin{minipage}{0.5\textwidth}
	Zur Lösung bitte \href{https://www.okuyakl.de/math/m8querL001/ll001.pdf}{hier klicken} oder den QR-Code scannen.\\
	Weitere Arbeitsblätter gibt es unter 
	
	\href{https://www.okuyakl.de}{www.okuyakl.de}
	\end{minipage}
	\hfill
	\begin{minipage}{0.4\textwidth}
		\includegraphics[width=1.5 cm]{../../viecher/zwe03}
		\includegraphics[width=3 cm]{qr001}
		\includegraphics[width=2 cm]{../../viecher/afanticon1}
		
	\end{minipage}}


\end{document}%L Lösung ------------------------------------------
