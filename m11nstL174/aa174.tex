
\documentclass[a4paper]{article}
\usepackage[pdftex]{graphicx}
\usepackage[utf8]{inputenc}
\usepackage{enumerate}
\usepackage{amssymb}
\usepackage{polynom}
\usepackage{href-ul}
\hypersetup{
	colorlinks=true,
	linkcolor=blue,
	urlcolor=blue}
\usepackage{geometry}
\geometry{a4paper, top=10mm, left=15mm, right=15mm, bottom=15mm,
headsep=0mm, footskip=12mm}

%Source: img_0174

% Start the document
\begin{document}
%Titel
{\bf Trainingsblatt Nullstellen }\\

{ \bf Die folgenden Methoden werden für dieses Trainingsblatt benötigt:}\

\renewcommand{\arraystretch}{2}
\begin{tabular}{lll}
(BIN) & Ausnutzen einer binomischen Formel & $x^2-36 =(x-6)(x+6)$\\
(LÖS) & Mitternachtsformel &
 $ax^2+bx+c=0 \leadsto x_{1/2}=\frac{-b \pm \sqrt{b^2-4ac}}{2a}$\\
(VIE) & Satz von Vieta & $x^2+3x+2=(x+a)(x+b)$\\
& & $a+b =3;~a\cdot b=2$\\
(RAT) & Raten, falls Nullstellen ganzzahlig & $x^3 +2x^2 -x-2=0; ~ x_1=1$ \\
(SUB) & Substitution, um Grad zu verkleinern & $x^4-4x^2+4=0;~u=x^2 \leadsto u^2-4u+4=0$\\
(AUS) & Ausklammern, falls möglich & $ 2x^3 +x^2 = x^2(2x-1)$\\
(POL) & Polynomdivision durch bekannte LSG & $ (x^3 +2x^2 -x-2):(x-1)=\dots$
\end{tabular}

\begin{enumerate}[1.]
%Aufgabe 1
{\bf \item Zerlege in Linearfaktoren. Nenne die verwendete Methode, soweit sie nicht angegeben ist.}

\begin{enumerate}[a)]
\item $$f(x)=x^2+18x+77 \stackrel{(\textnormal{VIE})}{=} \rule{90 pt}{0.5 pt}$$
\item $$ f(x)= 2x^2-14x \stackrel{(\rule{20 pt}{0.5 pt})}{=} \rule{90 pt}{0.5 pt}$$
\item $$ f(x)= 2x^2-2x-{3\over 2}  \stackrel{(\textnormal{LÖS})}{\Rightarrow} x_1=\rule{30 pt}{0.5 pt};x_2= \rule{30 pt}{0.5 pt} \Rightarrow f(x)=\qquad \underline{(\qquad)}\cdot \underline{(\qquad)}$$
\renewcommand{\baselinestretch}{2}\normalsize
\item $f(x)=2x^3-9x^2+7x+6; \quad \textnormal{bekannte Nst: 3} \\
(\rule{20 pt}{0.5 pt}):\quad  
(2x^3-9x^2+7x+6):(x-3)=  \rule{90 pt}{0.5 pt} \stackrel{(\textnormal{LÖS})}{\Rightarrow} x_2=\rule{50 pt}{0.5 pt};\quad x_3=\rule{50 pt}{0.5 pt}\\
\Rightarrow f(x)=\qquad (\rule{30 pt}{0.5 pt})\cdot (\rule{30 pt}{0.5 pt}) \cdot (\rule{30 pt}{0.5 pt})$
\end{enumerate}

{\bf \item Häufig müssen mehrere Methoden angewendet werden, um den Funktionsterm vollständig zu zerlegen. Nenne die verwendeten Methoden, soweit sie nicht angegeben sind.}

\begin{enumerate}[a)]
\renewcommand{\baselinestretch}{2}\normalsize
\item $f(x) = x^4 -13x^2 +36 \quad (SUB): $
\item $f(x)= 2x^4 + x^3 -10x^2  $
\item $f(x)=x^3+x^2-102x+360 $ Alle Nullstellen sind ganzzahlig, eine davon lautet $-12$.
\item $f(x)= 5x^6-30x^4-275x^2$
\item Bekannte Nullstellen sind -2 und 2. $f(x)=x^5+6x^4+5x^3-24x^2-36x$
\end{enumerate}
\end{enumerate} 

\fbox{
	\begin{minipage}{0.5\textwidth}
		Zur Lösung bitte \href{https://www.okuyakl.de/math/m11nstL174/ll174.pdf}{hier klicken} oder den QR-Code scannen.\\
	Weitere Arbeitsblätter gibt es unter 
	
	\href{https://www.okuyakl.de}{www.okuyakl.de}
	\end{minipage}
	\hfill
	\begin{minipage}{0.4\textwidth}
		\includegraphics[width=1.5 cm]{../../viecher/zwe03}
		\includegraphics[width=3 cm]{qr174}
		\includegraphics[width=2 cm]{../../viecher/afanticon1}
		
	\end{minipage}}
	
\end{document}%L Lösung ----------------------------------------------------------
