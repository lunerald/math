\documentclass[a4paper]{article}
\usepackage[pdftex]{graphicx}
\usepackage[utf8]{inputenc}
\usepackage{enumerate}
\usepackage{icomma}
\usepackage{amssymb}
\usepackage{geometry}
\geometry{a4paper, top=15mm, left=15mm, right=15mm, bottom=15mm,
	headsep=10mm, footskip=12mm}
\usepackage{href-ul}
\hypersetup{
	colorlinks=true,
	linkcolor=blue,
	urlcolor=blue}
\begin{document}
{\bf Funktionale Abbildungen}
\begin{enumerate}[1.]
{\bf \item Untersuchen Sie, welche Abbildung durch folgende Gleichung beschrieben wird:}
 $$ \left(
\begin{array}{c}
x'\\
y'
\end{array}
\right) = 
\left(
\begin{array}{cc}
0,60 & -0,80 \\
0,80 & 0,60 
\end{array}
\right)
\odot 
 \left(
\begin{array}{c}
x\\
y
\end{array}
\right)$$
Geben Sie die Matrix allgemein sowie gegebenenfalls den Verschiebevektor, die Gleichung der Spiegelachse, das Drehzentrum und den Drehwinkel an!

%Aufgabe 2
{\bf \item Die (unvollständige) Matrix }
$ M =  \left(
\begin{array}{cc}
-0,71 & z \\
0,71 & y 
\end{array}
\right)$ 
soll Spiegelmatrix sein, die einer Achsenspiegelung an einer Ursprungsgeraden $g$ zugeordnet wird. Bestimmen Sie $z$ und $y$ sowie durch Rechnung die Gleichung der Spiegelachse $g$.

%Aufgabe 3
{\bf \item Die Gerade $h: y=2x+2$ wird durch Achsenspiegelung an der Geraden 
$s: y=-\frac{2}{3}x$ auf die Gerade $h'$ abgebildet.} Bestimmen Sie mit Hilfe des Parameterverfahrens die Gleichung der Geraden  $h'$ .

{\bf \item Die Parabel $ p $ mit der Gleichung $ y = x^2 - 2x +4 $ wird durch Spiegelung an der y-Achse auf $ p' $ abgebildet.}

\begin{enumerate}[a)]
\item Führen Sie die Abbildung zeichnerisch durch.
\item Ermitteln Sie die Gleichung der Bildparabel $ p' $ durch Rechnung. 
\item Bestimmen Sie zeichnerisch und rechnerisch die Umkehrfunktion $ f $ zu $ p' $.
\end{enumerate}

\end{enumerate} 

\fbox{
	\begin{minipage}{0.5\textwidth}
		Zur Lösung bitte \href{https://www.okuyakl.de/math/m10rsabbL028/ll028.pdf}{hier klicken} oder den QR-Code scannen.\\
	Weitere Arbeitsblätter gibt es unter 
	
	\href{https://www.okuyakl.de}{www.okuyakl.de}
	\end{minipage}
	\hfill
	\begin{minipage}{0.4\textwidth}
		\includegraphics[width=1.5 cm]{../../viecher/zwe03}
		\includegraphics[width=3 cm]{qr028}
		\includegraphics[width=2 cm]{../../viecher/afanticon1}
		
	\end{minipage}}

\end{document}%Lösung-------------------------------------------
