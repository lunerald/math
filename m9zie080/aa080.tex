\documentclass[a4paper]{article}
\usepackage[pdftex]{graphicx}
\usepackage[utf8]{inputenc}
\usepackage{enumerate}
\usepackage{amssymb}
\usepackage{colortbl}
\usepackage{icomma}
\usepackage{siunitx}
\sisetup{locale=DE} 
\usepackage{geometry}
\geometry{a4paper, top=15mm, left=15mm, right=15mm, bottom=15mm,
	headsep=10mm, footskip=12mm}
\usepackage{href-ul}
\hypersetup{
	colorlinks=true,
	linkcolor=blue,
	urlcolor=blue}
\begin{document}
\begin{minipage}{0.5\textwidth}
{\bf Das Ziegenproblem}
\vspace{0.5 cm}

Bei einem Gewinnspiel gibt es einen Sportwagen zu gewinnen. Hierzu muss ein Kandidat die richtige von drei geschlossenen Türen auswählen. Hinter zwei Türen befindet sich jeweils eine Ziege und hinter einer Türe der Sportwagen. 
\end{minipage}
\hfill
\begin{minipage}{0.4\textwidth}
\includegraphics[width=7 cm]{ziege080}
\end{minipage}
\vspace{1.5 cm}

\noindent Das Spiel wird folgendermaßen durchgeführt: Der Kandidat zeigt auf eine Tür. Der Showmaster weiß, hinter welcher Türe sich das Auto befindet und öffnet daraufhin  
eine Tür mit einer Ziege dahinter. Der Kandidat bekommt die Möglichkeit, seine anfängliche Wahl zu ändern und die nun übrige Türe auszuwählen.
\vspace{1.5 cm}


\noindent  Zu diskutieren ist hierbei, ob der Kandidat von dieser Möglichkeit Gebrauch machen sollte. Berechne jeweils die Gewinnwahrscheinlichkeit, indem Du alle möglichen Fälle durchspielst. 
\vspace{1.5 cm}


\noindent Welche interessante und unerwartete Schlussfolgerung ergibt sich?
\vspace{1.5 cm}

\fbox{
	\begin{minipage}{0.5\textwidth}
		Weitere Arbeitsblätter gibt es unter 
		
		\href{https://www.okuyakl.de}{www.okuyakl.de}
	\end{minipage}
	\hfill
	\begin{minipage}{0.4\textwidth}
		\includegraphics[width=1.5 cm]{../../viecher/zwe03}
		\includegraphics[width=2 cm]{../../viecher/afanticon1}
		
\end{minipage}}

\end{document}

