\documentclass[a4paper]{article}
\usepackage[pdftex]{graphicx}
\usepackage[utf8]{inputenc}
\usepackage{enumerate}
\usepackage{amssymb}
\usepackage{colortbl}
\usepackage{icomma}
\usepackage{siunitx}
\sisetup{locale=DE}
\usepackage{geometry}
\geometry{a4paper, top=15mm, left=15mm, right=15mm, bottom=15mm,
	headsep=10mm, footskip=12mm}
\usepackage{href-ul}
\hypersetup{
	colorlinks=true,
	linkcolor=blue,
	urlcolor=blue}
\begin{document}
	\begin{minipage}{0.5\textwidth}
		{\bf The goat problem}
		\vspace{0.5 cm}
		
		There is a competition to win a sports car. To do this, a candidate must choose the correct one of three closed doors. There is a goat behind two doors and a sports car behind one door.
	\end{minipage}
	\hfill
	\begin{minipage}{0.4\textwidth}
		\includegraphics[width=7 cm]{ziege080}
	\end{minipage}
	\vspace{1.5 cm}
	
	\noindent The game is played as follows: The candidate points to a door. The showmaster knows which door the car is behind and then opens it
	a door with a goat behind it. The candidate is given the opportunity to change their initial choice and select the remaining door.
	\vspace{1.5 cm}
	
	
	\noindent What needs to be discussed here is whether the candidate should make use of this option. Calculate the probability of winning by playing through all possible cases.
	\vspace{1.5 cm}
	
	
	\noindent What interesting and unexpected conclusion emerges?
	\vspace{1.5 cm}
	
	\fbox{
		\begin{minipage}{0.5\textwidth}
			Additional worksheets are available at
			
			\href{http://www.okuyakl.com}{www.okuyakl.com}
		\end{minipage}
		\hfill
		\begin{minipage}{0.4\textwidth}
			\includegraphics[width=1.5 cm]{../../viecher/zwe03}
			\includegraphics[width=2 cm]{../../viecher/afanticon1}
			
	\end{minipage}}
	
\end{document}