\documentclass[a4paper]{article}
\usepackage[pdftex]{graphicx}
\usepackage[utf8]{inputenc}
\usepackage{enumerate}
\usepackage{icomma}
\usepackage{amssymb}
\usepackage{geometry}
\geometry{a4paper, top=15mm, left=15mm, right=15mm, bottom=15mm,
	headsep=10mm, footskip=12mm}
\usepackage{href-ul}
\hypersetup{
	colorlinks=true,
	linkcolor=blue,
	urlcolor=blue}
\begin{document}
	{\bf Wurzeln für Fortgeschrittene}-- nur Quadratwurzeln -- Bitte ohne Taschenrechner. 
	\begin{enumerate}[1.]
		{\bf \item Mache den Nenner rational und vereinfache} Es gilt $\mathbb{G}=\mathbb{R}^+$
		
		\renewcommand{\arraystretch}{3}
		\begin{tabular}{p{5 cm}p{5 cm}p{5 cm}}
			a) ${1 \over \sqrt{2}}$	& b) ${3 \over \sqrt{3}}$ & c) $ {v \over \sqrt{v}}$ \\
			d) ${5 \over \sqrt{10}}$ & e) ${\sqrt{48} \over \sqrt{3}\cdot \sqrt{14}}$ & f)$\frac{1}{\sqrt{8}-\sqrt{2}}$ \\
			 g) ${6 \over \sqrt{27}+ \sqrt{3}}$  & h) ${ \sqrt{a} \over b\sqrt{a}+c\sqrt{a}}$ & i) ${\sqrt{x^3}\over \sqrt{3x}}$ \\ 
			 j) ${a+b \over \sqrt{a+b}}$ & k) ${36 \over \sqrt{324}}$ & l) ${3 \over \sqrt{n}+\sqrt{n}+ \sqrt{n}} $\\
			\hline
		\end{tabular}
		
		{\bf \item Mache den Nenner mit der 3. binomischen Formel rational}
		
		\renewcommand{\arraystretch}{3}
		\begin{tabular}{p{5 cm}p{5 cm}p{5 cm}}
			a) ${1 \over \sqrt{3} -\sqrt{2} }$ & b) ${1 \over \sqrt{7} + \sqrt{3} }$ & c) ${ 1\over 2\sqrt{2} - \sqrt{6} }$ \\
			d) ${ 2 \over 3\sqrt{10} + 2 }$ & e) ${\sqrt{2} \over 6 - \sqrt{2} }$ & f) ${\sqrt{3}\over \sqrt{3} + 1}$ \\
			\hline
		\end{tabular}
		
		
		{\bf \item Bestimme die maximale Definitionsmenge.} Es gilt $\mathbb{G}=\mathbb{R}$
		
		\renewcommand{\arraystretch}{3}
		\begin{tabular}{p{5 cm}p{5 cm}p{5 cm}}
			a)$\sqrt{x}$ & b)${\sqrt{x-5}}$ & c)${\sqrt{3x+6}}$ \\
			d)${\sqrt{-x}}$ & e)${\sqrt{7-x}}$ & f) ${\sqrt{x^2}}$  \\
			g)${1 \over \sqrt{x}}$& h)${1 \over \sqrt{x^2}}$& i)${1 \over \sqrt{x^2 +1}}$ \\
			j) ${\sqrt{x^3}}$ & k) ${\sqrt{9x}}$ & l)${\sqrt{x-1}\over \sqrt{x-2}}$\\
			\hline
		\end{tabular}
		
	
			{\bf \item Bestimme zunächst die Definitionsmenge und dann die Lösungsmenge der Wurzelgleichungen.} \\ Es gilt $\mathbb{G}=\mathbb{R}$
			
			\renewcommand{\arraystretch}{3}
			\begin{tabular}{p{5 cm}p{5 cm}p{5 cm}}
				a)$\sqrt{x+1}=0 $& b)$\sqrt{2+x} =2$ & c)${\sqrt{2x} = \sqrt{x+3}}$ \\
				d)$\sqrt{x-4} =-1$ & e) ${\sqrt{2x} = \sqrt{x-4}}$ & f)  ${\sqrt{2x^2} = \sqrt{x^2-4}}$\\
				\hline
			\end{tabular}
			
			
			{\bf \item Ermittle die Gleichung der Geraden g.} Die dargestellten Vierecke sind Quadrate.\\ Ab hier ist der Taschenrechner wieder erlaubt
		
			\includegraphics[width=5 cm]{qger042}		
	\end{enumerate}
\fbox{
	\begin{minipage}{0.5\textwidth}
		Zur Lösung bitte \href{https://www.okuyakl.de/math/m9wuiL115/ll115.pdf}{hier klicken} oder den QR-Code scannen.\\
		Weitere Arbeitsblätter gibt es unter 
		
		\href{https://www.okuyakl.de}{www.okuyakl.de}
	\end{minipage}
	\hfill
	\begin{minipage}{0.4\textwidth}
		\includegraphics[width=1.5 cm]{../../viecher/zwe03}
		\includegraphics[width=3 cm]{qr115}
		\includegraphics[width=2 cm]{../../viecher/afanticon1}
		
	\end{minipage}}

\end{document}