
\documentclass[a4paper]{article}
\usepackage[pdftex]{graphicx}
\usepackage[utf8]{inputenc}
\usepackage{enumerate}
\usepackage{icomma}
\usepackage{amssymb}
\usepackage{geometry}
\geometry{a4paper, top=15mm, left=15mm, right=15mm, bottom=15mm,
	headsep=10mm, footskip=12mm}
\usepackage{href-ul}
\hypersetup{
	colorlinks=true,
	linkcolor=blue,
	urlcolor=blue}
\begin{document}
{\bf Goniometrische Gleichungen }\\

\begin{enumerate}[1.]
{\bf \item Bestimmen Sie rechnerisch die Lösungsmenge der folgenden goniometrischen Gleichungen unter Beachtung der Grundmenge 
	$ \mathbb{G} = [ 0^\circ ; 360^\circ]$}

\begin{enumerate}[a)]
\item $ 6 \cdot \sin{(\alpha + 60^\circ)} = 3 $
\item $ 2 \cdot \cos{(\alpha + 45^\circ)} = 2 $
\item $ \sin^2{\alpha}  = 0,75 $
\item $ \sin{\alpha} + \cos{\alpha} = 2 $
\item $ ( 1 - \cos^2{\alpha} ) = 0 $
\item $ ( \sin{\alpha} + 0,6 ) \cdot  ( \sin{\alpha} - 0,6 ) -0,28 = 0 $
\item $ 2 \cdot \sin{\alpha} \cdot \cos{\alpha} = 1 $
\item $  \cos{(\alpha)} =  \sin{\alpha} $
\item $ 4\sin^2{\alpha} + 4 \cdot \sin{\alpha} +1= 0 $
\item $ 2 \sin{\frac{\alpha}{2}} = 0 $
\item $ \frac{\sqrt{2}}{4 \sin{(\gamma)}} = \frac{\sqrt{2}}{3}$
\item $  \frac{\sin{\epsilon}}{ \sin{(\epsilon + 30^\circ )}} = 4 \sqrt{3} $
\end{enumerate}


\end{enumerate}

\fbox{
	\begin{minipage}{0.5\textwidth}
		Zur Lösung bitte \href{https://www.okuyakl.de/math/m10gonL082/ll082.pdf}{hier klicken} oder den QR-Code scannen.\\
	Weitere Arbeitsblätter gibt es unter 
	
	\href{https://www.okuyakl.de}{www.okuyakl.de}
	\end{minipage}
	\hfill
	\begin{minipage}{0.4\textwidth}
		\includegraphics[width=1.5 cm]{../../viecher/zwe03}
		\includegraphics[width=3 cm]{qr082}
		\includegraphics[width=2 cm]{../../viecher/afanticon1}
		
	\end{minipage}}

\end{document}%L L Ö S U N G ---------------------------------------------------------------
