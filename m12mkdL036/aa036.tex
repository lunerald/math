\documentclass[a4paper]{article}
\usepackage[pdftex]{graphicx}
\usepackage[utf8]{inputenc}
\usepackage{enumerate}
\usepackage{amssymb}
\usepackage{icomma}
\usepackage{href-ul}
\hypersetup{
	colorlinks=true,
	linkcolor=blue,
	urlcolor=blue}
\usepackage{siunitx}
\sisetup{locale=DE} 
\usepackage{geometry}
\geometry{a4paper, top=15mm, left=15mm, right=15mm, bottom=15mm,
	headsep=10mm, footskip=12mm}

\begin{document}
{\bf Muster-Kurvendiskussion}

\begin{enumerate}[1.]
{\bf \item Gegeben ist die Funktion $$f(x)=-x^3+3x^2$$
Führen Sie eine vollständige Kurvendiskussion durch. }
\begin{enumerate}[a)]
	\item Ermitteln Sie die Definitionsmenge, das Verhalten im Unendlichen und die Wertemenge.
	\item Untersuchen Sie die Funktion auf Symmetrie.
	\item Berechnen Sie alle Nullstellen und ihre Vielfachheiten.
	\item Bilden Sie die erste, die zweite und die dritte Ableitung.
	\item Untersuchen Sie das Monotonieverhalten und die Art und Lage der Extrempunkte mittels einer Monotonie\-tabelle.
	\item Bestimmen Sie das Krümmungsverhalten und den Wendepunkt.
	\item Geben Sie den Term der Wendetangente an.
	\item Zeichnen Sie den Graphen $G_f$ mit der Wendetangente anhand der gewonnenen Erkenntnisse in ein geeignetes Koordinatensystem.
	\item Der Graph $G_f$ und die $x$-Achse schließen ein Flächenstück ein. Berechnen Sie dessen Inhalt. 
\end{enumerate} 

\end{enumerate} 
\fbox{
	\begin{minipage}{0.5\textwidth}
		Zur Lösung bitte \href{https://www.okuyakl.de/math/m12mkdL036/ll036.pdf}{hier klicken} oder den QR-Code scannen.\\
	Weitere Arbeitsblätter gibt es unter 
	
	\href{https://www.okuyakl.de}{www.okuyakl.de}
	\end{minipage}
	\hfill
	\begin{minipage}{0.4\textwidth}
		\includegraphics[width=1.5 cm]{../../viecher/zwe03}
		\includegraphics[width=3 cm]{qr036}
		\includegraphics[width=2 cm]{../../viecher/afanticon1}
		
\end{minipage}}
\end{document}

