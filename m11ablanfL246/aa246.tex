\documentclass{article}
\usepackage[utf8]{inputenc}
\usepackage{graphicx}
\usepackage{enumerate}
\usepackage{icomma}
\usepackage{amssymb}
\usepackage{amsmath}
\usepackage{geometry}
\usepackage{href-ul}
\hypersetup{
	colorlinks=true,
	linkcolor=blue,
	urlcolor=blue}
\geometry{a4paper, top=15mm, left=15mm, right=15mm, bottom=15mm,
headsep=10mm, footskip=12mm}

\title{m11ablanfL246}

\begin{document}
\begin{enumerate}
\item {\bf Differenzierbarkeit}\\
{\bf  Gegeben ist die Funktion
 $$f(x)={1 \over |x-1|+1}$$}
\begin{enumerate}
\item Geben Sie an, bei welcher Stelle die Funktion $f$ eventuell nicht differenzierbar ist.
\item Zeigen Sie, dass die Funktion $f$ an dieser Stelle nicht differenzierbar ist.
\item Geben Sie eine Funktion an, die an der Stelle $x=-3$ nicht differenzierbar ist. 
\end{enumerate}
\item {\bf Aufgabe} \\
\includegraphics[width=10 cm]{nidiff246}
\begin{enumerate}
\item Geben Sie alle x-Werte an, an denen der Graph die Steigung 0 hat.
\item Geben Sie alle x-Werte an, an denen die Funktion nicht differenzierbar ist. 
\end{enumerate}
\item {\bf Ableitungsfunktion}\\
Die Ableitungsfunktion $f'$ einer Funktion $f$ ist diejenige Funktion, die jeder Stelle x, an der $f$ differenzierbar ist, die Ableitung $f'$ zuordnet:
$$f'(x)=\lim\limits_{h\to 0} \frac{f(x+h)-f(x)}{h}$$
\begin{enumerate}
\begin{minipage}{0.5\textwidth}
\item Gegeben ist die reinquadratische Funktion $f(x)=0,25x^2$. Zeichnen Sie die Tangenten $t_{-3}$, $t_{-1}$, $t_0$, $t_1$ und $t_3$ an den Graphen von $f$.
\item Bestimmen Sie rechnerisch die Ableitungsfunktion $f'(x)$
\item Berechnen Sie $f'(-3),\quad f'(-1), \quad f'(0), \quad f'(1), \quad f'(3)$ 
\end{minipage}
\begin{minipage}{0.5\textwidth}
\includegraphics[width=6 cm]{reiq246}
\end{minipage}

\begin{minipage}{0.5\textwidth}
\item Tragen Sie die Punkte\\
 $A(-3|f'(-3)), \quad B(-1|f'(-1)), \quad C(0|f'(0))$\\
 $D(1|f'(1)), \quad E(3|f'(3))$ \\
 in das Koordinatensystem.
 \item Tragen Sie den Graphen der Ableitungsfunktion $f'$ in das Koordinatensystem ein. 
\end{minipage}
\begin{minipage}{0.5\textwidth}
\includegraphics[width=6 cm]{leer246}
\end{minipage}
\end{enumerate}

\begin{minipage}{0.5\textwidth}
{\bf \item Die Ableitungsfunktion der Betragsfunktion} $$f(x)=|x|$$
Erinnerung: $f(x)=|x|$ ist in $x_0=0$ nicht differenzierbar.
\begin{enumerate}
\item Zeichnen Sie den Graphen der Ableitungsfunktion $f'$ in das Koordinatensystem
\item Geben Sie die Ableitungsfunktion $f'$ für $x \in \mathbb{R}^-$ bzw. $x \in \mathbb{R}^+$ an.
\end{enumerate}
\end{minipage}
\begin{minipage}{0.5\textwidth}
\includegraphics[width=6 cm]{abs246}
\end{minipage}

\end{enumerate}

\fbox{
	\begin{minipage}{0.5\textwidth}
		Zur Lösung bitte \href{https://www.okuyakl.de/math/m11ablanfL246/ll246.pdf}{hier klicken} oder den QR-Code scannen.\\
	Weitere Arbeitsblätter gibt es unter 
	
	\href{https://www.okuyakl.de}{www.okuyakl.de}
	\end{minipage}
	\hfill
	\begin{minipage}{0.4\textwidth}
		\includegraphics[width=1.5 cm]{../../viecher/zwe03}
		\includegraphics[width=3 cm]{qr246}
		\includegraphics[width=2 cm]{../../viecher/afanticon1}
		
	\end{minipage}}

\end{document}%L Lösung ---------------------------------------------------
