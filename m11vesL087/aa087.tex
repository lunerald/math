\documentclass[a4paper]{article}
\usepackage[pdftex]{graphicx}
\usepackage[utf8]{inputenc}
\usepackage{enumerate}
\usepackage{amssymb}
\usepackage{href-ul}
\hypersetup{
	colorlinks=true,
	linkcolor=blue,
	urlcolor=blue}
\usepackage{geometry}
\geometry{a4paper, top=15mm, left=15mm, right=15mm, bottom=15mm,
	headsep=10mm, footskip=12mm}

\begin{document}
{\bf Vektorprodukt, Spatprodukt}
\begin{enumerate}[1.]
{\bf \item Bilden Sie das Vektorprodukt}

\renewcommand{\arraystretch}{1}
\begin{tabular}{lll}
	a) $\left(\begin{array}{c} 5 \\-1 \\ 0 \end{array} \right)\times \left(\begin{array}{c} 3 \\ 4\\ 10\end{array} \right) $ & b) $\left(\begin{array}{c} -2 \\ 7\\ 4 \end{array} \right)\times \left(\begin{array}{c} 4 \\-14 \\ -8 \end{array} \right) $ & c) $\left(\begin{array}{c} 6 \\ -3 \\ 0,5\end{array} \right) \times \left(\begin{array}{c} 0 \\ 1\\ -4\end{array} \right)$
\end{tabular}

{\bf \item Berechnen Sie folgende Flächen}

\begin{enumerate}[a)]
	\item Parallelogramm $PQRS$ mit $\overrightarrow{PQ}=\left(\begin{array}{c} 3 \\-4 \\ 8\end{array} \right)$ und $\overrightarrow{PS}=\left(\begin{array}{c} 0 \\5 \\ 2\end{array} \right)$
	\item Dreieck $ABC$ mit  $A(0|1|0)$; $B(6|-7|0)$; $C(2|3|-2)$
\end{enumerate}

 
{\bf \item Bilden Sie das Spatprodukt aus den Vektoren}

\renewcommand{\arraystretch}{1}
\begin{tabular}{ll}
	a)$<\left(\begin{array}{c} 1 \\0 \\-2 \end{array} \right);\left(\begin{array}{c} 4 \\ 5 \\ 3\end{array} \right); \left(\begin{array}{c} 2 \\ 7\\ 0\end{array} \right)>$ & b) $<\left(\begin{array}{c} -3 \\ 1\\ 2\end{array}\right) ;\left(\begin{array}{c} 12 \\ 1\\ 0\end{array} \right);\left(\begin{array}{c} 4 \\ 1\\ -6\end{array} \right)> $ 
\end{tabular}

{\bf \item Berechnen Sie das Volumen...}
\begin{enumerate}[a)]
	\item ...der dreiseitigen Pyramide ABCS mit $A(6|0|2);\quad B(1|1|1);\quad C(0|7|2);\quad S(3|3|8)$
	\item ...der vierseitigen Pyramide ABCDS mit $\overrightarrow{AB}=\overrightarrow{DC}=\left(\begin{array}{c} 4 \\ 1\\ 4\end{array} \right); \quad \overrightarrow{BC}=\overrightarrow{AD}=\left(\begin{array}{c} 0 \\ -3\\ -7\end{array} \right) $ und $\overrightarrow{AS}=\left(\begin{array}{c} -2 \\ -2\\ 0\end{array} \right) $
\end{enumerate}

{\bf \item Für welches a wird das Volumen des Spates gleich null?}
	
$$<\left(\begin{array}{c} 2 \\-1 \\ 0 \end{array} \right); \left(\begin{array}{c} 3 \\ 4\\ 6\end{array} \right);\left(\begin{array}{c} a \\ 1\\ a\end{array} \right)>$$

Welche Lage haben dann die drei Vektoren?

\end{enumerate} 
\fbox{
	\begin{minipage}{0.5\textwidth}
		Zur Lösung bitte \href{https://www.okuyakl.de/math/m11vesL087/ll087.pdf}{hier klicken} oder den QR-Code scannen.\\
	Weitere Arbeitsblätter gibt es unter 
	
	\href{https://www.okuyakl.de}{www.okuyakl.de}
	\end{minipage}
	\hfill
	\begin{minipage}{0.4\textwidth}
		\includegraphics[width=1.5 cm]{../../viecher/zwe03}
		\includegraphics[width=3 cm]{qr087}
		\includegraphics[width=2 cm]{../../viecher/afanticon1}
		
\end{minipage}}
\end{document}