\documentclass[a4paper]{article}
\usepackage[pdftex]{graphicx}
\usepackage[utf8]{inputenc}
\usepackage{enumerate}
\usepackage{icomma}
\usepackage{siunitx}
\sisetup{locale=DE} 
\usepackage{amssymb}
\usepackage{geometry}
\geometry{a4paper, top=15mm, left=15mm, right=15mm, bottom=15mm,
	headsep=10mm, footskip=12mm}
\usepackage{href-ul}
\hypersetup{
	colorlinks=true,
	linkcolor=blue,
	urlcolor=blue}
\begin{document}
{\bf Exponentielles Wachstum und Zerfall}
\begin{enumerate}

{\bf \item Bei einer Kettenmail muss jeder Empfänger die Mail an 5 neue Personen weiterleiten}. Gib den mathematischen Funktionsterm hierfür an, wobei $y$ die Anzahl der Personen und $x$ die Anzahl der Durchgänge sein sollen.

 %Aufgabe 2
{\bf \item In einem See nimmt die Helligkeit pro Meter Wassertiefe um 20\% ab.} An der Oberfläche beträgt die Helligkeit 100 Einheiten.\\
Stelle die Funktion der Helligkeit $y$ in Abhängigkeit von der Wassertiefe $x$ in Metern dar.

%Aufgabe 3
{\bf \item Die Ausbreitung einer Infektionskrankheit lässt sich in den ersten Tagen mit dem Term $y=2,5^{0,5x}$ beschreiben,} wobei $y$ die Anzahl der Erkrankten und $x$ die vergangene Zeit in Tagen ist.

\begin{enumerate}[a)]
\item Erstelle eine Wertetabelle für $x \in [0;10], \quad \Delta x =1 $ und zeichne den Graphen in ein Koordinatensystem ein (Runde auf Ganze). Für die Zeich\-nung gilt: $ 0 \le x \le 10, \quad 1~d \hat{=}
\SI{1}{\centi\meter};\quad 0 \le y \le 100; \quad 10~\textnormal{Erkr.} \hat{=} \SI{1}{\centi\meter}$
\item Berechne den Wachstumsfaktor und um wie viel Prozent die Zahl der Erkrankten täglich zunimmt.
\item Entnehme der Zeichnung, nach welcher Zeit es 50 Erkrankte gibt.
\end{enumerate}

%Aufgabe 4
{\bf \item Eine Bakterienkultur wächst mit $y= 10 \cdot 3^{0,5x}$, wobei $y$ für die Anzahl der Bakterien und $x$ für die Zeit in Stunden ($h$) stehen.}

\begin{enumerate}[a)]
\item Zeichne den Graphen der Funktion in ein Koordinatensystem für\\
 $ 0 \le x \le 5$. Für die Zeichnung gilt: $\SI{1}{\hour} \hat{=}\SI{1}{\centi\meter};~ 20~\textnormal{Bakterien} \hat{=} \SI{1}{\centi\meter}$.
\item  Berechne den Wachstumsfaktor und um wie viel Prozent die Zahl der Bakterien stündlich zunimmt.
\item Lese aus der Zeichnung ab, wann die Kultur 100 Bakterien aufweist.
\end{enumerate} 

%Aufgabe 5
{\bf \item Wasserlinsen sind Pflanzen, die an der Wasseroberfläche von Teichen schwimmen und große Teile davon bedecken können.} Am 10. Juni um 12 Uhr mittags entdeckt Herr Grün eine $\SI{0,5}{\meter^2}$ große Ansammlung von Wasserlinsen auf seinem $\SI{20}{\meter^2}$ großen Gartenteich. Für die weitere Entwicklung ist anzunehmen, dass sich der mit Wasserlinsen bedeckte Flächeninhalt täglich um 35\% vergrößern wird. Dabei sind $x$ Tage nach der Entdeckung $y~\SI{}{\meter^2}$ Wasseroberfläche mit Wasserlinsen bedeckt.
 
\begin{enumerate}[a)]
\item Erstelle einen Funktionsterm für diese Entwicklung.
\item Tabellarisiere die Funktion für $x \in [0;8], \quad \Delta x = 1$ und zeichne den Graphen in ein Koordinatensystem ein (Runde auf zwei Stellen nach dem Komma). Für die Zeichnung gilt: \\
$ 0 \le x \le 10, \quad \SI{1}{\day}\hat{=}\SI{1}{\centi\meter};\quad 0 \le y \le 7\SI{}{\meter^2}; \quad \SI{1}{\meter^2} \hat{=} \SI{1}{\centi\meter}$
\item Nach einer Bestimmten Anzahl von Tagen seit der Entdeckung ist erstmals ein Fünftel der Wasseroberfläche des Gartenteichs mit Wasserlinsen bedeckt. Gib das zugehörige Datum mithilfe des Graphen an. 
\end{enumerate} 

%Aufgabe 6
{\bf \item Ein in der Medizin verwendetes radioaktives Präparat zerfällt mit \\
$y=5\cdot0,5^{0,5x}$,} wobei $y$ für die Masse des Präparates in $\SI{}{\milli\gram}$ und $x$ für die verstrichene Zeit in Tagen steht.
\begin{enumerate}[a)]
\item Zeichne den Graphen der Zerfallsgleichung in ein Koordinatensystem ein. Für die Zeichnung gilt: Längeneinheit $\SI{1}{\centi\meter}; \quad 0 \le x \le 8; \quad  0 \le y \le 6$.
\item  Berechne den Wachstumsfaktor und um wie viel Prozent die Menge pro Tag abnimmt.
\item Lese aus der Zeichnung ab, nach welcher Zeit noch $\SI{3}{\milli\gram}$ des verwendeten Präparates  im Körper vorhanden sind.
\end{enumerate} 

{\bf \item Eine Chemiefirma verpflichtet sich, ihren Schadstoffausstoß bis 2028 gegenüber dem Stand von 2016 um 14\% zu verringern.} Entscheide durch Rechnung, ob dieses Ziel durch eine jährliche prozentuale Abnahme von 1,25\% erreicht werden kann.
\end{enumerate}

\fbox{
	\begin{minipage}{0.5\textwidth}
		Zur Lösung bitte \href{https://www.okuyakl.de/math/m10expL029/ll029.pdf}{hier klicken} oder den QR-Code scannen.\\
	Weitere Arbeitsblätter gibt es unter 
	
	\href{https://www.okuyakl.de}{www.okuyakl.de}
	\end{minipage}
	\hfill
	\begin{minipage}{0.4\textwidth}
		\includegraphics[width=1.5 cm]{../../viecher/zwe03}
		\includegraphics[width=3 cm]{qr029}
		\includegraphics[width=2 cm]{../../viecher/afanticon1}
		
	\end{minipage}}

\end{document}%L L Ö S U N G -----------------------------------
