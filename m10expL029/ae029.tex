\documentclass[a4paper]{article}
\usepackage[pdftex]{graphicx}
\usepackage[utf8]{inputenc}
\usepackage{enumerate}
\usepackage{icomma}
\usepackage{siunitx}
\sisetup{locale=DE}
\usepackage{amssymb}
\usepackage{geometry}
\geometry{a4paper, top=15mm, left=15mm, right=15mm, bottom=15mm,
	headsep=10mm, footskip=12mm}
\usepackage{href-ul}
\hypersetup{
	colorlinks=true,
	linkcolor=blue,
	urlcolor=blue}
\begin{document}
	{\bf Exponential growth and decay}
	\begin{enumerate}
		
		{\bf \item In a chain email, each recipient has to forward the email to 5 new people}. Give the mathematical function term for this, where $y$ is the number of people and $x$ is the number of passes.
		
		%Exercise 2
		{\bf \item In a lake, the brightness decreases by 20\% per meter of water depth.} On the surface, the brightness is 100 units.\\
		Show the function of the brightness $y$ depending on the water depth $x$ in meters.
		
		%Task 3
		{\bf \item The spread of an infectious disease in the first few days can be described with the term $y=2.5^{0.5x}$,} where $y$ is the number of sick people and $x$ is the time that has passed in days.
		
		\begin{enumerate}[a)]
			\item Create a table of values for $x \in [0;10], \quad \Delta x =1 $ and draw the graph in a coordinate system (round to whole). The following applies to the drawing: $ 0 \le x \le 10, \quad 1~d \hat{=}
			\SI{1}{\centi\meter};\quad 0 \le y \le 100; \quad 10~\textnormal{Explanation} \hat{=} \SI{1}{\centi\meter}$
			\item Calculate the growth factor and by what percentage the number of sick people increases every day.
			\item Find out from the drawing after what time there will be 50 sick people.
		\end{enumerate}
		
		%Task 4
		{\bf \item A bacterial culture grows with $y= 10 \cdot 3^{0.5x}$, where $y$ is the number of bacteria and $x$ is the time in hours ($h$).}
		
		\begin{enumerate}[a)]
			\item Draw the graph of the function in a coordinate system for\\
			$ 0 \le x \le 5$. The following applies to the drawing: $\SI{1}{\hour} \hat{=}\SI{1}{\centi\meter};~ 20~\textnormal{bacteria} \hat{=} \SI{1} {\centi\meter}$.
			\item Calculate the growth factor and by what percentage the number of bacteria increases every hour.
			\item Read from the drawing when the culture has 100 bacteria.
		\end{enumerate}
		
		%Task 5
		{\bf \item Duckweed are plants that float on the surface of ponds and can cover large parts of them.} On June 10th at 12 noon, Mr. Green discovered a $\SI{0.5}{\meter^2} $ large collection of duckweed on his $\SI{20}{\meter^2}$ large garden pond. For further development it can be assumed that the area covered with duckweed will increase by 35\% every day. $x$ days after the discovery $y~\SI{}{\meter^2}$ water surface is covered with duckweed.
		
		\begin{enumerate}[a)]
			\item Create a function term for this expansion.
			\item Tabulate the function for $x \in [0;8], \quad \Delta x = 1$ and plot the graph in a coordinate system (round to two decimal places). The following applies to the drawing: \\
			$ 0 \le x \le 10, \quad \SI{1}{\day}\hat{=}\SI{1}{\centi\meter};\quad 0 \le y \le 7\SI{} {\meter^2}; \quad \SI{1}{\meter^2} \hat{=} \SI{1}{\centi \meter}$
			\item After a certain number of days since the discovery, a fifth of the water surface of the garden pond is covered with duckweed for the first time. Specify the corresponding date using the graph.
		\end{enumerate}
		
		%Task 6
		{\bf \item A radioactive preparation used in medicine decays with \\
			$y=5\cdot0.5^{0.5x}$,} where $y$ represents the mass of the specimen in $\SI{}{\milli\gram}$ and $x$ represents the elapsed time in days .
		\begin{enumerate}[a)]
			\item Draw the graph of the decay equation in a coordinate system. The following applies to the drawing: length unit $\SI{1}{\centi\meter}; \quad 0 \le x \le 8; \quad 0 \le y \le 6$.
			\item Calculate the growth factor and by what percentage the amount decreases per day.
			\item Read from the drawing how long $\SI{3}{\milli\gram}$ of the preparation used is still present in the body.
		\end{enumerate}
		
		{\bf \item A chemical company commits to reducing its pollutant emissions by 14\% by 2028 compared to 2016 levels.} Decide by calculation whether this goal can be achieved by an annual percentage decrease of 1.25\%.
	\end{enumerate}
	
	\fbox{
		\begin{minipage}{0.5\textwidth}
			For the solution, please \href{https://www.okuyakl.de/math/m10expL029/le029.pdf}{click here} or scan the QR code.\\
			Additional worksheets are available at
			
			\href{http://www.okuyakl.com}{www.okuyakl.com}
		\end{minipage}
		\hfill
		\begin{minipage}{0.4\textwidth}
			\includegraphics[width=1.5 cm]{../../viecher/zwe03}
			\includegraphics[width=3 cm]{qre029}
			\includegraphics[width=2 cm]{../../viecher/afanticon1}
			
	\end{minipage}}
	
\end{document}