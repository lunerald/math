\documentclass[a4paper]{article}
\usepackage[pdftex]{graphicx}
\usepackage[utf8]{inputenc}
\usepackage{enumerate}
\usepackage{icomma}
\usepackage{amssymb}
\usepackage{amsmath}
\usepackage{geometry}
\geometry{a4paper, top=15mm, left=15mm, right=15mm, bottom=15mm,
	headsep=10mm, footskip=12mm}
\usepackage{href-ul}
\hypersetup{
	colorlinks=true,
	linkcolor=blue,
	urlcolor=blue}

\begin{document}
	%Title
	{\bf School assignment from mathematics }\\
	\begin{enumerate}[1.]
		
		%Task 1
		
		
		\begin{minipage}{0.6\textwidth}
			{\bf \item Given is the power function $f(x)=x^{-5}$.} \\
			The following applies: $f \xrightarrow{\textnormal{x-axis; k=0.25}} f^{*}$
			\begin{enumerate}[a)]
				\item Calculate the equation of $f^*$
				\item Determine the definition and value set of $f^*$ as well as the symmetry properties and, if necessary, the asymptotes.
				\item The function $f^*$ is now with the vector
				$\vec{v} =\binom{3}{4} $ shifted in parallel. Calculate the equation of the resulting function $f^{**}$
			\end{enumerate}
		\end{minipage}
		\hfill
		\begin{minipage}{0.35\textwidth}
			\includegraphics[width=5 cm]{poff208}
		\end{minipage}
		
		%Exercise 2
		{\bf \item The graph of a power function $f_1(x)=x^4$ is mapped to $f_3$ as follows:}
		$$f_1(x)=x^4 \xrightarrow{\textnormal{x-axis; k=0.5}} f_2
		\xrightarrow{\vec{v} =\binom{4}{1}} f_3$$
		\begin{enumerate}[a)]
			\item Give the equations of the functions $f_2$ and $f_3$.
			\item Draw $f_1, f_2$ and $f_3$ in a coordinate system and specify the definition set and the value set of $f_3$.
		\end{enumerate}
		
		%Task 3
		{\bf \item Given is the function $f$ with the equation $ y=2(x-1)^{-2}+3$}
		\begin{enumerate}[a)]
			\item Specify the definition set and the value set and explain that the function has no zeros.
			\item Tabulate $f$ at an appropriate interval and draw the graph. For the drawing: length unit $1~cm;\quad -5 \le x \le 6;\quad -2 \le y \le 10$.
			\item The graph becomes $\vec{v} =\binom{-3}{-4} $
			pictured. Draw the image graph $h$ in the coordinate system for exercise 3. b).
			\item Calculate the equation for $h$ using the parametric method.
			$[\textnormal{ result}: h:~ y=2(x+2)^{-2}-1]$
			\item Give the equations of the asymptotes for $h$.
			\item Show mathematically that the hyperbola $h$ is axisymmetric with respect to the line with $x=-2$.\\
			{\it Solution instructions: Show that $f(-2+a) = f(-2-a)$.}
			\item Calculate the zeros of the hyperbola $h$ and check the values found using the diagram from exercise 3. c).
		\end{enumerate}
		
		%Task 4
		{\bf \item Calculate without calculator}
		
		a) $ (2+2,3)^0 $ \hspace{20 pt} b) $ 1+ \left({ 1 \over 3}\right)^{-2}$ \hspace{20 pt}
		c) $1 : 4^{-2}$ \hspace{20 pt}
		d) $ 2^{x+2} \cdot \left({ 1 \over 2}\right)^{x+2} $ \\
		e) $\left({ 2 \over a}\right)^3 \cdot (-a)^3$
	\end{enumerate}
	
	\fbox{
		\begin{minipage}{0.5\textwidth}
			For the solution, please \href{https://www.okuyakl.de/math/m10pofuL208/le208.pdf}{click here} or scan the QR code.\\
			Additional worksheets are available at
			
			\href{http://www.okuyakl.com}{www.okuyakl.com}
		\end{minipage}
		\hfill
		\begin{minipage}{0.4\textwidth}
			\includegraphics[width=1.5 cm]{../../viecher/zwe03}
			\includegraphics[width=3 cm]{qre208}
			\includegraphics[width=2 cm]{../../viecher/afanticon1}
			
	\end{minipage}}
	
\end{document}