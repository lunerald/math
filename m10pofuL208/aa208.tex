
\documentclass[a4paper]{article}
\usepackage[pdftex]{graphicx}
\usepackage[utf8]{inputenc}
\usepackage{enumerate}
\usepackage{icomma}
\usepackage{amssymb}
\usepackage{amsmath}
\usepackage{geometry}
\geometry{a4paper, top=15mm, left=15mm, right=15mm, bottom=15mm,
	headsep=10mm, footskip=12mm}
\usepackage{href-ul}
\hypersetup{
	colorlinks=true,
	linkcolor=blue,
	urlcolor=blue}
	
\begin{document}
%Titel
{\bf Schulaufgabe aus der Mathematik }\\
\begin{enumerate}[1.]

%Aufgabe 1


\begin{minipage}{0.6\textwidth}
{\bf \item Gegeben ist die Potenzfunktion $f(x)=x^{-5}$.} \\
Es gilt: $f \xrightarrow{\textnormal{x-Achse; k=0,25}}  f^{*}$
\begin{enumerate}[a)]
\item Ermittle rechnerisch die Gleichung von $f^*$
\item Bestimme von $f^*$ die Definitions- und Wertemenge sowie die Symmetrieeigenschaften und ggf. die Asymptoten.
\item Die Funktion $f^*$ wird nun mit dem Vektor 
$\vec{v} =\binom{3}{4} $ parallel verschoben. Ermittle rechnerisch die Gleichung der entstehenden Funktion $f^{**}$
\end{enumerate}
\end{minipage}
\hfill
\begin{minipage}{0.35\textwidth}
\includegraphics[width=5 cm]{poff208}
\end{minipage}

%Aufgabe 2
{\bf \item Der Graph einer Potenzfunktion $f_1(x)=x^4$ wird folgendermaßen auf $f_3$ abgebildet:}
$$f_1(x)=x^4 \xrightarrow{\textnormal{x-Achse; k=0,5}}  f_2  
\xrightarrow{\vec{v} =\binom{4}{1}} f_3$$
\begin{enumerate}[a)]
\item Gib die Gleichungen der Funktionen $f_2$ und $f_3$ an.
\item Zeichne $f_1, f_2$ und $f_3$ in ein Koordinatensystem und gib die Definitionsmenge und die Wertemenge von $f_3$ an.
\end{enumerate} 

%Aufgabe 3
{\bf \item Gegeben ist die Funktion $f$ mit der Gleichung $ y=2(x-1)^{-2}+3$}
\begin{enumerate}[a)]
\item Gib die Definitionsmenge und die Wertemenge an und begründe, dass die Funktion keine Nullstellen besitzt.
\item Tabellarisiere $f$ in einem geeigneten Intervall und zeichne den Graphen. Für die Zeichnung: Längeneinheit $1~cm;\quad -5 \le x \le 6;\quad -2 \le y \le 10$.
\item Der Graph wird mit $\vec{v} =\binom{-3}{-4} $
abgebildet. Zeichne den Bildgraphen $h$ in das Koordinatensystem zu Aufgabe 3. b) ein.
\item Ermittle rechnerisch die Gleichung zu $h$ mit Hilfe des Parameterverfahrens.
$[\textnormal{ Ergebnis}: h:~ y=2(x+2)^{-2}-1]$
\item Gib die Gleichungen der Asymptoten zu $h$ an.
\item Zeige rechnerisch, dass die Hyperbel $h$ achsensymmetrisch bezüglich der Geraden mit $x=-2$ ist.\\
 {\it Lösungsanleitung: Zeige, dass $f(-2+a) = f(-2-a)$ ist.}
\item Ermittle rechnerisch die Nullstellen der Hyperbel $h$ und überprüfe die gefundenen Werte anhand des Diagramms aus Aufgabe 3. c).
\end{enumerate} 

%Aufgabe 4
{\bf \item Berechne ohne Taschenrechner}

a) $ (2+2,3)^0 $ \hspace{20 pt} b) $ 1+ \left({ 1 \over 3}\right)^{-2}$ \hspace{20 pt} 
c) $1 : 4^{-2}$   \hspace{20 pt} 
d) $ 2^{x+2} \cdot  \left({ 1 \over 2}\right)^{x+2} $ \\
e) $\left({ 2 \over a}\right)^3 \cdot (-a)^3$
\end{enumerate} 

\fbox{
	\begin{minipage}{0.5\textwidth}
		Zur Lösung bitte \href{https://www.okuyakl.de/math/m10pofuL208/ll.pdf}{hier klicken} oder den QR-Code scannen.\\
	Weitere Arbeitsblätter gibt es unter 
	
	\href{https://www.okuyakl.de}{www.okuyakl.de}
	\end{minipage}
	\hfill
	\begin{minipage}{0.4\textwidth}
		\includegraphics[width=1.5 cm]{../../viecher/zwe03}
		\includegraphics[width=3 cm]{qr208}
		\includegraphics[width=2 cm]{../../viecher/afanticon1}
		
	\end{minipage}}

\end{document}%Lösung--------------------------------------------------------------------------------------------------
