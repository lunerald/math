\documentclass[a4paper]{article}
\usepackage[pdftex]{graphicx}
\usepackage[utf8]{inputenc}
\usepackage{enumerate}
\usepackage{icomma}
\usepackage{siunitx}
\sisetup{locale=DE} 
\usepackage{amssymb}
\usepackage{eurosym}
\usepackage{geometry}
\geometry{a4paper, top=15mm, left=15mm, right=15mm, bottom=15mm,
	headsep=10mm, footskip=12mm}
\usepackage{href-ul}
\hypersetup{
	colorlinks=true,
	linkcolor=blue,
	urlcolor=blue}
	
\begin{document}
{\bf Lineare Gleichungssysteme}
\begin{enumerate}[1.]
%Aufgabe 1
{\bf \item Löse das Gleichungssystem mit einer geeigneten Methode; $ \mathbb{G} =  \mathbb{R}
\times  \mathbb{R} $} 
\vspace{0.25 cm}

\begin{minipage}{0.5\textwidth}
	a)
	\begin{tabular}{cccc}
		\centering
		I. & $4y - 8x -24$ &=& 0\\
		II. & $17x + 9 - 5y$ &=& 0\\
	\end{tabular}
\end{minipage}
\hfill
\begin{minipage}{0.5\textwidth}
	b)
	\begin{tabular}{cccc}
	I.&	$6+6x$&=&$12y$\\
	II.&	$ -6x$ &=& $-4y-14$\\
	\end{tabular}
\end{minipage}


{\bf \item Gib zu folgenden Aufgaben jeweils ein lineares Gleichungssystem an und berechne es:}

\begin{enumerate}[a)]
\item Der Umfang eines gleichschenkligen Dreiecks beträgt $\SI{20}{\centi\meter}$. Seine Basis ist $\SI{2}{\centi\meter}$ länger als die Länge der Schenkel.
\item Ein Kaufmann zahlt zwei Rechnungen in bar. Vom Rechnungsbetrag der einen Rechnung zieht er 2\% Preisnachlass ab, von der anderen Rechnung 3\%. Zusammen sind es \EUR{144} an Preisnachlass. Behielte er von beiden Rechnungen 2,5\% Preisnachlass, würde er auf zusammen \EUR{140} Preisnachlass kommen.
\item Aus Orangennektar mit dem Fruchtgehalt von 30\% und Apfelsaft mit dem Fruchtgehalt von 80\% sollen 3 Liter Bowle mit dem Fruchtgehalt von 50\% gemischt werden. Wie viel Apfelsaft und wieviel Orangensaft müssen hierzu anteilig gemischt werden?
\end{enumerate}

{\bf \item Löse folgende lineare Gleichungssysteme mit dem angegebenen Verfahren.}
 
 \begin{minipage}{0.5\textwidth}
 	 a) Additionsverfahren \\
 	 $$
 	 \renewcommand{\arraystretch}{2}
 	 \begin{array}{rcl} 
 	 {1\over 3} x + {1\over 6} y &=& -2 \\
 	 \land \quad {2\over 3}y - {1 \over 2}x&=&-8 
 	 \end{array}
 	 $$
 \end{minipage}
 \hfill
 \begin{minipage}{0.5\textwidth}
 	 b) Determinantenverfahren\\
 	 $$
     \renewcommand{\arraystretch}{2}
 	 \begin{array}{rcl}
 	 -2x + 3y &=& 14 \\
 	 \land \quad 1,5x + 3,5y  &=& 1 
 	 \end{array}  
 	 $$	 
 \end{minipage}
 
{\bf \item Ein Parallelogramm hat die Fläche von $\SI{12}{\centi\meter^2}$ und den Umfang von $\SI{18}{\centi\meter}$.} Die Seite b ist halb so lang wie die Seite a. Berechne die Seitenlängen und die Höhe des Parallelogramms. Stelle dazu ein lineares Gleichungssystem auf.

{\bf \item Eine Kaffeeherstellerfirma will einen neuen Cappuchino auf den Markt bringen.} Es sollen 10 kg als Probe hergestellt werden. Für die Mischung verwendet man Kakaopulver zu \EUR{5} pro kg und Kaffeepulver zu \EUR{8} pro kg. Der Herstellungspreis für die 10 kg Probe liegt bei \EUR{68}. Berechne den Anteil an Kakaopulver und den Anteil an Kaffeepulver für diese Probe in kg.  Stelle dazu ein lineares Gleichungssystem auf.

{\bf \item Ein Rechteck hat einen Umfang von $\SI{28}{\centi\meter}$.} Verlängert man die kürzere Seite um  $\SI{2}{\centi\meter}$ und verkürzt man die längere Seite um $\SI{1}{\centi\meter}$, so erhält man ein neues Rechteck. der Flächeninhalt des neuen Rechtecks ist um $\SI{8}{\centi\meter^2}$ größer als der Flächeninhalt des ursprünglichen Rechtecks. \\
Wie lang sind die Seiten des ursprünglichen Rechtecks?

{\bf \item Ein gleichschenkliges Dreieck hat einen Umfang von $\SI{26}{\centi\meter}$.} Ein Schenkel ist fünfmal so lang wie der dritte Teil der Basis. \\
Wie lang sind die Schenkel und die Basis?

{\bf \item Die Differenz der Längen der beiden Grundseiten eines Trapezes beträgt  $\SI{4}{\centi\meter}$.} Die Höhe dieses Trapezes beträgt  $\SI{3}{\centi\meter}$, der Flächeninhalt $\SI{21}{\centi\meter^2}$\\
Berechne die Längen der beiden Grundseiten.

{\bf \item In einem Parallelogramm $ABCD$ ist der Winkel mit dem Maß $\alpha$ um $30^\circ$ größer als der Winkel mit dem Maß $\beta$.} Berechne die Maße der 4 Innenwinkel des Parallelogramms.

{\bf \item Eine zweiziffrige Zahl ist um 22 größer als ihre dreifache Quersumme.} Ihre Einerziffer ist um 1 kleiner als die Zehnerziffer. Wie lautet die Zahl? {\it Lege die Variablen fest, stelle die Gleichung auf und löse sie!}
\end{enumerate} 

\fbox{
	\begin{minipage}{0.5\textwidth}
		Zur Lösung bitte \href{https://www.okuyakl.de/math/m9lgsL043/ll043.pdf}{hier klicken} oder den QR-Code scannen.\\
	Weitere Arbeitsblätter gibt es unter 
	
	\href{https://www.okuyakl.de}{www.okuyakl.de}
	\end{minipage}
	\hfill
	\begin{minipage}{0.4\textwidth}
		\includegraphics[width=1.5 cm]{../../viecher/zwe03}
		\includegraphics[width=3 cm]{qr043}
		\includegraphics[width=2 cm]{../../viecher/afanticon1}
		
	\end{minipage}}

\end{document}%Lösung-------------------------------------------
