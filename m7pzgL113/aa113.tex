\documentclass[a4paper]{article}
\usepackage[pdftex]{graphicx}
\usepackage[utf8]{inputenc}
\usepackage{enumerate}
\usepackage{amssymb}
\usepackage{icomma}
\usepackage{siunitx}
\sisetup{                     
	locale=DE} 
\usepackage{geometry}
\geometry{a4paper, top=15mm, left=15mm, right=15mm, bottom=15mm,
	headsep=10mm, footskip=12mm}
\usepackage{href-ul}
\hypersetup{
	colorlinks=true,
	linkcolor=blue,
	urlcolor=blue}
	
\begin{document}
{\bf Potenzen}

\begin{enumerate}[1.]
{\bf \item Vereinfache, wenn möglich, folgende Potenzen mit dem ersten Potenzgesetz $a^m \cdot a^n = a^{m+n}$}

\renewcommand{\arraystretch}{3}
\begin{tabular}{p{5 cm}p{5 cm}p{5 cm}}
	a) $7^2 \cdot 7^3$	&b) $3^4 \cdot 3$  &c) $(-2)^2 \cdot (-2)^5$ \\
	d) $4^3 \cdot 4^{-1}$ &e) $\left({1\over 2}\right)^6 \cdot \left({1\over 2}\right)^5$	&f) $(-8)^{-3} \cdot (-8)^4$ \\
	g) $5^5 \cdot 5^{-8} \cdot 5$ &h) $x \cdot x \cdot x^2$ &i) $x^{-7}\cdot x^9$ \\
	j) $2^x \cdot 2^x$ &k) $x^2 +x^2$ &l) $3^{-6}\cdot 3^{-2}$ \\
\end{tabular}

{\bf \item Vereinfache folgende Potenzen mit dem zweiten Potenzgesetz ${a^m \over a^n} = a^{m-n}$}

\renewcommand{\arraystretch}{3}
\begin{tabular}{p{5 cm}p{5 cm}p{5 cm}}
	a)\Large $7^4 \over 7^2$	&b)\Large $9^3 \over 9$ &c)\Large $6 \over 6^{8}$ \\
	d)\Large $4^2 \over 4^{-6}$ &e)\Large$3^{-3} \over 3^{2}$&f) \Large$2^{-4}\over 2^{-10}$\\
	g)\Large  $x^9 \over x^3$ &h) \Large$ y\over y$ &i)\Large	$x \over x^{-1}$ \\
	j)\Large $2^{x+1} \over 2^x$&k)\Large $3^{2x} \over 3^{x}$&l)\Large $x^{-4} \over x^{-7}$
\end{tabular}

{\bf \item Wandle negative Potenzen in Brüche um und umgekehrt}

\renewcommand{\arraystretch}{3}
\begin{tabular}{p{5 cm}p{5 cm}p{5 cm}}
	a) $7^{-1}$	&b)\Large ${1\over 3^2}$ &c)\Large ${1 \over 2^{-1}}$ \\
	d) $8^{-3}$&e) $x^{-10}$	&f) \Large${1\over x^{12} }$\\
	g) $(-1)^{-1}$ &h) ${1\over 5^2\cdot 5}$ &i) $6^{-3}\cdot 6^{-2}$ \\
	j) $(-a)^{-2}$ &k) $2^{-x}$ &l) ${2\over 3}^{-1}$ 
\end{tabular}

{\bf \item Vereinfache folgende Potenzen mit dem dritten Potenzgesetz $(a^m)^n = a^{m\cdot n}$}

\renewcommand{\arraystretch}{3}
\begin{tabular}{p{5 cm}p{5 cm}p{5 cm}}
	a) $\left(2^2\right)^3 $	&b)  $\left(7^4 \right)^{-1} $&c)  $\left(4^4 \right)^5 $\\
	d)  $\left(1,5^2 \right)^2 $&e)	 $\left(0,1^{-5} \right)^2 $&f)  $\left(x^2 \right)^{-6} $\\
	g)   $\left(1^{10} \right)^{2} $&h)  $\left(2^x \right)^2 $&i)	 $\left(2^2 \right)^x $\\
	j)  $\left(\left({5\over 6}\right)^{2} \right)^{1} $&k)  $\left(10^7 \right)^{-3} $&l)  $\left(7^{-4} \right)^{-3} $
\end{tabular}

\newpage
{\bf \item Fasse zusammen nach den Regeln $a^nb^n = (ab)^n \qquad {a^n\over b^n} = ({a\over b})^n$}

\renewcommand{\arraystretch}{3}
\begin{tabular}{p{5 cm}p{5 cm}p{5 cm}}
	a) $3^4 \cdot 2^4 $	&b) \Large${12^4 \over 4^4}$ &c) $5^{-7}\cdot 2^{-7} $ \\
	d) $15^2 \cdot \left({1\over 3}\right)^2$ &e) $100^7 : 5^7 $ &f)\Large ${30^6 \over 5^6}$\\
	g)\Large ${1^{10} \over 2^{10}}$ &h) $7^2\cdot2^2\cdot3^2$ &i) $\left({1\over 3}\right)^4 \cdot 21^4$	\\
	j) $\left({1\over 4}\right)^2 \cdot \left({1\over 3}\right)^2$ &k) $\left({3\over 5}\right)^2 \cdot \left({10\over 3}\right)^2$ &l) $\left({1\over 4}\right)^{-1} \cdot \left({8\over 3}\right)^{-1}$ 
\end{tabular}

{\bf \item Schreibe als Potenz mit möglichst einfacher Basis}

\renewcommand{\arraystretch}{3}
\begin{tabular}{p{5 cm}p{5 cm}p{5 cm}}
	a) $9^2$ &b) $16^{-2}$ &c) $\left({1\over 4}\right)^3$ \\
	d) $36^{10} $ &e)$\left({25\over 49}\right)^{-1}$	&f) ${1\over 100^2} $\\
	g) $\left({64\over 36}\right)^3$ &h) $\left({18\over 8}\right)^{-1}$&i)	$8^3$\\
	j) $27^{-1}$ &k) $\left({32\over 162}\right)^2$&l) $\left({24\over 81}\right)^{-2}$
\end{tabular}


{\bf \item Fasse zusammen, wenn möglich. Entscheide, welche Gesetze Du anwendest} 

\renewcommand{\arraystretch}{3}
\begin{tabular}{p{5 cm}p{5 cm}p{5 cm}}
	a) $2^2 \cdot 8^2 \cdot 4$	&b) $7+7^2+7^3$ &c) $\left({16\over 9}\right)^{-1}\cdot \left({4\over 3}\right)^2$ \\
	d) $(10^5)^2 : 5^{10}$ & e) $\left({11\over 7}\right)^2 \cdot 121^{-1}$	&f) $\left({1\over 12}\right)^2 \cdot 3^4 \cdot 4^4$\\
	 g) $ 3^6 \cdot 5^3 $  &h) $ 1024 \cdot 2^{-9}$&i) $\left(5^2 \cdot 4^2 \right)^{-1}$ \\
	 j) $\left(32^{-1} \cdot 3^5\right)^2$ &k) $(-1)^{18}\cdot(-3)^4$ &l) $5^{2+3}\cdot {1\over 5^2}$
\end{tabular}

{\bf \item Bestimme $x$}

\renewcommand{\arraystretch}{2}
\begin{tabular}{p{5 cm}p{5 cm}p{5 cm}}
	a) $2^x = 32$	&b) $2^x = 16^2$ &c) $3^x ={1\over 9}$ \\
	d) $x^3 =-27$ & e) $x^4 = {1\over 16}$ &f) $10^x =1000$ \\
	g) $10^x = 0,01$ &h) $4^x =1$ & i)$\left(1\over 2\right)^x =4$
\end{tabular}
\end{enumerate} 

\fbox{
	\begin{minipage}{0.5\textwidth}
		Zur Lösung bitte \href{https://www.okuyakl.de/math/m7pzgL113/ll113.pdf}{hier klicken} oder den QR-Code scannen.\\
	Weitere Arbeitsblätter gibt es unter 
	
	\href{https://www.okuyakl.de}{www.okuyakl.de}
	\end{minipage}
	\hfill
	\begin{minipage}{0.4\textwidth}
		\includegraphics[width=1.5 cm]{../../viecher/zwe03}
		\includegraphics[width=3 cm]{qr113}
		\includegraphics[width=2 cm]{../../viecher/afanticon1}
		
	\end{minipage}}

\end{document}

