\documentclass[a4paper]{article}
\usepackage[pdftex]{graphicx}
\usepackage[utf8]{inputenc}
\usepackage{enumerate}
\usepackage{icomma}
\usepackage{siunitx}
\sisetup{locale=DE} 
\usepackage{href-ul}
\hypersetup{
	colorlinks=true,
	linkcolor=blue,
	urlcolor=blue}
\usepackage{amssymb}
\usepackage{geometry}
\geometry{a4paper, top=15mm, left=15mm, right=15mm, bottom=15mm,
	headsep=10mm, footskip=12mm}

\begin{document}
{\bf Kreisteile II}
\begin{enumerate}[1.]
 
\begin{minipage}{0.5\textwidth}
{\bf \item Die Abbildung zeigt ein gleichschenklig-rechtwinkliges Dreieck.}
\begin{enumerate}
\item Berechne den Inhalt der blauen Fläche in Abhängigkeit von a.
\item Zeige, dass die schraffierte Fläche zur blauen Fläche inhaltsgleich ist.  
\end{enumerate}
\end{minipage}
\begin{minipage}{0.5\textwidth}
\includegraphics[width=6 cm]{mond248}
\end{minipage}

\begin{minipage}{0.7\textwidth}
{\bf \item Nebenstehende Skizze zeigt einen Kreissektor mit dem Mittelpunktswinkel $\mu=70^\circ$ und dem Radius $r=\SI{4}{\centi\meter}$,} sowie ein Kreissegment, das durch den Kreisbogen und die Strecke $[AB]$ begrenzt wird. Berechnen Sie, um wie viel Prozent der Flächeninhalt des Kreissektors größer ist als der Flächeninhalt des Dreiecks $ABM$. 
\end{minipage}
\hfill
\begin{minipage}{0.25\textwidth}
\includegraphics[width=3.5 cm]{kreisek137}
\end{minipage}

\begin{minipage}{0.3\textwidth}
{\bf \item Bestimmen Sie Umfang $U$ und Flächeninhalt $A$ eines Spitzbogens mit Radius }
$r=\SI{1,5}{\meter}$ \\
\includegraphics[width=4.5 cm]{goti216}
\vspace{0.5 cm}
\end{minipage}
\hspace{0.3 cm}
\begin{minipage}{0.3\textwidth}
\includegraphics[width=4.5 cm]{dreikrei216}
{\bf \item Bestimmen Sie den Flächeninhalt $A$ der blau schattierten Figur in Abhängigkeit von $r$.}\\
\end{minipage}
\hspace{0.3 cm}
\begin{minipage}{0.3\textwidth}
{\bf \item Bestimmen Sie den Flächeninhalt des ,,Giftzahns" in Abhängigkeit von $r$.}
\includegraphics[width=5.5 cm]{gizz216}
\end{minipage}


\begin{minipage}{0.35\textwidth}
{\bf 	\item Aufgabe: Katzenauge}
	\begin{enumerate}[a)]
	\item Berechne den Flächeninhalt der blauen Fläche in Abhängigkeit von $a$.
	\item Berechne den Flächeninhalt  der blauen Fläche  für $a=\SI{8}{\centi\meter}$
\end{enumerate} 
\end{minipage}
\hfill
\begin{minipage}{0.6\textwidth}
	\includegraphics[width=9 cm]{auge204}
\end{minipage}

\end{enumerate} 

\fbox{
	\begin{minipage}{0.5\textwidth}
		Zur Lösung bitte \href{https://www.okuyakl.de/math/m10kreibL027/ll027.pdf}{hier klicken} oder den QR-Code scannen.\\
	Weitere Arbeitsblätter gibt es unter 
	
	\href{https://www.okuyakl.de}{www.okuyakl.de}
	\end{minipage}
	\hfill
	\begin{minipage}{0.4\textwidth}
		\includegraphics[width=1.5 cm]{../../viecher/zwe03}
		\includegraphics[width=3 cm]{qr027}
		\includegraphics[width=2 cm]{../../viecher/afanticon1}
		
	\end{minipage}}

\end{document}%Lösung-------------------------------------------
