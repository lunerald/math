\documentclass[a4paper]{article}
\usepackage[pdftex]{graphicx}
\usepackage[utf8]{inputenc}
\usepackage{enumerate}
\usepackage{icomma}
\usepackage{polynom}
\usepackage{amssymb}
\usepackage{href-ul}
\hypersetup{
	colorlinks=true,
	linkcolor=blue,
	urlcolor=blue}
\usepackage{geometry}
\geometry{a4paper, top=15mm, left=15mm, right=15mm, bottom=15mm,
	headsep=10mm, footskip=12mm}

\begin{document}
\begin{enumerate}[1.]
{\bf \item Führen Sie die Polynomdivision durch}


 $${\rm a)}\quad (4x^3-4x^2+7x-3):(2x-1)\hspace{2 cm} {\rm b)}\quad (3x^4+5x^3+7x^2+15x-6):(x^2+3)$$


{\bf \item Raten Sie eine Nullstelle und dividieren Sie durch den entsprechenden Linearfaktor}
$${\rm a)}\quad x^5-2x^3+3x^2-4x-8 \hspace{2 cm} {\rm b)}\quad x^4+2x^3-12x+9 $$

{\bf \item Wandeln Sie die unechten Brüche um in einen ganzrationalen Term plus einen echt gebrochenen Rest}

 $$ {\rm a)}\quad  {2x+1 \over x-1} \hspace{2 cm} {\rm b)}\quad {x^2+8x \over 2x-4} \hspace{2 cm} {\rm c)}\quad {x^3 \over x-3} \hspace{2 cm} {\rm d)}\quad {4x^3+x^2+6x \over 2x^2-3}$$

{\bf \item Gegeben sind folgende ganzrationale Funktionen:}\\
\includegraphics[width=14 cm]{poly094}

\renewcommand{\arraystretch}{2}
\begin{tabular}{|p{4 cm}|p{3.5 cm}|p{3.5 cm}|p{3.5 cm}|}
\hline
	                            &Funktion A &Funktion B &Funktion C \\
\hline
$\lim\limits_{x \to \infty}$	&   &   &   \\
\hline
$\lim\limits_{x \to -\infty}$	&   &   &   \\
\hline
Einfache Nullstellen           & & & \\
\hline
Doppelte Nullstellen           & & & \\
\hline
Dreifache Nullstellen          & & & \\
\hline
Grad  & & & \\
\hline
Markanter Punkt            & $P_A(0|\rule{30 pt}{0.5 pt})$ & $P_B(-2|\rule{30 pt}{0.5 pt})$ & $P_C(5|\rule{30 pt}{0.5 pt})$ \\
\hline
Vorfaktor a & & & \\
\hline
Möglicher Funktionsterm & & & \\
\hline
\end{tabular}


\end{enumerate} 
\fbox{
	\begin{minipage}{0.5\textwidth}
		Zur Lösung bitte \href{https://www.okuyakl.de/math/m11polL094/ll094.pdf}{hier klicken} oder den QR-Code scannen.\\
	Weitere Arbeitsblätter gibt es unter 
	
	\href{https://www.okuyakl.de}{www.okuyakl.de}
	\end{minipage}
	\hfill
	\begin{minipage}{0.4\textwidth}
		\includegraphics[width=1.5 cm]{../../viecher/zwe03}
		\includegraphics[width=3 cm]{qr094}
		\includegraphics[width=2 cm]{../../viecher/afanticon1}
		
\end{minipage}}
\end{document}