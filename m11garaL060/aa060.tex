\documentclass[a4paper]{article}
\usepackage[pdftex]{graphicx}
\usepackage[utf8]{inputenc}
\usepackage{enumerate}
\usepackage{icomma}
\usepackage{amssymb}
\usepackage{href-ul}
\hypersetup{
	colorlinks=true,
	linkcolor=blue,
	urlcolor=blue}
\usepackage{geometry}
\geometry{a4paper, top=15mm, left=15mm, right=15mm, bottom=15mm,
	headsep=10mm, footskip=12mm}

\begin{document}
{\bf Ganzrationale Funktionen}
\begin{enumerate}[1.]
%Aufgabe 1
{\bf \item Gegeben ist die Funktion $f(x) = \frac{3}{4} x^4 - 4x^3 + 6x^2 $ mit $ D_f = \mathbb{R} $ .}

\begin{enumerate}[a)]
\item Zeigen Sie, dass die Funktion genau eine ( und zwar eine doppelte) Nullstelle besitzt.
\item Ermitteln Sie Art und Lage der Punkte mit waagrechter Tangente. 
\end{enumerate}

{\bf \item Von einer ganzrationalen Funktion  $ f $ vom Grad 3 ist folgendes bekannt:}

\begin{itemize}
\item $G_f$ ist punktsymmetrisch zum Ursprung
\item der Punkt $ H(-1|4) $ ist ein Hochpunkt
\end{itemize}

\noindent Berechnen Sie den Funktionsterm von $f$.

{\bf \item Bilde die Ableitungsfunktion $f'$.}

\begin{enumerate}[a)]
\item $$ f(x)= 0,5 x^6 - 2,5x + 4 -{2 \over x^2}$$
\item $$ f(x)= 2 x^2 \cdot \sqrt{x}$$
\end{enumerate}

{\bf \item Gegeben ist die Funktion $f(x)=0,25x^3+3x^2+9x; \quad  D_f=\mathbb{R}$;}
\begin{enumerate}[a)]
\item Berechne, welchen Winkel die Tangente $t$ an den Graphen $G_f$ im Punkt $B(-4|-4)$ mit der x-Achse einschließt!
\item Überprüfe, ob der Graph von $f$ noch eine zu $t$ parallele Tangente besitzt.
\end{enumerate}

{\bf \item Gegeben ist die Funktion $ f(x)= -x^2+5x; \quad  D_f=\mathbb{R}$;}
Die Kurvennormalen $n_1$ und $n_2$ in den Punkten $N_1(0|0)$ und $N_2(5|0)$ des Graphen $G_f$ schneiden sich im Punkt $P$. Bestimme den Flächeninhalt des Dreiecks $N_1PN_2$.

{\bf \item Die gegebene Funktion $f$ ist eine nach unten geöffnete Normalparabel, die um 9 Einheiten nach oben verschoben ist.}  Der Graph von $f$ wird mit $G_f$ bezeichnet. 
\begin{enumerate}[a)]
\item Ermittle einen Funktionsterm.
\item Zwischen den Nullstellen wird $G_f$ ein Rechteck $ABCD$ einbeschrieben, wobei $A$ und $B$ auf der x-Achse, $C$ und $D$ auf $G_f$ liegen. Die x-Koordinate von $B$ sei $u$. Erstelle eine Zeichnung für $u=2$.
\item Bestimme $u$ so, dass die Rechtecksfläche maximal wird.
\end{enumerate}
\end{enumerate}

\fbox{
	\begin{minipage}{0.5\textwidth}
		Zur Lösung bitte \href{https://www.okuyakl.de/math/m11garaL060/ll060.pdf}{hier klicken} oder den QR-Code scannen.\\
	Weitere Arbeitsblätter gibt es unter 
	
	\href{https://www.okuyakl.de}{www.okuyakl.de}
	\end{minipage}
	\hfill
	\begin{minipage}{0.4\textwidth}
		\includegraphics[width=1.5 cm]{../../viecher/zwe03}
		\includegraphics[width=3 cm]{qr060}
		\includegraphics[width=2 cm]{../../viecher/afanticon1}
		
	\end{minipage}} 

\end{document}%Lösung-------------------------------------------
