
\documentclass[a4paper]{article}
\usepackage[pdftex]{graphicx}
\usepackage[utf8]{inputenc}
\usepackage{enumerate}
\usepackage{amssymb}
\usepackage{href-ul}
\hypersetup{
	colorlinks=true,
	linkcolor=blue,
	urlcolor=blue}
\usepackage{geometry}
\geometry{a4paper, top=15mm, left=15mm, right=15mm, bottom=15mm,
headsep=10mm, footskip=12mm}


%Source: img_0059

% Start the document
\begin{document}
%Titel
{\bf Mathematik -- Vektorrechnung Q11}\\
\begin{enumerate}[1.]

%Aufgabe 1

{\bf \item Geben Sie zwei Punkte P und Q an, so dass ihre Entfernung 6 LE beträgt.} P und Q sollen ungleich dem Koordinatenursprung sein.

%Aufgabe 2

{\bf \item Zeigen Sie, dass die Vektoren einen Würfel ABCDEFGH aufspannen, also dass die Vektoren die Kanten eines Würfels sind.}
$$ \vec{AB} = \left( 
\begin{array}{c}
{1}\\ {2}\\ {2}
\end{array}
\right) ; 
\hspace{15 pt}
 \vec{AD} = \left( 
\begin{array}{c}
{2}\\ {1}\\ {-2}
\end{array}
\right) ;
\hspace{15 pt}
 \vec{AE} = \left( 
\begin{array}{c}
{2}\\ {-2}\\ {1}
\end{array}
\right) $$

{\bf \item In einem dreidimensionalen Koordinatensystem mit dem Ursprung O sind die Punkte
$B(-3|4|0)$ und $D(-2|1|2)$ gegeben.}

\begin{enumerate}[a)]
\item Berechnen Sie die Koordinaten des Punktes $C$ so, dass das entstehende Viereck $OBCD$
ein Parallelogramm ist.[Zur Kontrolle $C(-5|5|2)$]
\item Das Parallelogramm $OBCD$ ist die Grundfläche einer geraden Pyramide mit der Spitze $S$.
Berechnen Sie den Inhalt der Grundfläche der Pyramide.
\item Bestimmen Sie die Koordinaten von $S$ so, dass die Höhe der Pyramide $5\sqrt{5}$ LE beträgt!
\item Die Pyramide rotiert nun um die $x_2$-Achse. Der Eckpunkt $D$ bewegt sich dabei auf einem Kreis. Geben Sie die Koordinaten des Mittelpunkts $N$ dieses Kreises an und berechnen Sie seinen Radius!
\end{enumerate}

{\bf \item Die Punkte $A(4|0|3)$, $B(0|5|0)$, $C(1|5|7)$ und $D(-3|0|4)$ sind Eckpunkte einer dreiseitigen Pyramide $ABCD$.}\
\begin{enumerate}[a)]
\item Zeichnen Sie  ein Schrägbild der Pyramide in einem 3D-Koordinatensystem. Wie verläuft die Kante $[AD]$ im Koordinatensystem?
\item Zeigen Sie, dass die Pyramide ein regelmäßiges Tetraeder ist.
\item Das Methanmolekül $CH_4$ ist tetraedrisch gebaut, wobei das $C$-Atom im räumlichen Mittelpunkt $M_C$ ist und die vier Bindungen zu den $H$-Atomen zu den Spitzen zeigen. Die Koordinaten des räumlichen Mittelpunktes sind das arithmetische Mittel der vier Eckpunkte. Berechnen Sie den Winkel zwischen zwei $C-H$-Bindungen.  
\item Berechnen Sie das Volumen und den Oberflächeninhalt des Tetraeders.
\item $M_1$, $M_2$, und $M_3$ sind die Mittelpunkte der Kanten $[AB]$, $[AC]$ und $[AD]$. Durch einen ebenen Schnitt, der durch die Mittelpunkte verläuft, wird von der Pyramide $ABCD$ die Pyramide $AM_1M_2M_3$ abgetrennt. Wie groß ist das Volumen und die Oberfläche der kleinen Pyramide im Verhältnis zur großen?[Hinweis:Zentrische Streckung] 
\item Berechnen Sie das Volumen und den Oberflächeninhalt des Pyramidenstumpfes.
\end{enumerate}
{\bf \item Weisen Sie nach, dass in jedem Parallelogramm die Summe der Quadrate über den vier Seiten ebenso groß ist wie die Summe der Quadrate über den beiden Diagonalen.}

\begin{minipage}{0.7\textwidth}
{\bf \item Projizieren Sie das Dreieck $ABC$ mit $A(2|-3|7)$, $B(1|4|9)$ und $C(-3|-4|5)$ senkrecht in die $x_1x_2$-Ebene.}
\begin{enumerate}[a)]
\item Geben Sie die Koordinaten der Punkte $A'$, $B'$ und $C'$ an.
\item Vergleichen Sie den Flächeninhalt des Dreiecks $A'B'C'$ mit dem des Dreiecks $ABC$.
\end{enumerate}
\end{minipage}
\begin{minipage}{0.3\textwidth}
\includegraphics[width= 5 cm]{proji337}
\end{minipage}

{\bf \item Eine Kugel mit dem Mittelpunkt $M(2|-1|7)$ geht durch den Punkt $A(8|2|5)$.}
\begin{enumerate}[a)]
\item Bestimmen Sie diejenigen Punkte $B_k(5|5|k)$ mit dem reellen Parameter $k$, die ebenfalls auf der Kugeloberfläche liegen.
\item Berechnen Sie den Flächeninhalt des Dreiecks $AB_9M$, wobei $B_9$ für den Punkt $(5|5|9)$ steht.
\end{enumerate}

\end{enumerate}

\fbox{
	\begin{minipage}{0.5\textwidth}
		Zur Lösung bitte \href{https://www.okuyakl.de/math/m11vecL337/ll337.pdf}{hier klicken} oder den QR-Code scannen.\\
	Weitere Arbeitsblätter gibt es unter 
	
	\href{https://www.okuyakl.de}{www.okuyakl.de}
	\end{minipage}
	\hfill
	\begin{minipage}{0.4\textwidth}
		\includegraphics[width=1.5 cm]{../../viecher/zwe03}
		\includegraphics[width=3 cm]{qr337}
		\includegraphics[width=2 cm]{../../viecher/afanticon1}
		
	\end{minipage}}
\end{document}%L Lösung ----------------------------------------------------------
