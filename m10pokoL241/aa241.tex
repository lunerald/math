
\documentclass[a4paper]{article}
\usepackage[pdftex]{graphicx}
\usepackage[utf8]{inputenc}
\usepackage{enumerate}
\usepackage{icomma}
\usepackage{amssymb}
\usepackage{colortbl}
\usepackage{geometry}
\geometry{a4paper, top=15mm, left=15mm, right=15mm, bottom=15mm,
	headsep=10mm, footskip=12mm}
\usepackage{href-ul}
\hypersetup{
	colorlinks=true,
	linkcolor=blue,
	urlcolor=blue}
% Start the document
\begin{document}
%Titel
\noindent{\bf Kartesische Koordinaten und Polarkoordinaten}\\

\noindent
\begin{minipage}{0.75\textwidth}
Jeder Punkt  $P(x|y)$ eines karte\-sischen Koordinatensystems kann auch als Wertepaar $P(r|\varphi)$ mit dem Abstand vom Ursprung $r$ und  dem Winkel $\varphi$ dargestellt werden. \\
Es gilt für  $P(r|\varphi) \Rightarrow P(x|y)$:
$$ x = r \cdot \cos \varphi \qquad ; \qquad y =r \cdot \sin \varphi$$
und für $P(x|y) \Rightarrow P(r|\varphi)$:
$$ r = \sqrt{x^2 + y^2} \qquad ; \qquad  \varphi = \arctan{y \over x}$$

\end{minipage}
\hfill
\begin{minipage}{0.25\textwidth}
\includegraphics[width=5 cm]{poko241}
\end{minipage}
\vspace{0.3 cm}

\noindent{\bf Rechne folgende Koordinaten um:}\\ Stelle, wenn möglich als exakte Zahl dar und runde sonst auf drei gültige Ziffern:
\vspace{0.3 cm}

\renewcommand{\arraystretch}{2}
\begin{tabular}{|p{5cm}|p{5 cm}|p{5 cm}|}
	\hline
 \rowcolor[gray]{.8} Kartesische Koordinaten	& Polarkoordinaten Gradmaß & Polarkoordinaten Bogenmaß \\
	\hline
$A(4|3)$	& &  \\
	\hline
	& $B(2|30^\circ)$ &  \\
	\hline
 $C(1,5|0)$	& &  \\
	\hline
	& & $D(2| {\pi \over 4})$ \\
	\hline
	& $E(8|90^\circ)$ &  \\
	\hline
	& & $F(4|3,14)$ \\
	\hline
$G(-5|-12)$ & &  \\
	\hline
	& & $H(12|{3 \over 2}\pi)$ \\
	\hline
	& $I(4,2|330^\circ)$ &  \\
	\hline
	& $J(3,2|275^\circ)$ &  \\
	\hline
$K(0|2)$	& &  \\
	\hline
	& & $L(12|{\pi \over 12})$\\
	\hline
$M(6|-8)$ & & \\
	\hline
		& $N(4|135^\circ)$ &\\
	\hline
$O(0|0)$		& & \\
	\hline
		& $P(1|200^\circ)$ &\\
	\hline
	 & & $Q(1|1)$\\
	 \hline
$R(1|1)$	& &  \\
	 \hline
	 & $S(1|1^\circ)$ & \\
	 \hline
\end{tabular}


\fbox{
	\begin{minipage}{0.5\textwidth}
		Zur Lösung bitte \href{https://www.okuyakl.de/math/m10pokoL241/ll241.pdf}{hier klicken} oder den QR-Code scannen.\\
	Weitere Arbeitsblätter gibt es unter 
	
	\href{https://www.okuyakl.de}{www.okuyakl.de}
	\end{minipage}
	\hfill
	\begin{minipage}{0.4\textwidth}
		\includegraphics[width=1.5 cm]{../../viecher/zwe03}
		\includegraphics[width=3 cm]{qr241}
		\includegraphics[width=2 cm]{../../viecher/afanticon1}
		
\end{minipage}}
%Ende
\end{document}