\documentclass[a4paper]{article}
\usepackage[pdftex]{graphicx}
\usepackage[utf8]{inputenc}
\usepackage{enumerate}
\usepackage{amssymb}
\usepackage{href-ul}
\hypersetup{
	colorlinks=true,
	linkcolor=blue,
	urlcolor=blue}
\usepackage{geometry}
\geometry{a4paper, top=15mm, left=15mm, right=15mm, bottom=15mm,
	headsep=10mm, footskip=12mm}

\begin{document}
	%Title
	{\bf exam questions from mathematics }\\
	\begin{enumerate}[1.]
		
		%Task 1
		{\bf \item The function $f(x)=1-(\ln{x})^2$ with domain $\mathbb{R}^+$ is given. The figure shows the graph $G_f$ of the function $f$.}
		
		\begin{enumerate}[a)]
			\begin{minipage}{0.6\textwidth}
				\item Determine the equation of the turning tangent w. $[y=-{2 \over e} \cdot x +2]$
				\item Show: the function $F(x)=-x (\ln{x}-1)^2 $ with $ D_F=\mathbb{R}^+$ is an antiderivative of $f$.
				\item Calculate the contents of the area bounded by $G_f$ and the x-axis in the first quadrant.
			\end{minipage}
			\hfill
			\begin{minipage}{0.4\textwidth}
				\includegraphics[width=5 cm]{elenint222}
			\end{minipage}
			\item Justify that the antiderivative $F$ is also the integral function
			$$ I(x)=\int\limits_0^x f(t)~{\rm d} t \quad \textnormal{with} ~ x\in \mathbb {R}^+$$.
			\item Calculate $\lim\limits_{x \to 0^+} F(x) $ and interpret the result geometrically using the graph.
		\end{enumerate}
		
		%Exercise 2
		{\bf \item The function $f(x)=(e^x-2)^2$ with domain $\mathbb{R}$ is given.
			Your graph is denoted by $G_f$ (see figure).}
		\begin{enumerate}[a)]
			\begin{minipage}{0.6\textwidth}
				\item The integral function defined in $\mathbb{R}$ is also given
				$$ I(x)=\int\limits_{\ln{2}}^x f(t)~{\rm d} t. $$
				Without using an integral-free representation of $I$, determine the monotonicity behavior of $I$. Show that $G_I$ has a terrace point and give its coordinates.
			\end{minipage}
			\hfill
			\begin{minipage}{0.4\textwidth}
				\includegraphics[width=5 cm]{eh2int222}
			\end{minipage}
			\item Show that the function $F(x)=0.5 e^{2x}-4e^x+4x$ with $D_F=\mathbb{R}$ is an antiderivative of $f$.
			\item The graph $G_f$ closes an area with the content $ using the straight lines in the I and II quadrants determined by the equations $y=4$ and \\ $x=u\quad \{u < 0 \}$ A(u)$ a. Determine $A(u)$.
			\item Determine $\lim\limits_{u \to -\infty}A(u)$ and interpret the result geometrically.
		\end{enumerate}
		
		%Task 3
		\begin{minipage}{0.6\textwidth}
			{\bf \item The figure on the right shows the graph $G_f$ of a completely rational function $f$ of the third degree with $ D_f=\mathbb{R}$.} The integral function is now considered
			$$I(x)= \int\limits_0^x f(t)~{\rm d} t \quad \textnormal{with} ~ D_I=\mathbb{R}$$
			One of the three graphs A, B, or C shown below represents the graph of $I$. Indicate which one it is and justify your answer by explaining why the other two graphs are not considered.
		\end{minipage}
		\hfill
		\begin{minipage}{0.4\textwidth}
			\includegraphics[width=5 cm]{gar3int222}
		\end{minipage}
	\end{enumerate}
	\includegraphics[width=4 cm]{ifunk1222}
	\includegraphics[width=4 cm]{ifunk2222}
	\includegraphics[width=4 cm]{ifunk3222}
	\vspace{0.5 cm}
	
	\fbox{
		\begin{minipage}{0.5\textwidth}
			For the solution, please \href{https://www.okuyakl.de/math/m12intapL222/le222.pdf}{click here} or scan the QR code.\\
			Additional worksheets are available at
			
			\href{http://www.okuyakl.com}{www.okuyakl.com}
		\end{minipage}
		\hfill
		\begin{minipage}{0.4\textwidth}
			\includegraphics[width=1.5 cm]{../../viecher/zwe03}
			\includegraphics[width=3 cm]{qre222}
			\includegraphics[width=2 cm]{../../viecher/afanticon1}
			
	\end{minipage}}
\end{document}