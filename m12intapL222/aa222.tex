\documentclass[a4paper]{article}
\usepackage[pdftex]{graphicx}
\usepackage[utf8]{inputenc}
\usepackage{enumerate}
\usepackage{amssymb}
\usepackage{href-ul}
\hypersetup{
	colorlinks=true,
	linkcolor=blue,
	urlcolor=blue}
\usepackage{geometry}
\geometry{a4paper, top=15mm, left=15mm, right=15mm, bottom=15mm,
	headsep=10mm, footskip=12mm}
%Source: img_0222

% Start the document
\begin{document}
%Titel
{\bf Prüfungsaufgaben aus der Mathematik }\\
\begin{enumerate}[1.]

%Aufgabe 1
{\bf \item Gegeben ist die Funktion $f(x)=1-(\ln{x})^2$ mit Definitionsbereich $\mathbb{R}^+$. Die Abbildung zeigt den Graphen $G_f$ der Funktion $f$.}

\begin{enumerate}[a)]
\begin{minipage}{0.6\textwidth}
\item Bestimmen Sie die Gleichung der Wendetangente w. $[y=-{2 \over e} \cdot x +2]$
\item Zeigen Sie: die Funktion $F(x)=-x (\ln{x}-1)^2 $ mit $ D_F=\mathbb{R}^+$ ist Stammfunktion von $f$.
\item Berechnen Sie den Inhalt der Fläche, die von $G_f$ und der x-Achse im ersten Quadranten begrenzt wird.
\end{minipage}
\hfill
\begin{minipage}{0.4\textwidth}
\includegraphics[width=5 cm]{elenint222}
\end{minipage}
\item Begründen Sie, dass die Stammfunktion $F$ zugleich die Integralfunktion 
$$ I(x)=\int\limits_0^x f(t)~{\rm d} t \quad \textnormal{mit} ~ x\in\mathbb{R}^+$$ ist.
\item Berechnen Sie $\lim\limits_{x \to 0^+} F(x) $ und deuten Sie das Ergebnis anhand des Graphen geometrisch.
\end{enumerate}

%Aufgabe 2
{\bf \item Gegeben ist die Funktion $f(x)=(e^x-2)^2$ mit Definitionsmenge $\mathbb{R}$.
Ihr Graph wird mit $G_f$ bezeichnet (siehe Abbildung).}
\begin{enumerate}[a)]
\begin{minipage}{0.6\textwidth}
\item Gegeben ist außerdem die in $\mathbb{R}$ definierte Integralfunktion 
$$ I(x)=\int\limits_{\ln{2}}^x f(t)~{\rm d} t. $$
Bestimmen Sie ohne Verwendung einer Integralfreien Darstellung von $I$ das Monotonieverhalten von $I$. Zeigen Sie, dass $G_I$ einen Terrassenpunkt besitzt und geben Sie dessen Koordinaten an.
\end{minipage}
\hfill
\begin{minipage}{0.4\textwidth}
\includegraphics[width=5 cm]{eh2int222}
\end{minipage}
\item Zeigen Sie, dass die Funktion $F(x)=0,5 e^{2x}-4e^x+4x$ mit $D_F=\mathbb{R}$ eine Stammfunktion von $f$ ist.
\item Der Graph $G_f$ schließt mit den durch die Gleichungen $y=4$ und \\ $x=u\quad \{u < 0 \}$ bestimmten Geraden im I. und II. Quadranten ein Flächenstück mit dem Inhalt $A(u)$ ein. Bestimmen Sie $A(u)$.
\item Ermitteln Sie $\lim\limits_{u \to -\infty}A(u)$ und deuten Sie das Ergebnis geometrisch. 
\end{enumerate}

%Aufgabe 3
\begin{minipage}{0.6\textwidth}
{\bf \item Die Abbildung rechts zeigt den Graphen $G_f$ einer ganzrationalen Funktion $f$ dritten Grades mit  $ D_f=\mathbb{R}$.} Betrachtet wird nun die Integralfunktion 
$$I(x)= \int\limits_0^x f(t)~{\rm d} t \quad \textnormal{mit} ~  D_I=\mathbb{R}$$
Eine der drei unten abgebildeten Graphen A, B, oder C stellt den Graphen von $I$ dar. Geben Sie an, welcher dies ist, und begründen Sie Ihre Antwort, indem Sie erklären, warum die beiden anderen Graphen nicht in Betracht kommen.
\end{minipage}
\hfill
\begin{minipage}{0.4\textwidth}
\includegraphics[width=5 cm]{gar3int222}
\end{minipage}
\end{enumerate} 
\includegraphics[width=4 cm]{ifunk1222}
\includegraphics[width=4 cm]{ifunk2222}
\includegraphics[width=4 cm]{ifunk3222}
\vspace{0.5 cm}

\fbox{
	\begin{minipage}{0.5\textwidth}
		Zur Lösung bitte \href{https://www.okuyakl.de/math/m12intapL222/ll222.pdf}{hier klicken} oder den QR-Code scannen.\\
	Weitere Arbeitsblätter gibt es unter 
	
	\href{https://www.okuyakl.de}{www.okuyakl.de}
	\end{minipage}
	\hfill
	\begin{minipage}{0.4\textwidth}
		\includegraphics[width=1.5 cm]{../../viecher/zwe03}
		\includegraphics[width=3 cm]{qr222}
		\includegraphics[width=2 cm]{../../viecher/afanticon1}
		
	\end{minipage}}
\end{document}
