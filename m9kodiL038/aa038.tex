\documentclass[a4paper]{article}
\usepackage[pdftex]{graphicx}
\usepackage[utf8]{inputenc}
\usepackage{enumerate}
\usepackage{icomma}
\usepackage{amssymb}
\usepackage{geometry}
\geometry{a4paper, top=15mm, left=15mm, right=15mm, bottom=15mm,
	headsep=10mm, footskip=12mm}
\usepackage{href-ul}
\hypersetup{
	colorlinks=true,
	linkcolor=blue,
	urlcolor=blue}
\begin{document}
{\bf Flächen im Koordinatensystem}
\begin{enumerate}[1.]

%Aufgabe 1
{\bf \item Bestimme rechnerisch den Flächeninhalt des Dreiecks} $ABC$ mit $A(-2|3), \\
B(2,5|-3)$ und $C(6|7,5)$.

{\bf \item Von dem Parallelogramm $AB_nCD_n$ kennt man die Eckpunkte $A(2|3)$ und $C(10|7)$.} Punkte $D_n$ liegen auf der Geraden $g$ mit der Gleichung \\ 
$ y= - 0,5x + 5 $. {\it Hinweis:} Die Koordinaten von $D_n$ sind dann $D_n(x|-0,5x+5)$

\begin{enumerate}[a)]
\item Bestimme den Flächeninhalt der Parallelogramme in Abhängigkeit von $x$.\\
$[ A(x) = ( -8x + 24 ) ~{\rm FE} ]$
\item Berechne die Koordinaten des Punktes $D_n$, wenn das Parallelogramm einen Flächeninhalt von 16 FE besitzt
\end{enumerate}

{\bf \item Ein Dreieck $ABC$ mit $A(2|2)$, $B(x_B|4)$ und $C(10|8)$ hat einen Flächeninhalt von 
19 FE.}\\ 
Berechne $x_B$ .

{\bf \item Durch die Punkte $A(6|3)$, $B_n \in g: \quad y= 2x$ und $C_n \in h:\quad y=-x+3 $ ist eine Schar von Dreiecken $AB_nC_n$ festgelegt.}
\begin{enumerate}[a)]
	\item Zeichne die Geraden g und h sowie das Dreieck $AB_1C_1$ für $x=2$ ein. Platzbedarf: $-3<x<7$ und $-3<y<7$.
	\item Zeige, dass für den Flächeninhalt $A(x)$ der Schardreiecke in Abhängigkeit der Abszisse x gilt: 
	$$A(x)=-1,5x^2+10,5x-9 ~{\rm FE}$$
	\item Berechne die Punkte $B_2$ und $C_2$ des Dreiecks $AB_2C_2$ mit dem maximalen Flächeninhalt.
\end{enumerate}

{\bf \item Punkte $C_n$ auf der Geraden } $g:\quad y=-0,4x+4,9$ sind Eckpunkte von Rechtecken $OB_nC_nD_n$, deren Seiten auf den Koordinatenachsen liegen. Der untere linke Eckpunkt eines solchen Rechtecks ist also der Ursprung. 
\begin{enumerate}[a)]
	\item Gib den Flächeninhalt der Rechtecke $OB_nC_nD_n$ in Abhängigkeit von $x$ an.
	\item Für welche Belegung von $x$ erhalten wir ein Rechteck $OB_1C_1D_1$ mit maximaler Fläche? Gib den maximalen Flächeninhalt an.
	\item Für welche Belegung von $x$ erhalten wir ein Quadrat? Zeichne die Gerade und das Quadrat $OB_2C_2D_2$ in ein Koordinatensystem.
	\item Rechts neben dem Quadrat $OB_2C_2D_2$ soll noch ein zweites Quadrat $B_2EC_3F$ unter der Geraden $g$ Platz finden. 
	Bestimme rechnerisch die Koordinaten von $C_3$ und zeichne auch dieses Quadrat ein.
\end{enumerate}

\end{enumerate} 

\fbox{
	\begin{minipage}{0.5\textwidth}
		Zur Lösung bitte \href{https://www.okuyakl.de/math/m9kodiL038/ll038.pdf}{hier klicken} oder den QR-Code scannen.\\
	Weitere Arbeitsblätter gibt es unter 
	
	\href{https://www.okuyakl.de}{www.okuyakl.de}
	\end{minipage}
	\hfill
	\begin{minipage}{0.4\textwidth}
		\includegraphics[width=1.5 cm]{../../viecher/zwe03}
		\includegraphics[width=3 cm]{qr038}
		\includegraphics[width=2 cm]{../../viecher/afanticon1}
		
	\end{minipage}}

\end{document}%Lösung-------------------------------------------
