\documentclass[a4paper]{article}
\usepackage[pdftex]{graphicx}
\usepackage[utf8]{inputenc}
\usepackage{enumerate}
\usepackage{amssymb}
\usepackage{href-ul}
\hypersetup{
	colorlinks=true,
	linkcolor=blue,
	urlcolor=blue}
\usepackage{geometry}
\geometry{a4paper, top=15mm, left=15mm, right=15mm, bottom=15mm,
	headsep=10mm, footskip=12mm}

\begin{document}
{\bf De Morgan'sche Regel, Venn-Diagramm}
\begin{enumerate}[1.]
%Aufgabe 1
{\bf \item Eine Hochschule bietet unter anderem die beiden Studiengänge Betriebs\-wirtschaftslehre (BWL) und Sozialpädagogik an.} Für eine Datenerhebung werden folgende Ereignisse betrachtet:\\
M: ,,Ein beliebig herausgegriffener Studienanfänger ist männlich" \\
B: ,,Ein beliebig herausgegriffener Studienanfänger studiert BWL" 

\begin{enumerate}[a)]
\item Beschreiben Sie das Ereignis $E_1=\overline{M} \cup B $ mit Worten.
\item Kennzeichnen Sie in einem VENN-Diagramm das Ereignis $E_2=\overline{\overline{B} \cup M}$.
Beschreiben Sie dieses Ereignis mit Worten.
\end{enumerate}

%Aufgabe 2
{\bf \item Ein Betrieb fertigt Kunststoffgehäuse. Die Qualität der Gehäuse wird durch regelmäßige Kontrollen am Ende der Fertigung auf Materialfehler und Formabweichungen überprüft.} Die Qualitätskontrolle wird somit durch zwei Ereignisse definiert:\\
M:,,Das geprüfte Gehäuse hat einen Materialfehler"\\
F: ,,Das geprüfte Gehäuse hat eine Formabweichung"

\begin{enumerate}[a)]
\item Beschreiben Sie mit Worten das Ereignis $E_3= F \cup M $.
\item Beschreiben Sie mit der Ereignisalgebra das Ereignis: ,,In der Qualitäts\-kon\-trolle tritt höchstens ein Fehler auf".
\item Kennzeichnen Sie in einem VENN-Diagramm das Ereignis 
$E_4=\overline{F \cap \overline{M}}$. Beschreiben Sie dieses Ereignis mit Worten.
\end{enumerate}


{\bf \item  $A$ und $B$ seien Ereignisse aus einem Ergebnisraum $\Omega$ eines Zufallsexperimentes. Es gelte:}
$$ P(A)=\frac{3}{4} ; \hspace{15 pt} P(\bar{B})=\frac{2}{3} ; \hspace{15 pt} P(A \cap B)= \frac{1}{6} \, . $$ 
Geben Sie die folgenden Ereignisse in der Mengenschreibweise an und berechnen Sie jeweils ihre Wahrscheinlichkeiten:
\begin{enumerate}[a)]
\item $E_1$: Mindestens eines der beiden Ereignisse tritt ein.
\item $E_2$: Höchstens eines der beiden Ereignisse tritt ein.
\end{enumerate}

\end{enumerate} 

\fbox{
	\begin{minipage}{0.5\textwidth}
		Zur Lösung bitte \href{https://www.okuyakl.de/math/m11stoL065/ll065.pdf}{hier klicken} oder den QR-Code scannen.\\
	Weitere Arbeitsblätter gibt es unter 
	
	\href{https://www.okuyakl.de}{www.okuyakl.de}
	\end{minipage}
	\hfill
	\begin{minipage}{0.4\textwidth}
		\includegraphics[width=1.5 cm]{../../viecher/zwe03}
		\includegraphics[width=3 cm]{qr065}
		\includegraphics[width=2 cm]{../../viecher/afanticon1}
		
	\end{minipage}}

\end{document}%Lösung-------------------------------------------
