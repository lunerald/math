\documentclass[a4paper]{article}
% Uncomment the following line to allow the usage of graphics (.png, .jpg)
\usepackage[pdftex]{graphicx}
% Allow the usage of utf8 characters
\usepackage[utf8]{inputenc}
\usepackage{enumerate}
\usepackage{amssymb}
\usepackage{eurosym}
\usepackage{icomma}
\usepackage{siunitx}
\sisetup{                     
	locale=DE,   } 
\usepackage{geometry}
\usepackage{href-ul}
\hypersetup{
	colorlinks=true,
	linkcolor=blue,
	urlcolor=blue}
\geometry{a4paper, top=15mm, left=15mm, right=15mm, bottom=15mm,
	headsep=10mm, footskip=12mm}

\begin{document}
{\bf Bruchteile, Brüche erweitern und kürzen}
\begin{enumerate}[1.]
\item Ergänze die fehlenden Zahlen:

\renewcommand{\arraystretch}{3}
\begin{tabular}{p{2.2 cm}p{2.2 cm}p{2.2 cm}p{2.2 cm}p{2.2 cm}p{2.2 cm}}
a) ${1\over 4} = { \over 12}$ &b) ${3 \over 9} = { 1 \over }$ &c) ${ \over 32}= { 3 \over 4}$ &d) ${14 \over }= {7 \over 8}$ &e) ${60 \over 100}= { \over 5}$ &f) ${27 \over 33}= {9 \over }$  \\
g) ${ \over 50}= {2 \over 5}$ &h) ${2 \over 11}= { 8 \over }$ &i) ${36 \over }= {6 \over 7}$ &j) ${9 \over 10}= { \over 110}$ &k) ${45 \over 105}= { \over 7}$ &l) ${8 \over 12}= {6 \over }$ \\
\end{tabular}

\item Wieviel sind...

\renewcommand{\arraystretch}{3}
\begin{tabular}{p{3.2 cm}p{3.2 cm}p{3.2 cm}p{3.2 cm}}
a) ${3 \over 8}$ von $\SI{160}{\kilogram}$	&b) ${1 \over 2}$ von $\SI{1,2}{\tonne}$ &c) ${6 \over 7}$ von \EUR{210} &d) ${3 \over 5}$ von $\SI{20}{\meter}$ \\
\end{tabular}

\item Wieviel ist das Ganze?

\renewcommand{\arraystretch}{3}
\begin{tabular}{p{3.2 cm}p{3.2 cm}p{3.2 cm}p{3.2 cm}}
a) ${1 \over 4}$ sind $\SI{22}{\centi\meter}$ &b) ${3 \over 2}$ sind $\SI{180}{\kilogram}$ &c) ${5 \over 11}$ sind \EUR{40} &d) ${3 \over 100}$ sind $\SI{12}{\kilo\meter}$ \\
\end{tabular}

\item Fülle die Lücken aus:

\renewcommand{\arraystretch}{3}
\begin{tabular}{p{4.3 cm}p{4.3 cm}p{4.3 cm}}
a)${4 \over 5}$ von 200 sind \rule{0.7 cm}{0.5 pt}	&b) ${2 \over 9}$ von \rule{0.7 cm}{0.5 pt} sind 40 &c) ${ \over 8}$ von 32 sind 24\\
d) ${9 \over }$ von 110 sind 99	&e) ${8 \over 3}$ von 12 sind  \rule{0.7 cm}{0.5 pt} &f) ${5 \over 6}$ von  \rule{0.7 cm}{0.5 pt} sind 225\\
g)${ \over 2}$ von 10 sind 45	&h) ${7 \over }$ von 400 sind 140&i) ${1 \over 12}$ von 72 sind \rule{0.7 cm}{0.5 pt}
\end{tabular}

\end{enumerate} 

\fbox{
	\begin{minipage}{0.5\textwidth}
	Zur Lösung bitte \href{https://www.okuyakl.de/math/m6bruL057/ll057.pdf}{hier klicken} oder den QR-Code scannen.\\
	Weitere Arbeitsblätter gibt es unter 
	
	\href{https://www.okuyakl.de}{www.okuyakl.de}
	\end{minipage}
	\hfill
	\begin{minipage}{0.4\textwidth}
		\includegraphics[width=1.5 cm]{../../viecher/zwe03}
		\includegraphics[width=3 cm]{qr057}
		\includegraphics[width=2 cm]{../../viecher/afanticon1}
		
	\end{minipage}}

\end{document}
