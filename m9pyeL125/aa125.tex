\documentclass[a4paper]{article}
\usepackage[pdftex]{graphicx}
\usepackage[utf8]{inputenc}
\usepackage{enumerate}
\usepackage{amssymb}
\usepackage{colortbl}
\usepackage{icomma}
\usepackage{siunitx}
\sisetup{locale=DE} 
\usepackage{geometry}
\geometry{a4paper, top=15mm, left=15mm, right=15mm, bottom=15mm,
	headsep=10mm, footskip=12mm}
\usepackage{href-ul}
\hypersetup{
	colorlinks=true,
	linkcolor=blue,
	urlcolor=blue}
\begin{document}
{\bf Berechne die fehlenden Größen des rechtwinkligen Dreiecks. Alle Angaben sind in Zentimetern }\\

\begin{minipage}{0.5\textwidth}
    \includegraphics{rwdreieck125}
   \end{minipage}
\hfill
\begin{minipage}{0.4\textwidth}
	
$$
\renewcommand{\arraystretch}{2}
\begin{array}{rl}
\textnormal{Satz des Pythagoras}: & a^2+b^2=c^2\\
\textnormal{Kathetensatz}: & a^2=c\cdot p \\
		     & b^2=c\cdot q \\
\textnormal{Höhensatz} & h^2 = p \cdot q
\end{array}
$$
\end{minipage}

\renewcommand{\arraystretch}{2}
\begin{tabular}{|>{\columncolor[gray]{.8}}p{1 cm}|p{2 cm}|p{2 cm}|p{2 cm}|p{2 cm}|p{2 cm}|p{2 cm}|}
	\hline
	\rowcolor[gray]{.8}	&$a$ &$b$ &$c$ &$p$ &$q$ & $h$\\
	\hline
	a) & 3 & 4 & & & &  \\
	\hline
	b) & 5 & & 13 & & &  \\
	\hline
	c) & 2,5 & & &1,5 & & \\
	\hline
	*d)& 4 & & & &7 &  \\
	\hline
	e) &6,5 & & & & & 6 \\
	\hline
	f)  & &8 &17 & & & \\
	\hline
	*g) & &7,5 & & 2& & \\
	\hline
	h) & &25 & & &24 & \\
	\hline
	i)  & &29 & & & &20 \\
	\hline
	j) & & &13 &4 & & \\
	\hline
	k) & & &30 & & 6 & \\
	\hline
	l)  & & &10 & & & 5\\
	\hline
	m) & & & & 16 & 4 & \\
	\hline
	n) & & & &13,5 & & 9\\
	\hline
	o)  & & & & &36 & 12\\
	\hline
\end{tabular}


\fbox{
	\begin{minipage}{0.5\textwidth}
		Zur Lösung bitte \href{https://www.okuyakl.de/math/m9pyeL125/ll125.pdf}{hier klicken} oder den QR-Code scannen.\\
		Weitere Arbeitsblätter gibt es unter 
		
		\href{https://www.okuyakl.de}{www.okuyakl.de}
	\end{minipage}
	\hfill
	\begin{minipage}{0.4\textwidth}
		\includegraphics[width=1.5 cm]{../../viecher/zwe03}
		\includegraphics[width=3 cm]{qr125}
		\includegraphics[width=2 cm]{../../viecher/afanticon1}
		
\end{minipage}}

\end{document}

