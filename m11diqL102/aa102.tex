\documentclass[a4paper]{article}
\usepackage[pdftex]{graphicx}
\usepackage[utf8]{inputenc}
\usepackage{enumerate}
\usepackage{icomma}
\usepackage{amssymb}
\usepackage{href-ul}
\hypersetup{
	colorlinks=true,
	linkcolor=blue,
	urlcolor=blue}
\usepackage{geometry}
\geometry{a4paper, top=15mm, left=15mm, right=15mm, bottom=15mm,
	headsep=10mm, footskip=12mm}

\begin{document}
	
{\bf	Berechnen von Tangenten an Funktionsgraphen}

\begin{enumerate}[1.]
\item Gegeben sei die Funktion $f(x)=-x^2+1; \qquad \mathbb{D}=\mathbb{R}$; ihr Graph ist $G_f$.
Berechnen Sie jeweils eine Gleichung der Tangente $t$ an $G_f$ im Punkt

\renewcommand{\arraystretch}{2}
\begin{tabular}{cccc}
a) $P(0|f(0))$	& b) $P(1|f(1))$ & c) $P(-2|f(-2))$ & d) $P(0,5|f(0,5))$ \\
\end{tabular}

\item Gegeben sei die Funktion $f(x)={2\over x}; \qquad \mathbb{D}=\mathbb{R}\setminus \{0\}$; ihr Graph ist $G_f$.
Berechnen Sie jeweils eine Gleichung der Tangente $t$ an $G_f$ im Punkt

\renewcommand{\arraystretch}{2}
\begin{tabular}{cccc}
	a) $P(1|f(1))$	& b) $P(-1|f(-1))$ & c) $P(2|f(2))$ & d) $P(0,5|f(0,5))$ \\
\end{tabular}

\item Gegeben sei die Funktion $f(x)=x^3-1; \qquad \mathbb{D}=\mathbb{R}$; ihr Graph ist $G_f$.
Berechnen Sie jeweils eine Gleichung der Tangente $t$ an $G_f$ im Punkt

\renewcommand{\arraystretch}{2}
\begin{tabular}{cccc}
	a) $P(1|f(1))$	& b) $P(-2|f(-2))$ & c) $P(0,5|f(0,5))$ & d) $P(0|f(0))$ \\
\end{tabular}

\item Gegeben sei die Funktion $f(x)={1\over 4}x^4; \qquad \mathbb{D}=\mathbb{R}$; ihr Graph ist $G_f$.
Berechnen Sie jeweils eine Gleichung der Tangente $t$ an $G_f$ im Punkt

\renewcommand{\arraystretch}{2}
\begin{tabular}{cccc}
	a) $P(0|f(0))$	& b) $P(1|f(1))$ & c) $P(2|f(2))$ & d) $P(0,5|f(0,5))$ \\
\end{tabular}

\vspace{1 cm}

\fbox{Anleitung:}

\fbox{

\begin{minipage}{0.45\textwidth}
	Mit Differenzialquotienten:
\begin{itemize}
	\item y-Koordinate von P bestimmen ($=f(x_0$))
	\item Den Differenzialquotienten aufstellen
	\item Differenzialquotienten umformen und kürzen
	\item Grenzwert ist gleich Tangentensteigung
	\item Gleichung der Tangente $y=mx+t$ aufstellen 
\end{itemize}
\end{minipage}}
\hfill
\fbox{
\begin{minipage}{0.45\textwidth}
    Mit Ableitungsfunktion:
\begin{itemize}
    \item Den Funktionsterm ableiten
    \item Die x-Koordinate in die Ableitungsfunktion $f'(x)$\\ einsetzen $\Rightarrow$ Tangentensteigung
    \item y-Koordinate von P bestimmen ($=f(x_0$))
    \item Gleichung der Tangente $y=mx+t$ aufstellen 
\end{itemize}

\end{minipage}}
    
\end{enumerate} 

\fbox{
	\begin{minipage}{0.5\textwidth}
		Zur Lösung bitte \href{https://www.okuyakl.de/math/m11diqL102/ll102.pdf}{hier klicken} oder den QR-Code scannen.\\
	Weitere Arbeitsblätter gibt es unter 
	
	\href{https://www.okuyakl.de}{www.okuyakl.de}
	\end{minipage}
	\hfill
	\begin{minipage}{0.4\textwidth}
		\includegraphics[width=1.5 cm]{../../viecher/zwe03}
		\includegraphics[width=3 cm]{qr102}
		\includegraphics[width=2 cm]{../../viecher/afanticon1}
		
\end{minipage}}
\end{document}