\documentclass[a4paper]{article}
% Uncomment the following line to allow the usage of graphics (.png, .jpg)
\usepackage[pdftex]{graphicx}
% Allow the usage of utf8 characters
\usepackage[utf8]{inputenc}
\usepackage{enumerate}
\usepackage{amssymb}
\usepackage{geometry}
\usepackage{href-ul}
\hypersetup{
	colorlinks=true,
	linkcolor=blue,
	urlcolor=blue}
\geometry{a4paper, top=15mm, left=15mm, right=15mm, bottom=15mm,
	headsep=10mm, footskip=12mm}

\begin{document}
{\bf Trainingsblatt Potenzen}
\begin{enumerate}[1.]
{\bf \item Nenne alle Primzahlen zwischen 1 und 20} 

{\bf \item Zerlege vollständig in Primfaktoren. Beispiel: $140=2^2\cdot 5 \cdot 7$}

\renewcommand{\arraystretch}{2}
\begin{tabular}{p{3 cm}p{3 cm}p{3 cm}p{3 cm}p{3 cm}}
	a) 12 & b) 50 & c) 216 & d) 72 & e) 440\\
	f) 153 & g) 294 & h) 325 & i) 101 & j) 909 
\end{tabular}

{\bf \item Bilde das kleinste gemeinsame Vielfache (kgV). Beispiel: kgV(4;6)=12}

\renewcommand{\arraystretch}{2}
\begin{tabular}{p{5 cm}p{5 cm}p{5 cm}}
	a) 6; 8 & b) 9; 12 & c) 20; 5 \\
	d) 8; 18 & e) 10; 35 & f) 11; 17 \\
	g) 3; 12; 8 & h) 4; 14; 6 & i) 24; 9; 2  
\end{tabular}

{\bf \item Finde den größten gemeinsamen Teiler (ggT). Beispiel: ggT(24;16)=8}

\begin{tabular}{p{5 cm }p{5 cm}p{5 cm}}
	a) 96; 80 & b) 165; 105  & c) 207; 333 \\
	d) 72; 200 & e) 132; 99 & f) 227;17 \\
	g) 42; 98; 28 & h) 45; 105; 75 & i) 144; 108; 18
\end{tabular}

{\bf \item Schreibe alle Zweierpotenzen von $2^0$ bis $2^{10}$ auf:}

\renewcommand{\arraystretch}{2}
\begin{tabular}{|p{1.1 cm}|p{1.1 cm}|p{1.1 cm}|p{1.1 cm}|p{1.1 cm}|p{1.1 cm}|p{1.1 cm}|p{1.1 cm}|p{1.1 cm}|p{1.1 cm}|p{1.1 cm}|}
	\hline
	$2^{10}$ & $2^9$ & $2^8$ & $2^7$ & $2^6$ & $2^5$ & $2^4$ & $2^3$ & $2^2$ & $2^1$ & $2^0$ \\ 
	\hline
	 & & & & & & & & & & \\
	 \hline
	
\end{tabular}

{\bf \item Übersetze Dezimal- und Binärzahlen durch Ausfüllen folgender Tabelle }

\renewcommand{\arraystretch}{2}
\begin{tabular}{|p{1.8cm}||p{0.8 cm}|p{0.8 cm}|p{0.8 cm}|p{0.8 cm}|p{0.8 cm}|p{0.8 cm}|p{0.8 cm}|p{0.8 cm}|p{0.8 cm}|p{0.8 cm}||p{1.8 cm}|}
	\hline
 Dezimalzahl & $2^9$ & $2^8$ & $2^7$ & $2^6$ & $2^5$ & $2^4$ & $2^3$ & $2^2$ & $2^1$ & $2^0$ & Binärzahl\\ 
 \hline
 168 & & & 1 & 0 & 1 & 0 & 1 & 0 & 0& 0 & 10101000\\
 \hline
 24 & & &  &  &  &  &  &  & & & \\
 \hline
   & & &  &  &  &  &  &  & & & 1101111000\\
 \hline
 666 & & &  &  &  &  &  &  & & & \\
 \hline
  1000 & & &  &  &  &  &  &  & & & \\
 \hline
     & & &  &  &  &  &  &  & & & 1000\\ 
 \hline
 227 & & &  &  &  &  &  &  & & & \\ 
 \hline  
\end{tabular}



\end{enumerate} 
\fbox{
	\begin{minipage}{0.5\textwidth}
		Zur Lösung bitte \href{https://www.okuyakl.de/math/m5pozL050/ll050.pdf}{hier klicken} oder den QR-Code scannen.\\
		Weitere Arbeitsblätter gibt es unter 
		
	\href{https://www.okuyakl.de}{www.okuyakl.de}
	\end{minipage}
	\hfill
	\begin{minipage}{0.4\textwidth}
		\includegraphics[width=1.5 cm]{../../viecher/zwe03}
		\includegraphics[width=3 cm]{qr050}
		\includegraphics[width=2 cm]{../../viecher/afanticon1}
		
	\end{minipage}}
\end{document}
