
\documentclass[a4paper]{article}
\usepackage[pdftex]{graphicx}
\usepackage[utf8]{inputenc}
\usepackage{enumerate}
\usepackage{icomma}
\usepackage{siunitx}
\sisetup{locale=DE} 
\usepackage{amssymb}
\usepackage{geometry}
\geometry{a4paper, top=15mm, left=15mm, right=15mm, bottom=15mm,
	headsep=10mm, footskip=12mm}
\usepackage{href-ul}
\hypersetup{
	colorlinks=true,
	linkcolor=blue,
	urlcolor=blue}
\begin{document}
%Titel
{\bf Trainingsblatt Kugelvolumen }\\
\begin{enumerate}[1.]

%Aufgabe 1
{\bf \item Kann man eine Styroporkugel mit einem Radius von $\SI{1}{\meter}$ alleine tragen?} Berechnen Sie die Masse unter der Annahme, dass $\SI{1}{\deci\meter}^3$ Styropor $\SI{30}{\gram}$ wiegt. 

%Aufgabe 2
{\bf \item Berechnen Sie den Durchmesser einer Kugel mit folgendem Volumen:} \\
a) $V=\SI{1}{\meter}^3$ \hspace{30 pt} b) $V=\SI{49}{\centi\meter}^3$

%Aufgabe 3
\begin{minipage}{0.75\textwidth}
{\bf \item Drei Tennisbälle befinden sich in einer zylinderförmigen Verpackung.} Dabei hat die Dose den gleichen Durchmesser $d$ wie ein Ball. Wieviel Prozent des Volumens entfällt auf die Zwischenräume? 
\end{minipage}
\hfill
\begin{minipage}{0.2\textwidth}
\includegraphics[width=4cm]{tendos217}
\end{minipage}

%Aufgabe 4
{\bf \item Eine Kugel, ein Zylinder und der Grundkreis eines Kegels haben denselben Radius.} Berechnen Sie die Höhe des Zylinders und des Kegels so, dass alle drei Körper gleiches Volumen haben.

%Aufgabe 5
{\bf \item Aus einem tropfenden Wasserhahn fällt alle 2 Sekunden ein (fast kugelförmiger) Tropfen mit einem Durchmesser von $\SI{4}{\milli\meter}$.} Wie viel Liter Wasser gehen dadurch im Laufe einer Woche verloren?

%Aufgabe 6
{\bf \item Der Radius eines kugelförmigen Ballons nimmt beim höher Steigen um 10\% zu.} Um wieviel Prozent nimmt  das Volumen zu?

%Aufgabe 7
{\bf \item Wie groß ist der Anteil des Kugelvolumens einer einbeschriebenen Kugel am Volumen des Würfels (in \%)?}

%Aufgabe 8
\begin{minipage}{0.7\textwidth}
{\bf \item  Die Wassermenge der Erde  beträgt etwa $ \SI{1,4e9}{\kilo\meter^3}$.}
\begin{enumerate}[a)]
	\item Wie tief wäre der weltumspannende Ozean, wenn die gesamte Erdoberfläche eben wäre ($r_{Erde}=\SI{6370}{\kilo\meter}$)?
    \item Der Jupitermond Europa ($r=\SI{1560}{\kilo\meter}$) ist etwas kleiner als der Erdmond,  enthält aber in seiner äußeren Hülle rund die doppelte Wassermenge der Erde; siehe Bild (Quelle: Wikipedia). Wie tief ist der dort vermutete Ozean einschließlich Eiskruste?
\end{enumerate}
\end{minipage}
\hspace{0.5 cm}
\begin{minipage}{0.3\textwidth}
	\includegraphics[width=4cm]{euro217}
\end{minipage}


{\bf \item Aufgabe}

\begin{enumerate}[a)]
	\item Zwei Kugeln aus Metall (Radius $r=1,5~a$) werden eingeschmolzen und damit eine neue Kugel gegossen. Berechne den Radius der neuen Kugel. (Ersatzergebnis: $2~a$)
	\item Wie verhält sich die Oberfläche der neuen Kugel zu der Gesamtoberfläche der beiden ursprünglichen Kugeln? Gib das Ergebnis in Prozent an.
\end{enumerate}

%Aufgabe 2
{\bf \item Eine Orange mit einem Außendurchmesser von $\SI{8}{\centi\meter}$ und einer Schalendicke von $\SI{6}{\milli\meter}$ soll ausgepresst werden.} Gehe davon aus, dass $70\%$ des Fruchtfleisches in Saft umgewandelt werden können. Wie viel Liter Saft ergeben 5 derartige Orangen?


%Aufgabe
\begin{minipage}{0.7\textwidth}
{\bf \item Kann ein Golfball schwimmen? Ein Golfball hat eine Masse von 45,9 g und einen Durchmesser von 42,7 mm.}
\begin{enumerate}[a)]
\item Berechne das Volumen und die Oberfläche dieses nahezu kugelförmigen Golfballs.
\item Berechne mithilfe des Volumens aus a) die Dichte des Balls und vergleiche sie mit der Dichte von Wasser.
\item Schwimmt (bei einem Fehlschlag) der Golfball im See eines Golfplatzes oder geht er unter?
\end{enumerate}
\end{minipage}
\begin{minipage}{0.3\textwidth}
\includegraphics[width=4 cm]{golf248}
\end{minipage}

\end{enumerate}

\fbox{
	\begin{minipage}{0.5\textwidth}
		Zur Lösung bitte \href{https://www.okuyakl.de/math/m10kuvolL217/ll217.pdf}{hier klicken} oder den QR-Code scannen.\\
	Weitere Arbeitsblätter gibt es unter 
	
	\href{https://www.okuyakl.de}{www.okuyakl.de}
	\end{minipage}
	\hfill
	\begin{minipage}{0.4\textwidth}
		\includegraphics[width=1.5 cm]{../../viecher/zwe03}
		\includegraphics[width=3 cm]{qr217}
		\includegraphics[width=2 cm]{../../viecher/afanticon1}
		
	\end{minipage}}
\end{document}%Lösung------------------------------------------------
