\documentclass[a4paper]{article}
\usepackage[pdftex]{graphicx}
\usepackage[utf8]{inputenc}
\usepackage{enumerate}
\usepackage{amssymb}
\usepackage{colortbl}
\usepackage{icomma}
\usepackage{siunitx}
\sisetup{locale=DE} 
\usepackage{geometry}
\geometry{a4paper, top=15mm, left=15mm, right=15mm, bottom=15mm,
	headsep=10mm, footskip=12mm}
\usepackage{href-ul}
\hypersetup{
	colorlinks=true,
	linkcolor=blue,
	urlcolor=blue}
\begin{document}
\begin{enumerate}[1.]
{\bf \item Gegeben sind die Punkte $O(0|0)$, $A(2|0)$, $B(6|-1)$. Es soll ein Quadrat rechnerisch konstruiert werden.}

\begin{enumerate}[a)]

\item Berechne den Vektor $\overrightarrow{AB}$ 
\item Stelle einen Vektor auf, der orthogonal zu $\overrightarrow{AB}$ ist und die gleiche Länge hat
\item Berechne damit einen möglichen Punkt $C$
\item Ergänze einen Punkt $D$ sodass das Quadrat ABCD vollständig wird.
\item Zeichne das Quadrat in das Schaubild und vergleiche Deine Ergebnisse.

\begin{minipage}{0.4\textwidth}
\item Stelle einen Funktionsterm für die Gerade $AB$ in Parameterform auf
\item Wandle die Parameterform in die Normalenform um
\item Welchen Winkel bildet die Gerade $AB$ mit der x-Achse? 
\item Bilde eine zu $AB$ senkrechte Hilfsgerade durch den Ursprung.
\item Bestimme den Lotfußpukt L des Ursprungs auf der Geraden $AB$
\item Bestimme den Abstand $|\overrightarrow{OL}|$des Ursprungs von der Geraden $AB$
\end{minipage}
\hfill
\begin{minipage}{0.5\textwidth}

   \includegraphics[width=8 cm]{core128}

\end{minipage}
\end{enumerate} 

\item Es sei M der Mittelpunkt des Quadrates ABCD. Bilde folgende Vektoren: 

\renewcommand{\arraystretch}{2}
\begin{tabular}{p{5 cm}p{5 cm}p{5 cm}}
	
	$\overrightarrow{AC}$	&	$\overrightarrow{AD}$	&	$\overrightarrow{AM}$	\\
	$\overrightarrow{OA}$	&	$\overrightarrow{BM}$	&	$\overrightarrow{BD}$	
	
\end{tabular}

\item Bilde folgende Skalarprodukte und interpretiere das Ergebnis:

\renewcommand{\arraystretch}{2}
\begin{tabular}{p{5 cm}p{5 cm}p{5 cm}}
	
$\overrightarrow{AC} \odot \overrightarrow{AD}$ & $\overrightarrow{AC}\odot \overrightarrow{AM}$ & $\overrightarrow{AC} \odot \overrightarrow{OA}$  \\
$\overrightarrow{AB} \odot \overrightarrow{AD}$ & $\overrightarrow{AM}\odot \overrightarrow{BM}$ & $\overrightarrow{AC} \odot \overrightarrow{BD}$  \\
\end{tabular}

\item Löse folgende Gleichungen: 

\renewcommand{\arraystretch}{2}
\begin{tabular}{p{5 cm}p{5 cm}p{5 cm}}
	
	${2 \choose -1} \odot {1 \choose c}=0$	& ${4\choose 2,5}\odot { c \choose 4}=0$&$ {0,1 \choose c}\odot {5 \choose 0.2}=0$\\
	
\end{tabular}

\end{enumerate} 

\fbox{
	\begin{minipage}{0.5\textwidth}
		Weitere Arbeitsblätter gibt es unter 
		
		\href{https://www.okuyakl.de}{www.okuyakl.de}
	\end{minipage}
	\hfill
	\begin{minipage}{0.4\textwidth}
		\includegraphics[width=1.5 cm]{../../viecher/zwe03}
		%\includegraphics[width=3 cm]{qr128}
		\includegraphics[width=2 cm]{../../viecher/afanticon1}
		
\end{minipage}}

\end{document}

