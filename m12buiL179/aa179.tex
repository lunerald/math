\documentclass[a4paper]{article}
\usepackage[pdftex]{graphicx}
\usepackage[utf8]{inputenc}
\usepackage{enumerate}
\usepackage{icomma}
\usepackage{amssymb}
\usepackage{href-ul}
\hypersetup{
	colorlinks=true,
	linkcolor=blue,
	urlcolor=blue}
\usepackage{geometry}
\geometry{a4paper, top=15mm, left=15mm, right=15mm, bottom=15mm,
	headsep=10mm, footskip=12mm}

\begin{document}
{\bf Bernoulli-Ketten und Binomialverteilung}

\begin{enumerate}[1.]
{\bf \item Eine Werbeaktion verspricht:},,in jedem siebten Überraschungsei ist eine Star-Wars-Figur". Sie öffnen sieben Überraschungseier. Wie groß ist die Wahrscheinlichkeit, dass Sie...
\begin{enumerate}[a)]
	\item  ...genau eine Star-Wars-Figur dabei finden ?
	\item  ...keine Star-Wars-Figur dabei finden ?
	\item  ...mindestens eine Star-Wars-Figur dabei finden ?
	\item  ...erst im siebten Ei eine Star-Wars-Figur finden?
	\item  ...zwei Star-Wars-Figuren dabei finden ?
	\item  ...genau zwei Star-Wars-Figur dabei finden und diese folgen direkt hintereinander?
\end{enumerate} 
{\bf \item Zwei Drittel der Senioren in Deutschland besitzen ein Mobiltelefon.} 
Bestimmen Sie die Wahrscheinlichkeit dafür, dass unter 30 zufällig ausgewählten
Senioren mindestens 17 und höchstens 23 ein
Mobiltelefon besitzen.

{\bf \item In einer Urne befinden sich vier rote und sechs blaue Kugeln.} Aus dieser
wird achtmal eine Kugel zufällig gezogen, die Farbe notiert und die Kugel
anschließend wieder zurückgelegt.
\begin{enumerate}[a)]
	\item Geben Sie einen Term an, mit dem die Wahrscheinlichkeit des Ereignisses
,,Es werden gleich viele rote und blaue Kugeln gezogen." berechnet
	werden kann.
	\item Beschreiben Sie im Sachzusammenhang jeweils ein Ereignis, dessen
	Wahrscheinlichkeit durch den angegebenen Term berechnet werden
	kann.

\renewcommand{\arraystretch}{2}
\begin{tabular}{p{6 cm}p{6 cm}}
i) $1-\left({3\over 5}\right)^8$ & ii) $\left({3\over 5}\right)^8 + 8\cdot {2\over 5}\cdot \left({3\over 5}\right)^7$ \\
\end{tabular}

\end{enumerate} 

{\bf \item Einer Studie zufolge besitzen 55\% der 16-jährigen Schüler ein Fernsehgerät.}
Geben Sie den Wert der Summe
$$\sum_{i=0}^{12} B(25; 0,55; i)$$ in Prozent an. Beschreiben Sie im Sachzusammenhang ein Ereignis, dessen
Wahrscheinlichkeit durch den angegebenen Term berechnet werden
kann.. 

{\bf \item In einer Umfrage haben ${2\over3}$ der Befragten ein Rauchverbot in Restaurants befürwortet}
Wie groß ist die Wahrscheinlichkeit, dass sich unter 10 zufällig
ausgewählten Personen höchstens 5 Befürworter befinden? 

{\bf \item Erwartungsgemäß treten 5\% der Flugreisenden ihre Reise nicht an}. Wie groß ist das Risiko einer Überbuchung, wenn bei einem Flugzeug mit 195 Sitzplätzen 200 Flugtickets verkauft werden ?

\begin{minipage}{0.7\textwidth}
	{\bf \item 10 Personen drehen nacheinander je einmal das
		Glücksrad.} Der Buchstabe A bedeutet den Gewinn einer vergoldeten Schwimmente.
	Bestimmen Sie die Wahrscheinlichkeiten folgender Ereignisse:
	\begin{enumerate}[a)]
		\item $E_1$: Genau 3 Personen erhalten je einen Gewinn.
		\item $E_2$: Mindestens 3 Personen erhalten einen Gewinn.
		\item $E_3$: 3 aufeinander folgende Personen erhalten einen Gewinn, die restlichen Personen erhalten nichts.
	\end{enumerate} 
	
\end{minipage}
\hfill
\begin{minipage}{0.3\textwidth}
	\includegraphics[width=5 cm]{glrad179}		
\end{minipage}

\end{enumerate} 
\fbox{
	\begin{minipage}{0.5\textwidth}
		Zur Lösung bitte \href{https://www.okuyakl.de/math/m12buiL179/ll179.pdf}{hier klicken} oder den QR-Code scannen.\\
	Weitere Arbeitsblätter gibt es unter 
	
	\href{https://www.okuyakl.de}{www.okuyakl.de}
	\end{minipage}
	\hfill
	\begin{minipage}{0.4\textwidth}
		\includegraphics[width=1.5 cm]{../../viecher/zwe03}
		\includegraphics[width=3 cm]{qr179}
		\includegraphics[width=2 cm]{../../viecher/afanticon1}
		
\end{minipage}}

\end{document}

