\documentclass[a4paper]{article}
\usepackage[pdftex]{graphicx}
\usepackage[utf8]{inputenc}
\usepackage{enumerate}
\usepackage{amssymb}
\usepackage{icomma}
\usepackage{siunitx}
\sisetup{                     
	locale=DE,   } 
\usepackage{geometry}
\geometry{a4paper, top=15mm, left=15mm, right=15mm, bottom=15mm,
	headsep=10mm, footskip=12mm}
\usepackage{href-ul}
\hypersetup{
	colorlinks=true,
	linkcolor=blue,
	urlcolor=blue}
	
\begin{document}

\begin{enumerate}[1.]
{\bf Geometrische Konstruktionen}\\

\begin{minipage}{0.5\textwidth}
{\bf \item Aufgabe }
\begin{enumerate}[a)]
\item Konstruiere die Winkelhalbierende des Winkels $\angle{CAD}$.
\item Ergänze das Dreieck $ACD$ durch Konstruktion so, dass ein Rechteck $ABCD$ entsteht.
\item Spiegele den Punkt $D$ an der Achse $AC$.
\end{enumerate}
\end{minipage}
\begin{minipage}{0.5\textwidth}
\includegraphics[width=5 cm]{drei249}
\end{minipage}

{\bf \item Der Platzwart des Fußballvereins hat vor einem wichtigen Spiel vergessen, den Elfmeterpunkt zu markieren.} Finde durch Konstruktion heraus, wo der Elfmeterpunkt gesetzt werden muss und markiere die Stelle farbig.

\includegraphics[width=12 cm]{tor249}

{\bf \item Wie lautet der Kongruenzsatz SsW in Worten?}

{\bf \item Untersuche und gib an, ob die nachfolgenden Dreiecke $\Delta A_1B_1C_1 $}und 
$ \Delta A_2B_2C_2 $ jeweils zueinander kongruent sind und begründe deine Antwort kurz. (Tipp: Planfigur)
$$ a_2 = b_1 ; \quad  \alpha_1 = \gamma_2 = \SI{70}{\degree}; \quad \beta_1 = \SI{50}{\degree} \quad \beta_2 = \SI{60}{\degree} $$

{\bf \item Begründe, ob mit folgenden Bestimmungsstücken ein Dreieck} $ABC$ konstruierbar ist und gib, falls möglich, eine Konstruktionsbeschreibung an!

\begin{enumerate}[a)]
\item $$ a= \SI{7}{\centi\meter}; \quad c=\SI{6}{\centi\meter}; \quad \gamma =\SI{100}{\degree}$$
\item $$ b= \SI{6,5}{\centi\meter}; \quad a= \SI{3,7}{\centi\meter}; \quad c = \SI{7}{\centi\meter}$$
\end{enumerate}

{\bf \item Beschreibe mit Worten, wie man den Mittelpunkt des Inkreises eines Dreiecks} und den zugehörigen Radius bestimmt.

{\bf \item Trage die Punkte $A(-2|1)$, $B(0|-1)$ und $C(5|0)$ in ein Koordinatensystem ein ($-4 < x <7 ; \quad -3 < y < 4 $)}

\begin{enumerate}[a)]
\item Konstruiere den Punkt $D$ so, dass das Viereck $ABCD$ ein Drachenviereck ist.
\item Konstruiere den Punkt $E$ so, dass das Viereck $ABCE$ ein gleich\-schenk\-liges Trapez ist.
\end{enumerate}

{\bf \item Beurteile die folgenden Aussagen mit richtig oder falsch!} Gib bei den falschen Aussagen ein Gegenbeispiel an.

\begin{enumerate}[a)]
\item Bei allen punktsymmetrischen Vierecken ergänzen sich benachbarte Winkel zu $180^\circ$.
\item Hat ein Viereck drei gleich große Winkel, so ist es ein Quadrat.
\item Sind bei einem Viereck die gegenüberliegenden Seiten jeweils gleich lang, so ist es achsensymmetrisch.
\item Sind bei einem Viereck die Diagonalen gleich lang, so ist es punktsymmetrisch.
\end{enumerate} 


{\bf \item Zeichne ein Koordinatensystem mit $-6<x<6 $ und  $-6<y<6 $.} Bearbeite dann folgende Aufgaben:
\begin{enumerate}[a)]
\item Zeichne in das Koordinatensystem die Punkte $A(-1|-1)$, $B(3|1)$, $C(2|3)$ sowie die Strecken $[AB]$ und $[CB]$ ein. Bezeichne den Schnittpunkt von  $[AB]$ mit der x-Achse mit 
,,$M$".
\item Zeichne die Gerade $g$ ein, die durch $C$ verläuft und zu $[AB]$ parallel ist. Zeichne die Gerade $h$ ein, die senkrecht auf $g$ steht und durch $A$ verläuft. 
\item Der Schnittpunkt von $g$ und $h$ heißt $D$. Gib die Koordinaten von $D$ an. Durch welche Quadranten verläuft die Strecke $[DA]$ ?
\item Ermittle den Abstand des Punktes $B$ von der Geraden $MC$.
\item Zeichne den Kreis um $M$, der durch den Punkt $D$ geht, ein.
\item Spiegele den Punkt $D$ an der Geraden $MC$. Gib die Koordinaten des Spiegelpunktes $D'$ an und erkläre die besondere Lage von $D'$ 
\end{enumerate}

\fbox{
	\begin{minipage}{0.5\textwidth}
		Zur Lösung bitte \href{https://www.okuyakl.de/math/m8dreiL053/ll053.pdf}{hier klicken} oder den QR-Code scannen.\\
	Weitere Arbeitsblätter gibt es unter 
	
	\href{https://www.okuyakl.de}{www.okuyakl.de}
	\end{minipage}
	\hfill
	\begin{minipage}{0.4\textwidth}
		\includegraphics[width=1.5 cm]{../../viecher/zwe03}
		\includegraphics[width=3 cm]{qr053}
		\includegraphics[width=2 cm]{../../viecher/afanticon1}
		
	\end{minipage}}

\end{enumerate} 

\end{document}%Lösung-------------------------------------------
