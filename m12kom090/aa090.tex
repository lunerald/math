\documentclass[a4paper]{article}
\usepackage[pdftex]{graphicx}
\usepackage[utf8]{inputenc}
\usepackage{enumerate}
\usepackage{mdwlist}
\usepackage{icomma}
\usepackage{amssymb}
\usepackage{href-ul}
\hypersetup{
	colorlinks=true,
	linkcolor=blue,
	urlcolor=blue}
\usepackage{geometry}
\geometry{a4paper, top=15mm, left=15mm, right=15mm, bottom=15mm,
	headsep=10mm, footskip=12mm}

\begin{document}
{\bf Kombinatorik, zum Teil Abituraufgaben } - {\it Die Lösungsvorschläge stammen aus eigener Berechnung }

\begin{enumerate}[1.]
{\bf \item Auf einer Silvesterparty sind 10 Gäste}. Um 12 Uhr wird mit Sekt angestoßen, dabei stößt jeder mit jedem einmal an. Wie oft erklingen dabei die Gläser? [45]
{\bf \item Wie viele mögliche Flug-Direkt-Verbindungen } gibt es zwischen den Hauptstädten aller 28 EU-Länder? [378]
{\bf \item Im Tierheim werden drei Hunde fotografiert:} ein Husky, ein Jack-Russel und eine Bulldogge. Anschließend werden die drei Fotos nebeneinander auf der Webseite veröffentlicht.
\begin{enumerate}[a)]
	\item Wie viele Anordnungen der Fotos sind möglich? [6]
	\item Wie viele Möglichkeiten gibt es, wenn jeder Hund entweder die Augen zu oder auf hat? [48]
\end{enumerate} 

{\bf \item Lunerald hat einen Geldkoffer gefunden. Leider ist er mit zwei je dreistelligen Zahlenschlössern verschlossen.}
\begin{enumerate}[a)]
\item Wie viele Kombinationen gibt es zum durchprobieren? [2000]
\item Wie viele Kombinationen gibt es, wenn die sechs Stellen ein Geburtsdatum sind (TT.MM.JJ {\it zur Vereinfachung gibt es keine Schaltjahre})? [36500]
\end{enumerate} 
{\bf \item In einer Fernsehshow} sitzen die 7 Kandidaten, 4 Frauen und 3 Männer, in einer Reihe. Wie viele Sitzanordnungen gibt es, wenn hinsichtlich der Personen unterschieden wird und 
\begin{enumerate}[a)]
\item die beiden Randplätze von Männern besetzt werden sollen, [720]
\item sich in der Reihe Männer und Frauen stets abwechseln sollen? [144]
\end{enumerate} 
{\bf \item Ein Spiel wird mit einer “Glückswand” durchgeführt.} Diese besteht aus 20 Feldern, auf die zufällig fünfmal die Zahl 200, viermal die Zahl 500 und dreimal die Zahl 1000 verteilt werden. Die übrigen Felder bleiben leer. Wie viele derartige Verteilungen gibt es? [3491888400]
{\bf \item In einem Kaufhaus sollen auf Grund verlängerter Ladenschlusszeiten } 12 neue Mitarbeiter eingestellt werden.
 In Abteilung A sind 5 Stellen zu besetzen, in Abteilung B 7 Stellen. Für Abteilung A bewerben sich 8 und für Abteilung B 10 Personen. Wie viele Möglichkeiten gibt es, die offenen Stellen zu besetzen, wenn die Stellen innerhalb jeder Abteilung 
	\begin{enumerate}[a)]
		\item nicht unterschieden werden, [56; 120]
		\item als verschieden angesehen werden? [6720; 604800]
	\suspend{enumerate} 
	 Bei der Begrüßung sitzen die 12 neuen Mitarbeiter, 8 Frauen und 4 Männer, in zwei Reihen mit je 6 Stühlen. Wie viele Sitzanordnungen gibt es, wenn nur nach Frauen und Männern unterschieden wird, und
	\resume{enumerate}
		\item in jeder Reihe zwei Männer sitzen, [225]
		\item die 4 Männer nebeneinander sitzen? [6]
	\end{enumerate}  

{\bf \item Ein Gärtner pflanzt 10 Tulpenzwiebeln} der rot blühenden Sorte und 10 Zwiebeln der gelb blühenden Sorte in zwei Reihen mit 8 und 12 Zwiebeln. Wie viele Möglichkeiten der Bepflanzung gibt es, wenn sich in einer der Reihen genau 4 Zwiebeln der gelb blühenden Tulpensorte befinden sollen? [78540]
{\bf \item Zum Jahrtausendwechsel hat die Firmenchefin, eine Hobbymathematikerin,} unter ihren Mitarbeitern ein Preisrätsel veranstaltet. Beantworten Sie die beiden dort gestellten Fragen: 
\begin{enumerate}[a)]
	\item Auf wie viele Arten kann man die Primfaktoren in der Primfaktordarstellung der Zahl 2000 anordnen? [35]
	\item Wie viele verschiedene Teiler hat die Zahl 2000? [20]
\end{enumerate} 

{\bf \item Vroni hat zu ihrer Geburtstagsparty} 3 Freundinnen und 4 Freunde eingeladen.
Für eine Pantomime werden aus den 8 Jugendlichen auf zufällige Weise
4 ausgewählt.
\begin{enumerate}[a)]
	\item  Wie viele Gruppenmöglichkeiten gibt es, wenn Vroni und
	Peter zusammen in der ausgewählten Gruppe sein sollen? [15] 
\suspend{enumerate} 


Auf der Tanzfläche tanzen nur Paare aus jeweils einem Mädchen und
einem Jungen. Wie viele verschiedene Zusammenstellungen der Paare
auf der Tanzfläche gibt es, wenn
\resume{enumerate}
	\item alle 8 Teilnehmer der Party mittanzen, [24]
	\item von den jungen Männern nur Max und Peter tanzen? [12]
\end{enumerate} 

\newpage
{\bf \item Ein Moderator lädt zu seiner Talkshow drei Politiker}, eine Journalistin und
zwei Mitglieder einer Bürgerinitiative ein. Für die Diskussionsrunde ist eine
halbkreisförmige Sitzordnung vorgesehen, bei der nach den Personen
unterschieden wird und der Moderator den mittleren Platz einnimmt.
\begin{enumerate}[a)]
	\item Berechnen Sie die Anzahl der möglichen Sitzordnungen, wenn keine weiteren Einschränkungen	berücksichtigt werden. [720]
	\item  Der Sender hat festgelegt, dass unmittelbar neben dem Moderator auf
	einer Seite die Journalistin und auf der anderen Seite einer der Politiker
	sitzen soll. Berechnen Sie unter Berücksichtigung dieser weiteren Einschränkung
	die Anzahl der möglichen Sitzordnungen. [144]
\end{enumerate} 

{\bf \item Der Kurs Theater und Film eines Gymnasiums} führt die Bühnenversion eines
	Romans auf.
    Für die Premiere wird die Aula der Schule bestuhlt; in der ersten Reihe werden
	acht Plätze für Ehrengäste reserviert. Bestimmen Sie die Anzahl der
	Möglichkeiten, die die fünf erschienenen Ehrengäste haben, sich auf die reservierten
	Plätze zu verteilen, wenn
	\begin{enumerate}[a)]
		\item die Personen nicht unterschieden werden; [56]
		\item die Personen unterschieden werden. [6720]
	\end{enumerate} 

{\bf \item An einem Musikwettbewerb nehmen zwölf Nachwuchsbands} aus ganz
	Deutschland teil. Genau zwei davon stammen aus Bayern. Die eine Hälfte der Bands singt ausschließlich Lieder
	mit englischen Texten, die andere ausschließlich Lieder mit deutschen
	Texten. In der ersten Runde des Wettbewerbs treten die zwölf Bands nacheinander
	mit jeweils einem Lied auf; die Reihenfolge der Auftritte wird ausgelost.
	Nach den ersten sechs Auftritten findet eine Pause statt.
	\begin{enumerate}[a)]
		\item Wie viele Möglichkeiten gibt es für die Reihenfolge der Auftritte? [479001600]
		\item Wie viele Möglichkeiten gibt es für die Reihenfolge der Auftritte,
		wenn nur danach unterschieden wird, ob eine Band aus Bayern stammt
		oder nicht? [66]
	\item  Berechnen Sie die Wahrscheinlichkeiten der folgenden Ereignisse.
 \begin{enumerate}[A]
 	\item ,,Die beiden bayerischen Bands treten vor der Pause auf.“ [0,227]
 	\item ,,Die beiden bayerischen Bands treten vor der Pause direkt nacheinander auf.“ [0,07575]
 	\item ,,Deutsch und englisch singende Bands treten abwechselnd auf.“ [0,00217]
 \end{enumerate} 
 \end{enumerate} 

{\bf \item An einem P-Seminar nehmen acht Mädchen und sechs Jungen teil}, darunter
Anna und Tobias. Für eine Präsentation wird per Los aus den Teilnehmerinnen
und Teilnehmern ein Team aus vier Personen zusammengestellt.
\begin{enumerate}[a)]
	\item Geben Sie zu jedem der folgenden Ereignisse einen Term an, mit dem
	die Wahrscheinlichkeit des Ereignisses berechnet werden kann.
	\begin{enumerate}[A]
		\item ,,Anna und Tobias gehören dem Team an.“ [0,066]
		\item ,,Das Team besteht aus gleich vielen Mädchen und Jungen.“ [0,420]
	\end{enumerate} 
	\item Beschreiben Sie im Sachzusammenhang ein Ereignis, dessen Wahrscheinlichkeit
	durch den folgenden Term berechnet werden kann: $ {14 \choose 4} - {6 \choose 4} \over {14 \choose 4}$
\end{enumerate} 
{\bf \item Ein Achterbahnzug besteht aus 12 Wagen,} davon sind vier rot, fünf gelb und drei blau.
\begin{enumerate}[a)]
	\item Wie viele verschiedene Wagenreihungen sind möglich? [27720]
	\item Wie viele verschiedene Wagenreihungen sind möglich, wenn die drei blauen Wagen direkt hintereinander fahren? [1260]
\end{enumerate} 

{\bf \item Die Klasse 7a eines Gymnasiums fährt ins Skilager.} Alle 11 Jungen und alle
	18 Mädchen der Klasse nehmen an dem einwöchigen Skikurs teil.
	In der Unterkunft stehen für die 18 Mädchen der Klasse ein Sechsbett-,
	ein Fünfbett-, ein Vierbett- und ein Dreibettzimmer zur Verfügung.
	\begin{enumerate}[a)]
		\item Wie viele verschiedene Möglichkeiten gibt es, die Mädchen so auf die vier
		Zimmer zu verteilen, dass jedes Zimmer voll besetzt ist? [514594080]
	\suspend{enumerate}
	Bei einem Abfahrtslauf der fortgeschrittenen Skifahrerinnen und
	Skifahrer der Klasse 7a werden die Startnummern von 1 bis 20 zufällig
	von den 20 Teilnehmern gezogen. Unter ihnen sind die 3 Mädchen aus
	dem Dreibettzimmer.
	Wie groß ist die Wahrscheinlichkeit, dass diese 3 Mädchen
	\resume{enumerate}
		\item unter den ersten zehn Startern sind? [0,105]
		\item aufeinander folgende Startnummern ziehen? [0,00263]
	\end{enumerate} 
	
	


\end{enumerate} 

\fbox{
	\begin{minipage}{0.5\textwidth}
		Weitere Arbeitsblätter gibt es unter 
		
		\href{https://www.okuyakl.de}{www.okuyakl.de}
	\end{minipage}
	\hfill
	\begin{minipage}{0.4\textwidth}
		\includegraphics[width=1.5 cm]{../../viecher/zwe03}
		%\includegraphics[width=3 cm]{qr128}
		\includegraphics[width=2 cm]{../../viecher/afanticon1}
		
\end{minipage}}

\end{document}

