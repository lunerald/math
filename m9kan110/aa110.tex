\documentclass[a4paper]{article}
\usepackage[pdftex]{graphicx}
\usepackage[utf8]{inputenc}
\usepackage{enumerate}
\usepackage{amssymb}
\usepackage{colortbl}
\usepackage{icomma}
\usepackage{siunitx}
\sisetup{locale=DE} 
\usepackage{geometry}
\geometry{a4paper, top=15mm, left=15mm, right=15mm, bottom=15mm,
	headsep=10mm, footskip=12mm}
\usepackage{href-ul}
\hypersetup{
	colorlinks=true,
	linkcolor=blue,
	urlcolor=blue}
\begin{document}
	
\begin{minipage}{0.5\textwidth}
		
{\bf Das Kaninchenproblem}
\vspace{1 cm}

Es soll auf einer Wiese an einem Fluss ein möglichst großes rechteckiges Kaninchengehege abgesteckt werden. Die Zaun\-länge beträgt $ \SI{20}{\meter}$; am Fluss entlang verläuft kein Zaun\-stück.
\vspace{0.5 cm}

\noindent{\bf Schritt 1:} \\Beschrifte die Seiten des Zauns mit sinnvollen Variablennamen.
\vspace{1.5 cm}

\end{minipage}
\hfill
\begin{minipage}{0.4\textwidth}
	\includegraphics[width=8 cm]{kani110}
\end{minipage}
\vspace{1 cm}


\noindent{\bf Schritt 2:}\\ Erstelle eine Zielfunktion. Hier ist es die Rechtecksfläche. Wie lautet die Formel dafür?
\vspace{1.5 cm}

\noindent{\bf Schritt 3:}\\ Formuliere eine Nebenbedingung. Hier ist es die Zaunlänge. Wie setzt sie sich zusammen? Formuliere eine Gleichung. 
 \vspace{1.5 cm}
 
\noindent{\bf Schritt 4:}\\ Löse die Nebenbedingung nach einer Variablen auf.
\vspace{1.5 cm}

\noindent{\bf Schritt 5:}\\ Setze die aufgelöste Nebenbedingung in die Zielfunktion ein. Du erhältst dann einen Funktionsterm, der nur noch eine Variable enthält. 
\vspace{1.5 cm}

\noindent{\bf Schritt 6:}\\ Multipliziere den Funktionsterm aus und bringe ihn auf eine quadratische Form.
\vspace{1.5 cm}

\noindent{\bf Schritt 7:}\\Wir suchen den Extremwert dieses Terms. Führe hierfür eine quadratische Ergänzung durch
\vspace{1.5 cm}

\noindent{\bf Schritt 8:}\\ Interpretiere das Ergebnis.
\vspace{1.5 cm}


\fbox{
	\begin{minipage}{0.5\textwidth}
		Weitere Arbeitsblätter gibt es unter 
		
		\href{https://www.okuyakl.de}{www.okuyakl.de}
	\end{minipage}
	\hfill
	\begin{minipage}{0.4\textwidth}
		\includegraphics[width=1.5 cm]{../../viecher/zwe03}
	%	\includegraphics[width=3 cm]{qr221}
		\includegraphics[width=2 cm]{../../viecher/afanticon1}
		
\end{minipage}}

\end{document}

