\documentclass[a4paper]{article}
\usepackage[pdftex]{graphicx}
\usepackage[utf8]{inputenc}
\usepackage{enumerate}
\usepackage{amssymb}
\usepackage{href-ul}
\hypersetup{
	colorlinks=true,
	linkcolor=blue,
	urlcolor=blue}
\usepackage{geometry}
\geometry{a4paper, top=15mm, left=15mm, right=15mm, bottom=15mm,
	headsep=10mm, footskip=12mm}

\begin{document}
{\bf Gebrochenrationale Funktionen}
\begin{enumerate}[1.]

%Aufgabe 3
{\bf \item Bestimme die Gleichung einer gebrochen rationalen Funktion, die die folgenden drei Bedingungen erfüllt:}

\begin{itemize}
\item ihr Graph schneidet die x-Achse bei $x=2$,
\item hat bei  $x=3$ einen Pol ohne Vorzeichenwechsel und
\item $y=3$ ist waagrechte Asymptote.
\end{itemize}
%Aufgabe 2
{\bf \item Gegeben ist die Funktion}
 $f(x)= { 1 \over x} + 3 ; \quad D_f=\mathbb{R}\setminus \{0\}$;\\
Berechne mit Hilfe des Differentialquotienten die Ableitung im Punkt $P(2|?)$.

{\bf \item Gegeben ist die Funktion $f(x) = \frac{x^3}{x^2-4} $. }

\begin{enumerate}[a)]
\item Ermitteln Sie die Definitionsmenge $ D_f $ und geben Sie - mit Hilfe geeigneter Rechnung oder Begründung - die Art der Definitionslücken an!
\item Untersuchen Sie die Funktion auf Symmetrie.
\item Bestimmen Sie die Gleichungen aller Asymptoten.
\item Zeigen Sie, dass für die Ableitung gilt : $ f'(x) =  \frac{x^4-12x^2}{{(x^2-4)}^2} $ 
\item Bestimmen Sie die Gleichung der Tangente an $ G_f$ im Punkt $ P (4|?) $. Unter welchem Winkel schneidet die Tangente die x-Achse?
\end{enumerate}

%Aufgabe 1
\begin{minipage}{0.55\textwidth}
{\bf \item Rechts ist der Graph einer gebrochen rationalen Funktion $f$ abgebildet.}
\begin{enumerate}[a)]
\item Geben Sie die Nullstellen, Definitionslücken und die Asymptoten der Funktion $f$ an.
\item Finden Sie durch Überlegen einen geeigneten Funktionsterm.
\item Weisen Sie mit Hilfe des in b) gefundenen Funktionsterms nach, dass der Graph von $f$ achsensymmetrisch zur y-Achse ist.
\end{enumerate} 
\end{minipage}
\hfill
\begin{minipage}{0.4\textwidth}
\includegraphics[width=5 cm]{gebra061}
\end{minipage}


%Aufgabe 2
{\bf \item Gegeben ist die Funktion $$f(x)= {x^2-9 \over 4x^4+4x^3-24x^2}$$}
 
\begin{enumerate}[a)]
\item Bestimmen Sie die maximale Definitionsmenge und geben Sie sämtliche Nullstellen von $f$ an.
\item Untersuchen Sie rechnerisch oder durch geeignete Überlegungen, um welche Art von Definitionslücken es sich handelt.
\item Skizzieren Sie $G_f$ unter Verwendung Ihrer bisherigen Ergebnisse.
\end{enumerate}

%Aufgabe 3
{\bf \item Gegeben ist die Funktion $$ f(x)= 1- {2 \over x}+{1 \over x^2}$$}

\begin{enumerate}[a)]
\item Berechnen Sie die Nullstellen der Funktion $f$, und stellen Sie die Gleichungen aller Asymptoten auf. Bestimmen Sie das Verhalten von $f$ in der Umgebung von $x=0$.
\item Skizzieren Sie den Graphen der Funktion.
\end{enumerate}

%Aufgabe 4
{\bf \item Gegeben ist die Funktion $$h(x)= {x^2+3x \over 4(x-1)}$$}

\begin{enumerate}[a)]
\item Bestimmen Sie die Definitionsmenge, sowie Art und Lage der Nullstellen von $h$ an.
\item Führen Sie alle nötigen Grenzwertbetrachtungen durch, damit Sie den Graphen von $h$ mit allen seinen Asymptoten skizzieren können.  
\end{enumerate}
\fbox{
	\begin{minipage}{0.5\textwidth}
		Zur Lösung bitte \href{https://www.okuyakl.de/math/m11gbrL061/ll061.pdf}{hier klicken} oder den QR-Code scannen.\\
	Weitere Arbeitsblätter gibt es unter 
	
	\href{https://www.okuyakl.de}{www.okuyakl.de}
	\end{minipage}
	\hfill
	\begin{minipage}{0.4\textwidth}
		\includegraphics[width=1.5 cm]{../../viecher/zwe03}
		\includegraphics[width=3 cm]{qr061}
		\includegraphics[width=2 cm]{../../viecher/afanticon1}
		
	\end{minipage}}

\end{enumerate} 

\end{document}%Lösung-------------------------------------------
