\documentclass[a4paper]{article}
\usepackage[pdftex]{graphicx}
\usepackage[utf8]{inputenc}
\usepackage{enumerate}
\usepackage{icomma}
\usepackage{amssymb}
\usepackage{geometry}
\geometry{a4paper, top=15mm, left=15mm, right=15mm, bottom=15mm,
	headsep=10mm, footskip=12mm}
\usepackage{href-ul}
\hypersetup{
	colorlinks=true,
	linkcolor=blue,
	urlcolor=blue}
\begin{document}
{\bf Laplace- Experimente}
\begin{enumerate}[1.]
{\bf \item Bei einem Zufallsexperiment wird vierzigmal gewürfelt und die gewürfelten Augenzahlen in der nachfolgenden Tabelle festgehalten.}

\begin{tabular}{c|c|c|c|c|c|c}
Augenzahl & 1 & 2 & 3 & 4 & 5 & 6\\
\hline
Anzahl der Würfe & 7 & 4 & 8 & & 6 & 
\end{tabular}

\begin{enumerate}[a)]
\item Gib die relative Häufigkeit der Augenzahl 3 an.
\item Die relative Häufigkeit der Augenzahl 4 beträgt 12,5\%. Wie oft wurde die 4 gewürfelt?
\item Gib die absolute Häufigkeit der Augenzahl 6 an.
\item Wie hoch ist die relative Häufigkeit für das Ereignis ,, Werfe höchstens eine Drei"? 
\end{enumerate}

{\bf \item Ein Zufallsexperiment besteht darin, dass ein Laplace-Würfel geworfen wird.} Entscheiden Sie jeweils, ob für $\Omega_1=\{1,~2,\dots,~6\}, \quad \Omega_2=\{\textnormal{keine}~6,~6\}$ und 
$\Omega_3=\{\textnormal{gerade Zahl, ungerade Zahl}\}$ ein Laplace- Experiment vorliegt.

{\bf \item Ein Würfel wird geworfen. Der Ergebnisraum ist $ \Omega=\{1,~2,\dots,~6\}$.} Berechnen Sie die Wahrscheinlichkeit folgender Ereignisse:
\begin{enumerate}[A:]
\item ,,Eine 6 wird geworfen"
\item ,,Eine gerade Zahl wird geworfen"
\item ,,Es wird höchstens eine 2 geworfen" 
\end{enumerate}

{\bf \item Das Geschlecht eines neugeborenen Kindes wird als Laplace-Experiment betrachtet, d.h. man geht davon aus, dass $P(w)=P(m)$.} \\
Wie groß ist die Wahrscheinlichkeit, dass eine Famlie mit zwei Kindern mindestens einen Jungen hat?

{\bf \item Ein Würfel wird zweimal geworfen. Mit welcher Wahrscheinlichkeit ist die Augensumme mindestens 10?}

{\bf \item Eine Laplace-Münze wird dreimal geworfen.}
\begin{enumerate}[a)]
\item Veranschaulichen Sie die Ausgänge mit Hilfe eines Baumdiagramms.
\item Geben Sie einen geeigneten Ergebnisraum an.
\item Berechnen Sie die Wahrscheinlichkeit folgender Ereignisse:
\begin{enumerate}[A:]
\item ,,Es erscheint dreimal das gleiche Ergebnis"
\item ,,Es fällt genau zweimal Wappen"
\item ,,Es erscheint höchstens zweimal Zahl"
\item ,,Der zweite Wurf zeigt ein Wappen"
\item ,,Nur der zweite Wurf zeigt ein Wappen"
\end{enumerate}
\end{enumerate} 

\end{enumerate} 

\fbox{
	\begin{minipage}{0.5\textwidth}
		Zur Lösung bitte \href{https://www.okuyakl.de/math/m9zufL059/ll059.pdf}{hier klicken} oder den QR-Code scannen.\\
	Weitere Arbeitsblätter gibt es unter 
	
	\href{https://www.okuyakl.de}{www.okuyakl.de}
	\end{minipage}
	\hfill
	\begin{minipage}{0.4\textwidth}
		\includegraphics[width=1.5 cm]{../../viecher/zwe03}
		\includegraphics[width=3 cm]{qr059}
		\includegraphics[width=2 cm]{../../viecher/afanticon1}
		
	\end{minipage}}
\end{document}%Lösung-------------------------------------------
