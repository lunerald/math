\documentclass[a4paper]{article}
\usepackage[pdftex]{graphicx}
\usepackage[utf8]{inputenc}
\usepackage{enumerate}
\usepackage{amssymb}
\usepackage{href-ul}
\hypersetup{
	colorlinks=true,
	linkcolor=blue,
	urlcolor=blue}
\usepackage{geometry}
\geometry{a4paper, top=12mm, left=15mm, right=15mm, bottom=12mm,
	headsep=10mm, footskip=12mm}

% Start the document
\begin{document}
%Titel
{\bf Trainingsblatt e- Funktion }\\

\noindent{\bf Gegeben sind folgende natürliche Exponentialfunktionen; von oben nach unten nach Asymptoten geordnet:}

\noindent\includegraphics[width=17 cm]{efun022}

%Aufgabe 1
\noindent{\bf 1. Ermitteln Sie die Parameter des jeweiligen Funktionsterms:}
$$ f(x)=\pm e^{\pm(x-b)} + c  $$
\renewcommand{\arraystretch}{2.5}
\begin{tabular}{|p{40 pt}|p{60 pt}|p{60 pt}|p{90 pt}|p{90 pt}|p{100 pt}|}
\hline
  & Exponent & Asymptote & x-Verschiebung & Wertemenge &  Funktionsterm  \\
   &   $+/-$ &  $y=c$ & $b$ & $\mathbb{W}=$  & $f(x)=$   \\
\hline
$f_1(x)$ & & & & &  \\
\hline
$f_2(x)$ & & & & &  \\
\hline
$f_3(x)$ & & & & &  \\
\hline
$f_4(x)$ & & & & &  \\
\hline
$f_5(x)$ & & & & &  \\
\hline
$f_6(x)$ & & & & &  \\
\hline
$f_7(x)$ & & & & &  \\
\hline
\end{tabular}

\fbox{
	\begin{minipage}{0.5\textwidth}
		Zur Lösung bitte \href{https://www.okuyakl.de/math/m11efunL022/ll022.pdf}{hier klicken} oder den QR-Code scannen.\\
	Weitere Arbeitsblätter gibt es unter 
	
	\href{https://www.okuyakl.de}{www.okuyakl.de}
	\end{minipage}
	\hfill
	\begin{minipage}{0.4\textwidth}
		\includegraphics[width=1.5 cm]{../../viecher/zwe03}
		\includegraphics[width=3 cm]{qr022}
		\includegraphics[width=2 cm]{../../viecher/afanticon1}
		
	\end{minipage}}

\end{document}%Lösung--------------------------------------------------------------
