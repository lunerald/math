
\documentclass[a4paper]{article}
\usepackage[pdftex]{graphicx}
\usepackage[utf8]{inputenc}
\usepackage{enumerate}
\usepackage[a4paper]{geometry}
\usepackage{amssymb}
\usepackage{href-ul}
\hypersetup{
	colorlinks=true,
	linkcolor=blue,
	urlcolor=blue}
%Source: img_0146

% Start the document
\begin{document}
%Titel
{\bf Integralberechnungen }\\
\begin{enumerate}[1.]
%Aufgabe 1
{\bf \item Bestimmen Sie das unbestimmte Integral unter Verwendung der Rechenregeln für Integrale:}

\begin{enumerate}[a)]
\begin{minipage}{0.3\textwidth}
\item $ \int (x-2)(x-3)~{\rm d} x =\dots$
\item $ \int {6 x^3 + 2x \over x^2} ~{\rm d} x =\dots$
\end{minipage}
\hfill
\begin{minipage}{0.3\textwidth}
\item $ \int x(x-3)^2~{\rm d} x =\dots$
\item $ \int { 1 \over 2} (1- \sin{x}) ~{\rm d} x =\dots$
\end{minipage}
\hfill
\begin{minipage}{0.3\textwidth}
\item $ \int 3 \ln{(x^2)}~{\rm d} x =\dots$
\item $ \int \sqrt[4]{x^3} ~{\rm d} x =\dots$
\end{minipage}
\end{enumerate}

%Aufgabe 2
{\bf \item Bestimmen Sie, falls möglich, das unbestimmte Integral mit Hilfe der Integrationsregel $\int f'(x) \cdot e^{f(x)}~{\rm d} x = e^{f(x)} + C $.}

\renewcommand{\arraystretch}{3}
\begin{tabular}{l|p{2 cm}|p{2 cm}|p{2 cm}|p{4 cm}|p{6 cm}}
\hline
 & Exponent der e-Funktion & Ableitung des Exponenten & Regel sofort anwendbar? & Ggf. Umformung, damit Regel anwendbar wird & Ergebnis \\
\hline
a) $ \int 2x\cdot e^{x^2+1} ~{\rm d} x$ & $x^2+1$ & & ja & entfällt & \\
\hline
b) $ \int \cos{x} \cdot e^{\sin{x}} ~{\rm d} x$ & & & & & \\
\hline
c) $ \int 6 \cdot e^{3x} ~{\rm d} x$ & & & & $=2 \cdot \int 3 \cdot  e^{3x}~{\rm d} x$ & \\
\hline
d) $ \int x \cdot e^{x+2} ~{\rm d} x$ & & & & & \\
\hline
e) $ \int e^{-x-1} ~{\rm d} x $ & & & & & \\
\hline
\end{tabular}


\newpage
%Aufgabe 2
{\bf \item Bestimmen Sie, falls möglich, das unbestimmte Integral mit Hilfe der Integrationsregel $\int {f'(x) \over f(x)}~{\rm d} x = \ln{|f(x)|} + C $.}

\renewcommand{\arraystretch}{3}
\begin{tabular}{l|p{2 cm}|p{2 cm}|p{2 cm}|p{4 cm}|p{6 cm}}
\hline
 & Nenner des Integranden & Ableitung des Nenners & Regel sofort anwendbar? & Ggf. Umformung, damit Regel anwendbar wird & Ergebnis \\
\hline
a) $ \int {2x \over x^2+3} ~{\rm d} x$ & $x^2+3$ & & ja & entfällt & \\
\hline
b) $ \int {12x-6 \over x^2-x +3} ~{\rm d} x$ & & & & $ =6 \cdot \int {2x-1 \over x^2-x +3} ~{\rm d} x$  & \\
\hline
c) $ \int {2x^2 \over 3 + x^2} ~{\rm d} x$ & & & & & \\
\hline
d) $ \int {\cos{x} \cdot \sin{x} \over (\sin{x})^2 + 7} ~{\rm d} x$ & & & & & \\
\hline
e) $ \int {e^x \over e^x+2} ~{\rm d} x $ & & & & & \\
\hline
\end{tabular}

\newpage
%Aufgabe 3
{\bf\item Bestimmen Sie, falls möglich, das unbestimmte Integral mit Hilfe der Integrationsregel 
$\int  f(ax+b)~{\rm d} x = {1 \over a} F(ax+b)+ C $.}

\renewcommand{\arraystretch}{3}
\begin{tabular}{l|p{2 cm}|p{2 cm}|p{2 cm}|p{8 cm}}
\hline
 & Äußere Funktion $f$ &  Stammfunk\-tion $F$ & Innere Funktion linear? & Berechnung und Ergebnis \\
\hline
a) $ \int 6 \cos{3x} ~{\rm d} x$ & $\cos{x}$ &  $\sin{x}+C$& ja & $= 6 \left( {1 \over 3} \sin{3x} \right) + C$ \\
\hline
b) $ \int \sqrt{2+e^x} ~{\rm d} x$ & & & nein & \\
\hline
c) $ \int \left( {1 \over 2} x - 1\right)^3 ~{\rm d} x$ & & & &  \\
\hline
d) $ \int{\sin{(4x+{\pi\over3})}} ~{\rm d} x$ & & & &  \\
\hline
e) $ \int \sqrt{2x-1}~{\rm d} x $ &  & & & \\
\hline
f) $  \int \ln{(2x+3)}~{\rm d} x $ & & & & \\
\hline
\end{tabular}

\end{enumerate} 

\fbox{
	\begin{minipage}{0.5\textwidth}
		Zur Lösung bitte \href{https://www.okuyakl.de/math/m12iregL189/ll189.pdf}{hier klicken} oder den QR-Code scannen.\\
	Weitere Arbeitsblätter gibt es unter 
	
	\href{https://www.okuyakl.de}{www.okuyakl.de}
	\end{minipage}
	\hfill
	\begin{minipage}{0.4\textwidth}
		\includegraphics[width=1.5 cm]{../../viecher/zwe03}
		\includegraphics[width=3 cm]{qr189}
		\includegraphics[width=2 cm]{../../viecher/afanticon1}
		
	\end{minipage}}
%Ende
\end{document}