\documentclass[a4paper]{article}
\usepackage[pdftex]{graphicx}
\usepackage[utf8]{inputenc}
\usepackage{enumerate}
\usepackage{amssymb}
\usepackage{colortbl}
\usepackage{icomma}
\usepackage{siunitx}
\sisetup{locale=DE} 
\usepackage{geometry}
\geometry{a4paper, top=15mm, left=15mm, right=15mm, bottom=15mm,
	headsep=10mm, footskip=12mm}
\usepackage{href-ul}
\hypersetup{
	colorlinks=true,
	linkcolor=blue,
	urlcolor=blue}
\begin{document}

{\bf Jahresgang der Tageslänge für Ingolstadt} - Modellierung mit der allgemeinen Sinusfunktion

$$f(x)=A \cdot \sin(b(x+c))+d$$
\begin{itemize}
\item $A$ ist die Amplitude. Hier ist x die Zeit in Tagen und A die halbe Differenz der maximalen und der minimalen Tageslänge $f$ in Stunden.
Der längste Tag ist zur Sommersonnenwende am 21 Juni. Sonnenaufgang ist dann um 4h11m, Sonnenuntergang um 20h33m Mitteleuropäische Normalzeit MEZ.
Der kürzeste Tag ist zur Wintersonnenwende am 21 Dezember. Sonnenaufgang ist dann um 8h16m, Sonnenuntergang um 16h20m MEZ.

\item $b$ ist die Wellenzahl. In ihr steckt die Periodenlänge P: $$b ={2 \pi \over P}$$
Wir nehmen zur Vereinfachung an, dass jeder Monat 30 Tage hat und das Jahr P=360 Tage.

\item $c$ ist die Verschiebung der Funktion in x-Richtung. Die Funktion ist verschoben um die negative Anzahl der Tage von Neujahr bis zum Frühlingsanfang am 21. März.


\item $d=12h$ ist die durchschnittliche Tageslänge.

\end{itemize}

\noindent
Stelle aus diesen Angaben einen Funktionsterm für die Tageslänge in Abhängigkeit von der Zeit im Jahresverlauf dar. 
Erstelle dann eine Wertetabelle, überlege eine sinnvolle Skalierung der Achsen und zeichne die Funktion hier ein:
\vspace{0.5 cm}

\includegraphics[width=15 cm]{plot127}

\fbox{
	\begin{minipage}{0.5\textwidth}
		Zur Lösung bitte \href{https://www.okuyakl.de/math/m10jasL127/ll127.pdf}{hier klicken} oder den QR-Code scannen.\\
		Weitere Arbeitsblätter gibt es unter 
		
		\href{https://www.okuyakl.de}{www.okuyakl.de}
	\end{minipage}
	\hfill
	\begin{minipage}{0.4\textwidth}
		\includegraphics[width=1.5 cm]{../../viecher/zwe03}
		\includegraphics[width=3 cm]{qr127}
		\includegraphics[width=2 cm]{../../viecher/afanticon1}
		
\end{minipage}}

\end{document}

