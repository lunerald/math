\documentclass[a4paper]{article}
\usepackage[pdftex]{graphicx}
\usepackage[utf8]{inputenc}
\usepackage{enumerate}
\usepackage{amssymb}
\usepackage{colortbl}
\usepackage{icomma}
\usepackage{siunitx}
\sisetup{locale=DE}
\usepackage{geometry}
\geometry{a4paper, top=15mm, left=15mm, right=15mm, bottom=15mm,
	headsep=10mm, footskip=12mm}
\usepackage{href-ul}
\hypersetup{
	colorlinks=true,
	linkcolor=blue,
	urlcolor=blue}
\begin{document}
	
	{\bf Annual cycle of day length for Ingolstadt} - Modelling with the general sine function
	
	$$f(x)=A \cdot \sin(b(x+c))+d$$
	\begin{itemize}
		\item $A$ is the amplitude. Here x is the time in days and A is half the difference between the maximum and minimum day length $f$ in hours.
		The longest day is the summer solstice on June 21st. Sunrise is then at 4h11m, sunset at 20h33m Central European Standard Time CET.
		The shortest day is the winter solstice on December 21st. Sunrise is then at 8h16m, sunset at 16h20m CET.
		
		\item $b$ is the wave number. It contains the period length P: $$b ={2 \pi \over P}$$
		For simplicity, we assume that each month has 30 days and the year P=360 days.
		
		\item $c$ is the shift of the function in the x direction. The function is shifted by the negative number of days from New Year's Day to the beginning of spring on March 21st.
		
		
		\item $d=12h$ is the average day length.
		
	\end{itemize}
	
	\noindent
	From this information, represent a functional term for the length of the day as a function of time over the course of the year.
	Then create a table of values, think about a sensible scaling of the axes and draw the function here:
	\vspace{0.5 cm}
	
	\includegraphics[width=15 cm]{plot127}
	
	\fbox{
		\begin{minipage}{0.5\textwidth}
			For the solution, please \href{https://www.okuyakl.de/math/m10jasL127/le127.pdf}{click here} or scan the QR code.\\
			Additional worksheets are available at
			
			\href{http://www.okuyakl.com}{www.okuyakl.com}
		\end{minipage}
		\hfill
		\begin{minipage}{0.4\textwidth}
			\includegraphics[width=1.5 cm]{../../viecher/zwe03}
			\includegraphics[width=3 cm]{qre127}
			\includegraphics[width=2 cm]{../../viecher/afanticon1}
			
	\end{minipage}}
	
\end{document}