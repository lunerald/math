\documentclass[a4paper]{article}
\usepackage[ngerman]{babel}
\usepackage{amsmath}
\usepackage{booktabs}
\usepackage[pdftex]{graphicx}
\usepackage[utf8]{inputenc}
\usepackage{enumerate}
\usepackage{icomma}
\usepackage{amssymb}
\usepackage{tikz}
\usepackage{geometry}
\geometry{a4paper, top=15mm, left=15mm, right=15mm, bottom=15mm,
	headsep=10mm, footskip=12mm}
\usepackage{href-ul}
\hypersetup{
	colorlinks=true,
	linkcolor=blue,
	urlcolor=blue}
\begin{document}
{\bf Vierfeldertafel und Baumdiagramm}
\begin{enumerate}[1.]

%Aufgabe 4
{\bf \item Ein Getränkeautomat ist defekt. Wirft man einen Euro ein, so ist die Wahrscheinlichkeit, dass man ein Getränk erhält,} 0,5. Die Wahrscheinlichkeit, dass der Automat ein Getränk und den Euro auswirft, ist ${1 \over 3}$. Die Wahrscheinlichkeit, dass man kein Getränk bekommt und den Euro zurückerhält, ist ${1\over 6}$
\begin{enumerate}[a)]
\item Gib einen Ergebnisraum an und erstelle eine Vierfeldertafel.
\item Wie groß ist die Wahrscheinlichkeit, dass man ein Getränk bekommt und es bezahlt hat?
\item Wie groß ist die Wahrscheinlichkeit, dass man entweder ein Getränk erhält oder seinen Euro zurückbekommt?(Nur eines von beiden)
\item Wie groß ist die Wahrscheinlichkeit, dass man ein Getränk bekommt, wenn der Automat den Euro einbehalten hat?
\item Wie groß ist die Wahrscheinlichkeit, dass der Automat den Euro wieder herausgibt, wenn man ein Getränk erhalten hat?
\end{enumerate}

{\bf \item Bei einem Fernsehsender treten Bildstörungen mit 10 \% Wahrscheinlichkeit auf. In diesem Fall kommt es dann mit 70 \% Wahrscheinlichkeit zu Tonstörungen.} Ist das Bild einwandfrei, so ist mit 95  \% Wahrscheinlichkeit auch der Ton in Ordnung. Berechne die Wahrscheinlichkeit für ein einwandfreies Bild, falls der Ton gestört ist.

{\bf \item Von 50 ärztlich untersuchten Grundschülern haben acht Kinder Haltungsschäden, fünf Kinder Karies und zwei Kinder sowohl Haltungsschäden als auch Karies.}

\begin{enumerate}[a)]
\item Berechnen Sie die relativen Häufigkeiten von \\ 
H: ,,Haltungsschäden" \\ K: ,,Karies"
\item Bestimmen Sie anhand der Vierfeldertafel die relativen Häufigkeiten\\
 $h(\overline{K} \cap \overline{H}),  h( K \cap \overline{H})$ und $ h(K \cup H)$.
\end{enumerate}

{\bf \item Von 60 Studenten einer Sprachschule haben 37 Englisch belegt, 28 Französisch und 17 beide Sprachen.}
\begin{enumerate}[a)]
\item Erstellen Sie je eine Vierfeldertafel für die absoluten und die relativen Häufigkeiten.
\item Bestimmen Sie die relativen Häufigkeiten für das Ereignis\\
 A = ,,Englisch oder Französisch belegt".
\end{enumerate}



	{\bf \item Die Marmeladenmanufaktur}

In einer kleinen Manufaktur werden Marmeladengläser produziert. Die Etikettiermaschine ist ein wenig unzuverlässig.

\begin{itemize}
	\item 80\,\% der Gläser enthalten \textbf{Erdbeermarmelade}, 20\,\% \textbf{Himbeermarmelade}.
	\item Bei Erdbeergläsern wird in 90\,\% der Fälle korrekt „Erdbeere“ etikettiert.
	\item Bei Himbeergläsern ist die Etikettierung in 95\,\% der Fälle korrekt.
\end{itemize}

\begin{enumerate}
	\item Erstelle eine Vierfeldertafel mit den Merkmalen: \textbf{Füllung} (Erdbeere / Himbeere) und \textbf{Etikett} (Erdbeere / Himbeere).
	\item Mit welcher Wahrscheinlichkeit greift jemand zu einem falsch etikettierten Glas?
	\item Wie wahrscheinlich ist es, dass ein Glas mit dem Etikett „Himbeere“ tatsächlich Himbeermarmelade enthält?
\end{enumerate}

	{\bf \item Die Party mit den Socken}

Auf einer WG-Party werden 100 Gäste gefragt, ob sie bunte Socken tragen und ob sie Mathe mögen. Die Ergebnisse:

\begin{itemize}
	\item 60 Personen tragen bunte Socken.
	\item 45 Personen mögen Mathe.
	\item 30 Personen tragen bunte Socken und mögen Mathe.
\end{itemize}

\begin{enumerate}
	\item Erstelle eine Vierfeldertafel mit den Merkmalen: \textbf{Socken} (bunt / nicht bunt) und \textbf{Mathe} (mag / mag nicht).
	\item Wie viele Gäste tragen keine bunten Socken und mögen Mathe nicht?
	\item Sind die Merkmale unabhängig?
\end{enumerate}

\newpage
{\bf \item Die Verschwörung der Würfel – oder doch nicht?}

Zwei Würfel, A und B, werden gleichzeitig geworfen. Man vermutet eine Verschwörung, doch die Wahrscheinlichkeiten sprechen (vielleicht) eine andere Sprache:

\begin{itemize}
	\item Würfel A zeigt in 70\,\% der Fälle eine gerade Zahl.
	\item Würfel B zeigt in 40\,\% der Fälle eine gerade Zahl.
\end{itemize}

Angenommen, die beiden Würfel verhalten sich \textbf{unabhängig} voneinander.

\begin{enumerate}
	\item Berechne die Wahrscheinlichkeit, dass beide Würfel gleichzeitig eine gerade Zahl zeigen.
	\item Erstelle eine Vierfeldertafel mit den Merkmalen: „Würfel A gerade / ungerade“ und „Würfel B gerade / ungerade“.
	\item Wie groß ist die Wahrscheinlichkeit, dass mindestens ein Würfel eine gerade Zahl zeigt?
\end{enumerate}


\end{enumerate} 
\fbox{
	\begin{minipage}{0.5\textwidth}
	Zur Lösung bitte \href{https://www.okuyakl.de/math/m10stoL037/ll037.pdf}{hier klicken} oder den QR-Code scannen.\\
Weitere Arbeitsblätter gibt es unter 

\href{https://www.okuyakl.de}{www.okuyakl.de}
	\end{minipage}
	\hfill
	\begin{minipage}{0.4\textwidth}
		\includegraphics[width=1.5 cm]{../../viecher/zwe03}
		\includegraphics[width=3 cm]{qr037}
		\includegraphics[width=2 cm]{../../viecher/afanticon1}
		
	\end{minipage}}

\end{document}
