\documentclass[a4paper]{article}
\usepackage[pdftex]{graphicx}
\usepackage[utf8]{inputenc}
\usepackage[ngerman]{babel}
\usepackage{amsmath}
\usepackage{booktabs}
\usepackage{enumerate}
\usepackage{icomma}
\usepackage{amssymb}
\usepackage{tikz}
\usepackage{href-ul}
\hypersetup{
	colorlinks=true,
	linkcolor=blue,
	urlcolor=blue}
\usepackage{geometry}
\geometry{a4paper, top=15mm, left=15mm, right=15mm, bottom=15mm,
	headsep=10mm, footskip=12mm}

\begin{document}
\begin{tikzpicture}(10,3)
	\draw[ultra thick](2,0) --(10,0) -- (10,1.3) --(2,1.3) -- (2,0);
	\draw[fill=black](2,0)-- (0,.6) -- (2,.6) -- (2,0);
	\node at (6,.6) {\large Lösungsblatt von \href{https://www.okuyakl.de}{www.okuyakl.de}};
\end{tikzpicture}
\vspace{0.5 cm}

\noindent {\bf Aufgabe 1. a)}\\
Der Ergebnisraum ist (Euro zurück = E; Getränk erhalten = G):
$$\Omega= \{ E \cap G ; E \cap \bar{G}; \bar{E} \cap G ; \bar{E} \cap \bar{G} \}$$
$$
\renewcommand{\arraystretch}{2}
\begin{array}{c|c|c|c}
       &   E        & \bar{E}   & \\
\hline
G      & {1\over 3} & {1\over 6}& {1\over 2} \\
\hline
\bar{G}& {1\over 6} & {1\over 3}& {1\over 2} \\
\hline
       & {1\over 2} & {1\over 2}& 1 
\end{array}
$$
\noindent {\bf Aufgabe 1. b)}\\
Die gesuchte Wahrscheinlichkeit ist 
$$P(G\cap \bar{E})= {1 \over 6}$$
\noindent {\bf Aufgabe 1. c)}\\
Die gesuchte Wahrscheinlichkeit ist für dieses ausschließende ,,Oder":
$$P( G \cap \bar{E})+ P(\bar{G} \cap E) = {1 \over 6 } + {1 \over 6} = { 1\over 3}$$
\noindent{\bf Aufgabe 1. d)}\\
Die gesuchte bedingte Wahrscheinlichkeit ist nach dem Satz von Bayes:
$$P_{\bar{E}}(G)={ P(\bar{E}\cap G) \over P(\bar{E})}= { {1 \over 6} \over {1 \over 2}}= {1 \over 3} $$
\noindent{\bf Aufgabe 1. e)}\\
Analog zu d) gilt:
$$P_G(E)={ P(G \cap E) \over P(G)}= { {1 \over 3} \over {1 \over 2}}= {2 \over 3} $$


\noindent {\bf Aufgabe 2.}\\
B: Bildstörung;  T: Tonstörung. \\
$P(B)=0,1$ \\ $P_B(T)=0,7$\\ $P_{\overline{B}}(\overline{T})=0,95$ \\
Dies ist ein Umkehrproblem, wir berechnen es mit der Formel von Bayes für die bedingte Wahrscheinlichkeit:

$$ P_T(\overline{B})={ P(T \cap \overline{B}) \over P(T)} \qquad(*) $$

Für die gesuchten Wahrscheinlichkeiten erstellen wir eine Vierfeldertafel:
$$ 
\begin{array}{cccc}
P(\overline{T} \cap \overline{B}) &= P(\overline{B}) \cdot P_{\overline{B}}(\overline{T})
&= 0,9 \cdot 0,95&=0,855 \\
 P(T \cap B)&= P(B) \cdot P_B(T)&= 0,1 \cdot 0,7 &= 0,07 
\end{array} 
$$

$$
\renewcommand{\arraystretch}{1.5}
\begin{array}{c|c|c|c}
              & B    & \overline{B} & \\
\hline
          T   & 0,07 & 0,045 & 0,115 \\
\hline
\overline{T}  & 0,03 & 0,855 & 0,885 \\
\hline
              & 0,1  & 0,9  & 1,00
\end{array}
$$

Also ist (*), die gesuchte Wahrscheinlichkeit:
$$ P_T(\overline{B})= {0,045 \over 0,115} = 0,3913 \approx 39\%$$ 

\noindent {\bf Aufgabe 3. a)}\\
$$h(H)={8\over 50}=0,16 \qquad h(K)={5\over 50}=0,10$$

\noindent {\bf Aufgabe 3. b)}\\
$$
\renewcommand{\arraystretch}{1.5}
\begin{array}{c|c|c|c}
&   H  & \overline{H} & \\
\hline
K            & 0,04 & 0,06 & 0,10 \\
\hline
\overline{K} & 0,12 & 0,78 & 0,90 \\
\hline
             & 0,16 & 0,84  & 1,00
\end{array}
$$
Wir lesen ab:
$$h(\overline{K}\cap \overline{H} )= 0,78 \qquad h(K \cap \overline{H})=0,06 \qquad 
h(K \cup H)= 0,12+0,04+0,06=0,22$$


\noindent {\bf Aufgabe 4. a)}\\
\begin{minipage}{0.4\textwidth}
Absolute Häufigkeiten:
$$
\renewcommand{\arraystretch}{1.5}
\begin{array}{c|c|c|c}
&   E  & \overline{E} & \\
\hline
F            & 17 & 11 & 28 \\
\hline
\overline{F} & 20 & 12 & 32 \\
\hline
             & 37 & 23  & 60
\end{array}
$$
\end{minipage}
\begin{minipage}{0.4\textwidth}
Relative Häufigkeiten:
$$
\renewcommand{\arraystretch}{1.5}
\begin{array}{c|c|c|c}
&   E  & \overline{E} & \\
\hline
F            & 0,28 & 0,18 & 0,46 \\
\hline
\overline{F} & 0,34 & 0,20 & 0,54 \\
\hline
             & 0,62 & 0,38 & 1
\end{array}
$$
\end{minipage}

\noindent {\bf Aufgabe 4. b)}\\
$$ h(A)= 1-h(\overline{E}\cap \overline{F})=0,80$$

\noindent {\bf Aufgabe 5}

\begin{center}
	\begin{tabular}{lcc|c}
		& \textbf{Etikett: Erdbeere} & \textbf{Etikett: Himbeere} & \textbf{Summe} \\
		\midrule
		\textbf{Füllung: Erdbeere} & 0{,}80 $\cdot$ 0{,}90 = 0{,}72 & 0{,}80 $\cdot$ 0{,}10 = 0{,}08 & 0{,}80 \\
		\textbf{Füllung: Himbeere} & 0{,}20 $\cdot$ 0{,}05 = 0{,}01 & 0{,}20 $\cdot$ 0{,}95 = 0{,}19 & 0{,}20 \\
		\midrule
		\textbf{Summe} & 0{,}73 & 0{,}27 & 1{,}00
	\end{tabular}
\end{center}

\begin{itemize}
	\item Falsch etikettierte Gläser: $0{,}08 + 0{,}01 = 0{,}09 = \boxed{9\,\%}$
	\item Bedingte Wahrscheinlichkeit:  
	\[
	P(\text{Himbeermarmelade} \mid \text{Etikett Himbeere}) = \frac{0{,}19}{0{,}27} \approx \boxed{70{,}37\,\%}
	\]
\end{itemize}

\noindent {\bf Aufgabe 6}

\begin{center}
	\begin{tabular}{lcc|c}
		& \textbf{Mag Mathe} & \textbf{Mag kein Mathe} & \textbf{Summe} \\
		\midrule
		\textbf{Bunte Socken} & 30 & 60$-$30 = 30 & 60 \\
		\textbf{Keine bunten Socken} & 45$-$30 = 15 & 100$-$30$-$30$-$15 = 25 & 40 \\
		\midrule
		\textbf{Summe} & 45 & 55 & 100
	\end{tabular}
\end{center}

\begin{itemize}
	\item Gäste ohne bunte Socken und ohne Matheliebe: $\boxed{25}$
	\item Unabhängigkeit: Prüfe $P(\text{Mag Mathe} \mid \text{Bunte Socken}) = \frac{30}{60} = 0{,}5$  
	und $P(\text{Mag Mathe}) = \frac{45}{100} = 0{,}45$  
	$\Rightarrow$ Nicht gleich $\Rightarrow$ \textbf{nicht unabhängig}
\end{itemize}


\noindent {\bf Aufgabe 7}

Da die Ereignisse unabhängig sind:
\[
P(A\text{ gerade} \cap B\text{ gerade}) = P(A\text{ gerade}) \cdot P(B\text{ gerade}) = 0{,}70 \cdot 0{,}40 = \boxed{0{,}28}
\]

{\bf Vierfeldertafel:}

\begin{center}
	\begin{tabular}{lcc|c}
		& \textbf{B gerade (0{,}40)} & \textbf{B ungerade (0{,}60)} & \textbf{Summe} \\
		\midrule
		\textbf{A gerade (0{,}70)} & $0{,}28$ & $0{,}70 - 0{,}28 = 0{,}42$ & $0{,}70$ \\
		\textbf{A ungerade (0{,}30)} & $0{,}40 - 0{,}28 = 0{,}12$ & $0{,}30 - 0{,}12 = 0{,}18$ & $0{,}30$ \\
		\midrule
		\textbf{Summe} & $0{,}40$ & $0{,}60$ & $1{,}00$
	\end{tabular}
\end{center}

{\bf  Wahrscheinlichkeit: mindestens ein Würfel zeigt gerade:}

Wir verwenden den Gegenwahrscheinlichkeitsansatz:

\[
P(\text{mindestens einer gerade}) = 1 - P(\text{beide ungerade}) = 1 - 0{,}18 = \boxed{0{,}82}
\]


\begin{center}
	\includegraphics[width=7 cm]{../../viecher/endcomic.pdf}
	
	Hier geht es zurück zum \href{https://www.okuyakl.de/math/m10stoL037/aa037.pdf}{Aufgabenblatt}
\end{center}

\end{document}

