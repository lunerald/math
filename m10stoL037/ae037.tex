\documentclass[a4paper]{article}
\usepackage[ngerman]{babel}
\usepackage{amsmath}
\usepackage{booktabs}
\usepackage[pdftex]{graphicx}
\usepackage[utf8]{inputenc}
\usepackage{enumerate}
\usepackage{icomma}
\usepackage{amssymb}
\usepackage{tikz}
\usepackage{geometry}
\geometry{a4paper, top=15mm, left=15mm, right=15mm, bottom=15mm,
	headsep=10mm, footskip=12mm}
\usepackage{href-ul}
\hypersetup{
	colorlinks=true,
	linkcolor=blue,
	urlcolor=blue}
\begin{document}
	{\bf four-field table and Tree diagram}
	\begin{enumerate}[1.]
		
		%Task 4
		{\bf \item A vending machine is broken. If you insert a euro, the probability that you get a drink is} 0.5. The probability that the machine dispenses a drink and the euro is ${1 \over 3}$. The probability that you don't get a drink and get the euro back is ${1 \over 6}$.
		\begin{enumerate}[a)]
			\item Specify a result space and create a four-field table.
			\item What is the probability that you get a drink and have paid for it?
			\item What is the probability that you either get a drink or get your euro back? (Only one of the two)
			\item What is the probability that you get a drink if the machine has kept the euro?
			\item What is the probability that the machine will dispense the euro after receiving a drink?
		\end{enumerate}
		
		{\bf \item Picture interference occurs on a television channel with a 10 \% probability. In this case, sound interference occurs with a 70 \% probability.} If the picture is perfect, the sound is also fine with a 95 \% probability. Calculate the probability of a perfect picture if the sound is distorted.
		
		{\bf \item Of 50 elementary school students examined by a doctor, eight children have postural deformities, five children have tooth decay, and two children have both postural deformities and tooth decay.}
		
		\begin{enumerate}[a)]
			\item Calculate the relative frequencies of \\
			H: "postural deformities" \\ K: "tooth decay"
			\item Using the four-field table, determine the relative frequencies \\
			$h(\overline{K} \cap \overline{H}), h( K \cap \overline{H})$, and $h(K \cup H)$.
		\end{enumerate}
		
		{\bf \item Of 60 students at a language school, 37 studied English, 28 French, and 17 both languages.}
		\begin{enumerate}[a)]
			\item Create a four-field table for the absolute and relative frequencies.
			\item Determine the relative frequencies for the event\\
			A = "English or French occupied".
		\end{enumerate}
		
		{\bf \item The Jam Factory}
		
		A small factory produces jars of jam. The labeling machine is somewhat unreliable.
		
		\begin{itemize}
			\item 80\,\% of the jars contain \textbf{strawberry jam}, 20\,\% \textbf{raspberry jam}.
			\item Strawberry jars are correctly labeled "strawberry" 90\,\% of the time.
			\item Raspberry jars are correctly labeled "strawberry" 95\,\% of the time.
		\end{itemize}
		
		\begin{enumerate}
			\item Create a four-field table with the attributes: \textbf{filling} (strawberry / raspberry) and \textbf{label} (strawberry / raspberry).
			\item What is the probability that someone will pick up an incorrectly labeled jar?
			\item How likely is it that a jar labeled "raspberry" actually contains raspberry jam?
		\end{enumerate}
		
		{\bf \item The Sock Party}
		
		At a shared apartment party, 100 guests are asked whether they wear colorful socks and whether they like math. The results:
		
		\begin{itemize}
			\item 60 people wear colorful socks.
			\item 45 people like math.
			\item 30 people wear colorful socks and like math.
		\end{itemize}
		
		\begin{enumerate}
			\item Create a four-field table with the characteristics: \textbf{socks} (colorful / not colorful) and \textbf{math} (likes / dislikes).
			\item How many guests do not wear colorful socks and do not like math?
			\item Are the characteristics independent?
		\end{enumerate}
		
		\newpage
		{\bf \item The Dice Conspiracy – Or Is It?}
		
		Two dice, A and B, are thrown simultaneously. A conspiracy is suspected, but the probabilities (perhaps) tell a different story:
		
		\begin{itemize}
			\item Dice A shows an even number 70% of the time.
			\item Dice B shows an even number 40% of the time.
		\end{itemize}
		
		Assume the two dice behave independently of each other.
		
		\begin{enumerate}
			\item Calculate the probability that both dice show an even number at the same time.
			\item Create a four-field table with the properties: "Dice A even/odd" and "Dice B even/odd."
			\item What is the probability that at least one die shows an even number?
		\end{enumerate}
		
	\end{enumerate}
	\fbox{
		\begin{minipage}{0.5\textwidth}
			For the solution, please click here or scan the QR code.\\
			More worksheets are available at
			
			\href{https://www.okuyakl.de}{www.okuyakl.de}
		\end{minipage}
		\hfill
		\begin{minipage}{0.4\textwidth}
			\includegraphics[width=1.5 cm]{../../viecher/zwe03}
			\includegraphics[width=3 cm]{qr037}
			\includegraphics[width=2 cm]{../../viecher/afanticon1}
			
	\end{minipage}}
	
\end{document}