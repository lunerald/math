\documentclass[a4paper]{article}
\usepackage[pdftex]{graphicx}
\usepackage[utf8]{inputenc}
\usepackage{enumerate}
\usepackage{icomma}
\usepackage{amssymb}
\usepackage{geometry}
\geometry{a4paper, top=15mm, left=15mm, right=15mm, bottom=15mm,
	headsep=10mm, footskip=12mm}
\usepackage{href-ul}
\hypersetup{
	colorlinks=true,
	linkcolor=blue,
	urlcolor=blue}
\begin{document}
{\bf Wurzeln}-- nur Quadratwurzeln -- Bitte ohne Taschenrechner. 
\begin{enumerate}[1.]

%Aufgabe 2


{\bf \item Ziehe die Wurzel, wenn möglich} Es gilt $\mathbb{G}=\mathbb{R}^+$

\renewcommand{\arraystretch}{2.8}
\begin{tabular}{p{5 cm}p{5 cm}p{5 cm}}
a)$\sqrt{2\cdot 8}$	& b) $\sqrt{3^2 +4^2}$ &c) $\sqrt{0,01}$ \\
d) $\sqrt{1000000}$ & e)$\sqrt{(-1)^2}$ & f) $\sqrt{2^{10}}$ \\
g)$\sqrt{a^4}$ & h)$\sqrt{36a^2b^2}$ &i)$\sqrt{a^2 + b^2}$ \\
j)$\sqrt{1 \over c^2 }$ &k)$\sqrt{u^{-2}}$ &l)$\sqrt{2u^3 \cdot 32u}$ \\
m)$\sqrt{(8+1)^2}$ & n)$\sqrt{(x+3)^2}$ & o) $\sqrt{p^2+2pq+q^2}$ \\
p) $\sqrt{121 \over 144}$ & q) $\sqrt{{2b^2 \over 512}}$ & r) $\sqrt{0,1}$ \\
\hline
\end{tabular}



{\bf \item Ziehe unter die Wurzel und fasse zusammen} Es gilt $\mathbb{G}=\mathbb{R}^+$

\renewcommand{\arraystretch}{2.8}
\begin{tabular}{p{5 cm}p{5 cm}p{5 cm}}
a) $3\sqrt{2}$ & b) $8\sqrt{3}$ & c) ${1\over 2}\sqrt{7}$\\ 
d) $z\sqrt{z}$& e) $a^2\sqrt{1\over a}$& f) $(x+1)\sqrt{x+1}$\\
\hline
\end{tabular}

{\bf \item Radiziere teilweise} Es gilt $\mathbb{G}=\mathbb{R}^+$

\renewcommand{\arraystretch}{2.8}
\begin{tabular}{p{5 cm}p{5 cm}p{5 cm}}
a) $\sqrt{8}$ &b) $\sqrt{48}$ &c) $\sqrt{1000}$ \\
d) $\sqrt{81+81}$ &e) $\sqrt{9^2 + 9^2 +9^2}$ &f) $\sqrt{8 \over 25}$  \\
g) $\sqrt{a^3}$ &h) $\sqrt{6 \over 2x^2}$ &i) $\sqrt{2 \cdot( a^2 +2ab +b^2)}$ \\
j) $\sqrt{x^3 \over y^3}$ &k) $\sqrt{r^2(r+1)}$ &l) $\sqrt{(x^3)^3}$ \\
\hline
\end{tabular}


{\bf \item Vereinfache so weit wie möglich.} Irrationale Zahlen bleiben stehen. 

\renewcommand{\arraystretch}{2.8}
\begin{tabular}{p{5 cm}p{5 cm}p{5 cm}}
a) $ \sqrt{6}\cdot \sqrt{24}$ & b) $2 \sqrt{3}\cdot\sqrt{27}$ & c) ${\sqrt{384}\over \sqrt{8}\cdot \sqrt{48}}$ \\
d) $(2\sqrt{2})^2$ & e) $\sqrt{2}+\sqrt{3}$ & f) $\sqrt{2} + 8 \sqrt{2}$ \\
g) $\sqrt{12}+\sqrt{3}$ & h) $2\sqrt{18}-\sqrt{98}$ & i) $ (\sqrt{8} + \sqrt{2})^2 $ \\
j)$(\sqrt{7}-\sqrt{2})(\sqrt{7}+\sqrt{2})$ & k)$(\sqrt{2}+\sqrt{3})^2$  & l)$(\sqrt{46}-\sqrt{6})(\sqrt{18}-\sqrt{138})$  \\
\hline
\end{tabular}

\end{enumerate} 

\fbox{
	\begin{minipage}{0.5\textwidth}
		Zur Lösung bitte \href{https://www.okuyakl.de/math/m9wurL044/ll044.pdf}{hier klicken} oder den QR-Code scannen.\\
	Weitere Arbeitsblätter gibt es unter 
	
	\href{https://www.okuyakl.de}{www.okuyakl.de}
	\end{minipage}
	\hfill
	\begin{minipage}{0.4\textwidth}
		\includegraphics[width=1.5 cm]{../../viecher/zwe03}
		\includegraphics[width=3 cm]{qr044}
		\includegraphics[width=2 cm]{../../viecher/afanticon1}
		
	\end{minipage}}

\end{document}%Lösung-------------------------------------------
