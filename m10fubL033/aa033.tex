\documentclass[a4paper]{article}
\usepackage[pdftex]{graphicx}
\usepackage[utf8]{inputenc}
\usepackage{enumerate}
\usepackage{icomma}
\usepackage{amssymb}
\usepackage{polynom}
\usepackage{geometry}
\geometry{a4paper, top=15mm, left=15mm, right=15mm, bottom=15mm,
	headsep=10mm, footskip=12mm}
\usepackage{href-ul}
\hypersetup{
	colorlinks=true,
	linkcolor=blue,
	urlcolor=blue}
\begin{document}
{\bf Funktionen}
\begin{enumerate}[1.]

{\bf \item Gegeben ist die Funktion $ g(x) = 0,2 x^3 - 1,4x^2 - 1,8x + 12,6 $}

\begin{enumerate}[a)]
\item Zeige, dass $ x=3 $  eine Nullstelle der Funktion $ g $ ist.
\item Bestimme alle weiteren Nullstellen der Funktion $ g $ .
\item Ermittle die Koordinaten des Schnittpunktes von $ g $ mit der y-Achse.
\item Bestimme das Verhalten des Graphen von $ g $ für $ x \longrightarrow \infty $ 
und $ x \longrightarrow -\infty $ . Gib eine kurze Begründung für dein Ergebnis an.
\end{enumerate}

 
\begin{minipage}{0.6\textwidth}
{\bf \item Der rechts abgebildete Funktionsgraph hat die y-Achse als senkrechte und die Gerade $y=0,5$ als waagrechte Asymptote.} Bei $x=1$ berührt er die x-Achse\\ 
Geben Sie einen möglichen Funktionsterm an!
\end{minipage}
\hfill
\begin{minipage}{0.4\textwidth}
\includegraphics[width=6 cm]{gebr134.jpeg}
\end{minipage}

%Aufgabe 2
{\bf \item Gegeben ist die Funktion }
$$ f(x) = \frac{2x^3-2x^2-8x+8}{x^2+2x-3}  $$
\begin{enumerate}[a)]
\item Zeigen Sie mit Hilfe einer Rechnung, dass gilt: $ D_f = \mathbb{R} \setminus \{-3;1\}$
\item Stellen Sie den Funktionsterm in vollständig faktorisierter Form dar.
\item Begründen Sie durch nachvollziehbare Grenzwertbetrachtungen, dass die Funktion $f$ eine hebbare Definitionslücke besitzt und berechnen Sie alle Nullstellen der Funktion $f$.
\item Begründen Sie durch nachvollziehbare Überlegungen, dass die Funktion $f$ eine Polstelle besitzt und geben Sie Lage und Art der Polstelle an.
\item Bestimmen Sie die Art und die Gleichungen sämtlicher Asymptoten der Funktion.
\end{enumerate}


%Aufgabe 4
{\bf \item Gegeben ist die Funktion $f$ mit $f(x)= \frac{1}{x+2}+5$.}
\begin{enumerate}[a)]
\item Bestimme die größtmögliche Definitionsmenge $D_f$ und die Gleichungen aller waagrechten und senkrechten Asymptoten des Funktionsgraphen $G_f$.
\item Ermittle diejenige x-Stelle, bei der der Funktionswert 7 ist.
\item Der Graph $G_g$ der Funktion $g$ entsteht aus $G_f$ durch Streckung in x-Richtung mit dem Faktor 0,5 und anschließender Verschiebung um +1 in y-Richtung. Ermittle rechnerisch den Funktionsterm $g(x)$ und die Glei\-chungen der waagrechten und senkrechten Asymptoten von $G_g$
\end{enumerate} 
\end{enumerate} 

\fbox{
	\begin{minipage}{0.5\textwidth}
		Zur Lösung bitte \href{https://www.okuyakl.de/math/m10fubL033/ll033.pdf}{hier klicken} oder den QR-Code scannen.\\
	Weitere Arbeitsblätter gibt es unter 
	
	\href{https://www.okuyakl.de}{www.okuyakl.de}
	\end{minipage}
	\hfill
	\begin{minipage}{0.4\textwidth}
		\includegraphics[width=1.5 cm]{../../viecher/zwe03}
		\includegraphics[width=3 cm]{qr033}
		\includegraphics[width=2 cm]{../../viecher/afanticon1}
		
	\end{minipage}}

\end{document}%Lösung-------------------------------------------
