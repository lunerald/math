\documentclass[a4paper]{article}
\usepackage[pdftex]{graphicx}
\usepackage[utf8]{inputenc}
\usepackage{enumerate}
\usepackage{icomma}
\usepackage{siunitx}
\sisetup{locale=DE} 
\usepackage{amssymb}
\usepackage{geometry}
\geometry{a4paper, top=15mm, left=15mm, right=15mm, bottom=15mm,
	headsep=10mm, footskip=12mm}
\usepackage{href-ul}
\hypersetup{
	colorlinks=true,
	linkcolor=blue,
	urlcolor=blue}
	
\begin{document}
{\bf Körperberechnungen}
\begin{enumerate}[1.]

%Aufgabe 1
{\bf \item Ein Prisma besitzt als Grundfläche eine Raute mit den Diagonalen $e$ und $f$. }

\begin{enumerate}[a)]
\item Berechne im Allgemeinen die Diagonallänge von $e$, wenn die Höhe $h$ des Prismas, das Volumen $V_{Pr}$ und die Diagonallänge von $f$ gegeben ist.
\item Nun sei $h=\SI{6,6}{\centi\meter}, \quad V_{Pr}= \SI{25,5}{\centi\meter^3} \quad \textnormal{und} \quad f=\SI{43}{\milli\meter}$
gegeben. Berechne die Diagonallänge von $e$. 
\end{enumerate}

{\bf \item Durch einen Zylinder aus Glas ($\rho_{Glas}=\SI{2,40}{\gram\per \centi\meter^3}$) wird eine prismenförmige 
Öffnung} mit einem regelmäßigen Sechseck als Grund-- und Deckfläche gefräst. Berechne auf zehntel Gramm gerundet die Masse des Restkörpers.

\includegraphics[width=12 cm]{prisma90.pdf}

\begin{minipage}{0.5\textwidth}
{\bf \item Von einer Pyramide mit quadratischer Grundfläche sind gegeben} ($a$: Kante der Grundfläche, $s:~$ Sei\-tenkante):
\begin{enumerate}[a)]
\item $h_k=\SI{7,8}{\centi\meter}, \quad h_a=\SI{9,4}{\centi\meter}$
\item $a=\SI{15}{\centi\meter},\quad h_a=\SI{18}{\centi\meter} $
\end{enumerate}
\end{minipage}
\hfill
\begin{minipage}{0.45\textwidth}
\includegraphics[width= 7 cm]{pyramide157.pdf}
\end{minipage}

Berechne die jeweils fehlenden Größen
 $a,~h_a,~h_k,~s,~V,~\textnormal{Mantelfläche}~ M,\\
\textnormal{und die Oberfläche}~O$

\end{enumerate} 

\fbox{
	\begin{minipage}{0.5\textwidth}
		Zur Lösung bitte \href{https://www.okuyakl.de/math/m9korL046/ll046.pdf}{hier klicken} oder den QR-Code scannen.\\
	Weitere Arbeitsblätter gibt es unter 
	
	\href{https://www.okuyakl.de}{www.okuyakl.de}
	\end{minipage}
	\hfill
	\begin{minipage}{0.4\textwidth}
		\includegraphics[width=1.5 cm]{../../viecher/zwe03}
		\includegraphics[width=3 cm]{qr046}
		\includegraphics[width=2 cm]{../../viecher/afanticon1}
		
	\end{minipage}}

\end{document}%Lösung-------------------------------------------
