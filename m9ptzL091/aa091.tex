\documentclass[a4paper]{article}
\usepackage[pdftex]{graphicx}
\usepackage[utf8]{inputenc}
\usepackage{enumerate}
\usepackage{amssymb}
\usepackage{geometry}
\geometry{a4paper, top=15mm, left=15mm, right=15mm, bottom=15mm,
	headsep=10mm, footskip=12mm}
\usepackage{href-ul}
\hypersetup{
	colorlinks=true,
	linkcolor=blue,
	urlcolor=blue}
\begin{document}
{\bf Potenzen mit rationalem Exponenten.} Für alle Aufgaben gilt: $\mathbb{G}=\mathbb{R}^+$
\begin{enumerate}[1.]

{\bf \item Ein DIN A0 Blatt} hat eine Fläche von $A_{A0}= 1 \rm{m^2}. $ Länge und Breite dieses Blatts verhalten sich wie $\sqrt{2} : 1 $\\
\begin{minipage}{0.5\textwidth}
	
	\begin{enumerate}[a)]
		\item Wie lang und wie breit ist das A0-Blatt? Schreibe die Lösung exakt als Potenz von 2 und als gerundete Kommazahl.
		
		Ein A1 Blatt ist genau die Hälfte des A0 Blattes und hat das gleiche Seitenverhältnis.
		Durch weiteres Teilen durch zwei erhält man die Formate A2, A3 und so weiter.
		
		\item Tabellarisiere die Längen, Breiten und die Flächeninhalte der A0 bis A4 Blätter.
		Schreibe die Lösung exakt als Potenz von 2 und als gerundete Kommazahl.
	\end{enumerate}
\end{minipage}
\hfill
\begin{minipage}{0.4\textwidth}
	\includegraphics[width=6 cm]{a4blatt091}
\end{minipage}
\end{enumerate} 

\renewcommand{\arraystretch}{3}
\begin{tabular}{|p{2.5 cm}|p{2.5 cm}|p{2.5 cm}|p{2.5 cm}|p{2.5 cm}|p{2.5 cm}|}
\hline
&A0 & A1 & A2 & A3 & A4 \\
\hline
Breite & & & & & \\
\hline
Länge  & & & & & \\
\hline
Flächeninhalt & & & & & \\
\hline
\end{tabular}


{\bf \item Schreibe ohne Bruch}

\renewcommand{\arraystretch}{3}
\begin{tabular}{p{9 cm}p{7 cm}}
	a)\Large \quad \quad ${1\over a^2}$	& b)\Large \quad ${c^{3}\over c^{4}}$ \\
	 c)\Large \quad ${1\over b^{-2}}$ & d)\Large \quad $ \left({2k\over n}\right)^5 $ \\
  e)\Large \quad ${1 \over {1\over a}} $ & f)\Large \quad ${{7 \over x} \over x^2}$ \\
\end{tabular}

{\bf \item Schreibe als Bruch}

\renewcommand{\arraystretch}{3}
\begin{tabular}{p{9 cm}p{7 cm}}
	a)\large \quad $(a^6)^{-1}\cdot b^2 $	& b)\large \quad ${u^2\cdot (vu)^{-1}} $\\
	 c)\large \quad $ (d^{-7})^2 \cdot c^2d^{10} $ &
	 d)\large \quad $(a+1)^{-2}\cdot (a+1) $\\
	  e)\large \quad $a^{-1} (0,5 b)^2 c^{-3} (2d)^4 $
\end{tabular}

{\bf \item Vereinfache folgende Terme:}

\renewcommand{\arraystretch}{3}
\begin{tabular}{p{9 cm}p{7 cm}}
a)\Large \quad ${k^2 \cdot (cd)^3 \over k^{-1} c^2d^4}$ & b)\Large \quad ${\left( {m\over n} \right)^2}\cdot {m^{-2} \over n}$ \\
 c)\Large \quad $ \left( \frac{ab^2c}{2a^2b^2}\right)^{-3} : \frac{4c^{-1}}{a} $ &
 d)\Large \quad ${{p^3\over q} \over pq^2}$ \\
  e)\Large \quad ${(ab)^{4}\over {b^3 \over a^3c}}$ & f) \Large \quad $\left({wz\over (wv)^3} \right)^{-2}$\\
\end{tabular}


{\bf \item Schreibe ohne Wurzel}

\renewcommand{\arraystretch}{3}
\begin{tabular}{p{9 cm}p{7 cm}}
a)\Large \quad $\sqrt[3]{a} $	& b)\Large \quad $\sqrt[4]{k^2} $\\
c)\Large \quad $\sqrt{(uv)^{-4}} $ & d)\Large \quad $\sqrt[5]{1\over m^2n^2} $\\
e)\Large \quad $\sqrt{4c^2 \over d^8} $& f)\Large \quad $(\sqrt[3]{kx})^9 $\\
\end{tabular}

{\bf \item Schreibe als Wurzel}

\renewcommand{\arraystretch}{3}
\begin{tabular}{p{9 cm}p{7 cm}}
a)\Large \quad $a^{3\over 2}$	& b)\Large \quad $x^{-{1\over 7}} $\\
c)\Large \quad $ \left({1\over x^3}\right)^{1\over 6}$ & d)\Large \quad $ {y^{0,5} \over y} $ \\
  e)\Large \quad $ z^{0,3} \cdot z^{-0,05} $& f)\Large \quad $ f^{5\over 8}\over  f^{1\over 8} $\\
\end{tabular}

\newpage

{\bf \item Vereinfache folgende Terme:}\\

\renewcommand{\arraystretch}{3}
\begin{tabular}{p{9 cm}p{7 cm}}
a)\Large \quad $ a \over \sqrt{a} $ & b)\Large \quad $\sqrt{8u}\cdot \sqrt{k^2 \over 2u} $ \\ 
c)\Large \quad $ \sqrt{x^3 \over y}\cdot {1\over x \sqrt{y}} $ & d)\Large \quad ${b^2c\over \sqrt{bc^2}}  $ \\
e) \Large \quad $\left(\sqrt[3]{1\over uvw^6}\right)^{-1} $ & f) \Large \quad ${\sqrt{p^3q^{-6}}\over p\sqrt{q}} $\\
g)\Large \quad $\sqrt{m\sqrt{m^2}} $ & h)\Large \quad $\sqrt{25p\over \sqrt{p}}$ \\ 
i)\Large \quad $ \sqrt[3]{\sqrt{x^{12}}} $ & j)\Large \quad $\sqrt{{\sqrt{y}\over \sqrt{y^{-3}}}} $ \\ k)\Large \quad $ \sqrt{n^4 + \sqrt{9n^8}} $ & l)\Large \quad $\sqrt{x\sqrt{x\sqrt{x}}} $ \\
\end{tabular}

{\bf \item Vereinfache durch Anwendung der Rechenregeln soweit wie möglich.} Im Ergebnis darf keine Wurzel mehr vorkommen. ($x\in \mathbb{R}^+$)
$$ \sqrt[3]{x^2} \cdot  \sqrt[3]{x} +  (\sqrt[5]{x})^{-10}$$


\fbox{
	\begin{minipage}{0.5\textwidth}
		Zur Lösung bitte \href{https://www.okuyakl.de/math/m9ptzL091/ll091.pdf}{hier klicken} oder den QR-Code scannen.\\
	Weitere Arbeitsblätter gibt es unter 
	
	\href{https://www.okuyakl.de}{www.okuyakl.de}
	\end{minipage}
	\hfill
	\begin{minipage}{0.4\textwidth}
		\includegraphics[width=1.5 cm]{../../viecher/zwe03}
		\includegraphics[width=3 cm]{qr091}
		\includegraphics[width=2 cm]{../../viecher/afanticon1}
		
\end{minipage}}

\end{document}