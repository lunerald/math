\documentclass[a4paper]{article}
\usepackage[pdftex]{graphicx}
\usepackage[utf8]{inputenc}
\usepackage{enumerate}
\usepackage{amssymb}
\usepackage{icomma}
\usepackage{siunitx}
\sisetup{locale=DE} 
\usepackage{href-ul}
\hypersetup{
	colorlinks=true,
	linkcolor=blue,
	urlcolor=blue}
\usepackage{geometry}
\geometry{a4paper, top=15mm, left=15mm, right=15mm, bottom=15mm,
	headsep=10mm, footskip=12mm}

\begin{document}
{\bf Muster-Kurvendiskussion e-Funktion}

\begin{enumerate}[1.]
{\bf \item Gegeben ist die Funktion $$f(x)=x\cdot e^x$$
Führen Sie eine vollständige Kurvendiskussion durch. }
\begin{enumerate}[a)]
	\item Ermitteln Sie die Definitionsmenge $\mathbb{D}_f$ und das Verhalten im Unendlichen.
	\item Untersuchen Sie die Funktion auf Symmetrie.
	\item Berechnen Sie alle Nullstellen.
	\item Bilden Sie die erste, die zweite und die dritte Ableitung.
	\item Untersuchen Sie das Monotonieverhalten und die Art und Lage des Extrempunktes mittels einer Monotonie\-tabelle. Bestimmen Sie die Wertemenge $\mathbb{W}_f$ von $f(x)$
	\item Bestimmen Sie das Krümmungsverhalten und den Wendepunkt.
	\item Geben Sie den Term der Wendetangente an.
	\item Zeichnen Sie den Graphen $G_f$ mit der Wendetangente anhand der gewonnenen Erkenntnisse in ein geeignetes Koordinatensystem.
	\item Zeigen Sie, dass $F(x)=(x-1)\cdot e^x$ eine Stammfunktion von $f(x)$ ist.
	\item Der Graph $G_f$, die $x$-Achse und die Wendetangente schließen ein Flächenstück ein. Berechnen Sie dessen Inhalt. 
\end{enumerate} 

\end{enumerate} 
\fbox{
	\begin{minipage}{0.5\textwidth}
		Zur Lösung bitte \href{https://www.okuyakl.de/math/m12mkeL069/ll069.pdf}{hier klicken} oder den QR-Code scannen.\\
	Weitere Arbeitsblätter gibt es unter 
	
	\href{https://www.okuyakl.de}{www.okuyakl.de}
	\end{minipage}
	\hfill
	\begin{minipage}{0.4\textwidth}
		\includegraphics[width=1.5 cm]{../../viecher/zwe03}
		\includegraphics[width=3 cm]{qr069}
		\includegraphics[width=2 cm]{../../viecher/afanticon1}
		
\end{minipage}}
\end{document}

