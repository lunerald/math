\documentclass[a4paper]{article}
\usepackage[pdftex]{graphicx}
\usepackage[utf8]{inputenc}
\usepackage{enumerate}
\usepackage{icomma}
\usepackage{siunitx}
\sisetup{locale=DE} 
\usepackage{amssymb}
\usepackage{geometry}
\geometry{a4paper, top=15mm, left=15mm, right=15mm, bottom=15mm,
	headsep=10mm, footskip=12mm}
\usepackage{href-ul}
\hypersetup{
	colorlinks=true,
	linkcolor=blue,
	urlcolor=blue}
	
\begin{document}
{\bf Flächen-Terme}
\begin{enumerate}[1.]

%Aufgabe 1
{\bf \item Ein Rechteck hat die Seitenlängen $\overline{AB}=\SI{12}{\centi\meter}$ und $\overline{AD}=\SI{5}{\centi\meter}$.} Verkürzt man $[AB]$ von $B$ aus um $x~\SI{}{\centi\meter}$ und verlängert gleichzeitig $[AD]$ über $D$ hinaus um ebenfalls $x~\SI{}{\centi\meter}$, so entstehen neue Rechtecke $AB_nC_nD_n$.

\begin{enumerate}[a)]
\item Zeichne das gegebene Rechteck und das Rechteck $AB_1C_1D_1$ für $x=2$.
\item Gib für $x$ eine sinnvolle Grundmenge an.
\item Berechne den Umfang der Rechtecke  $AB_nC_nD_n$ in Abhängigkeit von $x$. Was folgerst du aus dem Ergebnis?
\item Stelle den Flächeninhalt der Rechtecke $AB_nC_nD_n$ in Abhängigkeit von $x$ dar.
\item Berechne den maximalen Flächeninhalt und das zugehörige $x$.
\end{enumerate}

%Aufgabe 2
{\bf \item Ein Rechteck $ABCD$ hat die Seitenlängen $\overline{AB}=\SI{8}{\centi\meter}$ und $\overline{AD}=\SI{6}{\centi\meter}$.} Verlängert man $[AB]$ über $A$ und $B$ hinaus um {\bf jeweils} $x~\SI{}{\centi\meter}$ und verkürzt gleichzeitig $[AD]$ um $x~\SI{}{\centi\meter}$, so entstehen neue Rechtecke.

\begin{enumerate}[a)]
\item Zeichne das ursprüngliche Rechteck und das Rechteck $A_1B_1C_1D_1$ für $x=1,5$.
\item Gib für $x$ eine sinnvolle Grundmenge an.
\item Stelle den Umfang der Rechtecke  $A_nB_nC_nD_n$ in Abhängigkeit von $x$ dar.
\item Stelle den Flächeninhalt der Rechtecke  $A_nB_nC_nD_n$ in Abhängigkeit von $x$ dar.
\item Berechne den maximalen Flächeninhalt und das zugehörige $x$.
\end{enumerate}

%Aufgabe 3
{\bf \item Gegeben ist ein Quadrat $ABCD$ der Seitenlänge $\SI{8}{\centi\meter}$.} Verlängert man die eine Seite des Quadrats um $2x~\SI{}{\centi\meter}$ und verkürzt gleichzeitig die andere Seite um $x~\SI{}{\centi\meter}$, so erhält man Rechtecke $AB_nC_nD_n$.

\begin{enumerate}[a)]
\item Zeichne das gegebene Quadrat und berechne seinen Umfang und seinen Flächeninhalt.
\item Ergänze die Zeichnung von 3 a) mit dem Rechteck  $AB_1C_1D_1$ für $x=1$.
\item Gib für $x$ eine sinnvolle Grundmenge an.
\item  Stelle den Umfang der Rechtecke $AB_nC_nD_n$ in Abhängigkeit von $x$ dar.
\item  Stelle den Flächeninhalt der Rechtecke $AB_nC_nD_n$ in Abhängigkeit von $x$ dar.
\item Berechne den maximalen Flächeninhalt und das zugehörige $x$.
\end{enumerate}

%Aufgabe 4
{\bf \item Ein gleichschenkliges Dreieck $ABC$ hat die Basislänge $\overline{AB}=\SI{4}{\centi\meter}$ und die Höhe $\SI{10}{\centi\meter}$.} Verlängert man $[AB]$ über $A$ und $B$ hinaus um {\bf jeweils} $x~\SI{}{\centi\meter}$ und verkürzt gleichzeitig die Höhe um $x~\SI{}{\centi\meter}$, so entstehen neue Dreiecke $A_nB_nC_n$.

\begin{enumerate}[a)]
	\item Zeichne das gegebene Dreieck und berechne seinen Flächeninhalt.
	\item Ergänze die Zeichnung von 4 a) mit dem Dreieck  $A_1B_1C_1$ für $x=2$.
	\item Gib für $x$ eine sinnvolle Grundmenge an.
	\item  Stelle den Flächeninhalt der Dreiecke $A_nB_nC_n$ in Abhängigkeit von $x$ dar.
	\item Berechne den maximalen Flächeninhalt und das zugehörige $x$.
\end{enumerate}

%Aufgabe 4
{\bf \item Eine Raute $ABCD$ hat die eine Diagonallänge $\overline{AC}=\SI{6}{\centi\meter}$ und die andere Diagonallänge $\overline{BD}=\SI{8}{\centi\meter}$.} Verlängert man $[AC]$ über $A$ und $C$ hinaus um {\bf jeweils} $x~\SI{}{\centi\meter}$ und verkürzt gleichzeitig $[BD]$ beidseitig um $x~\SI{}{\centi\meter}$, so entstehen neue Rauten $A_nB_nC_nD_n$.

\begin{enumerate}[a)]
	\item Zeichne die gegebene Raute und berechne ihren Flächeninhalt.
	\item Ergänze die Zeichnung von 5 a) mit der Raute $A_1B_1C_1D_1$ für $x=1$.
	\item Gib für $x$ eine sinnvolle Grundmenge an.
	\item  Stelle den Flächeninhalt der Rauten $A_nB_nC_nD_n$ in Abhängigkeit von $x$ dar.
	\item Berechne den maximalen Flächeninhalt und das zugehörige $x$. Welche besondere Form hat diese Raute dann?
\end{enumerate}


\end{enumerate}

\fbox{
	\begin{minipage}{0.5\textwidth}
		Zur Lösung bitte \href{https://www.okuyakl.de/math/m8terL054/ll054.pdf}{hier klicken} oder den QR-Code scannen.\\
	Weitere Arbeitsblätter gibt es unter 
	
	\href{https://www.okuyakl.de}{www.okuyakl.de}
	\end{minipage}
	\hfill
	\begin{minipage}{0.4\textwidth}
		\includegraphics[width=1.5 cm]{../../viecher/zwe03}
		\includegraphics[width=3 cm]{qr054}
		\includegraphics[width=2 cm]{../../viecher/afanticon1}
		
	\end{minipage}}

\end{document}%Lösung-------------------------------------------
