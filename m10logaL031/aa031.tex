\documentclass[a4paper]{article}
\usepackage[pdftex]{graphicx}
\usepackage[utf8]{inputenc}
\usepackage{enumerate}
\usepackage{amssymb}
\usepackage{icomma}
\usepackage{eurosym}
\usepackage{geometry}
\geometry{a4paper, top=15mm, left=15mm, right=15mm, bottom=15mm,
	headsep=10mm, footskip=12mm}
\usepackage{href-ul}
\hypersetup{
	colorlinks=true,
	linkcolor=blue,
	urlcolor=blue}
\begin{document}
{\bf Logarithmen}
\begin{enumerate}[1.]
{\bf \item Löse folgende Exponentialgleichung: $$ 2 \cdot 1,2^{2x} = 5 \cdot 2^x $$}

%Aufgabe 4
{\bf \item Ein radioaktives Präparat enthält zu Beginn des Beobachtungszeitraumes 140 aktive Kerne,} die unter Abgabe radioaktiver Strahlung zu inaktiven Tochterkernen zerfallen. Es zerfallen in jeder Sekunde 1,5\% der aktiven Kerne. Wann ist die Radioaktivität des Materials auf 42\% des ursprünglichen Werts gesunken ?

{\bf \item Schreibe als Summe oder Differenz von einfachen Logarithmen} \\
 $ ( a, b \in \mathbb{R}^+; a \ne 1 ) $
\begin{enumerate}[a)]
\item $ \log_{5}{ \left( \sqrt[4]{5ab} \right)} $
\item $ \log_{a}{ \left( \frac{14a}{b^4} \right)} $
\end{enumerate}

{\bf \item Fasse soweit wie möglich zusammen } $ ( a, b, c \in \mathbb{R}^+; a \ne 1 ) $ \\
$$ \log_{a}{b^{-0,25}} - 0,5 \log_{a}{b} + 0,4 \log_{a}{ \left( \frac{\sqrt{b}}{b^2} \right)} $$

{\bf \item Der Bestand einer Population von Oryxantilopen wird durch die Funktion}
$$ f(t) = 20000 \cdot 3^{-0,3t} ; \quad D_f = [0;20] $$ 
($t =$ Anzahl der Jahre) beschrieben. Finde heraus, wann der Bestand nur noch 25\% des 
 Anfangsbestandes ausmacht.

{\bf \item Bestimme die Lösungsmenge der folgenden Exponentialgleichung über der Grundmenge }
$ \mathbb{G} = \mathbb{R} $ .
$$ 5^{2x+1} - 3 \cdot 5^{x+2} + 250 = 0$$

{\bf \item Berechne die Basis:} $$ \log_x{\sqrt[3]{0,5}} =  \frac{2}{3} $$

{\bf \item Vereinfache soweit wie möglich: }
$$ \log_5{(x^5 \cdot 5^{2x})} + 5 \cdot \log_5{\frac{25}{x}}$$

{\bf \item Aufgrund fortdauernder Wasserverschmutzung geht der Bestand einer Fischart in einem See jährlich um etwa 9\% zurück.} Zur Zeit befinden sich ca. 50000 Fische dieser Art im See. 

\begin{enumerate}[a)]
\item Wie groß ist der Bestand nach x Jahren?
\item Die Behörden wollen reagieren, wenn der Fischbestand um 20000 Exemplare zurückgegangen ist. Wie lange wird es dauern, bis sie Maßnahmen ergreifen?
\item Die Maßnahmen der Behörden führen dazu, dass der Fischbestand wieder um 6\% pro Jahr wächst. Nach welcher Zeit ist der ursprüngliche Fischbestand wieder erreicht?.
\end{enumerate}

{\bf \item Herr Mandelmeier erwirbt einen Neuwagen für \EUR {47900}.} Dieses Auto verliert jährlich 23\% seines Restwertes. Stellen Sie die passende Funktionsgleichung für den Restwert y \EUR{} nach x Jahren auf und berechnen Sie anschließend, wie alt das Fahrzeug beim Wiederverkauf ist, wenn der Eigentümer einen Verkaufspreis von \EUR {4550} erzielt. Runden Sie sinnvoll!

%Aufgabe 3
{\bf \item Ein Kapital $K_0$ wird mit $p\%$ verzinst. Die Zinsen werden jährlich dem Kapital $K_0$ zugeschlagen.} Zeigen Sie allgemein durch Rechnung, dass die (Maßzahl der) Zeit $x$ bis zur Ver-$n$-fachung des Kapitals $K_0$ nur vom Zinssatz $p$ und dem Faktor $n$ abhängt (das ,,Ergebnis" soll eine nach $x$ aufgelöste Gleichung sein).

{\bf \item Vereinfache soweit wie möglich}

\begin{enumerate}[a)]
\item $ \log_a{ \left( \frac{1}{a} \right)^2} \cdot \log_c{ \left( \frac{1}{c} \right)} -
\log_x{ \left( \frac{1}{x^3}\right)} \cdot \log_{b}1 $
\item $ \lg{a^5} - \lg{a^4 b} + \lg{\frac{b}{a}} + \lg{\sqrt{b^5}} $
\end{enumerate}

{\bf \item Frau Müller möchte einen bestimmten Betrag fest anlegen.} Die erste Hälfte des Betrages legt sie für 10 Jahre mit einem jährlichen Zinssatz von 5 \% an. Mit welchem Jahreszinssatz müsste sie die zweite Hälfte anlegen, damit die zweite Hälfte in 7 Jahren auf denselben Betrag anwächst wie die erste Hälfte in 10 Jahren?

\newpage
{\bf \item Bestimme $x$.}

\begin{enumerate}[a)]
\item $\frac{ 5^{x+1}}{2^x} = 78,125 $ 
\item $ \log_{\sqrt{x}}{\frac{1}{27}} = -3 $
\item $ \log_x7 = \frac{1}{3} $
\end{enumerate}

{\bf \item In einer Nährlösung befinden sich 14000 Bakterien vom Typ A und 5000 Bakterien vom Typ B.} Die Bakterien vom Typ A verdoppeln sich stündlich, während sich die Bakterien vom Typ B pro halber Stunde um 130\% vermehren. Nach welcher Zeit befinden sich gleich viele Bakterien von beiden Typen in der Nährlösung? 


{\bf \item Eine Chemiefirma verpflichtet sich, ihren Schadstoffausstoß bis 2028 gegenüber dem Stand von 2016 um 14\% zu verringern.} Entscheide durch Rechnung, ob dieses Ziel durch eine jährliche prozentuale Abnahme von 1,25\% erreicht werden kann.

{\bf \item Stelle den Term um und bestimme die Variable:}

\begin{enumerate}[a)]
\item $ 5^x = 0,04 $
\item $ \log_{7}{ \left( \frac{1}{\sqrt[3]{2401} } \right)}=c $
\end{enumerate}

{\bf \item Berechne:}

\begin{enumerate}[a)]
\item $ \log_{\sqrt{a}}{ \left(  a^2 \cdot \sqrt{a} \right)}$
\item $ \log_{a}{ \left( \sqrt[3]{ \frac{1}{a} } \right)} $
\end{enumerate}

{\bf \item Forme folgenden Wortlaut in eine logarithmische Gleichung um und löse sie dann:}\\
,, Mit welcher Zahl muss man 6 potenzieren, um 216 zu erhalten ? "
 
\end{enumerate} 

\fbox{
	\begin{minipage}{0.5\textwidth}
		Zur Lösung bitte \href{https://www.okuyakl.de/math/m10logaL031/ll031.pdf}{hier klicken} oder den QR-Code scannen.\\
	Weitere Arbeitsblätter gibt es unter 
	
	\href{https://www.okuyakl.de}{www.okuyakl.de}
	\end{minipage}
	\hfill
	\begin{minipage}{0.4\textwidth}
		\includegraphics[width=1.5 cm]{../../viecher/zwe03}
		\includegraphics[width=3 cm]{qr031}
		\includegraphics[width=2 cm]{../../viecher/afanticon1}
		
	\end{minipage}}

\end{document}%Lösung-------------------------------------------
