
\documentclass[a4paper]{article}
\usepackage[pdftex]{graphicx}
\usepackage[utf8]{inputenc}
\usepackage{enumerate}
\usepackage{icomma}
\usepackage{amssymb}
\usepackage{href-ul}
\hypersetup{
	colorlinks=true,
	linkcolor=blue,
	urlcolor=blue}
%Source: img_0180

% Start the document
\begin{document}
%Titel
{\bf Trainingsblatt Integrale }\\
\begin{enumerate}[1.]

%Aufgabe 1
{\bf \item Berechne den Wert des Integrals:}

\begin{tabular}{ccccc}
 a) $\int\limits_{0}^{1} x^2 ~{\rm d} x$ & b) $\int\limits_{-3}^{3} x^2 ~{\rm d} x $& 
 c) $\int\limits_{-1}^{5} x^2 ~{\rm d} x $ &  d) $\int\limits_{-3}^{6} {x^2 \over 6} ~{\rm d} x $ &
 e) $\int\limits_{0}^{2} x ~{\rm d} x $ \\
 f) $\int\limits_{-1}^{2} 2x ~{\rm d} x $ &  g) $\int\limits_{0}^{0,5} ~{\rm d} x $ &  
 h) $\int\limits_{-4}^{4} ~{\rm d} x $ &  i) $\int\limits_{-1}^{-3} -2 ~{\rm d} x $ & 
 j) $\int\limits_{-3}^{9} { 1 \over 3} ~{\rm d} x $ \\
\end{tabular}

\begin{tabular}{ccc}
 k) $\int\limits_{0}^{0,5} (8x-4) ~{\rm d} x $ &  l) $\int\limits_{-3}^{3} (x^3+2x^5) ~{\rm d} x $ &
 m) $\int\limits_{-1}^{-2} (6x-4x^3) ~{\rm d} x $ \\
 n) $\int\limits_{-1}^{1} |x^3| ~{\rm d} x $ &  o) $\left| \int\limits_{-3}^{3} x^2 ~{\rm d} x \right| $ &
 p)  $\int\limits_{0}^{\pi} (\sin x)^2 ~{\rm d} x+ \int\limits_{0}^{\pi}  (\cos x)^2 ~{\rm d} x $
\end{tabular}


\framebox{Integralwerte zu 1.: -6; -1; 0; $ {1\over 3} $;  $ {1\over 2}$; 2; 3; $\pi$; 4; 8; 13,5; 18; 42} 

%Aufgabe 2
{\bf \item Ermitteln Sie jeweils den Wert des Parameters $ k \in \mathbb{R}^+$.}\\
 a) $\int\limits_{-k}^{k} 3x^2 ~{\rm d} x=16  \qquad \textnormal{b)} ~\int\limits_{-2}^{4} kx ~{\rm d} x=18
\qquad \textnormal{c)} \int\limits_{-2}^{0} (-kx) ~{\rm d} x + \int\limits_{0}^{4} kx ~{\rm d} x =18 $\\
Vergleichen Sie die Teilaufgaben b) und c) und erläutern Sie die Ergebnisse.

\framebox{Parameterwerte zu 2.: 1,8; 2; 3}

%Aufgabe 3
\begin{minipage}{0.7\textwidth}
{\bf \item Der Graph $G_f$ der in ganz $\mathbb{R}$ integrierbaren Funktion $f$ (vgl. nebenstehende Abbildung) ist punktsymmetrisch zum Ursprung.} Geben Sie jeweils an, ob der Wert des Integrals kleiner als null, gleich null oder größer als null ist.
a) $\int\limits_{-1}^{1} f(x) ~{\rm d} x \quad$  b) $\int\limits_{-1}^{1,5} f(x) ~{\rm d} x $ \\
c) $\int\limits_{-1,5}^{1} f(x) ~{\rm d} x \quad $  d) $\int\limits_{a}^{1} f(x) ~{\rm d} x; ~ -1,5 \le a < -1$ 
\end{minipage}
\hfill
\begin{minipage}{0.3\textwidth}
\includegraphics[width=3.5 cm]{drei180}
\end{minipage}



%Aufgabe 4
{\bf \item Finden Sie durch Überlegen heraus, welche der neun Aussagen falsch sind und geben Sie jeweils die richtige Lösung an.}

\begin{tabular}{ccc}
a) $\int\limits_{0}^{4} x^4 ~{\rm d} x <0 $ & b) $\int\limits_{4}^{-4} x^4 ~{\rm d} x <0 $ & 
c) $\int\limits_{-2}^{4} 2 ~{\rm d} x = 8 $ \\
d) $\int\limits_{4}^{0} 2 ~{\rm d} x =-8 $  & e) $\int\limits_{-\pi}^{\pi} \sin{x} ~{\rm d} x =0 $ &
f) $\int\limits_{-\pi}^{\pi} \cos{x} ~{\rm d} x >0 $ \\
g) $\int\limits_{-2}^{4} 0 ~{\rm d} x =6 $ & h) $\int\limits_{3}^{-3} x ~{\rm d} x =0 $ &
 i) $\int\limits_{1}^{-1} (-x^2) ~{\rm d} x >0 $ 
\end{tabular}

\newpage

%Aufgabe 5
{\bf \item Die Abbildung zeigt den Graphen $G_f$ der Funktion} \\
$f(x)=(x-2)^2-1; \quad D_f=\mathbb{R}$

\begin{enumerate}[a)]
\begin{minipage}{0.6\textwidth}
\item Geben Sie je zwei Intervalle $[a;b]$ an mit:\\
i)  $\int\limits_{a}^{b} f(x) ~{\rm d} x >0 $ \\
ii)  $\int\limits_{a}^{b} f(x) ~{\rm d} x <0 $ 
\end{minipage}
\hfill
\begin{minipage}{0.35\textwidth}
\includegraphics[width=4 cm]{zwei180}
\end{minipage}


\item Berechnen Sie   $\int\limits_{0}^{2} f(x) ~{\rm d} x  $ sowie den Flächeninhalt des getönten Bereichs; vergleichen und erläutern Sie die beiden Ergebnisse.
\end{enumerate}

%Aufgabe 6
{\bf \item Finden Sie heraus, für welchen Wert / welche Werte von $b$ gilt:}\\

a) $\int\limits_{0}^{b} x ~{\rm d} x =8  \quad$ b) $\int\limits_{0}^{b} x ~{\rm d} x =18  \quad$
c) $\int\limits_{0}^{b} 2x^2 ~{\rm d} x =18  \quad$ d) $\int\limits_{0}^{b} x^2 ~{\rm d} x =576  \quad$

%Aufgabe 7
\includegraphics[width=12 cm]{profil180}

{\bf \item Die Abbildung (Längeneinheit: Meter) zeigt das sichelförmige Profil eines Stahlträgers, das von zwei Parabelbögen berandet wird.} Der obere Parabelbogen ist der Graph $G_f$ der Funktion $f$ mit\\ $f(x)=-0,0375x^2 + 2,4; \\
D_f=[x_A;x_B]$; der untere Parabelbogen ist der Graph $G_g$ der Funktion $g$ mit $g(x)= ax^2 +0,8; \quad D_f=[x_A;x_B]$.

\begin{enumerate}[a)]
\item  Ermitteln Sie zunächst die Spannweite $\overline{AB}$ des Trägers und dann $a$ .
\item berechnen Sie den Flächeninhalt des Trägerprofils.
\end{enumerate}

\end{enumerate} 

\fbox{
	\begin{minipage}{0.5\textwidth}
		Weitere Arbeitsblätter gibt es unter 
		
		\href{https://www.okuyakl.de}{www.okuyakl.de}
	\end{minipage}
	\hfill
	\begin{minipage}{0.4\textwidth}
		\includegraphics[width=1.5 cm]{../../viecher/zwe03}
		%\includegraphics[width=3 cm]{qr128}
		\includegraphics[width=2 cm]{../../viecher/afanticon1}
		
\end{minipage}}

%Ende
\end{document}