\documentclass[a4paper]{article}
\usepackage[pdftex]{graphicx}
\usepackage[utf8]{inputenc}
\usepackage{enumerate}
\usepackage{icomma}
\usepackage{amssymb}
\usepackage{href-ul}
\hypersetup{
	colorlinks=true,
	linkcolor=blue,
	urlcolor=blue}

\begin{document}
	%Title
	{\bf Training sheet Integrals }\\
	\begin{enumerate}[1.]
		
		%Task 1
		{\bf \item Calculate the value of the integral:}
		
		\begin{tabular}{ccccc}
			a) $\int\limits_{0}^{1} x^2 ~{\rm d} x$ & b) $\int\limits_{-3}^{3} x^2 ~{\rm d} x $&
			c) $\int\limits_{-1}^{5} x^2 ~{\rm d} x $ & d) $\int\limits_{-3}^{6} {x^2 \over 6} ~{\rm d} x $ &
			e) $\int\limits_{0}^{2} x ~{\rm d} x $ \\
			f) $\int\limits_{-1}^{2} 2x ~{\rm d} x $ & g) $\int\limits_{0}^{0.5} ~{\rm d} x $ &
			h) $\int\limits_{-4}^{4} ~{\rm d} x $ & i) $\int\limits_{-1}^{-3} -2 ~{\rm d} x $ &
			j) $\int\limits_{-3}^{9} { 1 \over 3} ~{\rm d} x $ \\
		\end{tabular}
		
		\begin{tabular}{ccc}
			k) $\int\limits_{0}^{0.5} (8x-4) ~{\rm d} x $ & l) $\int\limits_{-3}^{3} (x^3+ 2x^5) ~{\rm d} x $ &
			m) $\int\limits_{-1}^{-2} (6x-4x^3) ~{\rm d} x $ \\
			n) $\int\limits_{-1}^{1} |x^3| ~{\rm d} x $ & o) $\left| \int\limits_{-3}^{3} x^2 ~{\rm d} x \right| $ &
			p) $\int\limits_{0}^{\pi} (\sin x)^2 ~{\rm d} x+ \int\limits_{0}^{\pi} (\cos x)^2 ~{ \rm d} x $
		\end{tabular}
		
		
		\framebox{Integral values for 1.: -6; -1; 0; $ {1\over 3} $; ${1\over 2}$; 2; 3; $\pi$; 4; 8th; 13.5; 18; 42}
		
		%Exercise 2
		{\bf \item Determine the value of the parameter $ k \in \mathbb{R}^+$.}\\
		a) $\int\limits_{-k}^{k} 3x^2 ~{\rm d} x=16 \qquad \textnormal{b)} ~\int\limits_{-2}^{4} kx ~ {\rm d} x=18
		\qquad \textnormal{c)} \int\limits_{-2}^{0} (-kx) ~{\rm d} x + \int\limits_{0}^{4} kx ~{\rm d} x =18$\\
		Compare subtasks b) and c) and explain the results.
		
		\framebox{Parameter values for 2.: 1.8; 2; 3}
		
		%Task 3
		\begin{minipage}{0.7\textwidth}
			{\bf \item The graph $G_f$ of the function $f$, which can be integrated into the entire $\mathbb{R}$ (see figure opposite), is point-symmetrical to the origin.} Indicate in each case whether the value of the integral is less than zero , equal to zero or greater than zero.
			a) $\int\limits_{-1}^{1} f(x) ~{\rm d} x \quad$ b) $\int\limits_{-1}^{1.5} f(x) ~{\rm d} x $ \\
			c) $\int\limits_{-1.5}^{1} f(x) ~{\rm d} x \quad $ d) $\int\limits_{a}^{1} f(x) ~ {\rm d} x; ~ -1.5 \le a < -1$
		\end{minipage}
		\hfill
		\begin{minipage}{0.3\textwidth}
			\includegraphics[width=3.5 cm]{drei180}
		\end{minipage}
		
		
		
		%Task 4
		{\bf \item Find out by reflection which of the nine statements are false and give the correct solution for each.}
		
		\begin{tabular}{ccc}
			a) $\int\limits_{0}^{4} x^4 ~{\rm d} x <0 $ & b) $\int\limits_{4}^{-4} x^4 ~{\rm d} x <0$ &
			c) $\int\limits_{-2}^{4} 2 ~{\rm d} x = 8 $ \\
			d) $\int\limits_{4}^{0} 2 ~{\rm d} x =-8 $ & e) $\int\limits_{-\pi}^{\pi} \sin{x} ~ {\rm d} x =0 $ &
			f) $\int\limits_{-\pi}^{\pi} \cos{x} ~{\rm d} x >0 $ \\
			g) $\int\limits_{-2}^{4} 0 ~{\rm d} x =6 $ & h) $\int\limits_{3}^{-3} x ~{\rm d} x =0$ &
			i) $\int\limits_{1}^{-1} (-x^2) ~{\rm d} x >0 $
		\end{tabular}
		
		\newpage
		
		%Task 5
		{\bf \item The figure shows the graph $G_f$ of the function} \\
		$f(x)=(x-2)^2-1; \quad D_f=\mathbb{R}$
		
		\begin{enumerate}[a)]
			\begin{minipage}{0.6\textwidth}
				\item Specify two intervals $[a;b]$ with:\\
				i) $\int\limits_{a}^{b} f(x) ~{\rm d} x >0 $ \\
				ii) $\int\limits_{a}^{b} f(x) ~{\rm d} x <0 $
			\end{minipage}
			\hfill
			\begin{minipage}{0.35\textwidth}
				\includegraphics[width=4 cm]{zwei180}
			\end{minipage}
			
			
			\item Calculate $\int\limits_{0}^{2} f(x) ~{\rm d} x $ and the area of the tinted area; compare and explain the two results.
		\end{enumerate}
		
		%Task 6
		{\bf \item Find out which value(s) of $b$ is valid for:}\\
		
		a) $\int\limits_{0}^{b} x ~{\rm d} x =8 \quad$ b) $\int\limits_{0}^{b} x ~{\rm d} x = 18\quad$
		c) $\int\limits_{0}^{b} 2x^2 ~{\rm d} x =18 \quad$ d) $\int\limits_{0}^{b} x^2 ~{\rm d} x =576 \quad$
		
		%Task 7
		\includegraphics[width=12 cm]{profil180}
		
		{\bf \item The figure (length unit: meter) shows the crescent-shaped profile of a steel beam, which is bordered by two parabolic arcs.} The upper parabolic arc is the graph $G_f$ of the function $f$ with\\ $f(x)= -0.0375x^2 + 2.4; \\
		D_f=[x_A;x_B]$; the lower parabolic arc is the graph $G_g$ of the function $g$ with $g(x)= ax^2 +0.8; \quad D_f=[x_A;x_B]$.
		
		\begin{enumerate}[a)]
			\item First find the span $\overline{AB}$ of the beam and then $a$ .
			\item calculate the area of the support profile.
		\end{enumerate}
		
	\end{enumerate}
	
	\fbox{
		\begin{minipage}{0.5\textwidth}
			Additional worksheets are available at
			
			\href{http://www.okuyakl.com}{www.okuyakl.com}
		\end{minipage}
		\hfill
		\begin{minipage}{0.4\textwidth}
			\includegraphics[width=1.5 cm]{../../viecher/zwe03}
			%\includegraphics[width=3 cm]{qr128}
			\includegraphics[width=2 cm]{../../viecher/afanticon1}
			
	\end{minipage}}
	
	%End
\end{document}