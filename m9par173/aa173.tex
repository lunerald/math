
\documentclass[a4paper]{article}
\usepackage[pdftex]{graphicx}
\usepackage[utf8]{inputenc}
\usepackage{enumerate}
\usepackage{amssymb}
\usepackage{colortbl}
\usepackage{icomma}
\usepackage{siunitx}
\sisetup{locale=DE} 
\usepackage{geometry}
\geometry{a4paper, top=15mm, left=10mm, right=15mm, bottom=15mm,
	headsep=10mm, footskip=12mm}
\usepackage{href-ul}
\hypersetup{
	colorlinks=true,
	linkcolor=blue,
	urlcolor=blue}
\begin{document}

	
\begin{enumerate}[1.]
		
\begin{minipage}{0.6\textwidth}
{\bf \item Parabeln lassen sich einfach in y-Richtung verschieben.} Im rechten Schaubild sei die oberste Parabel $f_1$, die unterste $f_5$.
Der Funktionsterm ist dann $f(x)=x^2 + c \qquad c \in \mathbf{R}$; die Wertemenge $\mathbf{W_f} =[c \, ; +\infty [$
Lies $c$ aus dem Schaubild ab und fülle die Tabelle aus!
\vspace{0.5 cm}

\renewcommand{\arraystretch}{2}
\begin{tabular}{|>{\columncolor[gray]{.8}}p{1 cm}|p{1.8 cm}|p{2.5 cm}|p{2.5 cm}|p{2.5 cm}|}
	\hline
	\rowcolor[gray]{.8}	&Scheitel $S(0|c)$ &Funktionsterm & Wertemenge $W_f$ & Nullstellen $N(x|0)$\\
	\hline
 $f_1$ & & & & \\
	\hline
 $f_2$ & & & &\\
	\hline
 $f_3$ & & & & \\
	\hline
 $f_4$ & & & & \\
 \hline 
 $f_5$ & & & & \\
 \hline
 
\end{tabular}
\vspace{0.5 cm}

{\bf \item Im unteren Schaubild sind in y-Richtung gestreckte und ge\-stauchte Parabeln} abgebildet. Lies die Koordinaten markanter Punkte aus dem Schaubild ab, berechne $a$ mit $y=a\cdot x^2 $ und fülle die Tabelle aus! 
\vspace{0.5 cm}
	
\renewcommand{\arraystretch}{2}
\begin{tabular}{|>{\columncolor[gray]{.8}}p{1 cm}|p{2.5 cm}|p{1.8 cm}|p{2.5 cm}|}
	\hline
	\rowcolor[gray]{.8}	& Markanter Punkt $(x|y)$ & Öffnung a &Funktionsterm\\
	\hline
	$f_6$ & & &  \\
	\hline
	$f_7$ & & &\\
	\hline
	$f_8$ & & &\\
	\hline
	$f_9$ & & & \\
	\hline 
	$f_{10}$ & & & \\
	\hline
	
\end{tabular}


\end{minipage}
\hfill
\begin{minipage}{0.3\textwidth}
\includegraphics[width=6 cm]{pasch173}
\end{minipage}		


\end{enumerate} 
	
	
		\begin{minipage}{0.5\textwidth}
		\includegraphics[width=9 cm]{past173}
		\end{minipage}
		\hfill
		\fbox{\begin{minipage}{0.4\textwidth}
				Weitere Arbeitsblätter gibt es unter 
			
			\href{https://www.okuyakl.de}{www.okuyakl.de}
			\includegraphics[width=1.5 cm]{../../viecher/zwe03}
			\includegraphics[width=2 cm]{../../viecher/afanticon1}
		\end{minipage}}
	
\end{document}