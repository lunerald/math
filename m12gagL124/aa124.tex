\documentclass[a4paper]{article}
\usepackage[pdftex]{graphicx}
\usepackage[utf8]{inputenc}
\usepackage{enumerate}
\usepackage{amssymb}
\usepackage{icomma}
\usepackage{geometry}
\geometry{a4paper, top=15mm, left=15mm, right=15mm, bottom=15mm,
	headsep=10mm, footskip=12mm}
\usepackage{href-ul}
\hypersetup{
	colorlinks=true,
	linkcolor=blue,
	urlcolor=blue}
	
\begin{document}
	%Titel
	{\bf Mathematik -- Geraden im $\mathbb{R}^3$}\\
	\begin{enumerate}[1.]
	{\bf \item Aufgabe}\\ 
	\vspace{0.25 cm}\\
		Die Gerade g enthält die Punkte $A(2|0|-1)$ und $B(4|5|-3)$.
		\begin{enumerate}[a)] 
		   \item Entscheiden Sie, ob $C(-2|-10|3)$ zur Geraden g gehört. 
		   \item Entscheiden Sie, ob $D(-4|15|5)$ ein Punkt der Geraden g ist. 
    	\end{enumerate}
	{\bf \item Aufgabe}\\
	\vspace{0.25 cm}\\
		Auf der Verbindungsgeraden der Punkte $A(3|6|9)$ und $B(5|7|6)$ liegt $C(?|?|4,5)$. Bestimmen Sie 
		die Koordinaten von C.
	\vspace{0.25 cm}\\ 
	{\bf \item Aufgabe} \\
	\vspace{0.25 cm}\\
		Die Gerade g enthält die Punkte $A(2|0|-3)$ und $B(5|-2|-1)$. Die Gerade h enthält den Punkt 
		$C(4|3|0)$ und ist zur Geraden g parallel. Entscheiden Sie, ob der Punkt $D(-2|7|-4)$ zur Geraden h 
		gehört. 
	\vspace{0.25 cm}\\
	{\bf \item Aufgabe}\\
	\vspace{0.25 cm}\\
		Die Punkte $A(1|2|3)$, $B(4|b|4)$ und $C(7|6|c)$ liegen auf einer Geraden g. Bestimmen Sie b und c. 
		Geben Sie eine Parameterdarstellung der Geraden g an und entscheiden Sie mit ihrer Hilfe, ob
		$P(-2|0|1)$ zur Geraden g gehört. 
	\vspace{0.25 cm}\\
	{\bf \item Aufgabe}\\ 
	\vspace{0.25 cm}\\
		Die zur y-Achse parallele Gerade g geht durch $P(1|3|2)$. Bestimmen Sie a und c so, dass der Punkt 
		$Q(a|5|c)$ zur Geraden g gehört. 
	\vspace{0.25 cm}\\
	{\bf \item Aufgabe}\\
	\vspace{0.25 cm}\\ 
		Zur Geraden g gehören die Punkte $A(6|4|2)$ und $B(8|8|8)$. 
		\begin{enumerate}[a)]
		   \item Kann c so gewählt werden, dass der Punkt $C(3|2|c)$ zu g gehört? 
		   \item Kann d so gewählt werden, dass $D(3|-2|d)$ zur Geraden g gehört? 
	       \item Kann e so gewählt werden, dass
		$\left(
		\begin{array}{c}
		e \\ -3 \\ -4,5\\
		\end{array}
		\right)$
		ein Richtungsvektor von g ist? 
	    \end{enumerate}
    \end{enumerate}
    \vspace{0.7 cm}
	\fbox{
	\begin{minipage}{0.5\textwidth}
		Zur Lösung bitte \href{https://www.okuyakl.de/math/m12gagL124/ll124.pdf}{hier klicken} oder den QR-Code scannen.\\
	Weitere Arbeitsblätter gibt es unter 
	
	\href{https://www.okuyakl.de}{www.okuyakl.de}
	\end{minipage}
	\hfill
	\begin{minipage}{0.4\textwidth}
		\includegraphics[width=1.5 cm]{../../viecher/zwe03}
		\includegraphics[width=3 cm]{qr124}
		\includegraphics[width=2 cm]{../../viecher/afanticon1}
		
	\end{minipage}}
\end{document}
