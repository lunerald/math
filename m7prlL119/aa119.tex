\documentclass[a4paper]{article}
\usepackage[pdftex]{graphicx}
\usepackage[utf8]{inputenc}
\usepackage{enumerate}
\usepackage{amssymb}
\usepackage{icomma}
\usepackage{siunitx}
\sisetup{locale=DE} 
\usepackage{geometry}
\geometry{a4paper, top=15mm, left=15mm, right=15mm, bottom=15mm,
	headsep=10mm, footskip=12mm}
\usepackage{href-ul}
\hypersetup{
	colorlinks=true,
	linkcolor=blue,
	urlcolor=blue}
	
\begin{document}
{\bf Parallelverschiebung und Vektorrechnung}

\begin{enumerate}[1.]

{\bf \item Berechne folgende Terme:}

\renewcommand{\arraystretch}{2}
\begin{tabular}{p{5 cm}p{5 cm}p{5 cm}}
a) ${8 \choose -2} +{-1 \choose 5} $	& b) ${6 \choose 3} -{-3 \choose 1} $ &c) ${0 \choose 4} + \left[ {1 \choose 2}  + {-1 \choose 2}\right]$ \\
d) $- {11\choose 10} -2\cdot {-3 \choose 4} + { 7 \choose -7 }$ & e) $2\cdot {-7,5 \choose 0,5} +\left[{10 \choose 15} -{-2 \choose 5}\right]$	&f) $\left[ {0\choose 2}  + {-2 \choose 0}\right]-{5 \choose -1} $ \\
g) ${0,5 \choose 2} - \left[ {1,5 \choose 0}  + {2 \choose 2}\right]$ & h) $- \left[{10\choose 7} -2\cdot {0 \choose 4} \right]+ {-5 \choose 5 }$ & i) $ 2\cdot{4\choose 4} -\left[ {-6\choose 4} + { 9 \choose 0 }\right]$ \\
\end{tabular}
\vspace{0.5 cm}

{\bf \item Löse folgende Gleichungen:}

\renewcommand{\arraystretch}{2}
\begin{tabular}{p{5 cm}p{5 cm}p{5 cm}}
a) $\vec{x} - {3\choose 2}={7 \choose -1}$	&b) ${6\choose 9}+\vec{x}-{5\choose 5} = \vec{0}$ &c) $2 \cdot \vec{x} = {1 \choose 0}-{-7 \choose-8 }$ \\
d) ${2,5 \choose 3 }- \vec{x} = 0,5 \cdot {7 \choose -10 }$& e) ${0\choose 1}- \left[\vec{x}-{2\choose 2}\right] = \vec{0}$ &f) $0,5 \cdot \vec{x} + {8 \choose 6}={3 \choose3 }$ \\
\end{tabular}
\vspace{0.5 cm}

{\bf \item Gegeben sind folgende Punkte:} $A(-3|1)$ $B(0|2)$, $C(5|4)$ und $D(-2|-7)$
\begin{enumerate}[a)]
	\item Berechne die Vektoren $\overrightarrow{AB}$,  $\overrightarrow{AC}$,  $\overrightarrow{AD}$,  $\overrightarrow{BC}$,  $\overrightarrow{BD}$,  $\overrightarrow{CD}$ und ihre Gegenvektoren!
	\item  Folgende Linearkombinationen sind gegeben. Überlege, ob ein Vektor aus a) dargestellt wird und gib diesen dann an. 
	
\renewcommand{\arraystretch}{2}
	\begin{tabular}{p{3.5 cm}p{3.5 cm}p{3.5 cm}p{3.5 cm}}
	i) $\overrightarrow{AB} + \overrightarrow{BC}$	& ii) $\overrightarrow{AD} + \overrightarrow{DA}$ &iii) $\overrightarrow{BD}-\overrightarrow{CD}$  &iv) $\overrightarrow{BD} + \overrightarrow{AB}$\\
	v) $\overrightarrow{CD} + \overrightarrow{DA} + \overrightarrow{AC}$ & vi) $-\overrightarrow{AC} + \overrightarrow{AD} $ & vii) $\overrightarrow{BC} + \overrightarrow{CD} + \overrightarrow{DA}$ & viii) $\overrightarrow{BD} + \overrightarrow{AC} - \overrightarrow{AD} $
	\end{tabular}
\end{enumerate} 
 
\vspace{0.5 cm}

{\bf \item Berechne die fehlenden Koordinaten }($M$ = Mittelpunkt)

\renewcommand{\arraystretch}{2}
\begin{tabular}{p{5 cm}p{5 cm}p{5 cm}}
a) $A(-8|4)$ $B(0|2)$ $M(?|?)$ &b) $C(7|3)$ $D(?|?)$ $M(0|1)$&c) $E(?|?)$ $F(1|2)$ $M(1,5|3)$ \\
d) $G(?|0)$ $H(1|-2)$ $M(3|?)$ &e) $I(-5|5)$ $J(1|?)$ $M(?|3)$&f) $K(-4|?)$ $L(-2|2)$ $M(?|0)$ \\
\end{tabular}
\vspace{0.5 cm}

{\bf \item Fülle die Tabelle aus}

\renewcommand{\arraystretch}{2}
\begin{tabular}{|p{1 cm}|p{2 cm}|p{2 cm}|p{2 cm}|p{2 cm}|p{2 cm}|p{2 cm}|}
	\hline
$P$	&$(2|-1)$ & &$(-3|-3)$ & &$(9|0)$ & \\
	\hline
$Q$	&$(5|2)$ &$(-7|6)$& &$(0,5|1,5)$ & &$(-4|-7)$ \\
	\hline
$\overrightarrow{PQ}$	& &${0 \choose -1}$ & &${-1,5 \choose -2,5}$ &${-3 \choose 8}$  &\\
	\hline
$-\overrightarrow{PQ}$	& & &${-8 \choose 5}$ & &&${6 \choose 0}$  \\
	\hline
\end{tabular}
\vspace{0.5 cm}

{\bf \item Das Dreieck ABC mit } $A(-5|-0,5)$ $B(-2|0)$ und $C(-3,5|2)$ wird mit dem Vektor $\vec{a}={4 \choose 3}$ auf das Dreieck $A'B'C'$ abgebildet. Anschließend wird das Dreieck $A'B'C'$ mit dem Vektor $\vec{b}={-1 \choose -5}$ auf das Dreieck $A''B''C''$ verschoben. Zeichne das Dreieck $ABC$ in ein geeignetes Koordinatensystem und führe die Verschiebungen durch. Gib die Koordinaten des Dreieckpunkte $A''$ $B''$ und $C''$ an. 
\end{enumerate} 

\fbox{
	\begin{minipage}{0.5\textwidth}
		Zur Lösung bitte \href{https://www.okuyakl.de/math/m7prlL119/ll119.pdf}{hier klicken} oder den QR-Code scannen.\\
	Weitere Arbeitsblätter gibt es unter 
	
	\href{https://www.okuyakl.de}{www.okuyakl.de}
	\end{minipage}
	\hfill
	\begin{minipage}{0.4\textwidth}
		\includegraphics[width=1.5 cm]{../../viecher/zwe03}
		\includegraphics[width=3 cm]{qr119}
		\includegraphics[width=2 cm]{../../viecher/afanticon1}
		
	\end{minipage}}
\end{document}

