\documentclass[a4paper]{article}
\usepackage[pdftex]{graphicx}
\usepackage[utf8]{inputenc}
\usepackage{enumerate}
\usepackage{amssymb}
\usepackage{geometry}
\geometry{a4paper, top=10mm, left=15mm, right=15mm, bottom=10mm,
	headsep=10mm, footskip=12mm}
\usepackage{href-ul}
\hypersetup{
	colorlinks=true,
	linkcolor=blue,
	urlcolor=blue}
% Start the document
\begin{document}
%Titel
{\bf Trainingsblatt Sinusfunktion }\\

\noindent{\bf Gegeben sind folgende periodische Funktionen: $f_1(x)$ bis $f_5(x)$, von oben nach unten.}

\noindent\includegraphics[width=17 cm]{vs014}
\begin{enumerate}[1.]
%Aufgabe 1
{\bf \item Bestimme alle Parameter für eine Sinusfunktion und fülle die Tabelle aus:}

\renewcommand{\arraystretch}{1.8}
\begin{tabular}{|p{40 pt}|p{45 pt}|p{45 pt}|p{45 pt}|p{70 pt}|p{70 pt}|p{110 pt}|}
\hline
Funktion & Amplitude & Periode& Wellenzahl & x-Verschiebung & y-Verschiebung& Funktionsterm f(x)=\\
   &  $A$ &  $P$ &  $b$ &  ${c\over b}$ &  $d$ & $a\cdot \sin(bx+c)+d$\\
\hline
$f_1(x)$ & & & & & & \\
\hline
$f_2(x)$ & & & & & & \\
\hline
$f_3(x)$ & & & & & & \\
\hline
$f_4(x)$ & & & & & & \\
\hline
$f_5(x)$ & & & & & & \\ 
\hline
\end{tabular}

{\bf \item Gib diese Funktionen nun als Kosinusfunktion an:}

\begin{tabular}{|p{40 pt}|p{70 pt}|p{90 pt}|p{110 pt}|}
\hline
Funktion &Amplitude  $ A$& x-Verschiebung  ${c\over b}$ & Funktionsterm f(x)=\\
\hline
$f_1(x)$ & & & \\
\hline
$f_2(x)$ & &  & \\
\hline
$f_3(x)$ & & &\\
\hline
$f_4(x)$ & & &\\
\hline
$f_5(x)$ & & &\\ 
\hline
\end{tabular}

\end{enumerate}

\fbox{
	\begin{minipage}{0.5\textwidth}
	Zur Lösung bitte \href{https://www.okuyakl.de/math/m10sifuL014/ll014.pdf}{hier klicken} oder den QR-Code scannen.\\
Weitere Arbeitsblätter gibt es unter 

\href{https://www.okuyakl.de}{www.okuyakl.de}
	\end{minipage}
	\hfill
	\begin{minipage}{0.4\textwidth}
		\includegraphics[width=1.5 cm]{../../viecher/zwe03}
		\includegraphics[width=3 cm]{qr014}
		\includegraphics[width=2 cm]{../../viecher/afanticon1}
		
	\end{minipage}}
\end{document}%Lösung--------------------------------------------------------------
