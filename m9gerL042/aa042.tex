\documentclass[a4paper]{article}
\usepackage[pdftex]{graphicx}
\usepackage[utf8]{inputenc}
\usepackage{enumerate}
\usepackage{icomma}
\usepackage{amssymb}
\usepackage{geometry}
\geometry{a4paper, top=15mm, left=15mm, right=15mm, bottom=15mm,
	headsep=10mm, footskip=12mm}
\usepackage{href-ul}
\hypersetup{
	colorlinks=true,
	linkcolor=blue,
	urlcolor=blue}
	
\begin{document}
{\bf Geraden, lineare Funktionen}
\begin{enumerate}[1.]



{\bf \item Die Gerade $g$ wird durch die Punkte $A(-3|-1)$ und $B(2|3)$ festgelegt.}

\begin{enumerate}[a)]
\item Bestimme rechnerisch die Funktionsgleichung der Geraden $g$. \\
\item Zeichne die Gerade $g$ in ein Koordinatensystem.\\
(Längeneinheit $1~cm; \quad -4 \le x \le 5; \quad -5 \le y \le 4 $)
\item Berechne die fehlenden Koordinaten der Punkte $P$ und $Q$, die auf der Geraden $g$ liegen:
$$ P(x_p|5) \quad Q(-5|y_Q)$$
\item Prüfe rechnerisch, ob die Punkte $E(-1|0,7)$ und $F(20|17,4)$ auf dem Graphen von $g$ liegen. 

\item Gib die Funktionsgleichung der Geraden $h$ an, die orthogonal zu $g$ durch den Punkt 
$C(0|2)$ verläuft. Zeichne die Gerade $h$ in das Koordinatensystem aus Aufgabe b) ein. 

\item Ermittle rechnerisch die Funktionsgleichung der Geraden $k$, die ebenfalls orthogonal zu $g$ ist und durch den Punkt $D(-1|1)$ verläuft. Zeichne auch diese Gerade in das Koordinatensystem ein.

\item Gib rechnerisch die Nullstelle der Geraden $g$ an.

\item Erstelle die Funktionsgleichung der Geraden $n$, die parallel zu $g$ durch den Punkt $Y(0|-1)$ verläuft. Zeichne die Gerade $n$ in das Koordinatensystem aus b). 

\item Zeichne weitere Geraden ein:

\begin{tabular}{lll}
t: & $y=-3$ & \\
s: & $y+5 = 0,6x$ & (Tipp: forme nach $y=mx+t$ um)\\
u: & $y=-1,5(x-1)+2,5$ & (Tipp: löse die Klammer auf)\\
v: & $x=1$ & 
\end{tabular}
\end{enumerate}

{\bf \item Gegeben ist die Gerade $g_1$ mit $y = 0,4x + 4 $}

\begin{enumerate}[a)]
\item Zeichne die Gerade in ein Koordinatensystem ein.
\item Der y-Achsenabschnitt einer zu $g_1$ parallelen Gerade hat den Wert $1,5$. Zeichne diese Gerade $g_2$ in das gleiche Koordinatensystem ein und gib die zugehörige Funktionsgleichung an.
\item Eine weitere parallele Gerade verläuft durch den Punkt $R(-4,5|-5)$ Zeichne diese Gerade $g_3$ in das gleiche Koordinatensystem ein und bestimme die zugehörige Funktionsgleichung.
\item Überprüfe rechnerisch, ob der Punkt $A \left( \frac{7}{3}|4,9\bar{3}\right)$ auf der Geraden $g_1$ liegt.
\end{enumerate}

{\bf \item Bestimme die Gleichungen zu:}

\begin{enumerate}[a)]
\item $g_4$: Parallele zur x-Achse durch $P(-1|5)$
\item $g_5$: Winkelhalbierende des II. und IV. Quadranten
\item $g_5$: Gerade, die durch $D(-3|1)$ und $B(7|-2)$ verläuft 
\end{enumerate}

{\bf \item Gegeben ist die Gerade $g_6$ mit der Geradengleichung $ g_6: y = - \frac{3}{4}x + 2 $}

\begin{enumerate}[a)]
\item Berechne die Schnittpunkte $ S_y $ und $ S_x $ der Geraden $g_6$ mit beiden Koordinatenachsen.
\item Der Punkt $ H(x_H|-\frac{1}{4})$ liegt auf der Geraden $g_6$. Berechne die fehlende Koordinate $x_H$
\end{enumerate}




\fbox{
	\begin{minipage}{0.5\textwidth}
		Zur Lösung bitte \href{https://www.okuyakl.de/math/m9gerL042/ll042.pdf}{hier klicken} oder den QR-Code scannen.\\
	Weitere Arbeitsblätter gibt es unter 
	
	\href{https://www.okuyakl.de}{www.okuyakl.de}
	\end{minipage}
	\hfill
	\begin{minipage}{0.4\textwidth}
		\includegraphics[width=1.5 cm]{../../viecher/zwe03}
		\includegraphics[width=3 cm]{qr042}
		\includegraphics[width=2 cm]{../../viecher/afanticon1}
		
	\end{minipage}}
\end{enumerate} 

\end{document}%Lösung-------------------------------------------
