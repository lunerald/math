\documentclass[a4paper]{article}
\usepackage[pdftex]{graphicx}
\usepackage[utf8]{inputenc}
\usepackage{enumerate}
\usepackage{amssymb}
\usepackage{icomma}
\usepackage{siunitx}
\sisetup{                     
	locale=DE} 
\usepackage{eurosym}
\usepackage{geometry}
\geometry{a4paper, top=15mm, left=15mm, right=15mm, bottom=15mm,
	headsep=10mm, footskip=12mm}
\usepackage{href-ul}
\hypersetup{
	colorlinks=true,
	linkcolor=blue,
	urlcolor=blue}
\begin{document}
{\bf Gleichungen}
\begin{enumerate}[1.]
%Aufgabe 1
{\bf \item Ermittle jeweils die Lösungsmenge der Gleichung über der Grundmenge $\mathbb{Q}$.} Kürze vollständig!

\renewcommand{\arraystretch}{2}
\begin{tabular}{p{5.5 cm}p{5.5 cm}p{5.5 cm}}
	a) $-x +5 = -17 $ &  b) $ -31 -x = 44 $ &c) $ 4- (x+7) = 15$ \\
	d) $ ( 81 -x) +10 = 1$ &  e) $ 45 +(-7) = -(2-x)$ &f) $ (-3 +x -17) +21 =0 $ \\
	g) $ 12 + (-x + 8) = -10 $ &  h) $70 - (-x) =-(23 +17) $ &i) $-3 + (x-29) = -82 $ \\
\end{tabular}
\vspace{0.5 cm}

{\bf \item Berechne $x$}
\begin{enumerate}[a)]

\item $$ 7 \cdot \left(x - 14 \right) - 5 \cdot \left( 5 + x \right) = x - 122 $$
\item $$ 2-[3 \cdot (5x-11) + 7 ] = -2x + 6 - ( 2 - 5x ) $$
\item $$ \frac{22 + 8x}{18} =  \frac{10x - 11}{6} $$

\end{enumerate}

{\bf \item Dividiert man die Differenz aus einer Zahl und 2 durch 4,} so erhält man 3 mehr, als wenn man die Summe aus der Zahl und 2 durch 8 dividiert. {\it Lege die Variable fest, stelle die Gleichung auf und löse sie!}

{\bf \item Löse die Gleichungen und Ungleichungen über der angegebenen Grundmenge und gib die Lösungsmenge an !}

\begin{enumerate}[a)]
\item $ (x - 2)(4x + 2) +4 < (2x + 3)^2 \quad \mathbb{G} = \mathbb{Q} $
\item $ c + (2,5 + c - 0,5) \cdot 3 = - 14 + c \quad 
\mathbb{G} = \mathbb{N} $
\item $ ( 2x - 5 )^2+ x\cdot( 7 - 4x ) < 51 ; \quad \mathbb{G} = \mathbb{Q} $
\item $(x+4)(x-4)-(3-x)(2-x)=-2 ;\quad \mathbb{G}=\mathbb{Q}$
\item $(x-3)^2-12\ge-(1-x^2) ; \quad \mathbb{G}=\mathbb{Q}$
\end{enumerate}


{\bf \item   Bestimme die Lösungsmenge! $( \mathbb{G} =\mathbb{Q})$ }
$$2x \cdot x^2 + x^2 \cdot 3x + 3x -18 = 9 + 5x^3$$

{\bf \item Erstelle zu folgenden Textaufgaben je  eine Gleichung und löse sie:}

\begin{enumerate}[a)]
\item Eine Seite eines Rechtecks ist um $\SI{6}{\centi\meter}$ länger als die andere. Verkürzt man die längere Seite um $\SI{3}{\centi\meter}$ und verlängert man gleichzeitig die andere Seite um $\SI{2}{\centi\meter}$, so ist die Fläche des neuen Rechtecks um $\SI{4}{\centi\meter^2}$ größer als die Fläche des ursprünglichen Rechtecks. Wie lang sind die Seiten?
\item Wenn man die Summe aus einer Zahl und 3 vom Produkt der Zahl und 5 subtrahiert, so erhält man das Doppelte der Zahl. Wie heißt diese Zahl?
\end{enumerate}


\end{enumerate} 

\fbox{
	\begin{minipage}{0.5\textwidth}
		Zur Lösung bitte \href{https://www.okuyakl.de/math/m8gleiL051/ll051.pdf}{hier klicken} oder den QR-Code scannen.\\
	Weitere Arbeitsblätter gibt es unter 
	
	\href{https://www.okuyakl.de}{www.okuyakl.de}
	\end{minipage}
	\hfill
	\begin{minipage}{0.4\textwidth}
		\includegraphics[width=1.5 cm]{../../viecher/zwe03}
		\includegraphics[width=3 cm]{qr051}
		\includegraphics[width=2 cm]{../../viecher/afanticon1}
		
	\end{minipage}}

\end{document}