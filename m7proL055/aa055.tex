\documentclass[a4paper]{article}
\usepackage[pdftex]{graphicx}
\usepackage[utf8]{inputenc}
\usepackage{enumerate}
\usepackage{amssymb}
\usepackage{icomma}
\usepackage{eurosym}
\usepackage{geometry}
\geometry{a4paper, top=15mm, left=15mm, right=15mm, bottom=15mm,
	headsep=10mm, footskip=12mm}
\usepackage{href-ul}
\hypersetup{
	colorlinks=true,
	linkcolor=blue,
	urlcolor=blue}
	
\begin{document}

\begin{enumerate}[1.]

%Aufgabe 1
{\bf \item Aufgabe}

\begin{enumerate}[a)]
\item Gib die Brüche in Prozentschreibweise an: 
$$ \frac{34}{136}; \quad  \frac{28}{35}; \quad  \frac{13}{65}; \quad  \frac{7}{25}; \quad 
 \frac{3}{4}; $$
\item Berechne: \\
\begin{tabular}{rclcrcl}
6\%&von&350 m; &  \parbox[0 pt][3em][c]{0 cm}{} & 20\%&von&255 kg; \\
12\%&von&\EUR{500};  &  \parbox[0 pt][3em][c]{0 cm}{} & 90\%&von& 200 mg. \\
\end{tabular}
\end{enumerate}

%Aufgabe 2
{\bf \item In einem Kaufhaus findest du folgendes Schild:}

\begin{tabular}{|c|}
\hline
\parbox[0 pt][2em][c]{0 cm}{} 65\% reduziert: \\
 \parbox[0 pt][2em][c]{0 cm}{} Statt \EUR{50} jetzt nur noch   \EUR{21}\\
\hline
\end{tabular}

Stimmt die Aussage? Begründe deine Antwort durch Rechnung!

{\bf \item Wie berechnet man einen Prozentsatz, wenn der Grundwert und der Prozentwert gegeben sind?}
$$p\%=\dots$$

{\bf \item Eine Rechnung weist einen Warenwert von \EUR{1280} aus.}

\begin{enumerate}[a)]
\item Berechne den Endbetrag, wenn 19\% MWSt. hinzukommen.
\item Dem Käufer werden 2\% Barzahlungsrabatt vom Endbetrag eingeräumt. \\
Wie viel Euro muss er zahlen?
\end{enumerate}

{\bf \item Ein Verein hat 855 Mitglieder. Bei der Hauptversammlung sind 660 Mitglieder anwesend.} Nach Satzung gilt ein Antrag als angenommen, wenn mindestens 50\% aller eingetragenen Mitglieder zustimmen. Für den Antrag von Frau Finke stimmen 430 Anwesende. Ist er angenommen?

{\bf \item Im Schlussverkauf werden die Preise für Mäntel um 22\% und die für Anzüge um 30\% reduziert.} Ein Mantel kostete bisher \EUR{119,00} und ein Herrenanzug \EUR{169,00}. Berechne die neuen Preise.

{\bf \item Wie berechnet man die Zinsen für $t$ Tage,} wenn das Kapital $K$ und der Zinssatz $p\%$  gegeben ist? 
$$Z_t=\dots$$

{\bf \item Ein Jahr vor Beginn des Studiums ihrer Tochter legen die Eltern} so viel Geld an, dass Antonia jeweils aus den Jahreszinsen 12 gleiche Monatsraten von je \EUR{600} zur Verfügung stehen. Welches Kapital müssen die Eltern anlegen, wenn der Zinssatz 6\% beträgt?

{\bf \item Für ein Kapital von \EUR{90000} erhält Veronika monatlich \EUR{375} Zinsen.} Berechne den Zinssatz.

{\bf \item Herr Meinbach nimmt beim Hausbau folgende Darlehen auf:}\\

(1) \EUR{50000} zu 5,4\%,\hspace{15 pt} (2)  \EUR{70000} zu 5,6\%,\hspace{15 pt}
(3) \EUR{80000} zu 6,1\%,

\begin{enumerate}[a)]
\item Wie hoch sind die jährlichen Schuldzinsen insgesamt?
\item Außerdem muss er jedes Jahr 4\% der anfänglichen Darlehensbeträge als Tilgung leisten.  
Berechne die gesamte jährliche Belastung des Bauherrn.
\end{enumerate} 


{\bf \item Beim Schmelzen von Eis verringert sich das Volumen um 9\%.} Wieviel Liter Wasser entstehen aus 50 Liter Eis?

{\bf \item Das Schwimmbecken eines Bads kann durch zwei Leitungen gefüllt werden.} Mit Leitung A alleine dauert die Füllung 20 Stunden, mit der Leitung B alleine 12 Stunden. Wie lange dauert die Füllung, wenn beide Leitungen geöffnet sind?

\end{enumerate} 

\fbox{
	\begin{minipage}{0.5\textwidth}
		Zur Lösung bitte \href{https://www.okuyakl.de/math/m7proL055/ll055.pdf}{hier klicken} oder den QR-Code scannen.\\
	Weitere Arbeitsblätter gibt es unter 
	
	\href{https://www.okuyakl.de}{www.okuyakl.de}
	\end{minipage}
	\hfill
	\begin{minipage}{0.4\textwidth}
		\includegraphics[width=1.5 cm]{../../viecher/zwe03}
		\includegraphics[width=3 cm]{qr055}
		\includegraphics[width=2 cm]{../../viecher/afanticon1}
	\end{minipage}}

\end{document}%Lösung-------------------------------------------
