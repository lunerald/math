\documentclass[a4paper]{article}
\usepackage[pdftex]{graphicx}
\usepackage[utf8]{inputenc}
\usepackage{enumerate}
\usepackage{amssymb}
\usepackage{geometry}
\geometry{a4paper, top=15mm, left=15mm, right=15mm, bottom=15mm,
	headsep=10mm, footskip=12mm}
\usepackage{href-ul}
\hypersetup{
	colorlinks=true,
	linkcolor=blue,
	urlcolor=blue}
\begin{document}
{\bf Exponentialfunktion und Exponentialgleichungen}

\begin{enumerate}[1.]
{\bf \item Gegeben ist die allgemeine Exponentialfunktion $ f(x) = b \cdot a^x $}
\begin{enumerate}[a)]
\item Benennen und definieren Sie die Parameter $ a, b $ und geben Sie den Definitionsbereich an.
\item Begründen Sie Ihre Wahl des Zahlenbereichs für die Basis anhand von drei Argumenten für die Bereiche $ a < 0 ; \quad a = 0 \quad $ und $ \quad a = 1 $ (evtl. Rechenbeispiel).
\item Geben Sie jeweils eine Beispielfunktion an und begründen Sie ihre Wahl:

(i): Der Graph ist monoton fallend und verläuft nicht durch den Punkt $ P(0|1) $ 

(ii): Der Graph ist monoton steigend und verläuft durch den Punkt $ Q(0|2) $
\end{enumerate}

{\bf \item Finden Sie zu den folgenden Exponentialgleichungen alle Lösungen.} 

\begin{enumerate}[a)]
\item $ 4^{2x-5}=16 $
\item $ \frac{1}{3} \cdot 3^{x+1} + 3 = 30 $
\item $ 2^{x^2-x-7}=\frac{1}{32} $
\end{enumerate}

{\bf \item Die Punkte $ P(2|36) $ und $ Q(5|\frac{32}{3})$ liegen auf einer allgemeinen Exponentialfunktion $ f(x) = b \cdot a^x $.}\\
 Bestimmen Sie die Werte von $b$ und $a$ und geben Sie die passende Funktionsvorschrift an.


{\bf \item Füllen Sie die Tabelle für eine exponentielle Zunahme aus!} Hierbei ist $a$ der Wachstumsfaktor, $p$ die prozentuale Änderung pro  Zeiteinheit $t$, und $T$ sei die Verdoppelungszeit.


\renewcommand{\arraystretch}{2}
\begin{tabular}{|p{1,5 cm}|p{1,5 cm}|p{1,5 cm}|p{1,5 cm}|p{1,5 cm}|p{1,5 cm}|p{1,5 cm}|p{1,5 cm}|p{1,5 cm}|}
	\hline
$a$	& 1,02 &      &    & 1,35 &     &     & 2   &     \\
\hline
$p$ &      & 10\% &    &      & 7\% &     &     & 25\% \\
\hline
$T$ &      &      & 70 &      &     & 14  &     &  \\   
\hline
\end{tabular}

{\bf \item Füllen Sie die Tabelle für eine exponentielle Abnahme aus!} Hierbei ist $a$ der Wachstumsfaktor, $p$ die prozentuale Änderung pro  Zeiteinheit $t$, und $T$ sei die Halbwertszeit.


\renewcommand{\arraystretch}{2}
\begin{tabular}{|p{1,5 cm}|p{1,5 cm}|p{1,5 cm}|p{1,5 cm}|p{1,5 cm}|p{1,5 cm}|p{1,5 cm}|p{1,5 cm}|p{1,5 cm}|}
	\hline
	$a$	& 0,97 &      &    &  0,8 &     &     & 0,995 &     \\
	\hline
	$p$ &      & 8 \% &    &      & 50\%&     &       & 14\% \\
	\hline
	$T$ &      &      & 10 &      &     & 20  &       &  \\   
	\hline
\end{tabular}
\end{enumerate} 

\fbox{
	\begin{minipage}{0.5\textwidth}
		Zur Lösung bitte \href{https://www.okuyakl.de/math/m10expfuL030/ll030.pdf}{hier klicken} oder den QR-Code scannen.\\
	Weitere Arbeitsblätter gibt es unter 
	
	\href{https://www.okuyakl.de}{www.okuyakl.de}
	\end{minipage}
	\hfill
	\begin{minipage}{0.4\textwidth}
		\includegraphics[width=1.5 cm]{../../viecher/zwe03}
		\includegraphics[width=3 cm]{qr030}
		\includegraphics[width=2 cm]{../../viecher/afanticon1}
		
	\end{minipage}}

\end{document}%Lösung-------------------------------------------
