\documentclass[landscape]{article}
\usepackage[pdftex]{graphicx}
\usepackage[utf8]{inputenc}
\usepackage{enumerate}
\usepackage{amssymb}
\usepackage{href-ul}
\hypersetup{
	colorlinks=true,
	linkcolor=blue,
	urlcolor=blue}
\usepackage{geometry}
\geometry{a4paper, top=10mm, left=15mm, right=15mm, bottom=15mm,
	headsep=10mm, footskip=12mm}

\begin{document}
{\bf Grenzwerte gebrochen rationaler Funktionen I -- Verhalten im Unendlichen}
\vspace{0.5 cm}

\renewcommand{\arraystretch}{3}
\begin{tabular}{|c|c|c|c|c|c|c|c|}
	\hline
	& Funktionsterm & Zählergrad & Nennergrad & Fall & $\lim\limits_{x\to\pm\infty}$ & Polynomdivision liefert & Asymptote / Näherungsfunktion \\
	\hline
a) & ${2x^2 -1 \over x+3}$ & Z=2 & N=1 & Z=N+1 & $\pm\infty$ & $f(x)=2x-6+{17\over x+3}$ & $y=2x-6$ \\
\hline
b) & ${x+3 \over x^2 -1}$ & & & & & & \\
\hline
c) & ${x^3 +4x\over 2x^2-10x}$ & & & & & & \\
\hline
d) & ${(x+1)^2 \over 2(x+1)}$ & & & & & & \\
\hline
e) & ${0,1x^3 \over x^2+2x+1}$ & & & & & & \\
\hline
f) & ${x(x+5) \over 3x^2 }$ & & & & & & \\
\hline
g) & ${x^2-3x+2 \over x^3-2x^2 }$ & & & & & & \\
\hline
h) & ${tx \over x+1 }$ & & & & & & \\
\hline
i) &  & & & & & $f(x)= 3 + {2\over x-1}$ & \\
\hline
j) &  & & & & & $f(x)= 0,5x +1 + {1\over x-2}$ & \\
\hline
\end{tabular}

\fbox{
	\begin{minipage}{0.5\textwidth}
		Zur Lösung bitte \href{https://www.okuyakl.de/math/m11ligrL070/ll070.pdf}{hier klicken} oder den QR-Code scannen.\\
	Weitere Arbeitsblätter gibt es unter 
	
	\href{https://www.okuyakl.de}{www.okuyakl.de}
	\end{minipage}
	\hfill
	\begin{minipage}{0.4\textwidth}
		\includegraphics[width=1.5 cm]{../../viecher/zwe03}
		\includegraphics[width=3 cm]{qr070}
		\includegraphics[width=2 cm]{../../viecher/afanticon1}
		
\end{minipage}}
\end{document}

