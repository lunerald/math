\documentclass[a4paper]{article}
\usepackage[pdftex]{graphicx}
\usepackage[utf8]{inputenc}
\usepackage{enumerate}
\usepackage{amssymb}
\usepackage{icomma}
\usepackage{colortbl}
\usepackage{mdwlist}
\usepackage{siunitx}
\sisetup{locale=DE}
\usepackage{geometry}
\geometry{a4paper, top=15mm, left=15mm, right=15mm, bottom=15mm,
	headsep=10mm, footskip=12mm}
\usepackage{href-ul}
\hypersetup{
	colorlinks=true,
	linkcolor=blue,
	urlcolor=blue}
\begin{document}
	{\bf The tangent}
	
	\begin{enumerate}[1.]
		{\bf \item Calculate the missing sizes}
		
		\begin{minipage}{0.65\textwidth}
			\renewcommand{\arraystretch}{2}
			\begin{tabular}{|>{\columncolor[gray]{.8}}p{1.5 cm}|p{1.5 cm}|p{1.5 cm}|p{1.5 cm}|p{1.5 cm}|p {1.5cm}|}
				\hline
				\rowcolor[gray]{.8} &a) &b) &c) & d) & e) \\
				\hline
				$a$ & $\SI{4,2}{\centi \meter}$ &$\SI{2,8}{\meter}$ &$\SI{36}{\centi \meter}$ && $a $\\
				\hline
				$b$ & & &$\SI{3.5}{\deci\meter}$ &$\SI{1.2}{\kilo\meter}$ &$\sqrt{3}\cdot a$ \\
				\hline
				$\alpha$ & & $6^\circ$& & & \\
				\hline
				$\beta$ &$71^\circ$ & & &$8 \cdot \alpha$ & \\
				\hline
			\end{tabular}
		\end{minipage}
		\hfill
		\begin{minipage}{0.35\textwidth}
			\includegraphics[width=6 cm]{tan100}
		\end{minipage}
		
		\begin{minipage}{0.7\textwidth}
			{\bf \item A rectangle has side lengths $a=\SI{8,0}{\centi \meter}$ and $b=\SI{5}{\centi \meter}$.}
			\begin{enumerate}[a)]
				\item Calculate the angles that the diagonals enclose with the sides.
				\item Calculate the angle that both diagonals form with each other.
			\end{enumerate}
		\end{minipage}
		\hfill
		\begin{minipage}{0.3\textwidth}
			\includegraphics[width=4 cm]{rech100}
		\end{minipage}
		
		\begin{minipage}{0.7\textwidth}
			{\bf \item The picture on the right shows a section of the \rule{3 cm}{0.5 pt} flag}.\\ Draw the diagonals into the rhombus, measure their lengths and \underbar{calculate}\\ their interior angles .
		\end{minipage}
		\hfill
		\begin{minipage}{0.3\textwidth}
			\includegraphics[width=4 cm]{flag100}
		\end{minipage}
		
		{\bf \item Earth's moon } has a diameter of about 3500 kilometers. In the night sky it appears at a size of $0.5^\circ$.
		\begin{enumerate*}
			\item What distance is it from Earth then?
			\item During a solar eclipse, the moon and sun are congruent in the sky. What is the diameter of the Sun if we know that its distance from the Earth is $\SI{150e6}{\kilo\meter}$?
			\suspend{enumerate*}
			
			{\bf A parallax second or a parsec (1 pc )} in astronomy is the distance from which a\\ cosmic observer sees the radius of the Earth's orbit around the Sun at a viewing angle of one arcsecond.\\ (1 arcsecond or 1" = ${1\over 3600}^\circ$ )
			
			
			\resume{enumerate*}
			\item What is this distance in kilometers?
			\item How many light years is this if one light year is $\SI{9.5e12}{\kilo\meter}$?
		\end{enumerate*}
		
		
		\begin{minipage}{0.8\textwidth}
			{\bf \item Lunerald is going on a bike ride to the Alps.} He can already see his destination on the horizon.\\ He stretches out his hand and can use it to cover up the $\SI{1731}{\meter}$ high Duke's stand.
			\begin{enumerate}[a)]
				\item How far away is it from the mountains if a hand's breadth means a viewing angle of $8^\circ$?
				\item Finally he is at the foot of the mountains. The road rises. How big is the incline angle that he has to cycle up?
			\end{enumerate}
			
		\end{minipage}
		\hfill
		\begin{minipage}{0.2\textwidth}
			\includegraphics[width=3 cm]{vstg100}
		\end{minipage}
		
		
		\begin{minipage}{0.5\textwidth}
			{\bf \item Forester's Triangle:} The height of trees can be determined indirectly. To do this, measure the horizontal distance to the tree and keep the following arrangement in mind:
		\end{minipage}
		\hfill
		\begin{minipage}{0.5\textwidth}
			\includegraphics[width=8 cm]{forst100}
		\end{minipage}
		\begin{enumerate}[a)]
			\item At what angle can you see the tree if $x=\SI{70}{\centi\meter}$ and $y=\SI{28}{\centi\meter}$?
			\item How tall is the tree when it is $\SI{30}{\meter}$ away from the viewer?
			\item Solve problem b) without tangent using the four-line theorem.
		\end{enumerate}
		
		{\bf(Please turn $\longrightarrow$)}
		\newpage
		
		\begin{minipage}{0.7\textwidth}
			{\bf \item Line slopes $m$ and intersection angles} of lines with the x-axis have a simple connection. The following applies:
			$$ m = \tan \alpha$$
			Given are the lines\\ $g: y=0.5x \qquad h: y=-3x+5 \qquad \rm{and} \qquad f: y=-0.4x+5$
		\end{minipage}
		\hfill
		\begin{minipage}{0.3\textwidth}
			\includegraphics[width=5 cm]{stei100}
		\end{minipage}
		\begin{enumerate}[a)]
			\item Draw the lines in a coordinate system
			\item Calculate the angle at which the lines intersect the x-axis.
			\item Calculate the interior angles of the triangle formed.
		\end{enumerate}
		
		\begin{minipage}{0.5\textwidth}
			{\bf \item Given are right triangles $A_nB_nC$ (see figure). }
			The points $A_n(x|-0.5x+3)$ lie on the line $g$ with $y= -0.5x+3$.
			Furthermore: $B_n(4|y_A); ~C(4|4)$
			\begin{enumerate}[a)]
				\item Calculate the coordinates of $A_1$ and $B_1$ for $x_1=2$
				\item Calculate the measure $\alpha_1$ of the angle $\angle B_1A_1C$.
				\item Show that the measure $\alpha$ of the angle $\angle B_nA_nC$ holds:
				$$ \tan \alpha = {1+ 0.5x \over 4-x}$$
				\item In triangle $A_2B_2C$ the following applies: $\alpha_2=14.04^\circ$. \\
				Calculate the coordinates of points $A_2$ and $B_2$.
			\end{enumerate}
			
			\end{minipage}
			\hfill
			\begin{minipage}{0.45\textwidth}
			\includegraphics[width=6 cm]{dreieck100}
			\end{minipage}
			
			
			
			\fbox{
			\begin{minipage}{0.5\textwidth}
				For the solution, please \href{https://www.okuyakl.de/math/m10tanL100/le100.pdf}{click here} or scan the QR code.\\
				Additional worksheets are available at
				
				\href{http://www.okuyakl.com}{www.okuyakl.com}
			\end{minipage}
			\hfill
			\begin{minipage}{0.4\textwidth}
				\includegraphics[width=1.5 cm]{../../viecher/zwe03}
				\includegraphics[width=3 cm]{qre100}
				\includegraphics[width=2 cm]{../../viecher/afanticon1}
				
				\end{minipage}}
				
		\end{enumerate}
	\end{document}