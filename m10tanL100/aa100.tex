\documentclass[a4paper]{article}
\usepackage[pdftex]{graphicx}
\usepackage[utf8]{inputenc}
\usepackage{enumerate}
\usepackage{amssymb}
\usepackage{icomma}
\usepackage{colortbl}
\usepackage{mdwlist}
\usepackage{siunitx}
\sisetup{locale=DE} 
\usepackage{geometry}
\geometry{a4paper, top=15mm, left=15mm, right=15mm, bottom=15mm,
	headsep=10mm, footskip=12mm}
\usepackage{href-ul}
\hypersetup{
	colorlinks=true,
	linkcolor=blue,
	urlcolor=blue}
\begin{document}
{\bf Der Tangens}

\begin{enumerate}[1.]
{\bf \item Berechne die fehlenden Größen}

\begin{minipage}{0.65\textwidth}
	\renewcommand{\arraystretch}{2}
	\begin{tabular}{|>{\columncolor[gray]{.8}}p{1.5 cm}|p{1.5 cm}|p{1.5 cm}|p{1.5 cm}|p{1.5 cm}|p{1.5 cm}|}
		\hline
		\rowcolor[gray]{.8}	&a) &b) &c) & d) & e) \\
		\hline
		$a$ & $\SI{4,2}{\centi\meter}$ &$\SI{2,8}{\meter}$ &$\SI{36}{\centi\meter}$ && $a$\\
		\hline
		$b$ & & &$\SI{3,5}{\deci\meter}$ &$\SI{1,2}{\kilo\meter}$  &$\sqrt{3}\cdot a$ \\
		\hline
		$\alpha$ & & $6^\circ$& & & \\
		\hline
		$\beta$ &$71^\circ$ &  & &$8 \cdot \alpha$ & \\
		\hline
	\end{tabular}
\end{minipage}
\hfill
\begin{minipage}{0.35\textwidth}
	\includegraphics[width=6 cm]{tan100}		
\end{minipage}

\begin{minipage}{0.7\textwidth}
{\bf \item Ein Rechteck hat die Seitenlängen $a=\SI{8,0}{\centi\meter}$ und $b=\SI{5}{\centi\meter}$.}
	\begin{enumerate}[a)]
		\item Berechne die Winkel, die die Diagonalen mit den Seiten einschließen.
		\item Berechne den Winkel, den beide Diagonalen miteinander einschließen.
	\end{enumerate} 
\end{minipage}
\hfill
\begin{minipage}{0.3\textwidth}
	\includegraphics[width=4 cm]{rech100}		
\end{minipage}

\begin{minipage}{0.7\textwidth}
{\bf \item Das Bild rechts zeigt eine Ausschnitt aus der \rule{3 cm}{0.5 pt} Flagge}.\\ Zeichne die Diagonalen in die Raute ein, miss ihre Längen und \underbar{berechne}\\ damit ihre Innenwinkel.
\end{minipage}
\hfill
\begin{minipage}{0.3\textwidth}
	\includegraphics[width=4 cm]{flag100}		
\end{minipage}

{\bf \item Der Erdmond } hat einen Durchmesser von etwa 3500 Kilometern. Am Nachthimmel erscheint er in einer Größe von $0,5^\circ$. 
\begin{enumerate*}
	\item Welche Entfernung von der Erde hat er dann?
	\item Bei einer Sonnenfinsternis sind Mond und Sonne am Himmel deckungsgleich. Welchen Durchmesser hat die Sonne, wenn man weiß, dass ihr Abstand zur Erde $\SI{150e6}{\kilo\meter}$ beträgt ?
\suspend{enumerate*}

{\bf Eine Parallaxensekunde oder ein Parsec (1 pc )} ist in der Astronomie die Entfernung, aus welcher ein\\ kosmischer Beobachter den Radius der Erdbahn um die Sonne unter einem Blickwinkel von einer Bogensekunde sieht.\\  (1 Bogensekunde oder 1" = ${1\over 3600}^\circ$ )


\resume{enumerate*}
    \item Wie groß ist diese Entfernung in Kilometern?
    \item Wie viele Lichtjahre sind dies, wenn ein Lichtjahr $\SI{9,5e12}{\kilo\meter}$ sind?
\end{enumerate*}  


\begin{minipage}{0.8\textwidth}
	{\bf \item Lunerald macht eine Radtour zu den Alpen.} Am Horizont sieht er schon sein Ziel.\\ Er streckt die Hand aus und kann damit genau den $\SI{1731}{\meter}$ hohen Herzogstand verdecken.
\begin{enumerate}[a)]
	\item Wie weit ist er vom Gebirge entfernt, wenn eine Handbreit einen Sichtwinkel von $8^\circ$ bedeutet?
	\item Endlich ist er am Fuß der Berge. Die Straße steigt an. Wie groß ist nun der Steigungswinkel, den er hochradeln muss?
\end{enumerate} 	
	
\end{minipage}
\hfill
\begin{minipage}{0.2\textwidth}
	\includegraphics[width=3 cm]{vstg100}		
\end{minipage}


\begin{minipage}{0.5\textwidth}
{\bf \item Försterdreieck:} Die Höhe von Bäumen lässt sich indirekt bestimmen. Hierzu misst man den horizontalen Abstand zum Baum und hält sich folgende Anordnung vor die Augen:
\end{minipage}
\hfill
\begin{minipage}{0.5\textwidth}
	\includegraphics[width=8 cm]{forst100}		
\end{minipage}
\begin{enumerate}[a)]
	\item Unter welchem Winkel sieht man den Baum, wenn $x=\SI{70}{\centi\meter}$ und $y=\SI{28}{\centi\meter}$ ist ?
	\item Wie hoch ist der Baum dann, wenn er $\SI{30}{\meter}$ vom Betrachter entfernt ist ?
	\item Löse die Aufgabe b) ohne Tangens mit dem Vierstreckensatz.
\end{enumerate} 

{\bf(Bitte wenden $\longrightarrow$)}
\newpage

\begin{minipage}{0.7\textwidth}
	{\bf \item Geradensteigungen $m$ und Schnittwinkel} von Geraden mit der x-Achse haben einen einfachen Zusammenhang. Es gilt: 
	$$ m = \tan \alpha$$
	Gegeben sind die Geraden\\ $g: y=0,5x \qquad h: y=-3x+5 \qquad \rm{und} \qquad f: y=-0,4x+5$
\end{minipage}
\hfill
\begin{minipage}{0.3\textwidth}
	\includegraphics[width=5 cm]{stei100}		
\end{minipage}
\begin{enumerate}[a)]
	\item Zeichne die Geraden in ein Koordinatensystem
	\item Berechne den jeweiligen Winkel, mit dem die Geraden die x-Achse schneiden.
	\item Berechne die Innenwinkel des gebildeten Dreiecks.  
\end{enumerate} 

\begin{minipage}{0.5\textwidth}
	{\bf \item Gegeben sind rechtwinklige Dreiecke $A_nB_nC$ (siehe Abbildung). }
	Die  Punkte $A_n(x|-0,5x+3)$ liegen auf der Ge\-raden $g$ mit $y= -0,5x+3$.
	Weiterhin gilt: $B_n(4|y_A); ~C(4|4)$
\begin{enumerate}[a)]
	\item Berechne die Koordinaten von $A_1$ und $B_1$ für $x_1=2$
	\item Berechne das Maß $\alpha_1$ des Winkels $\angle B_1A_1C$.
	\item Zeige, dass für das Maß $\alpha$ des Winkels  $\angle B_nA_nC$ gilt:
	$$ \tan \alpha = {1+ 0,5x \over 4-x}$$
	\item Im Dreieck $A_2B_2C$ gilt: $\alpha_2=14,04^\circ$. \\
	Berechne die Koordinaten der Punkte $A_2$ und $B_2$.
\end{enumerate}	
	
\end{minipage}
\hfill
\begin{minipage}{0.45\textwidth}
	\includegraphics[width=6 cm]{dreieck100}
\end{minipage}



\fbox{
	\begin{minipage}{0.5\textwidth}
		Zur Lösung bitte \href{https://www.okuyakl.de/math/m10tanL100/ll100.pdf}{hier klicken} oder den QR-Code scannen.\\
	Weitere Arbeitsblätter gibt es unter 
	
	\href{https://www.okuyakl.de}{www.okuyakl.de}
	\end{minipage}
	\hfill
	\begin{minipage}{0.4\textwidth}
		\includegraphics[width=1.5 cm]{../../viecher/zwe03}
		\includegraphics[width=3 cm]{qr100}
		\includegraphics[width=2 cm]{../../viecher/afanticon1}
		
\end{minipage}}

\end{enumerate}
\end{document}

