\documentclass[a4paper]{article}
\usepackage[pdftex]{graphicx}
\usepackage[utf8]{inputenc}
\usepackage{enumerate}
\usepackage{icomma}
\usepackage{siunitx}
\sisetup{locale=DE} 
\usepackage{amssymb}
\usepackage{geometry}
\geometry{a4paper, top=15mm, left=15mm, right=15mm, bottom=15mm,
	headsep=10mm, footskip=12mm}
\usepackage{href-ul}
\hypersetup{
	colorlinks=true,
	linkcolor=blue,
	urlcolor=blue}
\begin{document}
{\bf Dreiecksgeometrie und Zentrische Streckung}
\begin{enumerate}[1.]

{\bf \item Gegeben ist das Dreieck $ABC$ mit $ A(1|1) $, $ B(10|-2) $ und $ C(7|4)$}
\begin{enumerate}[a)]
%Aufgabe 1
\item  Zeichne das Dreieck in ein geeignetes Koordinatensystem ein. \\ 
$ ( 0 \le x \le 11, \quad -3  \le y \le 5 ) $

\item  Zeige rechnerisch, dass es sich um ein gleichschenkliges Dreieck handelt

\item Beschreibe mit Worten, wie man die Koordinaten des Umkreismittelpunktes berechnen kann.

\item Der Umkreismittelpunkt ist $ M(5,5|-0.5) $. Berechne den Radius des Umkreises

\item Um welchen Faktor ist der Flächeninhalt des Umkreises größer als der Flächeninhalt des
 Dreiecks ? 
\end{enumerate}

{\bf \item Sind folgende Aussagen wahr oder falsch? }

Bei einer zentrischen Streckung mit dem Zentrum $ Z $  und dem Streckungsfaktor $ k $ gilt:

\begin{enumerate}[a)]
\item Wenn der Bildpunkt zwischen $ Z $ und dem Urpunkt liegt, ist $ k $ negativ.
\item Wenn der Bildpunkt zwischen $ Z $ und dem Urpunkt liegt, gilt $ 0 < k < 1 $.
\item Jede Gerade durch $ Z $ ist eine Fixgerade
\item Sie ist für $ k \neq -1 $ und $ k \neq 1 $ eine Kongruenzabbildung 
\end{enumerate}

{\bf \item Gegeben ist die Gerade $ g: \quad y = 0,75x + 2 $ und $ P(2|3,5) \in g $ .}
\begin{enumerate}[a)]
\item Die Gerade wird durch zentrische Streckung am Zentrum $ Z(2|2) $ und mit dem Streckungsfaktor $ k = 3 $ auf $ g' $ abgebildet. Zeichne $ g $ und $ g' $ in ein Koordinatensystem ein. $ ( 0 \le x \le 7 , \quad 0  \le y \le 7 ) $
\item Berechne die Länge der Strecke $ \overline{ZP'} $
\item Berechne anschließend die Gleichung der Geraden  $ g' $
\item Die Punkte $ A(4|2,5) $, $ B(6,5|2,5) $ und $ C(4|y_c) $ legen das Dreieck ABC fest. $ C $ liegt auf der Geraden $ g $ . Zeichne das Dreieck in das Koordinatensystem und berechne dessen Flächeninhalt nachvollziehbar. Um welches besondere Dreieck handelt es sich?
\item Das Dreieck $ ABC $ wird am Zentrum $ S(0|0) $ mit $ k = -2 $ zentrisch gestreckt. Berechne den Flächeninhalt des Dreiecks  $A'B'C'$ .
\end{enumerate}

%Aufgabe 4
{\bf \item Die Grundfläche eines Pyramidenstumpfs ist neunmal so groß wie die Deckfläche, seine Höhe beträgt $\SI{6,0}{\centi\meter}$.} Berechne die Höhe der Pyramidenspitze über der Grundfläche.

\end{enumerate} 

\fbox{
	\begin{minipage}{0.5\textwidth}
	Zur Lösung bitte \href{https://www.okuyakl.de/math/m9zesL041/ll041.pdf}{hier klicken} oder den QR-Code scannen.\\
Weitere Arbeitsblätter gibt es unter 

\href{https://www.okuyakl.de}{www.okuyakl.de}
	\end{minipage}
	\hfill
	\begin{minipage}{0.4\textwidth}
		\includegraphics[width=1.5 cm]{../../viecher/zwe03}
		\includegraphics[width=3 cm]{qr041}
		\includegraphics[width=2 cm]{../../viecher/afanticon1}
		
	\end{minipage}}

\end{document}%Lösung-------------------------------------------
