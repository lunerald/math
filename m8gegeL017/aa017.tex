\documentclass[a4paper]{article}
\usepackage[pdftex]{graphicx}
\usepackage[utf8]{inputenc}
\usepackage{enumerate}
\usepackage{amssymb}
\usepackage{geometry}
\geometry{a4paper, top=15mm, left=15mm, right=15mm, bottom=15mm,
	headsep=10mm, footskip=12mm}
\usepackage{href-ul}
\hypersetup{
	colorlinks=true,
	linkcolor=blue,
	urlcolor=blue}
	
% Start the document
\begin{document}
%Titel
{\bf Trainingsblatt Geraden }\\
{\bf Gegeben sind folgende Geraden: $g_1$ bis $g_7$, Y-Achsenabschnitt von oben nach unten.}

\noindent\includegraphics[width=17 cm]{vs017}

%Aufgabe 1
\noindent{\bf Bestimme alle Parameter für eine Gerade und fülle die Tabelle aus:}\\
\renewcommand{\arraystretch}{3}
\begin{tabular}{|p{40 pt}|p{80 pt}|p{100 pt}|p{100 pt}|p{80 pt}|}
\hline
Gerade & Steigung $m$ & Y-Achsenabschnitt $t$  & Funktionsterm $y=$ & Nullstelle\\
\hline
$g_1:$ & & & & \\
\hline
$g_2:$ & & & & \\
\hline
$g_3:$ & & & & \\
\hline
$g_4:$ & & & & \\
\hline
$g_5:$ & & & & \\
\hline
$g_6:$ & & & & \\
\hline
$g_7:$ & & & & \\
\hline
\end{tabular}

\fbox{
	\begin{minipage}{0.5\textwidth}
		Zur Lösung bitte \href{https://www.okuyakl.de/math/m8gegeL017/ll017.pdf}{hier klicken} oder den QR-Code scannen.\\
	Weitere Arbeitsblätter gibt es unter 
	
	\href{https://www.okuyakl.de}{www.okuyakl.de}
	\end{minipage}
	\hfill
	\begin{minipage}{0.4\textwidth}
		\includegraphics[width=1.5 cm]{../../viecher/zwe03}
		\includegraphics[width=3 cm]{qr017}
		\includegraphics[width=2 cm]{../../viecher/afanticon1}
		
	\end{minipage}}

\end{document}%Lösung---------------------------------------------------------------
