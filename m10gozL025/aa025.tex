
\documentclass[a4paper]{article}
\usepackage[pdftex]{graphicx}
\usepackage[utf8]{inputenc}
\usepackage{enumerate}
\usepackage{icomma}
\usepackage{amssymb}
\usepackage{amsmath}
\usepackage{geometry}
\geometry{a4paper, top=15mm, left=15mm, right=15mm, bottom=15mm,
	headsep=10mm, footskip=12mm}
\usepackage{href-ul}
\hypersetup{
	colorlinks=true,
	linkcolor=blue,
	urlcolor=blue}
% Start the document
\begin{document}

%Titel
{\bf Trainingsblatt Trigonometrische Funktionen }\\
\begin{enumerate}[1.]

%Aufgabe 1
{\bf \item Bestimmen Sie alle Lösungen folgender Gleichungen über der Grundmenge $\mathbb{G}=[0;2\pi]$}

\begin{enumerate}[a)]

\begin{minipage}{0.4\textwidth}
\item $$ \tan^2x = {1 \over 4-4\sin^2 x}$$
\item $$\sin x \cos x = \sqrt{(1-\sin x)(1+\sin x)}$$
\item $$ { 1 \over \tan^2 x} = { 3 \over 4 \sin^2 x}$$
\end{minipage}
\hspace{1 cm}
\begin{minipage}{0.4\textwidth}
\item $$ \sin x + {1 \over 2} = { 1 \over 2 \sin x} $$ 
\item $$ \sin^2 x + {1 \over 2}\sqrt{2} \sin x + \cos x = {\sin x \over \tan x} $$
\end{minipage}
\end{enumerate}

\begin{minipage}{0.4\textwidth}
{\bf \item Sinus- und Kosinusfunktion lassen sich aufgrund ihrer Symmetrie ineinander umwandeln.} Benutzen Sie dieses Schaubild, um die folgenden Terme zu vereinfachen:
\end{minipage}
\begin{minipage}{0.6\textwidth}
\setlength{\unitlength}{1 cm}
\begin{picture}(9,4)
\includegraphics[width=9 cm]{scfu025}
\put(-0.2,0.5){$\sin x$}
\put(0.2,2.5){$\cos x$}
\end{picture}
\end{minipage}

\begin{enumerate}[a)]
\begin{minipage}{0.4\textwidth}
\item $$ \sin (x-\pi) + \sin x $$
\item $$ 2 \cos (-x) - \cos x $$
\item $$ \sin \left(x+{\pi \over 2}\right) + \cos (2\pi -x)$$
\item $$ \cos  \left({\pi \over 2}-x\right) + \sin (x+\pi) $$
\end{minipage}
\hspace{1 cm}
\begin{minipage}{0.4\textwidth}
\item $$ \sin \left({\pi \over 2}-x\right) + \cos (x + \pi) $$
\item $$ \sin (\pi - x) - \sin (-x)$$
\item $$ \cos \left(x+{\pi \over 2}\right)+ \cos \left(x-{\pi \over 2}\right)$$   
\end{minipage}
\end{enumerate}

{\bf \item Vereinfachen Sie:}

\begin{enumerate}[a)]
\item $$\sin^4 x + 2 \sin^2 x \cos^2 x + \cos^4 x $$
\item $$ {\tan x \over \sqrt{1 + \tan^2 x}}$$
\end{enumerate}

{\bf \item Es gilt angenähert: $\sin{(\frac{\pi}{5})} \approx 0,5878 $.\\
	Bestimme die Lösungen der folgenden Gleichung über der Grundmenge}\\
$\mathbb{G} = [-\pi; 5\pi] $:
$$ \sin{x} = -0,5878 $$

{\bf \item Skizziere eine Periode des Graphen der Funktion $ f(x) = 2\sin{ \left( \frac{1}{2}x -  \frac{\pi}{4} \right)}$}

\end{enumerate}

\fbox{
	\begin{minipage}{0.5\textwidth}
		Zur Lösung bitte \href{https://www.okuyakl.de/math/m10gozL025/ll025.pdf}{hier klicken} oder den QR-Code scannen.\\
	Weitere Arbeitsblätter gibt es unter 
	
	\href{https://www.okuyakl.de}{www.okuyakl.de}
	\end{minipage}
	\hfill
	\begin{minipage}{0.4\textwidth}
		\includegraphics[width=1.5 cm]{../../viecher/zwe03}
		\includegraphics[width=3 cm]{qr025}
		\includegraphics[width=2 cm]{../../viecher/afanticon1}
		
	\end{minipage}}

\end{document}%Lösung--------------------------------------------------------------------------------------------------
