\documentclass[a4paper]{article}
\usepackage[pdftex]{graphicx}
\usepackage[utf8]{inputenc}
\usepackage{enumerate}
\usepackage{icomma}
\usepackage{siunitx}
\sisetup{locale=DE} 
\usepackage{amssymb}
\usepackage{geometry}
\geometry{a4paper, top=15mm, left=15mm, right=15mm, bottom=15mm,
	headsep=10mm, footskip=12mm}
\usepackage{href-ul}
\hypersetup{
	colorlinks=true,
	linkcolor=blue,
	urlcolor=blue}
	
\begin{document}
{\bf Trainingsblatt Parabeln}
\begin{enumerate}[1.]
%Aufgabe 1
{\bf \item Die Parabel $p_1$ mit dem Öffnungsfaktor $a= 0,5$ verläuft durch die Punkte $P(1|1)$ und $Q(-6|-2,5)$.}

\begin{enumerate}[a)]
\item Ermittle rechnerisch die Gleichung der Parabel $p_1$ und zeichne die Parabel $p_1$ in ein Koordinatensystem ein $(-8 \le x \le 2; \quad -8 \le y \le 4 )$.\\
$[\textnormal{ Teilergebnis}: ~ p_1: ~y=0,5x^2+3x-2,5]$
\item Die Gleichung der Parabel $p_2$ hat die allgemeine Form $y=-0,25x^2 -2,25x-2,5$
\item Berechne die Scheitelkoordinaten und zeichne die Parabel $p_2$ ebenfalls in das Koordinatensystem von Aufgabe 1. a) ein.
\item Bestimme rechnerisch die Koordinaten der Schnittpunkte $A$ und $C$ der beiden Parabeln $p_1$ und $p_2$.  \\
$[\textnormal{ Teilergebnis}:~x_A=-7; \quad x_C=0]$
\item Die Punkte $A$, $B_n$ und $C$ bilden Dreiecke $AB_nC$, wobei gilt: 
$$B_n \in p_1 ~\textnormal{mit} ~ x \in ]-7;0[ ~\textnormal{und}~ x \in \mathbb{R}$$
Zeichne die Dreiecke $AB_1C$ für $x_1=-4$ und $AB_2C$ für $x_2=-1$ in das Koordinatensystem von Aufgabe 1. a) ein.
\item Bestimme rechnerisch den Flächeninhalt der Dreiecke $AB_nC$ in Abhäng\-igkeit von $x$.\\
$[\textnormal{ Teilergebnis}:~ A(x)=(-1,75x^2-12,25x) FE]$
\item Das Dreieck $AB_3C$ besitzt den größten Flächeninhalt. Berechne den zugehörigen Wert für $x_3$ und gib $A_{max}$ an.
\end{enumerate}

%Aufgabe 2
{\bf \item Eine Seite eines Rechtecks ist $\SI{6}{\centi\meter}$ lang, die Länge der anderen Seite beträgt $\SI{8}{\centi\meter}$.} Man erhält neue Rechtecke, indem man die kürzere Seite um $2x~\SI{}{\centi\meter}$ verlängert und die längere Seite um $x~\SI{}{\centi\meter}$ verkürzt. Es gilt: $ 0<x<8$. 
\begin{enumerate}[a)]
\item Berechne den Flächeninhalt $A(x)$ in Abhängigkeit von $x$.\\
$[\textnormal{ Teilergebnis}:~ A(x)=(-2x^2+10x+48)\SI{}{\centi\meter}^2]$
\item Bestimme rechnerisch, bei welchem $x$ der Flächeninhalt $\SI{20}{\centi\meter^2}$ beträgt.
\end{enumerate}

{\bf \item Gegeben ist die Parabelschar $$p_a \quad \textnormal{mit} \quad y = x^2 -2ax + a^2 + 0,5a - 2; \quad \mathbb{G} =  \mathbb{R} \times  \mathbb{R} $$}

\begin{enumerate}[a)]
\item Bestimme die Scheitelpunktsform der Gleichung in Abhängigkeit von a. [Zwischenergebnis:
$ S_a(a|0,5a-2)$]
\item Gib die Scheitelkoordinaten der Punkte $S_a$ für $a \in \{-2; 0; -3 \}$  an und zeichne die zugehörigen Parabeln in ein Koordinatensystem ein.
\item Bestimme rechnerisch die Gleichung des Trägergraphen aller Scheitelpunkte der Parabelschar $p_a$.
\item Der Punkt $P(8|62)$ liegt auf den Parabeln $p_6$ und $p_7$ der Schar. Bestimme die zugehörigen Gleichungen in der Normalform. [2 Lösungen!]
\item Bestimme die Gleichung der nach oben geöffneten Normalparabel, die durch die Punkte $A(-1,5|-0,75)$ und $B(1|8)$ verläuft. Gehört die Parabel zur oben genannten Schar? Begründe!
\end{enumerate}

{\bf \item Gegeben ist die quadratische Funktion $f$ mit $f(x)= -1,5x^2+6x-4,5$}

\begin{enumerate}[a)]
\item Bestimme die Koordinaten des Scheitels der zugehörigen Parabel und gib die Wertemenge zur Funktion $f$ an.
\item Überprüfe, ob der Punkt $P(-11|-25)$ auf, oberhalb oder unterhalb der Parabel liegt. 
\end{enumerate}

{\bf \item Der Graph der Funktion $h$ mit $h(x)=4(x+1,5)^2-3 $ wird an der x-Achse und anschließend an der y-Achse gespiegelt.} Gib eine Gleichung für die entstandene Parabel an. Begründe.

\begin{minipage}{0.45\textwidth}
{\bf \item Die Burg Quadreck wird von Banditen angegriffen.} Zur Verteidigung werden Pfeile eingesetzt. Die Flugbahn der Pfeile entspricht einer Parabel. Der Punkt $A(50|20)$ ist der Abschusspunkt der Pfeile und liegt auf dem vordersten Turm. Ihren höchsten Punkt erreicht die Flugbahn der Pfeile im Punkt $S(80|31,25)$.
\end{minipage}
\hfill
\begin{minipage}{0.45\textwidth}
\includegraphics[width=7 cm]{burg039}
\end{minipage}

\begin{enumerate}[a)]
\item Bestimme die quadratische Funktion $g$, die die Flugbahn der Pfeile beschreibt in der Form  
$g(x)=a(x-b)^2+c$ 
\item Ermittle rechnerisch, bei welcher x-Koordinate der Pfeil auf den Boden trifft. Wie weit ist der Pfeil geflogen?
\end{enumerate} 

{\bf \item Von einer Parabel sind nur die Nullstellen $x_1=-4,5$ und $x_2=3,5$ bekannt.} Wie lautet die x-Koordinate des Scheitels?

\end{enumerate} 

\fbox{
	\begin{minipage}{0.5\textwidth}
		Zur Lösung bitte \href{https://www.okuyakl.de/math/m9paraL039/ll039.pdf}{hier klicken} oder den QR-Code scannen.\\
	Weitere Arbeitsblätter gibt es unter 
	
	\href{https://www.okuyakl.de}{www.okuyakl.de}
	\end{minipage}
	\hfill
	\begin{minipage}{0.4\textwidth}
		\includegraphics[width=1.5 cm]{../../viecher/zwe03}
		\includegraphics[width=3 cm]{qr039}
		\includegraphics[width=2 cm]{../../viecher/afanticon1}
		
	\end{minipage}}

\end{document}%Lösung-------------------------------------------
