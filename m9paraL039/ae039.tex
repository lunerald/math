\documentclass[a4paper]{article}
\usepackage[pdftex]{graphicx}
\usepackage[utf8]{inputenc}
\usepackage{enumerate}
\usepackage{icomma}
\usepackage{siunitx}
\sisetup{locale=DE}
\usepackage{amssymb}
\usepackage{geometry}
\geometry{a4paper, top=15mm, left=15mm, right=15mm, bottom=15mm,
	headsep=10mm, footskip=12mm}
\usepackage{href-ul}
\hypersetup{
	colorlinks=true,
	linkcolor=blue,
	urlcolor=blue}

\begin{document}
	{\bf Training sheet parabolas}
	\begin{enumerate}[1.]
		%Task 1
		{\bf \item The parabola $p_1$ with the opening factor $a= 0.5$ passes through the points $P(1|1)$ and $Q(-6|-2.5)$.}
		
		\begin{enumerate}[a)]
			\item Calculate the equation of the parabola $p_1$ and draw the parabola $p_1$ in a coordinate system $(-8 \le x \le 2; \quad -8 \le y \le 4 )$.\\
			$[\textnormal{ partial result}: ~ p_1: ~y=0.5x^2+3x-2.5]$
			\item The equation of the parabola $p_2$ has the general form $y=-0.25x^2 -2.25x-2.5$
			\item Calculate the vertex coordinates and also draw the parabola $p_2$ in the coordinate system of exercise 1. a).
			\item Calculate the coordinates of the intersection points $A$ and $C$ of the two parabolas $p_1$ and $p_2$. \\
			$[\textnormal{ partial result}:~x_A=-7; \quad x_C=0]$
			\item The points $A$, $B_n$ and $C$ form triangles $AB_nC$, where:
			$$B_n \in p_1 ~\textnormal{with} ~ x \in ]-7;0[ ~\textnormal{and}~ x \in \mathbb{R}$$
			Draw the triangles $AB_1C$ for $x_1=-4$ and $AB_2C$ for $x_2=-1$ in the coordinate system of exercise 1. a).
			\item Calculate the area of the triangles $AB_nC$ depending on $x$.\\
			$[\textnormal{ partial result}:~ A(x)=(-1.75x^2-12.25x) FE]$
			\item The triangle $AB_3C$ has the largest area. Calculate the corresponding value for $x_3$ and give $A_{max}$.
		\end{enumerate}
		
		%Exercise 2
		{\bf \item One side of a rectangle is $\SI{6}{\centi \meter}$ long, the length of the other side is $\SI{8}{\centi \meter}$.} New rectangles are obtained , by lengthening the shorter side by $2x~\SI{}{\centi\meter}$ and shortening the longer side by $x~\SI{}{\centi\meter}$. The following applies: $0<x<8$.
		\begin{enumerate}[a)]
			\item Calculate the area $A(x)$ depending on $x$.\\
			$[\textnormal{ partial result}:~ A(x)=(-2x^2+10x+48)\SI{}{\centi\meter}^2]$
			\item Determine mathematically at which $x$ the area is $\SI{20}{\centi\meter^2}$.
		\end{enumerate}
		
		{\bf \item Given is the set of parabolas $$p_a \quad \textnormal{with} \quad y = x^2 -2ax + a^2 + 0.5a - 2; \quad \mathbb{G} = \mathbb{R} \times \mathbb{R} $$}
		
		\begin{enumerate}[a)]
			\item Determine the vertex form of the equation depending on a. [Intermediate result:
			$S_a(a|0.5a-2)$]
			\item Give the vertex coordinates of the points $S_a$ for $a \in \{-2; 0; -3 \}$ and draw the associated parabolas in a coordinate system.
			\item Calculate the equation of the support graph of all vertices of the parabolic family $p_a$.
			\item The point $P(8|62)$ lies on the parabolas $p_6$ and $p_7$ of the family. Determine the corresponding equations in normal form. [2 solutions!]
			\item Determine the equation of the upward-opening normal parabola that passes through the points $A(-1.5|-0.75)$ and $B(1|8)$. Does the parable belong to the group mentioned above? Reasons!
		\end{enumerate}
		
		{\bf \item Given is the quadratic function $f$ with $f(x)= -1.5x^2+6x-4.5$}
		
		\begin{enumerate}[a)]
			\item Determine the coordinates of the vertex of the associated parabola and specify the set of values for the function $f$.
			\item Check whether the point $P(-11|-25)$ is on, above or below the parabola.
		\end{enumerate}
		
		{\bf \item The graph of the function $h$ with $h(x)=4(x+1.5)^2-3 $ is mirrored on the x-axis and then on the y-axis.} Enter one Equation for the resulting parabola. Reasons.
		
		\begin{minipage}{0.45\textwidth}
			{\bf \item Quadreck Castle is being attacked by bandits.} Arrows are used for defense. The trajectory of the arrows corresponds to a parabola. The point $A(50|20)$ is the firing point of the arrows and is on the front tower. The trajectory of the arrows reaches its highest point at point $S(80|31.25)$.
		\end{minipage}
		\hfill
		\begin{minipage}{0.45\textwidth}
			\includegraphics[width=7 cm]{burg039}
		\end{minipage}
		
		\begin{enumerate}[a)]
			\item Determine the quadratic function $g$ that describes the trajectory of the arrows in the form
			$g(x)=a(x-b)^2+c$
			\item Calculate at which x coordinate the arrow hits the ground. How far did the arrow fly?
		\end{enumerate}
		
		{\bf \item Only the zeros $x_1=-4.5$ and $x_2=3.5$ are known for a parabola.} What is the x-coordinate of the vertex?
		
	\end{enumerate}
	
	\fbox{
		\begin{minipage}{0.5\textwidth}
			For the solution, please \href{https://www.okuyakl.de/math/m9paraL039/le039.pdf}{click here} or scan the QR code.\\
			Additional worksheets are available at
			
			\href{http://www.okuyakl.com}{www.okuyakl.com}
		\end{minipage}
		\hfill
				\begin{minipage}{0.4\textwidth}
				\includegraphics[width=1.5 cm]{../../viecher/zwe03}
				\includegraphics[width=3 cm]{qre039}
				\includegraphics[width=2 cm]{../../viecher/afanticon1}
				
				\end{minipage}}
				
		\end{document}