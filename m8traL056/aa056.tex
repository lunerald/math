\documentclass[a4paper]{article}
\usepackage[pdftex]{graphicx}
\usepackage[utf8]{inputenc}
\usepackage{enumerate}
\usepackage{icomma}
\usepackage{siunitx}
\sisetup{locale=DE} 
\usepackage{amssymb}
\usepackage{geometry}
\geometry{a4paper, top=15mm, left=15mm, right=15mm, bottom=15mm,
	headsep=10mm, footskip=12mm}
\usepackage{href-ul}
\hypersetup{
	colorlinks=true,
	linkcolor=blue,
	urlcolor=blue}
	
\begin{document}
{\bf Trapeze}
\begin{enumerate}[1.]

{\bf \item Die vier Punkte $P(-1,5|-2)$, $A(x|y)$, $U(5|3)$ und $L(0|3)$ legen das Trapez $PAUL$ mit den Parallelseiten $[PA]$ und $[UL]$, dessen Flächeninhalt  $\SI{30}{\centi\meter^2}$ beträgt, fest.}

\begin{enumerate}[a)]
\item Zeichne die Punkte $P$, $U$ und $L$ in ein Koordinatensystem ( Einheit 1 cm) ein, ermittle die Koordinaten des Punktes $A$, gib dessen Koordinaten an und vervollständige das Trapez.
\item Das Parallelogramm $PA'U'L$ hat einen Flächeninhalt von $\SI{15}{\centi\meter^2}$ . Berechne die x-Koordinaten der Punkte $A'(x|-2)$ und $U'(x|3)$. Zeichne das Parallelogramm mit blauer Farbe ein.
\item Berechne den Bruchteil der Trapezfläche $PAUL$, die im II. und III. Quadranten liegt. Gib sie in Prozent an.
\end{enumerate}

\begin{minipage}{0.6\textwidth}
{\bf \item Stelle einen Term für den Flächeninhalt der nebenstehenden Figur auf.}
\end{minipage}
\begin{minipage}{0.4\textwidth}
\includegraphics[width=4 cm]{haus249}
\end{minipage}

{\bf \item Vom Trapez ABCD ist die Grundseite $g_1=\SI{6,6}{\centi\meter}$, die Höhe $h=\SI{7,5}{\centi\meter}$ und der Flächeninhalt $A=\SI{30}{\centi\meter^2}$ gegeben.} Berechne die Länge der Seite $g_2$.
\end{enumerate} 

\fbox{
	\begin{minipage}{0.5\textwidth}
		Zur Lösung bitte \href{https://www.okuyakl.de/math/m8traL056/ll056.pdf}{hier klicken} oder den QR-Code scannen.\\
	Weitere Arbeitsblätter gibt es unter 
	
	\href{https://www.okuyakl.de}{www.okuyakl.de}
	\end{minipage}
	\hfill
	\begin{minipage}{0.4\textwidth}
		\includegraphics[width=1.5 cm]{../../viecher/zwe03}
		\includegraphics[width=3 cm]{qr056}
		\includegraphics[width=2 cm]{../../viecher/afanticon1}
		
	\end{minipage}}

\end{document}%Lösung-------------------------------------------
